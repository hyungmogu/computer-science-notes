\documentclass[12pt]{article}
\usepackage[margin=2.5cm]{geometry}
\usepackage{enumerate}
\usepackage{amsfonts}
\usepackage{amsmath}
\usepackage{fancyhdr}
\usepackage{amsmath}
\usepackage{amssymb}
\usepackage{amsthm}
\usepackage{mdframed}
\usepackage{graphicx}
\usepackage{subcaption}
\usepackage{adjustbox}
\usepackage{listings}
\usepackage{xcolor}
\usepackage{booktabs}
\usepackage[utf]{kotex}

\definecolor{codegreen}{rgb}{0,0.6,0}
\definecolor{codegray}{rgb}{0.5,0.5,0.5}
\definecolor{codepurple}{rgb}{0.58,0,0.82}
\definecolor{backcolour}{rgb}{0.95,0.95,0.92}

\lstdefinestyle{mystyle}{
    backgroundcolor=\color{backcolour},
    commentstyle=\color{codegreen},
    keywordstyle=\color{magenta},
    numberstyle=\tiny\color{codegray},
    stringstyle=\color{codepurple},
    basicstyle=\ttfamily\footnotesize,
    breakatwhitespace=false,
    breaklines=true,
    captionpos=b,
    keepspaces=true,
    numbers=left,
    numbersep=5pt,
    showspaces=false,
    showstringspaces=false,
    showtabs=false,
    tabsize=1
}

\lstset{style=mystyle}

\begin{document}
\title{CSC209 Week 2 Notes}
\author{Hyungmo Gu}
\maketitle

\section*{Shell Programming 1 of 6}
\begin{itemize}
    \item \textit{*.\textless EXTENSION\textgreater}
    \begin{itemize}
    \item returns all items under the extension

    \begin{lstlisting}[language=bash]
    # computer-science-notes folder
    >>> echo *.pdf
    common_mistakes_in_proofs.pdf sample.pdf
    \end{lstlisting}

    \end{itemize}
    \item \textit{read}
    \begin{itemize}
    \item store values in variable interactively``

    \begin{lstlisting}[language=bash]
    >>> read x y
    >>> hello world 2
    >>> echo $x
    hello
    >>> echo $y
    world 2
    \end{lstlisting}

    \end{itemize}
    \item \textit{``}
    \begin{itemize}
    \item store commands in variable

    \begin{lstlisting}[language=bash]
    >>> i = `expr 4 + 1`
    >>> echo $i
    5
    \end{lstlisting}
    \end{itemize}
\end{itemize}

\bigskip

\section*{Shell Programming 2 of 6}

\bigskip

\begin{itemize}
    \item \textit{test}
    \begin{itemize}
    \item checks file types and compares values
    \item used in if and while statement to check condition
    \item \textbf{test \textless EXPRESSION \textgreater} is equivalent to
    \textbf{[ EXPRESSION ]}

    \begin{lstlisting}[language=bash]
    >>> test 2 -lt 3
    >>> echo $?
    0
    \end{lstlisting}

    \item has the following numeric comparison operators
    \begin{enumerate}[1.]
    \item \textbf{-lt:} less than
    \item \textbf{-gt:} greater than
    \item \textbf{-eq:} equal to
    \item \textbf{-ne:} not equal
    \item \textbf{-le:} less than
    \item \textbf{-ge:} greater than
    \end{enumerate}

    \item has the following file testing operators
    \begin{enumerate}[1.]
    \item \textbf{-f file:} file exists and is a plain file
    \item \textbf{-d file:} file exists and is a directory
    \item \textbf{-s file:} file exists and is a plain file of non-zero size
    \end{enumerate}

    \end{itemize}

\end{itemize}

\end{document}