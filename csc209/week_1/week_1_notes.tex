\documentclass[12pt]{article}
\usepackage[margin=2.5cm]{geometry}
\usepackage{enumerate}
\usepackage{amsfonts}
\usepackage{amsmath}
\usepackage{fancyhdr}
\usepackage{amsmath}
\usepackage{amssymb}
\usepackage{amsthm}
\usepackage{mdframed}
\usepackage{graphicx}
\usepackage{subcaption}
\usepackage{adjustbox}
\usepackage{listings}
\usepackage{xcolor}
\usepackage{booktabs}
\usepackage[utf]{kotex}

\definecolor{codegreen}{rgb}{0,0.6,0}
\definecolor{codegray}{rgb}{0.5,0.5,0.5}
\definecolor{codepurple}{rgb}{0.58,0,0.82}
\definecolor{backcolour}{rgb}{0.95,0.95,0.92}

\lstdefinestyle{mystyle}{
    backgroundcolor=\color{backcolour},
    commentstyle=\color{codegreen},
    keywordstyle=\color{magenta},
    numberstyle=\tiny\color{codegray},
    stringstyle=\color{codepurple},
    basicstyle=\ttfamily\footnotesize,
    breakatwhitespace=false,
    breaklines=true,
    captionpos=b,
    keepspaces=true,
    numbers=left,
    numbersep=5pt,
    showspaces=false,
    showstringspaces=false,
    showtabs=false,
    tabsize=1
}

\lstset{style=mystyle}

\begin{document}
\title{CSC209 Week 1 Notes}
\author{Hyungmo Gu}
\maketitle

\section*{The Unix Command Line}
\begin{itemize}
\item cat
\begin{itemize}
    \item \textit{cat hello} : reads file

    \begin{lstlisting}[language=Python]
    >>> cat hello
    Hello World!
    \end{lstlisting}
    \item \textit{cat hello1 \textgreater\:hello2} : transfer contents from file \textit{hello1} to \textit{hello2}

    \begin{lstlisting}[language=Python]
    >>> cat hello > hello1
    >>> cat hello1
    Hello World!
    \end{lstlisting}

    \begin{itemize}
        \item \textit{hello2} created if doesn't exist
    \end{itemize}
\end{itemize}
\item rm
\begin{itemize}
    \item \textit{-i} in \textit{rm -i} : creates prompt
    \item \textit{-r} in \textit{rm -r} : removes files recursively
    \item \textit{-f} in \textit{rm -r -f} : supresses prompt (also removes files recursively).

    \bigskip

    Don't use \textit{-f} in \textit{rm -r -f} unnecess necessary.
\end{itemize}
\item alias
\begin{itemize}
    \item A shell alias is a shortcut to reference a command
    \item avoids typing long commands
    \begin{lstlisting}[language=Python]
    >>> alias rm='rm -i'
    >>> rm  hello
    remove hello?
    \end{lstlisting}
    \item \textit{unalias} - remove alias
\end{itemize}
\end{itemize}

\bigskip

\section*{Software Tools Part 1 of 2}
\begin{itemize}
    \item Software Tool Principles
    \begin{enumerate}[1.]
        \item Does one thing well
        \item Are small
        \item Interfaced cleanly
        \item Expect the output of every program to become the input of another
        \item Program inputs easy to generate or type

        \bigskip

        i.e. [NO]: binary input tpye.
        \item Use regular expressions for all pattern matching
    \end{enumerate}
\end{itemize}

\bigskip

\section*{Software Tools Part 2 of 2}
\begin{itemize}
    \item Software Tools in Unix
    \begin{enumerate}[1.]
        \item \textit{grep}
        \begin{itemize}
        \item outputs lines which match a pattern

        \begin{lstlisting}
        >>> grep '1' sample.txt
        1
        10
        \end{lstlisting}
        \end{itemize}

        \item \textit{head, tail}
        \begin{itemize}
        \item Returns the first or last $n$ lines of file

        \begin{lstlisting}
        >>> head -10 sample.txt
        1
        2
        3
        4
        5
        \end{lstlisting}

        \end{itemize}

        \item \textit{sort}
        \begin{itemize}
            \item Sorts the input
            \item \textit{-n} does numeric sorting in file
        \begin{lstlisting}
        >>> sort -10 sample.txt
        1
        10
        2
        3
        4
        5
        6
        7
        8
        9
        \end{lstlisting}

        \end{itemize}
        \item \textit{uniq}
        \begin{itemize}
            \item Collapses adjacent identucal lines
        \end{itemize}
        \begin{lstlisting}
        >>> uniq sample2.txt
        Hot
        Cold
        Hot
        Cold
        Warm
        \end{lstlisting}
        \item \textit{sed}
        \begin{itemize}
            \item Collapses adjacent identucal lines
            \item \textit{s} in \textit{s/\textless FROM \textgreater/ \textless TO \textgreater} means subtitution
        \end{itemize}

        \bigskip

        \begin{lstlisting}
        >>> cat sample3.txt
        Hello
        Hello
        >>> sed 's/Hello/hi' sample3.txt
        hi
        hi
        \end{lstlisting}

        \item \textit{diff}
        \begin{itemize}
            \item Compares two files in a text-oriented way
        \end{itemize}

        \begin{lstlisting}
        diff -c sample4_1.txt sample4_2.txt
        *** sample4_1.txt       2020-05-08 14:16:47.000000000 +0900
        --- sample4_2.txt       2020-05-08 14:16:59.000000000 +0900
        ***************
        *** 1 ****
        ! Hello beautiful world
        \ No newline at end of file
        --- 1 ----
        ! Hello wonderful world.
        \ No newline at end of file
        \end{lstlisting}

        \item \textit{comm}
        \begin{itemize}
            \item Compares two files in a text-oriented way
        \end{itemize}

        \begin{lstlisting}
        comm sample6_1.txt sample6_2.txt
        001 Fred Flintstone
                002 80
        002 Wilma Flintstone
                003 90
        003 Betty Rubble
                004 75
        004 Barney Rubble
                005 0
        005 Pebbles Flintstone
        006 Bam-Bam Rubble      007 68%
        \end{lstlisting}


        \item \textit{join}
        \begin{itemize}
            \item Performs database join of two files

        \end{itemize}

        \begin{lstlisting}
        >>> join sample6_1.txt sample6_2.txt
        002 Wilma Flintstone 80
        003 Betty Rubble 90
        004 Barney Rubble 75
        005 Pebbles Flintstone 0
        \end{lstlisting}

    \end{enumerate}
\end{itemize}

\end{document}