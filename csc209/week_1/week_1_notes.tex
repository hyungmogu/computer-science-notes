\documentclass[12pt]{article}
\usepackage[margin=2.5cm]{geometry}
\usepackage{enumerate}
\usepackage{amsfonts}
\usepackage{amsmath}
\usepackage{fancyhdr}
\usepackage{amsmath}
\usepackage{amssymb}
\usepackage{amsthm}
\usepackage{mdframed}
\usepackage{graphicx}
\usepackage{subcaption}
\usepackage{adjustbox}
\usepackage{listings}
\usepackage{xcolor}
\usepackage{booktabs}
\usepackage[utf]{kotex}

\definecolor{codegreen}{rgb}{0,0.6,0}
\definecolor{codegray}{rgb}{0.5,0.5,0.5}
\definecolor{codepurple}{rgb}{0.58,0,0.82}
\definecolor{backcolour}{rgb}{0.95,0.95,0.92}

\lstdefinestyle{mystyle}{
    backgroundcolor=\color{backcolour},
    commentstyle=\color{codegreen},
    keywordstyle=\color{magenta},
    numberstyle=\tiny\color{codegray},
    stringstyle=\color{codepurple},
    basicstyle=\ttfamily\footnotesize,
    breakatwhitespace=false,
    breaklines=true,
    captionpos=b,
    keepspaces=true,
    numbers=left,
    numbersep=5pt,
    showspaces=false,
    showstringspaces=false,
    showtabs=false,
    tabsize=1
}

\lstset{style=mystyle}

\begin{document}
\title{CSC209 Week 1 Notes}
\author{Hyungmo Gu}
\maketitle

\section*{The Unix Command Line}
\begin{itemize}
\item cat
\begin{itemize}
    \item \textit{cat hello} : reads file

    \begin{lstlisting}[language=Python]
    >>> cat hello
    Hello World!
    \end{lstlisting}
    \item \textit{cat hello1 \textgreater\:hello2} : transfer contents from file \textit{hello1} to \textit{hello2}

    \begin{lstlisting}[language=Python]
    >>> cat hello > hello1
    >>> cat hello1
    Hello World!
    \end{lstlisting}

    \begin{itemize}
        \item \textit{hello2} created if doesn't exist
    \end{itemize}
\end{itemize}
\item rm
\begin{itemize}
    \item \textit{-i} in \textit{rm -i} : creates prompt
    \item \textit{-r} in \textit{rm -r} : removes files recursively
    \item \textit{-f} in \textit{rm -r -f} : supresses prompt (also removes files recursively).

    \bigskip

    Don't use \textit{-f} in \textit{rm -r -f} unnecess necessary.
\end{itemize}
\item alias
\begin{itemize}
    \item A shell alias is a shortcut to reference a command
    \item avoids typing long commands
    \begin{lstlisting}[language=Python]
    >>> alias rm='rm -i'
    >>> rm  hello
    remove hello?
    \end{lstlisting}
    \item \textit{unalias} - remove alias
\end{itemize}
\end{itemize}
\end{document}