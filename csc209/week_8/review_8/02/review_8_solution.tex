\documentclass[12pt]{article}
\usepackage[margin=2.5cm]{geometry}
\usepackage{enumerate}
\usepackage{amsfonts}
\usepackage{amsmath}
\usepackage{fancyhdr}
\usepackage{amsmath}
\usepackage{amssymb}
\usepackage{amsthm}
\usepackage{mdframed}
\usepackage{graphicx}
\usepackage{subcaption}
\usepackage{adjustbox}
\usepackage{listings}
\usepackage{xcolor}
\usepackage{booktabs}
\usepackage[utf]{kotex}
\usepackage{hyperref}

\definecolor{codegreen}{rgb}{0,0.6,0}
\definecolor{codegray}{rgb}{0.5,0.5,0.5}
\definecolor{codepurple}{rgb}{0.58,0,0.82}
\definecolor{backcolour}{rgb}{0.95,0.95,0.92}

\lstdefinestyle{mystyle}{
    backgroundcolor=\color{backcolour},
    commentstyle=\color{codegreen},
    keywordstyle=\color{magenta},
    numberstyle=\tiny\color{codegray},
    stringstyle=\color{codepurple},
    basicstyle=\ttfamily\footnotesize,
    breakatwhitespace=false,
    breaklines=true,
    captionpos=b,
    keepspaces=true,
    numbers=left,
    numbersep=5pt,
    showspaces=false,
    showstringspaces=false,
    showtabs=false,
    tabsize=1
}

\lstset{style=mystyle}

\pagestyle{fancy}
\renewcommand{\headrulewidth}{0.4pt}
\lhead{CSC 209}
\rhead{Review 8 Solution}

\begin{document}
\title{CSC 209 Review 8 Solution}
\maketitle

\bigskip

\begin{enumerate}[1.]
    \item

\begin{lstlisting}[language=c]
    int *my_malloc (int n) {
        int *res;

        res = malloc(n * sizeof(int));
        if (res == NULL) {
            perror("Allocation failed.");
        }

        return res;
    }
\end{lstlisting}

    \bigskip

    Please see \texttt{question\_1.c} for details.

    \item

\begin{lstlisting}[language=c]
    char *duplicate(char *str) {
        char *res;

        res = malloc(strlen(str) + 1);
        if (res == NULL) {
            return res;
        }

        strcpy(res, str);

        return res;
    }
\end{lstlisting}

    \bigskip

    Please see \texttt{question\_2.c} for details.

    \item

\begin{lstlisting}[language=c]
    int *create_array(int n, int initial_value) {
        int *p, *res;

        res = malloc(n * sizeof(int));

        if (res == NULL) {
            return res;
        }

        for (p = res; p < res + n; p++) {
            *p = initial_value;
        }

        return res;
    }
\end{lstlisting}

    \bigskip

    Please see \texttt{question\_3.c} for details.

    \item


    \begin{lstlisting}[language=c]
        int main(void) {
            struct point {int x, y};
            struct rectangle {struct point upper_left, lower_right};
            struct rectangle *p;

            p = malloc(sizeof(struct rectangle));

            p.upper_left.x = 10;
            p.upper_left.y = 25;

            p.lower_right.x = 20;
            p.lower_right.y = 15;

            printf("%d %d\n", p.upper_left.x, p.upper_left.y);
            printf("%d %d\n", p.lower_right.x, p.lower_right.y);

            free(p);

            return 0;
        }

    \end{lstlisting}


    Please see \texttt{question\_4.c} for details.

    \item

    \texttt{b)}, \texttt{c)} and \texttt{d)} are legal.

    \bigskip

    \begin{mdframed}
    \underline{\textbf{Correct Solution}}

    \bigskip

    \texttt{b)}, \texttt{c)} are legal.
    \end{mdframed}

    \bigskip

    \underline{\textbf{Notes}}

    \begin{itemize}
        \item \textbf{The -$>$ Operator}
        \begin{itemize}
            \item doesn't carry over to accessing nested members. Only works when
            struct is a pointer

            \bigskip

            \underline{\textbf{Example}}

            \bigskip

            \texttt{p-$>$upper\_left.x}
        \end{itemize}
    \end{itemize}

    \item

\begin{lstlisting}[language=c]
    struct node *delete_from_list(struct node *list, int n)
    {
        struct node *cur = list, *temp;

        if (cur->value == n) {
            list = cur->next;
            return list;
        }

        for (cur = list; cur != NULL; cur = cur -> next) {

            if (cur->next != NULL && cur->next->value == n) {
            break;
            }
        }

        if (cur == NULL) {
            return list;
        }

        temp = cur->next;
        cur->next = cur->next->next;

        free(temp);
        return list;
    }
\end{lstlisting}

    \item

    It's incorrect because it's deleting the node before moving to next.

    \bigskip

    To fix this bug, p must move to the next node before removing the current.

\begin{lstlisting}[language=c]
    struct node *temp;
    p = first;
    while (p != NULL) {
        temp = p;
        p = p -> next
        remove(temp);
    }
\end{lstlisting}

    \item

\begin{lstlisting}[language=c]
    struct node *temp;
    p = first;
    while (p != NULL) {
        temp = p;
        p = p -> next
        remove(temp);
    }
\end{lstlisting}

    \item

\begin{lstlisting}[language=c]
    struct node *temp;
    p = first;
    while (p != NULL) {
        temp = p;
        p = p -> next
        remove(temp);
    }
\end{lstlisting}

    \item

    Please see file \texttt{question\_8/stack.h}, \texttt{question\_8/stack.c},
    \texttt{question\_8/calc.c} for details.


\end{enumerate}

\end{document}