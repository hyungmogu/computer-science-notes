\documentclass[12pt]{article}
\usepackage[margin=2.5cm]{geometry}
\usepackage{enumerate}
\usepackage{amsfonts}
\usepackage{amsmath}
\usepackage{fancyhdr}
\usepackage{amsmath}
\usepackage{amssymb}
\usepackage{amsthm}
\usepackage{mdframed}
\usepackage{graphicx}
\usepackage{subcaption}
\usepackage{adjustbox}
\usepackage{listings}
\usepackage{xcolor}
\usepackage{booktabs}
\usepackage[utf]{kotex}
\usepackage{hyperref}

\definecolor{codegreen}{rgb}{0,0.6,0}
\definecolor{codegray}{rgb}{0.5,0.5,0.5}
\definecolor{codepurple}{rgb}{0.58,0,0.82}
\definecolor{backcolour}{rgb}{0.95,0.95,0.92}

\lstdefinestyle{mystyle}{
    backgroundcolor=\color{backcolour},
    commentstyle=\color{codegreen},
    keywordstyle=\color{magenta},
    numberstyle=\tiny\color{codegray},
    stringstyle=\color{codepurple},
    basicstyle=\ttfamily\footnotesize,
    breakatwhitespace=false,
    breaklines=true,
    captionpos=b,
    keepspaces=true,
    numbers=left,
    numbersep=5pt,
    showspaces=false,
    showstringspaces=false,
    showtabs=false,
    tabsize=1
}

\lstset{style=mystyle}

\pagestyle{fancy}
\renewcommand{\headrulewidth}{0.4pt}
\lhead{CSC 209}
\rhead{Review 8 Solution}

\begin{document}
\title{CSC 209 Review 8 Solution}
\maketitle

\bigskip

\begin{enumerate}[1.]
    \item

\begin{lstlisting}[language=c]
    int *my_malloc (int n) {
        int *res;

        res = malloc(n * sizeof(int));
        if (res == NULL) {
            perror("Allocation failed.");
        }

        return res;
    }
\end{lstlisting}

    \bigskip

    Please see \texttt{question\_1.c} for details.

    \item

\begin{lstlisting}[language=c]
    char *duplicate(char *str) {
        char *res;

        res = malloc(strlen(str) + 1);
        if (res == NULL) {
            return res;
        }

        strcpy(res, str);

        return res;
    }
\end{lstlisting}

    \bigskip

    Please see \texttt{question\_2.c} for details.

    \item

\begin{lstlisting}[language=c]
    int *create_array(int n, int initial_value) {
        int *p, *res;

        res = malloc(n * sizeof(int));

        if (res == NULL) {
            return res;
        }

        for (p = res; p < res + n; p++) {
            *p = initial_value;
        }

        return res;
    }
\end{lstlisting}

    \bigskip

    Please see \texttt{question\_3.c} for details.

    \item


    \begin{lstlisting}[language=c]
        int main(void) {
            struct point {int x, y};
            struct rectangle {struct point upper_left, lower_right};
            struct rectangle *p;

            p = malloc(sizeof(struct rectangle));

            p.upper_left.x = 10;
            p.upper_left.y = 25;

            p.lower_right.x = 20;
            p.lower_right.y = 15;

            printf("%d %d\n", p.upper_left.x, p.upper_left.y);
            printf("%d %d\n", p.lower_right.x, p.lower_right.y);

            free(p);

            return 0;
        }

    \end{lstlisting}


    Please see \texttt{question\_4.c} for details.

    \item

    \texttt{b)}, \texttt{c)} and \texttt{d)} are legal.

    \bigskip

    \begin{mdframed}
    \underline{\textbf{Correct Solution}}

    \bigskip

    \texttt{b)}, \texttt{c)} are legal.
    \end{mdframed}

    \bigskip

    \underline{\textbf{Notes}}

    \begin{itemize}
        \item \textbf{The -$>$ Operator}
        \begin{itemize}
            \item doesn't carry over to accessing nested members. Only works when
            struct is a pointer

            \bigskip

            \underline{\textbf{Example}}

            \bigskip

            \texttt{p-$>$upper\_left.x}
        \end{itemize}
    \end{itemize}

    \item

\begin{lstlisting}[language=c]
    struct node *delete_from_list(struct node *list, int n)
    {
        struct node *cur = list, *temp;

        if (cur->value == n) {
            list = cur->next;
            return list;
        }

        for (cur = list; cur != NULL; cur = cur -> next) {

            if (cur->next != NULL && cur->next->value == n) {
            break;
            }
        }

        if (cur == NULL) {
            return list;
        }

        temp = cur->next;
        cur->next = cur->next->next;

        free(temp);
        return list;
    }
\end{lstlisting}

    \item

    It's incorrect because it's deleting the node before moving to next.

    \bigskip

    To fix this bug, p must move to the next node before removing the current.

\begin{lstlisting}[language=c]
    struct node *temp;
    p = first;
    while (p != NULL) {
        temp = p;
        p = p -> next
        remove(temp);
    }
\end{lstlisting}

    \item

    Please see file \texttt{question\_8/stack.h}, \texttt{question\_8/stack.c},
    \texttt{question\_8/calc.c} for details.

    \item

    True.

    \bigskip

    By definition, \texttt{(\&x)$->$a} is the same as \texttt{(*(\&x)).a}.

    \bigskip

    Since \texttt{(*(\&x)) = x}, we can write \texttt{(\&x)$->$a} is the same
    as x.a.

    \item

\begin{lstlisting}[language=c]
    void print_part(struct part *p)
    {
        printf("Part number: %d\n", p->number);
        printf("Part name: %s\n", p->name);
        printf("Quantity on hand: %d\n", p->on_hand);
    }
\end{lstlisting}

    \item

    Please see \texttt{question\_11.c} for details.

    \item

\begin{lstlisting}[language=c]
    struct node {
        int value;
        struct node *next;
    };

    struct node *find_last(struct node *list, int n)
    {
        struct node *res = NULL, *p;

        for (p = list; p != NULL; p = p->next) {
            if (p->value == n) {
                res = p;
            }
        }

        return res;
    }
\end{lstlisting}

    \bigskip

    Please see file \texttt{question\_12.c} for details.

    \item

\begin{lstlisting}[language=c]
    struct node {
        int value;
        struct node *next;
    };

    struct node *insert_into_ordered_list(struct node *list, struct node *new_node)
    {
        struct node *cur = list, *prev = NULL;

        if (list == NULL) {
            list = new_node;
            return list;
        }

        while (cur != NULL && cur->value <= new_node ->value) {
            prev = cur;
            cur = cur->next;
        }

        prev->next = new_node;
        new_node->next = cur;
        return list;
    }
\end{lstlisting}

    \item

\begin{lstlisting}[language=c]
    struct node *delete_from_list(struct node **list, int n)
    {
        struct node *cur, *prev;

        for (cur = *list, prev = NULL;
                cur != NULL && cur->value != n;
                prev = cur, cur = cur->next)

                ;

        if (cur == NULL) {
            return (*list);
        }

        if (prev == NULL) {
            *list = (*list)->next;
        } else {
            prev->next = cur->next;
        }

        free(cur);
        return (*list);
    }
\end{lstlisting}

    \bigskip

    Please see \texttt{question\_14.c} for details.

    \item

    It returns the value of \texttt{n} that is equal to \texttt{i * i + i - 12}.

    \bigskip

    Here, the value of \texttt{n} is 3.

    \bigskip

    Please see \texttt{question\_15.c} for details.

    \item


\begin{lstlisting}[language=c]
    int sum(int (*f)(int), int start, int end)
    {
        int res = 0;
        for (int i = start; i <= end; i++) {
            res += (*f)(i);
        }

        return res;
    }
\end{lstlisting}

    \bigskip

    Please see \texttt{question\_16.c} for details.

    \item

\begin{lstlisting}[language=c]
    qsort(&a[50], 50, sizeof(a[0]), compare_parts);
\end{lstlisting}

    \item

\begin{lstlisting}[language=c]
    int compare_parts(const void *p, const void *q)
    {
        return ((struct part *) q)->number - ((struct part *) p)->number;
    }
\end{lstlisting}

    \item

\begin{lstlisting}[language=c]
    struct {
        char *cd_name;
        void (*cmd_pointer)(void);
    } file_cmd[] = {
        {"new", new_cmd},
        {"open", open_cmd},
        {"close", close_cmd},
        {"close all", close_all_cmd},
        {"save", save_cmd},
        {"save as", save_as_cmd},
        {"save all", save_all_cmd},
        {"print", print_cmd},
        {"exit", exit_cmd}
    } ;


    void run_cmd(const char *cmd)
    {
        int cmd_cnt = sizeof(file_cmd)/sizeof(file_cmd[0]);
        char cmd_cpy[21], *p;

        strcpy(cmd_cpy, cmd);
        if (strlen(cmd) == 21) {
            cmd_cpy[20] = '\0';
        }

        for (p = cmd_cpy; *p != '\0'; p++)
        {
            *p = tolower(*p);
        }

        for (int i = 0; i < cmd_cnt; i++) {
            if (strcmp((file_cmd[i]).cmd_name, cmd_cpy) == 0) {
                return (file_cmd[i]).cmd_pointer();
            }
        }
    }

    void new_cmd(void)
    {
        printf("new_cmd\n");
    }
    void open_cmd(void)
    {
        printf("open_cmd\n");
    }
    void close_cmd(void)
    {
        printf("close_cmd\n");
    }
    void close_all_cmd(void)
    {
        printf("close_all_cmd\n");
    }
    void save_cmd(void)
    {
        printf("save_cmd\n");
    }
    void save_as_cmd(void)
    {
        printf("save_as_cmd\n");
    }
    void save_all_cmd(void)
    {
        printf("save_all_cmd\n");
    }
    void print_cmd(void)
    {
        printf("print_cmd\n");
    }
    void exit_cmd(void)
    {
        printf("exit_cmd\n");
    }
\end{lstlisting}

    \bigskip

    Please see \texttt{question\_19.c} for details.

    \item

    Please see \texttt{question\_20.c} for details.

    \item

    Please see \texttt{question\_21.c} for details.



\end{enumerate}

\end{document}