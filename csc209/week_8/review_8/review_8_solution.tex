\documentclass[12pt]{article}
\usepackage[margin=2.5cm]{geometry}
\usepackage{enumerate}
\usepackage{amsfonts}
\usepackage{amsmath}
\usepackage{fancyhdr}
\usepackage{amsmath}
\usepackage{amssymb}
\usepackage{amsthm}
\usepackage{mdframed}
\usepackage{graphicx}
\usepackage{subcaption}
\usepackage{adjustbox}
\usepackage{listings}
\usepackage{xcolor}
\usepackage{booktabs}
\usepackage[utf]{kotex}
\usepackage{hyperref}

\definecolor{codegreen}{rgb}{0,0.6,0}
\definecolor{codegray}{rgb}{0.5,0.5,0.5}
\definecolor{codepurple}{rgb}{0.58,0,0.82}
\definecolor{backcolour}{rgb}{0.95,0.95,0.92}

\lstdefinestyle{mystyle}{
    backgroundcolor=\color{backcolour},
    commentstyle=\color{codegreen},
    keywordstyle=\color{magenta},
    numberstyle=\tiny\color{codegray},
    stringstyle=\color{codepurple},
    basicstyle=\ttfamily\footnotesize,
    breakatwhitespace=false,
    breaklines=true,
    captionpos=b,
    keepspaces=true,
    numbers=left,
    numbersep=5pt,
    showspaces=false,
    showstringspaces=false,
    showtabs=false,
    tabsize=1
}

\lstset{style=mystyle}

\pagestyle{fancy}
\renewcommand{\headrulewidth}{0.4pt}
\lhead{CSC 209}
\rhead{Review 8 Solution}

\begin{document}
\title{CSC 209 Review 8 Solution}
\maketitle

\bigskip

\begin{enumerate}[1.]
    \item

    I need to create a wrapper function \texttt{my\_malloc} that does the following:

    \bigskip

    \begin{itemize}
        \item ask \texttt{my\_malloc} it to allocate \texttt{n} bytes
        \item call \texttt{malloc}
        \item test \texttt{malloc} doesn't have a null pointer
        \item return pointer from \texttt{malloc}
    \end{itemize}

    \bigskip

    The solution to this problem is:

\begin{lstlisting}[language=c]
    void *my_malloc(int n) {
        void *p;

        p = malloc(n);

        if (!p) {
            printf("ERROR: Malloc allocation failed");
        }

        return p;
    }
\end{lstlisting}

    \underline{\textbf{Notes}}

    \begin{itemize}
        \item Learned that void function can return value
        \item \textbf{Dynamic Storage Allocation}

        \begin{itemize}
            \item Allows to allocate storage during program execution
            \item Allows to create data structures and shink and grow array as needed
            \item e.g. \texttt{malloc}, \texttt{calloc}, \texttt{realloc}
        \end{itemize}
        \item \textbf{Memory Allocation Functions}

        \begin{itemize}
            \item \texttt{malloc} - Allocates a block of memory but doesn't initialize it

            \begin{itemize}
                \item doesn't initialize the allocated memory
                \item more efficient than \texttt{calloc}
                \item accessing the content $\to$ segmentation fault (accessing value at invalid mem. location)
                or garbage values
            \end{itemize}

            \item \texttt{calloc} - Allocates a block of memory and clears it

            \begin{itemize}
                \item allocates memory and \underline{initializes the memory block} to zero
                \item accessing the content of blocks would return 0
            \end{itemize}

            \item \texttt{realloc} - Resizes a previously allocated block of memory
        \end{itemize}

        \item \textbf{Null Pointer}

        \begin{itemize}
            \item is returned when it fails to allocate a block of memory large
            enough to satisfy the request
        \end{itemize}

        \bigskip

        \underline{\textbf{Example}}

        \begin{center}
        \includegraphics[width=\linewidth]{images/review_8_solution_1.png}
        \end{center}
    \end{itemize}

    \item

    I need to write a function named \texttt{duplicate} that uses dynamic storage
    allocation to create a copy of a string.

    \bigskip

    The requirements of the function are

    \begin{itemize}
        \item \texttt{duplicate} allocates space for a string of the same length as \texttt{str}
        \item \texttt{duplicate} copies the contents of str into the new string
        \item \texttt{duplicate} returns a pointer to it
        \item \texttt{duplicate} returns a null pointer if the memory allocation fails
    \end{itemize}

    \bigskip

    The solution to this problem is:

    \bigskip

\begin{lstlisting}[language=c]
    #include <stdio.h>
    #include <stdlib.h>  // malloc
    #include <string.h>  // strlen

    char *duplicate(const char *str);

    int main(void) {
        char s[] = "hello world", *p;

        p = duplicate (s);

        printf("Duplicate: %s\n", p);

        free(p);
        return 0;
    }


    char *duplicate(const char *str) {
        char *p, *q;
        const char *r;

        int n = strlen(str);

        p = (char *)malloc(n + 1);

        if (!p) {
            return p;
        }

        r = str;
        q = p;
        while (r < str + n) {
            *q = *r;
            q++;
            r++;
        }

        *q = '\0';

        return p;
    }
\end{lstlisting}

    \bigskip

    \begin{mdframed}
    \underline{\textbf{Correct Solution:}}

\begin{lstlisting}[language=c]
    #include <stdio.h>
    #include <stdlib.h>  // malloc
    #include <string.h>  // strlen

    char *duplicate(const char *str);

    int main(void) {
        char s[] = "hello world", *p, *q;
        ;
        p = duplicate (s);

        printf("Duplicate: %s\n", p);

        free(p);
        return 0;
    }


    char *duplicate(const char *str) {
        char *p, *q;
        const char *r;

        int n = strlen(str);

        p = (char *)malloc(n + 1);

        if (!p) {
            p = ((void*)0);
            return p;
        }

        r = str;
        q = p;
        while (r < str + n) {
            *q = *r;
            q++;
            r++;
        }

        *q = '\0';

        return p;
    }
\end{lstlisting}

    \end{mdframed}

    \underline{\textbf{Note}}

    \begin{itemize}
        \item Null pointer has value \texttt{((void*)0)}
        \item \texttt{const} tag in parameter prevetns the function from modifying
        what it's pointer variable is pointing to.

        \begin{itemize}
            \item value is modifiable
            \item changes the parameter to pass by value
        \end{itemize}
    \end{itemize}

    \item

\begin{lstlisting}[language=c]
    int *create_array(int n, int initial_value) {
        int *array;

        array = malloc(n * sizeof(int));

        if (array == NULL) {
            return array;
        }

        for(int i = 0; i < n; i++){
            array[i] = initial_value;
        }

        return array
    }
\end{lstlisting}


    \bigskip

    \underline{\textbf{Notes}}

    \begin{itemize}
        \item \textbf{Dynamically Allocated Arrays}
        \begin{itemize}
            \item \textbf{Syntax:}

            \bigskip

            \texttt{int *a;}

            \texttt{a = malloc(n * sizeof(int));}

            \bigskip
            \item returns null pointer if allocation fails
        \end{itemize}
    \end{itemize}

    \item

    \bigskip

\begin{lstlisting}[language=c]
    #include <stdio.h>
    #include <stdlib.h>
    #include <string.h>

    struct point {int x, y;};
    struct rectangle {struct point upper_left, lower_right;};

    int main(void) {

        struct rectangle *p;

        p = malloc(sizeof(struct rectangle));

        p->upper_left.x = 10;
        p->upper_left.y = 25;
        p->lower_right.x = 20;
        p->lower_right.y = 15;

        printf("%d %d %d %d",
            p->upper_left.x,
            p->upper_left.y,
            p->lower_right.x,
            p->lower_right.y
        );

        return 0;
    }
\end{lstlisting}

    \underline{\textbf{Notes}}

    \begin{itemize}
        \item -$>$ doesn't carry over to accessing nested members. Only works when
        struct is a pointer

        \bigskip

        \underline{\textbf{Example}}

        \bigskip

        \texttt{p-$>$upper\_left.x}

        \bigskip
        \item \textbf{Linked Lists}

        \begin{itemize}
            \item \textbf{Declaring Node Type}

            \begin{itemize}
                \item \textbf{Syntax (Node structure):}

                \begin{center}
                \includegraphics[width=\linewidth]{images/review_8_solution_2.png}
                \end{center}

            \end{itemize}

            \item \textbf{Creating a Node}

            \begin{itemize}
                \item \textbf{Syntax (Allocating using malloc):}

                \bigskip

                \texttt{struct node *new\_node;}

                \texttt{new\_node = malloc(sizeof(struct node));}

                \bigskip

                \item Assigning value

                \bigskip

                \texttt{(*new\_node).value = 10;}

                \bigskip

            \end{itemize}
            \item \textbf{-$>$ Operator}

            \begin{itemize}
                \item is a short form of \texttt{(*STRUCT\_NAME).MEMBER\_NAME}

                \bigskip

                \underline{\textbf{Example}}

                \bigskip

                \texttt{(*new\_node).value = 10;}

                \bigskip

                Is the same as

                \bigskip

                \texttt{new\_node-$>$value = 10;}
            \end{itemize}
        \end{itemize}
    \end{itemize}

    \item

    \texttt{b)} and \texttt{c)} are legal

    \bigskip

    \item

\begin{lstlisting}[language=c]
    struct node *delete_from_list(struct node *list, int n)
    {
        struct node *curr, *to_be_freed;

        for (curr = list; curr != NULL && curr->value != n; curr = curr->next) {
            if (curr->next != NULL && curr->next->value == n) {
                to_be_freed = curr->next;
                curr->next = curr->next->next;
                free(to_be_freed);

                return list;
            }
        }


        return list;

    }
\end{lstlisting}

    \bigskip

    \underline{\textbf{Notes}}

    \begin{itemize}
        \item \textbf{Searching a Linked List}

        \begin{itemize}
            \item \textbf{Syntax:} \texttt{for (p = first; p != NULL; p = p -$>$next)}

            \bigskip

            \underline{\textbf{Example:}}

            \bigskip

            \begin{center}
            \includegraphics[width=0.9\linewidth]{images/review_8_solution_3.png}
            \end{center}
        \end{itemize}
        \item \textbf{Deleting Node from a List}

        \begin{itemize}
            \item Steps

            \begin{enumerate}[1.]
                \item Locate the node to be deleted

                \begin{itemize}
                    \item \textbf{Syntax (Searching for the node of value \texttt{n} to be deleted):}

                    \bigskip

                    \begin{center}
                    \includegraphics[width=0.8\linewidth]{images/review_8_solution_4.png}
                    \end{center}
                \end{itemize}

                \item Alter the previous node so that it "bypasses" the deleted node

                \begin{center}
                \includegraphics[width=0.6\linewidth]{images/review_8_solution_5.png}
                \end{center}

                \item Call \texttt{free} to reclaim the space occupied by the deleted code

                \begin{center}
                \includegraphics[width=0.6\linewidth]{images/review_8_solution_6.png}
                \end{center}
            \end{enumerate}

            \bigskip

            Putting together, we have

            \begin{center}
            \includegraphics[width=\linewidth]{images/review_8_solution_7.png}
            \end{center}
        \end{itemize}
    \end{itemize}

    \item

    The statement is incorrect because it removes the current node before its
    pointer moves to the next node.

    \bigskip

\end{enumerate}

\end{document}