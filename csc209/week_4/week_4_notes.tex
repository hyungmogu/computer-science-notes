\documentclass[12pt]{article}
\usepackage[margin=2.5cm]{geometry}
\usepackage{enumerate}
\usepackage{amsfonts}
\usepackage{amsmath}
\usepackage{fancyhdr}
\usepackage{amsmath}
\usepackage{amssymb}
\usepackage{amsthm}
\usepackage{mdframed}
\usepackage{graphicx}
\usepackage{subcaption}
\usepackage{adjustbox}
\usepackage{listings}
\usepackage{xcolor}
\usepackage{booktabs}
\usepackage[utf]{kotex}

\definecolor{codegreen}{rgb}{0,0.6,0}
\definecolor{codegray}{rgb}{0.5,0.5,0.5}
\definecolor{codepurple}{rgb}{0.58,0,0.82}
\definecolor{backcolour}{rgb}{0.95,0.95,0.92}

\lstdefinestyle{mystyle}{
    backgroundcolor=\color{backcolour},
    commentstyle=\color{codegreen},
    keywordstyle=\color{magenta},
    numberstyle=\tiny\color{codegray},
    stringstyle=\color{codepurple},
    basicstyle=\ttfamily\footnotesize,
    breakatwhitespace=false,
    breaklines=true,
    captionpos=b,
    keepspaces=true,
    numbers=left,
    numbersep=5pt,
    showspaces=false,
    showstringspaces=false,
    showtabs=false,
    tabsize=1
}

\lstset{style=mystyle}

\pagestyle{fancy}
\renewcommand{\headrulewidth}{0.4pt}
\lhead{Hyungmo Gu} % Left side of header.
\rhead{CSC209 Week 4 Notes}

\begin{document}
\title{CSC209 Week 4 Notes}
\author{Hyungmo Gu}
\maketitle

\section*{Introduction to arrays in C 1 of 3}

\bigskip

\begin{itemize}
    \item Array
    \begin{itemize}
    \item \textbf{Syntax:} \textless TYPE \textgreater VAR\_NAME[ARRAY\_SIZE]

    \begin{lstlisting}[language=c]
    #include <stdio.h>

    int main() {
        float daytime_high[4];
    }
    \end{lstlisting}
    \end{itemize}
\end{itemize}

\bigskip

\section*{Introduction to arrays in C 2 of 3}

\bigskip
\begin{itemize}
    \item Accessing Array Elements

    \begin{itemize}
    \item C doesn't check if an array access is within the bounds of array
    \item Overwrites memory location if exists
    \begin{lstlisting}[language=c]
    #include <stdio.h>

    int main() {
        float daytime_high[4] = {1,2,3};
        daytime_high[5] = 999;
    }
    \end{lstlisting}

    \item Segmentation fault occurs if suitable memory location doesn't exist.

    \begin{lstlisting}[language=c]
    #include <stdio.h>

    int main() {
        int daytime_high[4] = {1,2,3};
        daytime_high[3000] = 999;
    }
    \end{lstlisting}
    \end{itemize}
\end{itemize}

\bigskip

\section*{Introduction to arrays in C 3 of 3}

\bigskip
\begin{itemize}
    \item Iterating Over Arrays

    \begin{itemize}
    \item For loop

    \begin{itemize}
    \item `\textless` is used over `\textless=' for the end condition,i.e. $i$
    \textless 4 in for ($i = 0$; $i$ \textless 4; $i$++).

    \begin{lstlisting}[language=c]
    #include <stdio.h>

    int main() {
        float daytime_high[4] = {16.0, 12.8, 14.6, 19.1};

        float average_temp = 0;

        int i;
        for (i = 0; i < 4; i++) {
            printf("Adding element %d with value %f.\n", index, daytime_high[i]);
            average_temp += daytime_high[i];
        }

        average_temp = average_temp / 4;
        printf("average %f\n", average_temp);

        return 0;
    }
    \end{lstlisting}
    \end{itemize}

    \item Constants

    \begin{itemize}
    \item Combines multiple repeating values into one
    \item Used to increase maintainability and readibility

    \begin{lstlisting}[language=c]
    #include <stdio.h>
    #define DAYS 4 // <-- HERE!!

    int main() {
        float daytime_high[DAYS] = {16.0, 12.8, 14.6, 19.1};

        float average_temp = 0;

        int i;
        for (i = 0; i < DAYS; i++) {
            printf("Adding element %d with value %f.\n", index, daytime_high[i]);
            average_temp += daytime_high[i];
        }

        average_temp = average_temp / DAY;
        printf("average %f\n", average_temp);

        return 0;
    }
    \end{lstlisting}
    \end{itemize}

    \end{itemize}

\end{itemize}

\bigskip

\section*{Pointers in C 1 of 7}

\bigskip

\begin{itemize}
    \item Address in C
    \begin{itemize}
    \item \textbf{\& \textless VARIABLE\_NAME \textgreater}
    \item Returns memory location of variable

    \begin{lstlisting}[language=c]
    #include <stdio.h>
    #define DAYS 4

    int main() {
        int i;
        i = 5;
        printf("Value of i: %d\n", i);
        printf("Address of i: %p\n", &i);
    }
    \end{lstlisting}

    \end{itemize}
    \item Pointer
    \begin{itemize}
    \item \textbf{\textless TYPE \textgreater* \textless VARIABLE\_NAME \textgreater}
    \item Is used to store memory addresses

    \begin{lstlisting}[language=c]
    #include <stdio.h>
    #define DAYS 4

    int main() {
        int *pt;
        pt = &i;

        printf("value of pt: %p\n", pt);
        printf("Address of pt: %p\n", &pt);

        printf("Value pointed to by pt: %d\n", *pt);
    }
    \end{lstlisting}
    \end{itemize}
\end{itemize}

\bigskip

\section*{Pointers in C 2 of 7}

\bigskip

\begin{itemize}
    \item Assigning to Deferenced Pointers
    \begin{itemize}
    \item \textbf{Syntax:} $TYPE* POINTER\_NAME$
    \item \textbf{$\text{TYPE *\textless POINTER\_NAME \textgreater} = \text{VARIABLE\_NAME}$}
    \begin{itemize}
        \item Stores memory location of variable to pointer
        \item is the same as

    \begin{lstlisting}[language=c]
    <TYPE> *<POINTER_NAME>;
    <POINTER_NAME> = VARIABLE_NAME
    \end{lstlisting}

    \end{itemize}

    \item \textbf{*\textless POINTER\_NAME \textgreater = VALUE}
    \begin{itemize}
        \item changes the value pointed by pointer
    \end{itemize}

    \bigskip

    \underline{\textbf{Example:}}

    \begin{lstlisting}[language=c]
    #include <stdio.h>
    #define DAYS 4

    int main() {
        int i = 7;
        int *pt;
        pt = &i; // <- stores memory location of i, i.e. 0x7ffeeab32a28
        *pt = 9; // <- changes the value of i to 9

        printf("Value of i: %d\n", i);
        printf("Address of i: %p\n", &i);

        printf("pt points to %d\n", *pt);

        return 0;
    }
    \end{lstlisting}

    \end{itemize}
\end{itemize}

\bigskip

\section*{Pointers in C 3 of 7}

\bigskip

\begin{itemize}
    \item Pointers as Parameters to Functions
    \begin{itemize}
        \item \textbf{Syntax:} ... \textless FUNCTION\_TYPE \textgreater(\textless TYPE \textgreater *\textless VARIABLE\_NAME \textgreater)
        \item Passes variable to function by reference
        \item Changing values of variable inside function affects the variable
        outside of function

    \begin{lstlisting}[language=c]
    #include <stdio.h>

    void apply_late_penalty(char *grade_ptr) {
        if (*grade_ptr != 'F') {
            (*grade_ptr)++;
        }
    }

    int main() {
        char grade_moe = 'B';
        apply_late_penalty(&grade_moe)

        return 0;
    }
    \end{lstlisting}
    \end{itemize}
\end{itemize}

\bigskip

\section*{Pointers in C 4 of 7}

\bigskip

\begin{itemize}
    \item Passing Arrays as Parameters
    \begin{itemize}
        \item \textbf{Syntax:} ... \textless FUNCTION\_TYPE \textgreater(\textless TYPE \textgreater *\textless ARRAY\_VARIABLE\_NAME \textgreater)

    \begin{lstlisting}[language=c]
    #include <stdio.h>

    int sum(int *A, int size) {
        ...
    }

    int main() {
        ...
        printf("total is %d\n", sum(scores, 4));

        return 0;
    }
    \end{lstlisting}
        \item Passes first element of array to function by reference
        \item Size of array needs to be passed indenpendently.
        \begin{itemize}
            \item \textit{sizeof} measures size of pointer value, not the array
        \end{itemize}
        \item Array elements are also passed by reference

        \begin{lstlisting}[language=c]
        #include <stdio.h>

        void change(int *A) {
            A[0] = 50;
        }

        int main() {
            int scores[4] = {4,5,-1,12};
            ...
            change(scores);
            printf("First element in array has value %d\n", scores[0]); // <- returns 50, instead of 4
            return 0;
        }
        \end{lstlisting}
    \end{itemize}
\end{itemize}

\bigskip

\section*{Pointers in C 5 of 7}

\bigskip

\begin{itemize}
    \item Pointer Arithmetic
    \begin{itemize}
        \item Moves the memory location by $x$ amount, where $x$ is int.
        \item \textbf{General Formula:} $p = p + (\text{sizeof(type)} \cdot n)$, given

    \begin{lstlisting}[language=c]
    type k;
    type *p = &k;

    int n;
    p = p + n;
    \end{lstlisting}

        \bigskip

        \item Example
        \begin{itemize}
            \item $\text{int* var1}+1:$ increases memory size by 4
            \item $\text{char* var2}+1:$ increases memory size by 3
        \end{itemize}
    \end{itemize}
    \item Pointer Arithmetic in Array
    \begin{itemize}
        \item Moves the memory location by $x$ amount, where $x$ is int.
        \item \textbf{General Formula:} $p[k] == p + k$, given

    \begin{lstlisting}[language=c]
    int n = //Arbitrary positive int value;
    type A[n];
    type *p = &A;

    int k = //Arbitrary int value with size less than n;
    print(*p);
    print(*(p+k));
    \end{lstlisting}

    \end{itemize}
\end{itemize}

\end{document}