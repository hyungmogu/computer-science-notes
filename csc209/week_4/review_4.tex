\documentclass[12pt]{article}
\usepackage[margin=2.5cm]{geometry}
\usepackage{enumerate}
\usepackage{amsfonts}
\usepackage{amsmath}
\usepackage{fancyhdr}
\usepackage{amsmath}
\usepackage{amssymb}
\usepackage{amsthm}
\usepackage{mdframed}
\usepackage{graphicx}
\usepackage{subcaption}
\usepackage{adjustbox}
\usepackage{listings}
\usepackage{xcolor}
\usepackage{booktabs}
\usepackage[utf]{kotex}
\usepackage{hyperref}

\definecolor{codegreen}{rgb}{0,0.6,0}
\definecolor{codegray}{rgb}{0.5,0.5,0.5}
\definecolor{codepurple}{rgb}{0.58,0,0.82}
\definecolor{backcolour}{rgb}{0.95,0.95,0.92}

\lstdefinestyle{mystyle}{
    backgroundcolor=\color{backcolour},
    commentstyle=\color{codegreen},
    keywordstyle=\color{magenta},
    numberstyle=\tiny\color{codegray},
    stringstyle=\color{codepurple},
    basicstyle=\ttfamily\footnotesize,
    breakatwhitespace=false,
    breaklines=true,
    captionpos=b,
    keepspaces=true,
    numbers=left,
    numbersep=5pt,
    showspaces=false,
    showstringspaces=false,
    showtabs=false,
    tabsize=1
}

\lstset{style=mystyle}

\pagestyle{fancy}
\renewcommand{\headrulewidth}{0.4pt}
\lhead{CSC 209}
\rhead{Worksheet Review 4}

\begin{document}
\title{CSC 209 Review 4}
\maketitle

\bigskip

\section{Exercises}

\begin{enumerate}[1.]
    \item \textbf{K.K. 11.1:} If $i$ is a variable and $p$ points to $i$,
    which of the following expressions are aliases for $i$?

    \begin{enumerate}[a)]
        \item \texttt{*p}
        \item \texttt{*\&p}
        \item \texttt{*i}
        \item \texttt{*\&i}
        \item \texttt{\&p}
        \item \texttt{\&*p}
        \item \texttt{\&i}
        \item \texttt{\&*i}
    \end{enumerate}

    \item \textbf{K.K 11.2:} If $i$ is an int variable and $p$ and $q$ are pointers to \texttt{int}
    which of the following assignments are legal?

    \begin{enumerate}[a)]
        \item \texttt{p = i;}
        \item \texttt{*p = \&i;}
        \item \texttt{\&p =q;}
        \item \texttt{p = \&q;}
        \item \texttt{p = *\&q;}
        \item \texttt{p = q;}
        \item \texttt{p = *q;}
        \item \texttt{*p = q;}
        \item \texttt{*p = *q;}
    \end{enumerate}

    \item \textbf{K.K.11.3:} The following function supposedly computes the sum and average of the
    numbers in the array \texttt{a}, which has length \texttt{n. avg} and sum point to variables
    that the function should modify. Unfortunately, the function contains several errors; find
    and correct them

\begin{lstlisting}[language=c]
    void avg_sum(double a[], int n, double *avg, double *sum)
    {
        int i;

        sum = 0.0;
        for (i = 0; i < n; i++)
            sum += a[i];
        avg = sum / n;
    }
\end{lstlisting}

    \item \textbf{K.K.11.4:} Write the following function

    \bigskip

    \texttt{void swap(int *p, int *q);}

    \bigskip

    When passed the addresses of two variables, \texttt{swap} should exchange the values of the variables:

    \bigskip

    \texttt{swap(\&i, \&j); /* Exchanges values of i and j */}

    \bigskip

    \item \textbf{K.K.11.5:} Write the following function

    \bigskip

    \texttt{void split\_time(long total\_sec, int *hr, int *min, int *sec);}

    \bigskip

    \texttt{total\_sec} is a time represented as the number of seconds since midnight.
    hr, min and sec are pointers to variables in which the function will store the
    equivalent time in hours (0-23), minutes (0-59) and seconds (0-59), respectively.

    \bigskip

    \item \textbf{K.K.11.6:} Write the following function:


    \bigskip

    \texttt{void find\_two\_largest(int a[], int n, int *largest, int *second\_largest);}

    \bigskip

    When passed an array \texttt{a} of length \texttt{n}, the function will search
    \texttt{a} for its largest and second-largest elements, storing them in the
    variables pointed to by \texttt{largest} and \texttt{second\_largest} respectively.


    \item \textbf{K.K.11.7:} Write the following function:

    \bigskip

    \texttt{void split\_date (int day\_of\_year, int year, int *month, int *day);}

    \bigskip

    \texttt{day\_of\_year} is an integer between 1 and 366, specifying a particular day within
    the year designated by \texttt{year}. \texttt{month} and \texttt{day} point to variables in which the
    function will store the equivalent month (1-12) and day within that month
    (1-31).

    \item \textbf{K.K.11.8:} Write the following function:

    \bigskip

    \texttt{int *find\_largest(int a[], int n)}

    \bigskip

    When passed an array \texttt{a} of length \texttt{n}, the function will return a pointer to
    the array's largest element.
\end{enumerate}

\section{Programming Projects}


\begin{enumerate}[1.]
    \item Modify Programming Project 7 from Chapter 2 so that it includes the following
    function:

    \bigskip

    \texttt{void pay\_amount(int dollars, int *twenties, int *tens, int *fives, int *ones);}

    \bigskip

    The function determines the smallest number of \$20, \$10, \$5 and \$1 bills
    necessary to pay the amount represented by the \texttt{dollars} parameter. The \texttt{twenties}
    parameter points to a variable in which the function will store the number of
    \$20 bills required. The \texttt{tens}, \texttt{fives} and \texttt{ones} parameters are similar.

    \bigskip

    \item Modify Programming Project 8 from Chapter 5 so that it includes the
    following function:

    \bigskip

    \texttt{void find\_closest\_flight(int desired\_time, int *departure\_time, int *arrival\_time);}

    \bigskip

    This function wil find the flight whose departure time is closest to \texttt{desired\_time}
    (expressed in minutes sinces midnight). It will store the departure and arrival times of
    this flight (also expressed in minutes since midnight) in the variables pointed to by
    \texttt{departure\_time} and \texttt{arrival\_time}, respectively.

    \item Modify Programming Project 3 from Chapter 6 so that it includes the
    following function:

    \bigskip

    \texttt{void reduce(int numerator, int denominator, int *reduced\_numerator,\\
    int *reduced\_denominator);}

    \bigskip

    \texttt{numerator} and \texttt{denominator} are the numerator and denominator
    of a fraction. \texttt{reduced\_numerator} and \texttt{reduced\_denominator} are pointers to
    variables in which the function will store the numerator and denominator of
    the fraction once it has been reduced to lowest terms.

    \item Modify the \texttt{poker.c} program of Section 10.5 by moving all external variables
    into main and modifying functions so that they communicate by passing arguments.
    The \texttt{analyze\_hand} function needs to change the straight, flush, four, three
    and pairs variables, so it will have to be passed pointers to those variables.

\end{enumerate}

\end{document}