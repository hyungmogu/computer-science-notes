\documentclass[12pt]{article}
\usepackage[margin=2.5cm]{geometry}
\usepackage{enumerate}
\usepackage{amsfonts}
\usepackage{amsmath}
\usepackage{fancyhdr}
\usepackage{amsmath}
\usepackage{amssymb}
\usepackage{amsthm}
\usepackage{mdframed}
\usepackage{graphicx}
\usepackage{subcaption}
\usepackage{adjustbox}
\usepackage{listings}
\usepackage{xcolor}
\usepackage{booktabs}
\usepackage[utf]{kotex}

\definecolor{codegreen}{rgb}{0,0.6,0}
\definecolor{codegray}{rgb}{0.5,0.5,0.5}
\definecolor{codepurple}{rgb}{0.58,0,0.82}
\definecolor{backcolour}{rgb}{0.95,0.95,0.92}

\lstdefinestyle{mystyle}{
    backgroundcolor=\color{backcolour},
    commentstyle=\color{codegreen},
    keywordstyle=\color{magenta},
    numberstyle=\tiny\color{codegray},
    stringstyle=\color{codepurple},
    basicstyle=\ttfamily\footnotesize,
    breakatwhitespace=false,
    breaklines=true,
    captionpos=b,
    keepspaces=true,
    numbers=left,
    numbersep=5pt,
    showspaces=false,
    showstringspaces=false,
    showtabs=false,
    tabsize=1
}

\lstset{style=mystyle}

\pagestyle{fancy}
\renewcommand{\headrulewidth}{0.4pt}
\lhead{Hyungmo Gu}
\rhead{CSC209 Week 7 Notes}

\begin{document}
\title{CSC209 Week 7 Notes}
\author{Hyungmo Gu}
\maketitle

\section*{Unix Filesystem 1 of 1}
\begin{itemize}
    \item is a file system supported by many unix and unix-like
    operating systems
    \begin{itemize}
        \item i.e. Linux, Mac
    \end{itemize}

    \begin{center}
    \includegraphics[width=\linewidth]{images/week_7_notes_1.png}
    \end{center}
    \item ls and Contents of Inodes (File Attributes)

    \begin{center}
    \includegraphics[width=\linewidth]{images/week_7_notes_2.png}
    \end{center}

    \bigskip

    \begin{itemize}
        \item \textbf{Permission of Modes:} The permissions assigned to the file
        for the owner, the group and all

        \begin{center}
        \includegraphics[width=0.7\linewidth]{images/week_7_notes_3.png}
        \end{center}

        \bigskip

        Here,

        \begin{itemize}
            \item \textbf{r} means read
            \item \textbf{w} means write
            \item \textbf{x} means executable
        \end{itemize}

        \item \textbf{Number of Links:} Number of Symlinks associated to this file
        \item \textbf{Owner:} the owner of the file
        \item \textbf{Group:} associated group for the file
        \item \textbf{Size:} the size of the file in bytes
        \item \textbf{Modification Date:} the date file was last modified
        \item \textbf{File Name:} Name associated to this file
    \end{itemize}

    \item Symlinks
    \begin{itemize}
        \item \textbf{Syntax:} ln -s \textless Source \textgreater \textless Destination \textgreater
        \item is like shortcut in Windows
        \item can cross filesystems across different users, i.e. $\backslash$moegu to $\backslash$mylove

    \begin{lstlisting}[language=bash]
    >>> echo hello > symlink_example_source.txt
    >>> ln -s symlink_example_source.txt symlink_example_link.txt
    >>> cat symlink_example_link.txt # <- symlinked file
    hello
    \end{lstlisting}

    \end{itemize}
    \item Other notes
    \begin{itemize}
        \item Max size of single file is 2GB
    \end{itemize}
\end{itemize}

\end{document}