\documentclass[12pt]{article}
\usepackage[margin=2.5cm]{geometry}
\usepackage{enumerate}
\usepackage{amsfonts}
\usepackage{amsmath}
\usepackage{fancyhdr}
\usepackage{amsmath}
\usepackage{amssymb}
\usepackage{amsthm}
\usepackage{mdframed}
\usepackage{graphicx}
\usepackage{subcaption}
\usepackage{adjustbox}
\usepackage{listings}
\usepackage{xcolor}
\usepackage{booktabs}
\usepackage[utf]{kotex}
\usepackage{hyperref}

\definecolor{codegreen}{rgb}{0,0.6,0}
\definecolor{codegray}{rgb}{0.5,0.5,0.5}
\definecolor{codepurple}{rgb}{0.58,0,0.82}
\definecolor{backcolour}{rgb}{0.95,0.95,0.92}

\lstdefinestyle{mystyle}{
    backgroundcolor=\color{backcolour},
    commentstyle=\color{codegreen},
    keywordstyle=\color{magenta},
    numberstyle=\tiny\color{codegray},
    stringstyle=\color{codepurple},
    basicstyle=\ttfamily\footnotesize,
    breakatwhitespace=false,
    breaklines=true,
    captionpos=b,
    keepspaces=true,
    numbers=left,
    numbersep=5pt,
    showspaces=false,
    showstringspaces=false,
    showtabs=false,
    tabsize=1
}

\lstset{style=mystyle}

\pagestyle{fancy}
\renewcommand{\headrulewidth}{0.4pt}
\lhead{CSC 209}
\rhead{Review 6 Solution}

\begin{document}
\title{CSC 209 Review 6 Solution}
\maketitle

\bigskip

\section{Exercises}

\begin{enumerate}[1.]
    \item

    I need to write which of the supplied function calls don't work and explain why.

    \bigskip

    \begin{itemize}
        \item \texttt{b)} String format in \texttt{printf} expects character constant, but string literal is used
        \item \texttt{c)} String format in \texttt{printf} expects string but character constrant is used
        \item \texttt{e)} The first argument in \texttt{printf} expects pointer but character constrant (an integer) is used isntead
        \item \texttt{h)} The first argument in \texttt{putchar} expects a character, but string literal (a pointer to character) is used
        \item \texttt{i)} The first argument in \texttt{puts} expects a pointer to character, but character constant (an integer) is used
    \end{itemize}

    \underline{\textbf{Notes}}

    \begin{itemize}
        \item \textbf{putchar}

        \begin{itemize}
            \item \textbf{Syntax:} \texttt{int putchar(int char)}
            \item Writes a character (an unsigned char) specified by the argument char to stdout.
            \item Does not append a new line to the output
            \item Is similar to printf but for character
        \end{itemize}

        \item \textbf{puts}

        \begin{itemize}
            \item \textbf{Syntax:} \texttt{int puts(const char *str)}
            \item Writes a string to stdout up to but not including the null character
            \item Appends a newline character to the output.
            \item Is similar to printf but for string
        \end{itemize}

        \item \textbf{Character Constant}

        \begin{itemize}
            \item \textbf{Syntax:} \texttt{' ... '}
            \item Is represented by an \underline{integer}
        \end{itemize}

        \item \textbf{String Literal}
        \begin{itemize}
            \item \textbf{Syntax:} \texttt{" ... "}
            \item Has a sequence of characters inside
            \item Ends with \texttt{$\backslash$0}
            \item Is represented by a \underline{pointer}

            \bigskip

            \underline{\textbf{Example}}

            \bigskip

            "When you come to a fork in the road, take it"
        \end{itemize}

        \item \textbf{Escape Squences in String Literal}

        \begin{itemize}
            \item A common example is `\texttt{$\backslash$n}'

            \begin{itemize}
                \item causes the cursor to advance to the next line
            \end{itemize}
        \end{itemize}
    \end{itemize}
\end{enumerate}


\end{document}