\documentclass[12pt]{article}
\usepackage[margin=2.5cm]{geometry}
\usepackage{enumerate}
\usepackage{amsfonts}
\usepackage{amsmath}
\usepackage{fancyhdr}
\usepackage{amsmath}
\usepackage{amssymb}
\usepackage{amsthm}
\usepackage{mdframed}
\usepackage{graphicx}
\usepackage{subcaption}
\usepackage{adjustbox}
\usepackage{listings}
\usepackage{xcolor}
\usepackage{booktabs}
\usepackage[utf]{kotex}

\definecolor{codegreen}{rgb}{0,0.6,0}
\definecolor{codegray}{rgb}{0.5,0.5,0.5}
\definecolor{codepurple}{rgb}{0.58,0,0.82}
\definecolor{backcolour}{rgb}{0.95,0.95,0.92}

\lstdefinestyle{mystyle}{
    backgroundcolor=\color{backcolour},
    commentstyle=\color{codegreen},
    keywordstyle=\color{magenta},
    numberstyle=\tiny\color{codegray},
    stringstyle=\color{codepurple},
    basicstyle=\ttfamily\footnotesize,
    breakatwhitespace=false,
    breaklines=true,
    captionpos=b,
    keepspaces=true,
    numbers=left,
    numbersep=5pt,
    showspaces=false,
    showstringspaces=false,
    showtabs=false,
    tabsize=1
}

\lstset{style=mystyle}

\pagestyle{fancy}
\renewcommand{\headrulewidth}{0.4pt}
\lhead{Hyungmo Gu} % Left side of header.
\rhead{CSC209 Week 2 Notes}

\begin{document}
\title{CSC209 Week 3 Notes}
\author{Hyungmo Gu}
\maketitle

\section*{Shell Programming 4 of 6}

\bigskip

\begin{itemize}
    \item I/O redirection
    \begin{itemize}
    \item \textbf{\textless:} sends \textit{stdin} from file to variable or command
    \begin{lstlisting}[language=bash]
    >>> var=`<sample.txt`
    >>> echo $var
    hello!
    hi!
    it's a pleasure to meet you
    \end{lstlisting}

    \item \textbf{\textgreater\textgreater:} appends line to file\
    \item \textbf{2\textgreater\&1:} redirects \textit{stderr} to the same file as \textit{stdout}

    \begin{lstlisting}[language=bash]
    >>> cat hello > output.txt 2>&1
    >>> cat output.txt
    cat: hello: No such file or directory
    \end{lstlisting}

    \end{itemize}
    \item File Descriptors
    \begin{itemize}
    \item \textbf{0:} Standard Input
    \item \textbf{1:} Standard Output
    \begin{itemize}
    \item Channels all output

    \begin{lstlisting}[language=bash]
    cat sample.txt
    hello!
    hi!
    it's a pleasure to meet you
    >>> cat sample.txt> output.log
    >>> cat sample.txt1> output.log # identical
    >>> cat output.log
    hello!
    hi!
    it's a pleasure to meet you
    \end{lstlisting}
    \end{itemize}

    \item \textbf{2:} Standard Error
    \begin{itemize}
    \item Channels all error output

    \begin{lstlisting}[language=bash]
    >>> cat 2> error.log
    \end{lstlisting}
    \end{itemize}

    \end{itemize}
    \item \textit{\$\#}
    \begin{itemize}
    \item Returns number of command line arguements
    \end{itemize}

    \begin{lstlisting}[language=bash]
    >>> cat arg_number_check_example.sh
    if [ $# -ne 2 ]
    then
        echo usage: arg_number_check_example.sh x y >& 2
        exit 1
    fi

    expr $1 + $2
    >>> arg_number_check_example.sh 2
    usage: arg_number_check_example.sh x y
    \end{lstlisting}

    \item \textit{\$*}
    \begin{itemize}
    \item Means all command line arguements
    \item all arguements passed are treated as one
    \end{itemize}

    \begin{lstlisting}[language=bash]
    >>> sh dollar_star_example.sh hello world hi
    cat: hello world hi: No such file or directory
    \end{lstlisting}

    \item \textit{\$@}
    \begin{itemize}
    \item Also means all command line arguements
    \item each arguement sparated a space is treated independently
    \item Works like for loop
    \end{itemize}
    \begin{lstlisting}[language=bash]
    >>> sh dollar_at_example.sh hello world hi
    cat: hello: No such file or directory
    cat: world: No such file or directory
    cat: hi: No such file or directory
    \end{lstlisting}

    \item \textit{\$\{x\}}
    \begin{itemize}
    \item Works like template literal in javascript
    \item Works with arguements as well!
    \item Is useful when using in loop
    \end{itemize}

    \begin{lstlisting}[language=bash]
    >>> cat dolloar_curly_x_example.sh
    a="hello"
    cat ${a}2
    >>> sh dolloar_curly_x_example.sh
    cat: hello2: No such file or directory
    \end{lstlisting}
    \begin{lstlisting}[language=bash]
    >>> x = "hello"
    >>> sed -n ${x}p file
    sed: 1: "hellop": extra characters at the end of h command
    \end{lstlisting}
\end{itemize}

\bigskip

\section*{Introduction to arrays in C 1 of 3}

\bigskip

\begin{itemize}
    \item
\end{itemize}

\end{document}