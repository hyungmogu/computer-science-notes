\documentclass[12pt]{article}
\usepackage[margin=2.5cm]{geometry}
\usepackage{enumerate}
\usepackage{amsfonts}
\usepackage{amsmath}
\usepackage{fancyhdr}
\usepackage{amsmath}
\usepackage{amssymb}
\usepackage{amsthm}
\usepackage{mdframed}
\usepackage{graphicx}
\usepackage{subcaption}
\usepackage{adjustbox}
\usepackage{listings}
\usepackage{xcolor}
\usepackage{booktabs}
\usepackage[utf]{kotex}
\usepackage{soul}
\usepackage{hyperref}

\definecolor{codegreen}{rgb}{0,0.6,0}
\definecolor{codegray}{rgb}{0.5,0.5,0.5}
\definecolor{codepurple}{rgb}{0.58,0,0.82}
\definecolor{backcolour}{rgb}{0.95,0.95,0.92}

\lstdefinestyle{mystyle}{
    backgroundcolor=\color{backcolour},
    commentstyle=\color{codegreen},
    keywordstyle=\color{magenta},
    numberstyle=\tiny\color{codegray},
    stringstyle=\color{codepurple},
    basicstyle=\ttfamily\footnotesize,
    breakatwhitespace=false,
    breaklines=true,
    captionpos=b,
    keepspaces=true,
    numbers=left,
    numbersep=5pt,
    showspaces=false,
    showstringspaces=false,
    showtabs=false,
    tabsize=1
}

\lstset{style=mystyle}

\pagestyle{fancy}
\renewcommand{\headrulewidth}{0.4pt}
\rhead{CSC369 Assignment 3}

\begin{document}
\title{CSC369 Assignment 3 - File Systems}
\maketitle

\bigskip

\section{Introduction}

\bigskip

\noindent In this assignment, you will explore the implementation of a particular file
system, \hl{ext2}, and will write tools to modify \hl{ext2}-format virtual disks. To do
this work, you will need to be comfortable working with binary data and will
need to learn about the \hl{ext2 filesystem}.

\bigskip

\noindent This assignment contains some bonus features. While you cannot get more than
100\% on this assignment, implementing a bonus will compensate for any possible
marks lost in another section of the assignment.

\bigskip

\section{Requirements}

\bigskip

Your task is to write five programs (in C) that operate on an ext2 formatted
virtual disk. The executables must be named \textit{ext2\_ls}, \textit{ext2\_cp},
\textit{ext2\_mkdir}, \textit{ext2\_ln}, and \textit{ext2\_rm} and must take the
specified arguments.

\bigskip

\begin{itemize}
    \item \textbf{ext2\_ls:} This program takes two command line arguments. The
    first is the name of an ext2 formatted virtual disk. The second is an
    absolute path on the ext2 formatted disk. The program should work like ls -1
    (that's number one "1", not lowercase letter "L"), printing each directory
    entry on a separate line. If the flag "-a" is specified (after the disk
    image argument), your program should also print the . and .. entries. In other
    words, it will print one line for every directory entry in the directory
    specified by the absolute path. If the path does not exist, print "No such
    file or directory", and return an ENOENT. Directories passed as the second
    argument may end in a "/" - in such cases the contents of the last directory
    in the path (before the "/") should be printed (as ls would do). Additionally,
    the path (the last argument) may be a file or link. In this case, your program
    should simply print the file/link name (if it exists) on a single line, and
    refrain from printing the . and ...

    \item \textbf{ext2\_cp:} This program takes three command line arguments. The
    first is the name of an ext2 formatted virtual disk. The second is the path
    to a file on your native operating system, and the third is an absolute path
    on your ext2 formatted disk. The program should work like cp, copying the
    file on your native file system onto the specified location on the disk.
    If the specified file or target location does not exist, then your program
    should return the appropriate error (ENOENT). Please read the specifications
    of ext2 carefully, some things you will not need to worry about (like permissions,
    gid, uid, etc.), while setting other information in the inodes may be important
    (e.g., \textit{i\_dtime}).

    \item \textbf{ext2\_mkdir:} This program takes two command line arguments.
    The first is the name of an ext2 formatted virtual disk. The second is an
    absolute path on your ext2 formatted disk. The program should work like mkdir,
    creating the final directory on the specified path on the disk. If any component
    on the path to the location where the final directory is to be created does
    not exist or if the specified directory already exists, then your program
    should return the appropriate error (ENOENT or EEXIST). Again, please read
    the specifications to make sure you're implementing everything correctly
    (e.g., directory entries should be aligned to 4B, entry names are not
    null-terminated, etc.).

    \item \textbf{ext2\_ln:} This program takes three command line arguments. The
    first is the name of an ext2 formatted virtual disk. The other two are absolute
    paths on your ext2 formatted disk. The program should work like ln, creating
    a link from the first specified file to the second specified path. If the
    source file does not exist (ENOENT), if the link name already exists
    (EEXIST), or if either location refers to a directory (EISDIR), then your
    program should return the appropriate error. Note that this version of ln
    only works with files. Additionally, this command may take a "-s" flag,
    after the disk image argument. When this flag is used, your program must
    create a symlink instead (other arguments remain the same). If in doubt
    about correct operation of links, use the ext2 specs and ask on the discussion
    board.

    \item \textbf{ext2\_rm:} This program takes two command line arguments. The
    first is the name of an ext2 formatted virtual disk, and the second is an
    absolute path to a file or link (not a directory) on that disk. The program
    should work like rm, removing the specified file from the disk. If the file
    does not exist or if it is a directory, then your program should return the
    appropriate error. Once again, please read the specifications of ext2
    carefully, to figure out what needs to actually happen when a file or link
    is removed (e.g., no need to zero out data blocks, must set \textit{i\_dtime} in the
    inode, removing a directory entry need not shift the directory entries after
    the one being deleted, etc.).

    \bigskip

    Bonus(5\% extra): Implement an additional "-r" flag (after the disk image argument),
    which allows removing directories as well. In this case, you will have to recursively
    remove all the contents of the directory specified in the last argument. If "-r" is
    used with a regular file or link, then it should be ignored (the \textit{ext2\_rm}
    operation should be carried out as if the flag had not been entered). If you
    decide to do the bonus, make sure first that your \textit{ext2\_rm} works, then create
    a new copy of it and rename it to \textit{ext2\_rm\_bonus.c}, and implement the
    additional functionality in this separate source file.
\end{itemize}

All of these programs should be minimalist. Don't implement what isn't specified:
only provide the required functionality and specified errors. (For example, don't
implement wildcards. Also, can't delete directories? Too bad! Unless you want the
bonus!)

\bigskip

You will find it very useful for these programs to share code. You will want a
function that performs a path walk, for example. You will also want a function
that opens a specific directory entry and writes to it.


\bigskip

\section{Learning about System}

\bigskip

Here are several sample virtual disk images:

\bigskip

\begin{itemize}
    \item \href{https://www.teach.cs.toronto.edu//~csc369h/summer/assignments/new-a3/images/emptydisk.img}{emptydisk:} An empty virtual disk.
    \item \href{https://www.teach.cs.toronto.edu//~csc369h/summer/assignments/new-a3/images/onefile.img}{onefile:} A single text file has been added to emptydisk.
    \item \href{https://www.teach.cs.toronto.edu//~csc369h/summer/assignments/new-a3/images/deletedfile.img}{deletedfile:} The file from onefile has been removed.
    \item \href{https://www.teach.cs.toronto.edu//~csc369h/summer/assignments/new-a3/images/hardlink.img}{onedirectory:} A single directory containing a text file has been added to emptydisk.
    \item \href{https://www.teach.cs.toronto.edu//~csc369h/summer/assignments/new-a3/images/hardlink.img}{hardlink:} A hard link to the textfile in onedirectory was added.
    \item \href{https://www.teach.cs.toronto.edu//~csc369h/summer/assignments/new-a3/images/deleteddirectory.img}{deleteddirectory:} A recursive remove was used to remove the directory and file from onedirectory.
    \item \href{https://www.teach.cs.toronto.edu//~csc369h/summer/assignments/new-a3/images/twolevel.img}{twolevel:} The root directory contains a directory called level1 and a file called afile. level1 contains a directory called level2, and level2 contains a file called bfile.
    \item \href{https://www.teach.cs.toronto.edu//~csc369h/summer/assignments/new-a3/images/largefile.img}{largefile:} A file larger than 13KB (13440 bytes) is in the root directory. This file requires the single indirect block in the inode.
\end{itemize}

\bigskip

These disks were each created and formatted in the same way (on an ubuntu virtual machine):

\bigskip

\begin{lstlisting}[language=Java,caption={lesson\_01/Prompter.java}]
% dd if=/dev/zero of=~/DISKNAME.img bs=1024 count=128
% mke2fs -N 32 DISKNAME.img
% sudo mount -o loop ~/DISKNAME.img /home/reid/mntpoint
% cd /home/reid/mntpoint
% ...... normal linux commands to add/remove files/directories/links .....
% cd ~
% umount /home/reid/mntpoint
\end{lstlisting}

\bigskip

Since we are creating images with mke2fs, the disks are formatted with the \href{https://en.wikipedia.org/wiki/Ext2}{ext2
file system}. You may wish to read about this system before doing some exploration.
The \href{http://en.wikipedia.org/wiki/Ext2}{wikipedia page for ext2} provides a
good overview, but the \href{http://wiki.osdev.org/Ext2}{Ext2 wiki} and Dave
Poirer's \href{http://www.nongnu.org/ext2-doc/index.html}{Second Extended File System}
article provide more detail on how the system places data onto a disk. It's a good
reference to keep on hand as you explore.

\bigskip

We are restricting ourselves to some simple parameters, so you can make the
following assumptions when you write your code:

\bigskip

\begin{itemize}
    \item A disk is 128 blocks where the block size is 1024 bytes.
    \item There is only one block group.
    \item There are 32 inodes.
    \item You do not have to worry about permissions or modified time fields in
    the \hl{inodes}. You should set the type (in \textit{i\_mode}), \textit{i\_size},
    \textit{i\_links\_count}, \textit{i\_blocks(disk sectors)}, and the \textit{i\_block}
    array.
\end{itemize}

We will not test your code on anything other than disk images that follow this
specification, or on corrupted disk images.

Other tips:

\begin{itemize}
    \item Inode and disk block numbering starts at 1 instead of 0.
    \item The root inode is inode number 2 (at index 1)
    \item The first 11 inodes are reserved.
    \item There is always a lost+found directory in the root directory.
    \item Disk sectors are 512 bytes. (This is relevant for the i\_blocks field of the inode.)
    \item You should be able to handle directories that require more than one block.
    \item You should be able to handle a file that needs a single indirection
    \item Although you can construct your own structs from the information in the documentation above, you are welcome to use the ext2.h file that I used for the test code. I took out a bunch of components that we aren't using, but there are still quite a few fields that are irrelevant for our purposes.
\end{itemize}

\bigskip

However, you will probably also want to explore the disk images to get an intuitive sense of how they are structured. (The next three exercises will also help you explore the disk images and get started on the assignment.)

There are two good ways to interface with these images. The first way is to interact with it like a user by mounting the file system so that you can use standard commands (mkdir, cp, rm, ln) to interact with it. Details of how to do this are below. The second way is to interact with the disk as if it is a flat binary file. Use xxd to create hex dumps, diff to look for differences between the dumps, and your favorite text editor to view the diffs. For example (YMMV):

% diff <(xxd emptydisk.img) <(xxd onefile.img) > empty-onefile.diff
% vimdiff empty-onefile.diff
You should be able to use a combination of these techniques to understand how files are placed on disk and how they are removed. For example, you can create a new disk image, use mount to place files of various sizes on it, unmount it, and then use xxd and diff to see how the image differs from the other images you have.


\bigskip

\subsection{Mounting about File System}

\bigskip

\section{Submission}

\end{document}