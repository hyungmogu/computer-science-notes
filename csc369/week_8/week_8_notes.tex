\documentclass[12pt]{article}
\usepackage[margin=2.5cm]{geometry}
\usepackage{enumerate}
\usepackage{amsfonts}
\usepackage{amsmath}
\usepackage{fancyhdr}
\usepackage{amsmath}
\usepackage{amssymb}
\usepackage{amsthm}
\usepackage{mdframed}
\usepackage{graphicx}
\usepackage{subcaption}
\usepackage{adjustbox}
\usepackage{listings}
\usepackage{xcolor}
\usepackage{booktabs}
\usepackage[utf]{kotex}
\usepackage{hyperref}

\definecolor{codegreen}{rgb}{0,0.6,0}
\definecolor{codegray}{rgb}{0.5,0.5,0.5}
\definecolor{codepurple}{rgb}{0.58,0,0.82}
\definecolor{backcolour}{rgb}{0.95,0.95,0.92}

\lstdefinestyle{mystyle}{
    backgroundcolor=\color{backcolour},
    commentstyle=\color{codegreen},
    keywordstyle=\color{magenta},
    numberstyle=\tiny\color{codegray},
    stringstyle=\color{codepurple},
    basicstyle=\ttfamily\footnotesize,
    breakatwhitespace=false,
    breaklines=true,
    captionpos=b,
    keepspaces=true,
    numbers=left,
    numbersep=5pt,
    showspaces=false,
    showstringspaces=false,
    showtabs=false,
    tabsize=1
}

\lstset{style=mystyle}

\pagestyle{fancy}
\renewcommand{\headrulewidth}{0.4pt}
\lhead{Hyungmo Gu}
\rhead{CSC369 Week 8 Notes}

\begin{document}
\title{CSC369 Week 8 Notes}
\author{Hyungmo Gu}
\maketitle

\begin{itemize}
    \item File Systems
    \begin{itemize}
        \item Is the part of operating system dealing with files $^{[2]}$
        \item Controls how data is stored and retrieved. $^{[1]}$
        \begin{itemize}
            \item Without a file system, data placed in a storage medium is one
            large body of data with no way to tell where it stops and the next begins
        \end{itemize}
    \end{itemize}

    \begin{center}
    \includegraphics[width=0.8\linewidth]{images/week_8_notes_1_1.png}
    \end{center}

    \bigskip

    \underline{\textbf{Refernces:}}

    \bigskip

    \begin{enumerate}[1)]
        \item Wikipedia: File Systems, \href{https://en.wikipedia.org/wiki/Paging}{link}
        \item Tanebaum AS, Boss H. 2015. Modern Operating Systems. 4th Edition. New Jersy: Pearson Education, Inc.
    \end{enumerate}
    \item File Concept
    \begin{itemize}
        \item Files
        \begin{itemize}
            \item Are logical units of information created by processes $^{[1]}$
            \item Is named collection of data with some attributes
            \begin{enumerate}[1.]
                \item Name
                \item Owner
                \item Location
                \item Size
                \item Protection
                \item Creation Time
                \item Time of Last Access
            \end{enumerate}
        \end{itemize}

    \end{itemize}

    \bigskip

    \underline{\textbf{Refernces:}}

    \bigskip

    \begin{enumerate}[1)]
        \item Tanebaum AS, Boss H. 2015. Modern Operating Systems. 4th Edition. New Jersy: Pearson Education, Inc.
    \end{enumerate}
    % \item File Types
    % \item Conceptual File Operation
    % \item File Access Methods
    % \item Handling Operation on Files
    % \item Shared Open Files
    \item Directories
    \begin{itemize}
        \item Are file system files for maintaining the structure of the file
        system $^{[1]}$
        \item Serves multiple purposes
        \begin{itemize}
            \item \textit{All} $\to$ Stores information about files (owner, permission, etc)
            \item \textit{Users} $\to$ provides a structured way to organize files
            \item \textit{File System} $\to$ provides a convinent naming interface
            that allows the implementation to separate \textbf{logical file} organization
            from \textbf{physical file} placement on the disk

            \bigskip

            \begin{itemize}
                \item \textbf{Logical files:} Is a channel that connects the program
                to the physical file (Stream) $^{[2]}$
                \item \textbf{Physical files:} A collection of bits stored in the
                secondary storage $^{[2]}$

                \bigskip

                \underline{\textbf{Example:}}

                \bigskip

                FILE* output;

                output = fopen("sample.txt", "w");

                \bigskip

                Here, output is the logical file and sample.txt is the physical file
            \end{itemize}

            \begin{center}
            \includegraphics[width=0.8\linewidth]{images/week_8_notes_1_2.png}
            \end{center}
        \end{itemize}
    \end{itemize}

    \bigskip

    \underline{\textbf{Refernces:}}

    \bigskip

    \begin{enumerate}[1)]
        \item Tanebaum AS, Boss H. 2015. Modern Operating Systems. 4th Edition. New Jersy: Pearson Education, Inc.
        \item Kumar, S. (2010). \textit{File structures} [PowerPoint Slides]. Slide Share \href{https://www.slideshare.net/shyamujaco/file-structures}{link}
    \end{enumerate}
    \item What is a Directory at the OS Level?
    \item Operations on Directories
    \item Example Directory Operations
    % \item Path Name Translation
    % \item Possible Directory Implementations
    % \item File Links
    \item Symbolic vs Hard Links

    \begin{center}
    \includegraphics[width=0.8\linewidth]{images/week_8_notes_1_3.png}
    \end{center}

    \begin{itemize}
        \item \textbf{Inode}
        \begin{itemize}
            \item Is a database structure in a UNIX-style file system that describes
            a file system object such as a file or a directory $^{[1]}$
            \item Contains disk block location of the object's data $^{[1]}$
            \item Is a numerical equivalent of a full address $^{[2]}$
        \end{itemize}
        \item \textbf{Symbolic Link:}
        \begin{itemize}
            \item Is directory entry containing "true" path to the file
            \item Is a shortcut that reference to a file instead of inode value $^{[2]}$
        \end{itemize}

        \begin{center}
            \includegraphics[width=0.8\linewidth]{images/week_8_notes_1_4.png}
            \includegraphics[width=0.8\linewidth]{images/week_8_notes_1_5.png}
            \includegraphics[width=0.8\linewidth]{images/week_8_notes_1_6.png}
        \end{center}

        \item \textbf{Hard Link:}
        \begin{itemize}
            \item Is a direct reference to a file via its inode $^{[2]}$
            \item Is second directory entry identical to first
        \end{itemize}

        \begin{center}
        \includegraphics[width=0.8\linewidth]{images/week_8_notes_1_10.png}
        \includegraphics[width=0.8\linewidth]{images/week_8_notes_1_11.png}
        \includegraphics[width=0.8\linewidth]{images/week_8_notes_1_12.png}
        \includegraphics[width=0.8\linewidth]{images/week_8_notes_1_13.png}
        \end{center}
    \end{itemize}

    \bigskip

    \underline{\textbf{Refernces:}}

    \bigskip

    \begin{enumerate}[1)]
        \item Wikipedia: inode, \href{https://en.wikipedia.org/wiki/Inode}{link}
        \item Andrew. (2018, January 16). \textit{Hard links and Symbolic links — A comparison}. Medium. \href{https://medium.com/@307/hard-links-and-symbolic-links-a-comparison-7f2b56864cdd}{link}
    \end{enumerate}
    % \item Issues with Acyclic Graphs
    \item File Sharing
    \item Protection

    \begin{itemize}
        \item File systems implement some kind of protection system
        \begin{itemize}
            \item Who can access a file
            \item How they can access it
        \end{itemize}
        \item Protection system dictates whether given \color{green}\textbf{action}
        \color{black}\:by a given \color{orange}\textbf{subject}\color{black}\:on
        a given \color{red}\:\textbf{object}\color{black}\: should be allowed
        \begin{itemize}
            \item You can read and/or write your files, but others cannot
            \item You can read "etc/motd", but you cannot write it
        \end{itemize}
    \end{itemize}

    \begin{center}
    \includegraphics[width=0.8\linewidth]{images/week_8_notes_1_7.png}
    \end{center}
    \item Types of Access
    \item Representing Protection
    \item ACLs and Capabilities
    \item File System Implementation
    \item Directory Implementation
    \item Disk Layout Strategies
    \item Contiguous Allocation
    \item Linked Allocation
    \item Indexed Allocation: Unix Inodes
    \begin{itemize}
        \item Each inode contains 15 block pointers
        \begin{itemize}
            \item First 12 are direct block pointers
            \begin{itemize}
                \item Stops here if files are small
            \end{itemize}
            \item Then single, double and triple indirect
        \end{itemize}
    \end{itemize}

    \begin{center}
    \includegraphics[width=0.8\linewidth]{images/week_8_notes_1_8.png}
    \end{center}

    \underline{\textbf{Refernces:}}

    \bigskip

    \begin{enumerate}[1)]
        \item Wikipedia: Inode Pointer Structure, \href{https://en.wikipedia.org/wiki/Inode_pointer_structure}{link}
        \item Udacity (2015). \textit{Inode Structure} [online]. Available at: \href{https://www.youtube.com/watch?v=tMVj22EWg6A}{link} (Accessed May 28th, 2020)
    \end{enumerate}
    \item Unix Inodes and Path Search
    \begin{itemize}
        \item Unix Inodes
        \begin{itemize}
            \item Is what we see on typing `ls -li' command in terminal

            \begin{center}
            \includegraphics[width=\linewidth]{images/week_8_notes_1_9.png}
            \end{center}
            \item Describes where on the disk the blocks for a file are placed
            \item inode information is loaded to main memory $^{[1]}$
            \begin{itemize}
                \item Only for the corresponding files that are open
                \item NOT all are loaded
            \end{itemize}
        \end{itemize}
    \end{itemize}

    \bigskip

    \underline{\textbf{Refernces:}}

    \bigskip

    \begin{enumerate}[1)]
        \item Tanebaum AS, Boss H. 2015. Modern Operating Systems. 4th Edition. New Jersy: Pearson Education, Inc.
    \end{enumerate}
    \item File Buffer Cache
    \begin{itemize}
        \item Reads information from disk only once and then stores retrieved file blocks
        in memory until no longer needed $^{[]}$
        \begin{itemize}
            \item Because reading from disk is slow
            \item Is common to read same part of disk multiple times

            \bigskip

            \underline{\textbf{Example:}}

            \bigskip

            \begin{enumerate}[1.]
                \item Reading email message, read the message for an edit, and
                read the message again when copying to folder
            \end{enumerate}
        \end{itemize}
    \end{itemize}

    \bigskip

    \underline{\textbf{Refernces:}}

    \bigskip

    \begin{enumerate}[1)]
        \item Linux System Administrators Guide: Chapter 6. Memory Management, \href{https://www.tldp.org/LDP/sag/html/buffer-cache.html}{link}
    \end{enumerate}
    % \item Caching Writes
    % \item Read Ahead
\end{itemize}

\end{document}