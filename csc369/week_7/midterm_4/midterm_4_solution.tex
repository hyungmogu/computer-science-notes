\documentclass[12pt]{article}
\usepackage[margin=2.5cm]{geometry}
\usepackage{enumerate}
\usepackage{amsfonts}
\usepackage{amsmath}
\usepackage{fancyhdr}
\usepackage{amsmath}
\usepackage{amssymb}
\usepackage{amsthm}
\usepackage{mdframed}
\usepackage{graphicx}
\usepackage{subcaption}
\usepackage{adjustbox}
\usepackage{listings}
\usepackage{xcolor}
\usepackage{courier}
\usepackage[utf]{kotex}
\usepackage{hyperref}
\usepackage{soul}

\definecolor{codegreen}{rgb}{0,0.6,0}
\definecolor{codegray}{rgb}{0.5,0.5,0.5}
\definecolor{codepurple}{rgb}{0.58,0,0.82}
\definecolor{backcolour}{rgb}{0.95,0.95,0.92}

\lstdefinestyle{mystyle}{
    backgroundcolor=\color{backcolour},
    commentstyle=\color{codegreen},
    keywordstyle=\color{magenta},
    numberstyle=\tiny\color{codegray},
    stringstyle=\color{codepurple},
    basicstyle=\ttfamily\footnotesize,
    breakatwhitespace=false,
    breaklines=true,
    captionpos=b,
    keepspaces=true,
    numbers=left,
    numbersep=5pt,
    showspaces=false,
    showstringspaces=false,
    showtabs=false,
    tabsize=1
}

\lstset{style=mystyle}

\pagestyle{fancy}
\renewcommand{\headrulewidth}{0.4pt}
\lhead{CSC 369}
\rhead{Midterm 4 Solution}

\begin{document}
\title{CSC 369 Midterm 4 Solution}

\bigskip

\begin{enumerate}[1.]
    \item

    \begin{enumerate}[a)]

        \item
        \begin{enumerate}[1)]
            \item 4 inode blocks. 1 for the file \texttt{c}, and 3 for the
            directdories \texttt{/}, \texttt{a}, \texttt{b}

            \item 3 directory blocks - one for root \texttt{/}, one for \texttt{a},
            the other for \texttt{b}

            \item 1 single indirect block as far as we know. The file definitely has more than
            12 blocks (\# of data blocks pointed by direct pounters), but less than 1036 (\# of data blocks pointed
            by direct pointers and single indirect pointers). We are reading block 1034.

            \item 1 data block for file \texttt{c}
        \end{enumerate}

        \item

        All of the above

        \bigskip

        \underline{\textbf{Notes}}

        \begin{itemize}
            \item \textbf{Inode}

            \begin{itemize}
                \item Is short form of \textbf{index node}
                \item describes a file system object such as file or data
                \item contains all information about a file/directory, including
                \begin{itemize}
                    \item File Type,
                    \item Size
                    \item Time information (e.g time created, time modified)
                    \item Location of data blocks residing on disk
                \end{itemize}

            \end{itemize}
        \end{itemize}

    \end{enumerate}
\end{enumerate}


\end{document}