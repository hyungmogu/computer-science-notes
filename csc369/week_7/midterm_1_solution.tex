\documentclass[12pt]{article}
\usepackage[margin=2.5cm]{geometry}
\usepackage{enumerate}
\usepackage{amsfonts}
\usepackage{amsmath}
\usepackage{fancyhdr}
\usepackage{amsmath}
\usepackage{amssymb}
\usepackage{amsthm}
\usepackage{mdframed}
\usepackage{graphicx}
\usepackage{subcaption}
\usepackage{adjustbox}
\usepackage{listings}
\usepackage{xcolor}
\usepackage{courier}
\usepackage[utf]{kotex}
\usepackage{hyperref}

\definecolor{codegreen}{rgb}{0,0.6,0}
\definecolor{codegray}{rgb}{0.5,0.5,0.5}
\definecolor{codepurple}{rgb}{0.58,0,0.82}
\definecolor{backcolour}{rgb}{0.95,0.95,0.92}

\lstdefinestyle{mystyle}{
    backgroundcolor=\color{backcolour},
    commentstyle=\color{codegreen},
    keywordstyle=\color{magenta},
    numberstyle=\tiny\color{codegray},
    stringstyle=\color{codepurple},
    basicstyle=\ttfamily\footnotesize,
    breakatwhitespace=false,
    breaklines=true,
    captionpos=b,
    keepspaces=true,
    numbers=left,
    numbersep=5pt,
    showspaces=false,
    showstringspaces=false,
    showtabs=false,
    tabsize=1
}

\lstset{style=mystyle}

\pagestyle{fancy}
\renewcommand{\headrulewidth}{0.4pt}
\lhead{CSC 369}
\rhead{Midterm 1 Solution}

\begin{document}
\title{CSC 369 Midterm 1 Solution}

\bigskip

\begin{enumerate}[1.]
    \item

    \begin{enumerate}[a)]
        \item Trap instruction is run in user mode, and privileged operation is
        run in kernel mode
        \bigskip

        \underline{\textbf{Notes}}

        \begin{itemize}
            \item \textbf{Previliged Instructions}

            \begin{itemize}
                \item Is the instruction that can run only in \textbf{kernel mode}
                \item Attempt at execution in \textbf{user mode} $\to$ treated as an illegal operation \& will not run.
            \end{itemize}

            \item \textbf{Trap}

            \begin{itemize}
                \item Is a special hardware instruction
                \item Is a software generated interrupt $^{[4]}$
                \item Is a type of synchronous interrupt $^{[1]}$
                \item Is caused by an exceptional condition $^{[1]}$

                \begin{enumerate}[1.]
                    \item Division by zero $^{[1]}$
                    \item Invalid memory access (segmentation fault) $^{[1]}$
                    \item Previleged instruction by \textbf{user mode} code $^{[2]}$
                \end{enumerate}
                \item Usually results in a switch to \textbf{kernel mode} $\to$ Operating system performs action $\to$
                Returns control to oroginal process
            \end{itemize}

            \item \textbf{Trap Instruction}

            \begin{itemize}
                \item Is executed when a user wants to invoke a service from the operating system (i.e. reading hard drive)
                in \textbf{user mode}
                \item Raise (the processor) privilege level to kernel mode
            \end{itemize}

            \item \textbf{User Mode}

            \begin{itemize}
                \item Is restricted
                \item Executing code has no ability to \textit{directly} access
                hardware or reference memory $^{[3]}$
                \item Crashes are always recoverable $^{[3]}$
                \item Is where most of the code on our computer / applications are executed $^{[3]}$
            \end{itemize}

            \item \textbf{Kernel Mode}
            \begin{itemize}
                \item Is previleged (non-restricted)
                \item Executing code has complete and unrestricted access to the underlying hardware $^{[3]}$
                \item Is generally reserved for the lowest-level, most trusted functions of the operating
                system $^{[3]}$
                \item Is fatal to crash; it will halt the entire PC (i.e the blue screen of death) $^{[3]}$
            \end{itemize}
        \end{itemize}

        \bigskip

        \underline{\textbf{References}}

        \begin{enumerate}[1)]
            \item Wikipedia, Trap (computing), \href{https://en.wikipedia.org/wiki/Trap_(computing)#:~:text=In%20computing%20and%20operating%20systems,zero%2C%20invalid%20memory%20access).}{link}
            \item University of Utah, CS5460: Operating Systems Lecture 3 - OS Organization, \href{https://my.eng.utah.edu/~cs5460/slides/Lecture03.pdf}{link}
            \item Coding Horror, Understanding User and Kernel Mode, \href{https://blog.codinghorror.com/understanding-user-and-kernel-mode/}{link}
            \item ETH Zurich, Programming in Systems, \href{link}{link}
        \end{enumerate}

        \item

        No. Lock uses a variable with binary states 0 (acquired) and 1 (available), where as
        semaphore uses counter variable that can have value greater than 1 to
        keep track of the amount of resource remaining.

        \bigskip

        \underline{\textbf{Notes}}

        \begin{itemize}
            \item \textbf{Locks}

            \begin{itemize}
                \item Is a variable with two boolean states

                \begin{itemize}
                    \item 1 - (available/unlock/free)
                    \item 0 - (acquired/locked/held)
                \end{itemize}

                \item Has two operations

                \begin{enumerate}[1.]
                    \item \texttt{acquire()}

                    \bigskip

                    \begin{center}
                    \includegraphics[width=0.7\linewidth]{images/midterm_1_solution_4.png}
                    \end{center}

                    \bigskip

                    \item \texttt{release()}

                    \bigskip

                    \begin{center}
                    \includegraphics[width=0.7\linewidth]{images/midterm_1_solution_5.png}
                    \end{center}

                    \bigskip
                \end{enumerate}
                \item Is put around critical section to ensure critical section executes
                as if it's a single atomic instruction

                \begin{center}
                \includegraphics[width=\linewidth]{images/midterm_1_solution_1.png}
                \end{center}
                \item Can only be released by the thread that acquired it
                \item Is used to protect shared resource (e.g. from race condition in files and data structure) $^{[2]}$
            \end{itemize}

            \item \textbf{Semaphore}

            \begin{itemize}
                \item Is an abstract data types suitable for synchronization problems $^{[2]}$
                \item Has variable \texttt{count} that allows arbitrary resource count $^{[1]}$
                \item Has two atomic operations

                \begin{enumerate}[1.]
                    \item (\texttt{wait/P/decrement}) - block until \texttt{count $>$ 0} then decrement variable

                    \begin{center}
                    \includegraphics[width=0.7\linewidth]{images/midterm_1_solution_2.png}
                    \end{center}

                    \item (\texttt{signal/V/increment}) - increment \texttt{count}, unblock a waiting thread

                    \begin{center}
                    \includegraphics[width=0.7\linewidth]{images/midterm_1_solution_3.png}
                    \end{center}
                \end{enumerate}

                \item Can be signaled by any thread $^{[2]}$
            \end{itemize}
        \end{itemize}

        \bigskip

        \underline{\textbf{References}}

        \begin{enumerate}[1)]
            \item Wikipedia, Semaphore (programming), \href{https://en.wikipedia.org/wiki/Semaphore_(programming)}{link}
            \item Stack Overflow, Difference between binary semaphore and mutex, \href{https://stackoverflow.com/questions/62814/difference-between-binary-semaphore-and-mutex}{link}
        \end{enumerate}

        \item

        If both access are read, then concurrency error will not occur.

        \bigskip

        \underline{\textbf{Notes}}

        \begin{itemize}
            \item What is concurrency error? Where and when does it occur?

            \item \textbf{Concurrency}

            \begin{itemize}
                \item Is the ability of different parts or units of a program, algorithm,
                or problem to be \underline{executed out of order}, \underline{without affecting the final
                outcome}. $^{[1]}$
            \end{itemize}

            \item \textbf{Concurrency Error}

            \begin{itemize}
                \item Two types of concurrency errors $^{[3]}$

                \begin{enumerate}[1.]
                    \item \textbf{Deadlock:} A situation wherein two or more processes
                    are never able to proceed because each is waiting for the others
                    to do something

                    \bigskip

                    Key: Circular wait

                    \bigskip

                    \item \textbf{Race Condition:} a timing dependent error
                    involving shared state

                    \begin{itemize}
                        \item \textbf{Data Race:} Concurrent accesses to a shared variable
                        and at least one access is a write
                        \item \textbf{Atomicity Bugs:} Code does not enforce the atomicity
                        programmers intended for a group of memory access
                        \item \textbf{Order Bugs:} Code does not enforce the order programmers
                        intended for a group of memory access
                    \end{itemize}
                \end{enumerate}
            \end{itemize}

            \item \textbf{Thread}

            \begin{itemize}
                \item Is the smallest sequence of programmed instructions that can be managed independently
                by a schdeduler $^{[2]}$

                \begin{center}
                \includegraphics[width=0.4\linewidth]{images/midterm_1_solution_6.png}
                \includegraphics[width=\linewidth]{images/midterm_1_solution_7.png}
                \end{center}

                \item A thread is bound to a single process
                \item A process can have multiple threads
            \end{itemize}
        \end{itemize}

        \bigskip

        \underline{\textbf{References}}

        \begin{enumerate}[1)]
            \item Wikipedia, Concurrency (computer science), \href{https://en.wikipedia.org/wiki/Concurrency_(computer_science)}{link}
            \item Wikipedia, Thread, \href{https://en.wikipedia.org/wiki/Thread_(computing)#:~:text=In%20computer%20science%2C%20a%20thread,part%20of%20the%20operating%20system.}{link}
            \item Columbia University, Concurrency Errors, \href{https://www.cs.columbia.edu/~junfeng/13fa-w4118/lectures/l11-concurrency-error.pdf}{link}
        \end{enumerate}


        \item

        \bigskip

        \underline{\textbf{Notes}}

        \begin{itemize}
            \item \textbf{Virtualization of CPU}

            \begin{itemize}
                \item
            \end{itemize}
            \item \textbf{Limited Direct Execution}

            \begin{itemize}
                \item Idea: Just run the program you want to run on the CPU,
                but first make sure to set up the hardware so as to limit what
                process can do without OS assistance
                \item baby proofs the CPU by

                \bigskip

                \begin{enumerate}[1.]
                    \item Setting up trap handlers
                    \item Starts an interrupt timer
                    \item Run processes in a restricted mode
                \end{enumerate}

                \bigskip

                \underline{\textbf{Example}}

                \bigskip

                Baby proofing a room:

                \bigskip

                \begin{itemize}
                    \item Locking cabinets containing dangerous stuff and covering electrical sockets.
                    \item When room is readied, let your baby roam free in knowledge that all the dangerous
                    aspect of the room is restricted
                \end{itemize}
            \end{itemize}

            \item \textbf{Trap Handlers}

            \begin{itemize}
                \item Is instruction that tells the hardware what to run when certain exceptions occur

                \bigskip

                \underline{\textbf{Example}}

                \bigskip

                What code to run when

                \begin{enumerate}[1.]
                    \item Hard disk interrupt occurs
                    \item Keyboard interrupt occrs
                    \item Program makes a system call?
                \end{enumerate}

            \end{itemize}

            \item \textbf{Timer Interrupt}

            \begin{itemize}
                \item Is a hardware mechanism that ensures the user program does not run forever
                \item Is emitted at regular intervals by a timer chip $^{[1]}$
            \end{itemize}
        \end{itemize}

        \bigskip

        \underline{\textbf{References}}

        \begin{enumerate}[1)]
            \item Wikibooks, Operating System Design/Processes/Interrupt, \href{https://en.wikibooks.org/wiki/Operating_System_Design/Processes/Interrupt#:~:text=Perhaps%20the%20most%20important%20interrupt,processor%20executing%20a%20specific%20instruction.}{link}
        \end{enumerate}
    \end{enumerate}
\end{enumerate}

\end{document}