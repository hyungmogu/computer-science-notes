\documentclass[12pt]{article}
\usepackage[margin=2.5cm]{geometry}
\usepackage{enumerate}
\usepackage{amsfonts}
\usepackage{amsmath}
\usepackage{fancyhdr}
\usepackage{amsmath}
\usepackage{amssymb}
\usepackage{amsthm}
\usepackage{mdframed}
\usepackage{graphicx}
\usepackage{subcaption}
\usepackage{adjustbox}
\usepackage{listings}
\usepackage{xcolor}
\usepackage{courier}
\usepackage[utf]{kotex}
\usepackage{hyperref}
\usepackage{soul}

\definecolor{codegreen}{rgb}{0,0.6,0}
\definecolor{codegray}{rgb}{0.5,0.5,0.5}
\definecolor{codepurple}{rgb}{0.58,0,0.82}
\definecolor{backcolour}{rgb}{0.95,0.95,0.92}

\lstdefinestyle{mystyle}{
    backgroundcolor=\color{backcolour},
    commentstyle=\color{codegreen},
    keywordstyle=\color{magenta},
    numberstyle=\tiny\color{codegray},
    stringstyle=\color{codepurple},
    basicstyle=\ttfamily\footnotesize,
    breakatwhitespace=false,
    breaklines=true,
    captionpos=b,
    keepspaces=true,
    numbers=left,
    numbersep=5pt,
    showspaces=false,
    showstringspaces=false,
    showtabs=false,
    tabsize=1
}

\lstset{style=mystyle}

\pagestyle{fancy}
\renewcommand{\headrulewidth}{0.4pt}
\lhead{CSC 369}
\rhead{Midterm 3 Solution}

\begin{document}
\title{CSC 369 Midterm 3 Solution}

\bigskip

\begin{enumerate}[1.]
    \item

    \begin{enumerate}[a)]

        \item
        Yes, they are part of system call's Application Programming Interface,
        and they are the only way to interact between computer program and OS kernel.
        \bigskip

        \underline{\textbf{Notes}}

        \begin{itemize}
            \item \textbf{System Calls}
            \begin{itemize}
                \item Is issued by a client
                \item Is the only entry points into the kernel system
                \item Provides services via API or Application Program Interface
                \item Has five different types of calls

                \begin{center}
                \includegraphics[width=0.8\linewidth]{../images/midterm_3_solution_1.png}
                \end{center}
            \end{itemize}

            \bigskip

            \underline{\textbf{Example}}

            \bigskip

            \texttt{open()}, \texttt{read()}, \texttt{write()}, \texttt{close()}, \texttt{mkdir()} are other examples of system calls
        \end{itemize}

        \bigskip

        \underline{\textbf{References}}

        \begin{enumerate}[1)]
            \item Tutorials Point, Types of System Calls, \href{https://www.tutorialspoint.com/different-types-of-system-calls}{link}
        \end{enumerate}

        \item

        It is user's responsibility to keep track of allocated blocks of heap memory,
        and memory leak occurs if user fails to deallocate allocated blocks of heap memory


        \bigskip

        \underline{\textbf{Notes}}

        \begin{itemize}
            \item \textbf{Memory API}

            \begin{itemize}
                \item Has two types of memory

                \begin{enumerate}[1.]
                    \item \textbf{Stack}

                    \begin{itemize}
                        \item Is also called \textbf{automatic memory}
                        \item Allocations and deallocations are managed by compiler
                        \item Deallocates memory by the end of function call
                    \end{itemize}

                    \item \textbf{Heap}

                    \begin{itemize}
                        \item Is long-lived
                        \item Allocation and deallocation are managed by user
                        \item Creates \textbf{memory leak} if memory not freed
                        \item \textbf{valgrind} is a useful heap memoery debugging tool \href{https://www.valgrind.org/docs/manual/quick-start.html}{link}
                    \end{itemize}
                \end{enumerate}

                \item \texttt{malloc()}
                \begin{itemize}
                    \item Is a C library call
                    \item \textbf{Syntax:} \texttt{void *malloc(size\_t size)}
                    \item Allocates a block of \texttt{size} bytes to \textbf{heap memory}
                    and if successful, returns a pointer to it
                    \item Returns \texttt{NULL} if memory allocation is unsuccessful
                \end{itemize}

                \bigskip

                \underline{\textbf{Example}}

                \bigskip

                \texttt{int *x = malloc(10 * sizeof(int));}

                \bigskip
                \item \texttt{free()}
                \begin{itemize}
                    \item Is a C library call
                    \item Frees heap memory that is no longer in use
                \end{itemize}

                \bigskip

                \underline{\textbf{Example}}

                \bigskip

                \texttt{int *x = malloc(10 * sizeof(int));}\\
                \texttt{...}\\
                \texttt{free(x);}

                \bigskip

                \item \texttt{brk(), sbrk(), mmap()}

                \begin{itemize}
                    \item Are system calls for memory management
                \end{itemize}

            \end{itemize}

            \item \textbf{Buffer overflow}
            \begin{itemize}
                \item is an error that occurs when not enough heap memory is allocated

                \bigskip

                \begin{center}
                \includegraphics[width=0.8\linewidth]{../images/midterm_3_solution_2.png}
                \end{center}
            \end{itemize}
        \end{itemize}

        \item

        If the access by two threads are both about reading the stored value (as opposed to write),
        then concurrency error will not occur.

        \item

        \bigskip

        \underline{\textbf{Notes}}

        \begin{itemize}
            \item \textbf{Coarse-grained-locking}

            \begin{itemize}
                \item Is one big lock that is used any time any critical section is accessed
                \item Is easy to write
                \item Is easy to prove correctness
                \item No fault-tolerance but deadlock-free
                \item Perfoms poorly when contention (the need for performance due to load) is high
                \begin{itemize}
                    \item No concurrent access
                \end{itemize}

                \bigskip

                \underline{\textbf{Example}}

                \bigskip

                \begin{center}
                \includegraphics[width=\linewidth]{../images/midterm_3_solution_10.png}
                \end{center}
            \end{itemize}
            \item \textbf{Fine-grained-locking}

            \begin{itemize}
                \item Uses different locks to often protect different data and data strutures
                \item Allows more threads to be in locked code at once

                \bigskip

                \underline{\textbf{Example}}

                \bigskip

                \begin{center}
                \includegraphics[width=\linewidth]{../images/midterm_3_solution_11.png}
                \end{center}

            \end{itemize}
            \item \textbf{Hand-over-hand locking}

            \begin{itemize}
                \item Idea: instead of having a single lock for the entire list, a lock per node
                of the list is added; when traversing the list, the list grabs the next node's lock,
                and releases the current node's lock
                \item Is a fine-grained-locking
                \item Holds at most 2 locks at a time

                \bigskip

                \underline{\textbf{Example}}

                \bigskip

                \begin{center}
                \includegraphics[width=\linewidth]{../images/midterm_3_solution_3.png}
                \includegraphics[width=\linewidth]{../images/midterm_3_solution_4.png}
                \includegraphics[width=\linewidth]{../images/midterm_3_solution_5.png}
                \includegraphics[width=\linewidth]{../images/midterm_3_solution_6.png}
                \includegraphics[width=\linewidth]{../images/midterm_3_solution_7.png}
                \includegraphics[width=\linewidth]{../images/midterm_3_solution_8.png}
                \includegraphics[width=\linewidth]{../images/midterm_3_solution_9.png}
                \end{center}

            \end{itemize}
        \end{itemize}

        \bigskip

        \underline{\textbf{References}}

        \begin{enumerate}[1)]
            \item Techion, Linked Lists: The Role of Locking, \href{http://www.cs.technion.ac.il/~erez/courses/seminar/talks/05.pdf}{link}
        \end{enumerate}

    \end{enumerate}


\end{enumerate}

\end{document}