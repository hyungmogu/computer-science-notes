\documentclass[12pt]{article}
\usepackage[margin=2.5cm]{geometry}
\usepackage{enumerate}
\usepackage{amsfonts}
\usepackage{amsmath}
\usepackage{fancyhdr}
\usepackage{amsmath}
\usepackage{amssymb}
\usepackage{amsthm}
\usepackage{mdframed}
\usepackage{graphicx}
\usepackage{subcaption}
\usepackage{adjustbox}
\usepackage{listings}
\usepackage{xcolor}
\usepackage{courier}
\usepackage[utf]{kotex}
\usepackage{hyperref}

\definecolor{codegreen}{rgb}{0,0.6,0}
\definecolor{codegray}{rgb}{0.5,0.5,0.5}
\definecolor{codepurple}{rgb}{0.58,0,0.82}
\definecolor{backcolour}{rgb}{0.95,0.95,0.92}

\lstdefinestyle{mystyle}{
    backgroundcolor=\color{backcolour},
    commentstyle=\color{codegreen},
    keywordstyle=\color{magenta},
    numberstyle=\tiny\color{codegray},
    stringstyle=\color{codepurple},
    basicstyle=\ttfamily\footnotesize,
    breakatwhitespace=false,
    breaklines=true,
    captionpos=b,
    keepspaces=true,
    numbers=left,
    numbersep=5pt,
    showspaces=false,
    showstringspaces=false,
    showtabs=false,
    tabsize=1
}

\lstset{style=mystyle}

\pagestyle{fancy}
\renewcommand{\headrulewidth}{0.4pt}
\lhead{CSC 369}
\rhead{Midterm 2 Solution}

\begin{document}
\title{CSC 369 Midterm 2 Solution}

\bigskip

\begin{enumerate}[1.]
    \item

    \begin{enumerate}[a)]
        \item False
        \item True
        \item
    \end{enumerate}

    \bigskip

    \underline{\textbf{Notes}}

    \bigskip

    \begin{itemize}
        \item \textbf{User Mode}

        \begin{itemize}
            \item Is restricted
            \item Executing code has no ability to \textit{directly} access
            hardware or reference memory $^{[1]}$
            \item Crashes are always recoverable $^{[1]}$
            \item Is where most of the code on our computer / applications are executed $^{[3]}$
        \end{itemize}

        \item \textbf{Kernel Mode}
        \begin{itemize}
            \item Is previleged (non-restricted)
            \item Executing code has complete and unrestricted access to the underlying hardware $^{[3]}$
            \item Is generally reserved for the lowest-level, most trusted functions of the operating
            system $^{[1]}$
            \item Is fatal to crash; it will halt the entire PC (i.e the blue screen of death) $^{[3]}$
        \end{itemize}

        \item \textbf{Interrupt}

        \begin{itemize}
            \item Are signals sent to the CPU by external devices, normally I/O devices. $^{[2]}$
            \item Tells the CPU to stop its current activities and execute the appropriate part of the operating system (\textbf{Interrupt Handler}). $^{[2]}$
            \item Has three different types $^{[2]}$

            \begin{enumerate}[1)]
                \item \textbf{Hardware Interupts}

                \begin{itemize}
                    \item Are generated by hardware devices to signal that they need some attention from the OS.
                    \item May be due to receiving some data

                    \bigskip

                    \underline{\textbf{Examples}}

                    \bigskip

                    \begin{itemize}
                        \item Keystrokes on the keyboard
                        \item Receiving data on the ethernet card
                    \end{itemize}

                    \bigskip

                    \item May be due to completing a task which the operating system previous requested

                    \bigskip

                    \underline{\textbf{Examples}}

                    \bigskip

                    Transfering data between the hard drive and memory
                \end{itemize}

                \bigskip

                \item \textbf{Software Interupts}

                \bigskip

                \begin{itemize}
                    \item Are generated by programs when a system call is requested
                \end{itemize}

                \bigskip

                \item \textbf{Traps}

                \bigskip

                \begin{itemize}
                    \item Are generated by the CPU itself
                    \item Indicate that some error or condition occured for which assistance from the operating system is needed
                \end{itemize}

                \bigskip
            \end{enumerate}
        \end{itemize}

        \item \textbf{Content Switch}

        \begin{itemize}
            \item Is switching from running a user level process to the OS kernel and often
            to other user processes before the current process is resumed
            \item Happens during a timer interrupt or system call
            \item Saves the following states for a process during a context switch
            \begin{itemize}
                \item Stack Pointer
                \item Program Counter
                \item User Registers
                \item Kernel State
            \end{itemize}
            \item May hinder performance
        \end{itemize}

        \item \textbf{System Call}

        \bigskip

        \underline{\textbf{Example}}

        \begin{itemize}
            \item \texttt{yield()}
            \begin{itemize}
                \item Is a system call
                \item Causes the calling thread to relinquish the CPU
                \item Places the current thread at the end of the run queue
                \item Schedules another thread to run
            \end{itemize}
        \end{itemize}

        \item \textbf{Threads}

        \begin{itemize}
            \item
        \end{itemize}
    \end{itemize}

    \bigskip

    \underline{\textbf{References}}

    \begin{enumerate}[1)]
        \item Coding Horror, Understanding User and Kernel Mode, \href{https://blog.codinghorror.com/understanding-user-and-kernel-mode/}{link}
        \item Kansas State University, Basics of How Operating Systems Work, \href{http://faculty.salina.k-state.edu/tim/ossg/Introduction/OSworking.html#:~:text=Interrupts%20are%20signals%20sent%20to,part%20of%20the%20operating%20system.&text=Hardware%20Interupts%20are%20generated%20by,some%20attention%20from%20the%20OS.}{link}
        \item Kansas State University, Glossary, \href{http://faculty.salina.k-state.edu/tim/ossg/glossary.html#term-context-switch}{link}
    \end{enumerate}
\end{enumerate}

\end{document}