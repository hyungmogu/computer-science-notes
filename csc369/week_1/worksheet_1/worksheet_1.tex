\documentclass[12pt]{article}
\usepackage[margin=2.5cm]{geometry}
\usepackage{enumerate}
\usepackage{amsfonts}
\usepackage{amsmath}
\usepackage{fancyhdr}
\usepackage{amsmath}
\usepackage{amssymb}
\usepackage{amsthm}
\usepackage{mdframed}
\usepackage{graphicx}
\usepackage{subcaption}
\usepackage{adjustbox}
\usepackage{listings}
\usepackage{xcolor}
\usepackage{booktabs}
\usepackage[utf]{kotex}
\usepackage{hyperref}

\definecolor{codegreen}{rgb}{0,0.6,0}
\definecolor{codegray}{rgb}{0.5,0.5,0.5}
\definecolor{codepurple}{rgb}{0.58,0,0.82}
\definecolor{backcolour}{rgb}{0.95,0.95,0.92}

\lstdefinestyle{mystyle}{
    backgroundcolor=\color{backcolour},
    commentstyle=\color{codegreen},
    keywordstyle=\color{magenta},
    numberstyle=\tiny\color{codegray},
    stringstyle=\color{codepurple},
    basicstyle=\ttfamily\footnotesize,
    breakatwhitespace=false,
    breaklines=true,
    captionpos=b,
    keepspaces=true,
    numbers=left,
    numbersep=5pt,
    showspaces=false,
    showstringspaces=false,
    showtabs=false,
    tabsize=1
}

\lstset{style=mystyle}

\pagestyle{fancy}
\renewcommand{\headrulewidth}{0.4pt}
\lhead{CSC 369}
\rhead{Worksheet 1}

\begin{document}
\title{CSC 369 Worksheet 1}
\maketitle

\bigskip

Source: \href{http://pages.cs.wisc.edu/~remzi/OSTEP/cpu-intro.pdf}{link}

\bigskip

\begin{enumerate}[1.]
    \item Run process-run.py with the following flags: -l 5:100,5:100.
    What should the CPU utilization be (e.g., the percent of time the
    CPU is in use?) Why do you know this? Use the -c and -p flags to
    see if you were right

    \item  Now run with these flags: ./process-run.py -l 4:100,1:0.
    These flags specify one process with 4 instructions (all to use the
    CPU), and one that simply issues an I/O and waits for it to be done.
    How long does it take to complete both processes? Use -c and -p
    to find out if you were right.

    \item Switch the order of the processes: -l 1:0,4:100. What happens
    now? Does switching the order matter? Why? (As always, use -c
    and -p to see if you were right)

    \item We’ll now explore some of the other flags. One important flag is
    -S, which determines how the system reacts when a process issues an I/O. With the flag set to SWITCH ON END, the system
    will NOT switch to another process while one is doing I/O, instead waiting until the process is completely finished. What happens when you run the following two processes (-l 1:0,4:100
    -c -S SWITCH ON END), one doing I/O and the other doing CPU
    work?

    \item Now, run the same processes, but with the switching behavior set
    to switch to another process whenever one is WAITING for I/O (-l
    1:0,4:100 -c -S SWITCH ON IO). What happens now? Use -c
    and -p to confirm that you are right.

    \item  One other important behavior is what to do when an I/O completes. With -I IO RUN LATER, when an I/O completes, the process that issued it is not necessarily run right away; rather, whatever
    was running at the time keeps running. What happens when you
    run this combination of processes? (Run ./process-run.py -l
    3:0,5:100,5:100,5:100 -S SWITCH ON IO -I IO RUN LATER
    -c -p) Are system resources being effectively utilized?

    \item Now run the same processes, but with -I IO RUN IMMEDIATE set,
    which immediately runs the process that issued the I/O. How does
    this behavior differ? Why might running a process that just completed an I/O again be a good idea?

    \item  Now run with some randomly generated processes: -s 1 -l 3:50,3:50
    or -s 2 -l 3:50,3:50 or -s 3 -l 3:50,3:50. See if you can
    predict how the trace will turn out. What happens when you use
    the flag -I IO RUN IMMEDIATE vs. -I IO RUN LATER? What happens when you
    use -S SWITCH ON IO vs. -S SWITCH ON END?
\end{enumerate}

\end{document}