\documentclass[12pt]{article}
\usepackage[margin=2.5cm]{geometry}
\usepackage{enumerate}
\usepackage{amsfonts}
\usepackage{amsmath}
\usepackage{fancyhdr}
\usepackage{amsmath}
\usepackage{amssymb}
\usepackage{amsthm}
\usepackage{mdframed}
\usepackage{graphicx}
\usepackage{subcaption}
\usepackage{adjustbox}
\usepackage{listings}
\usepackage{xcolor}
\usepackage{booktabs}
\usepackage[utf]{kotex}
\usepackage{soul}

\definecolor{codegreen}{rgb}{0,0.6,0}
\definecolor{codegray}{rgb}{0.5,0.5,0.5}
\definecolor{codepurple}{rgb}{0.58,0,0.82}
\definecolor{backcolour}{rgb}{0.95,0.95,0.92}

\lstdefinestyle{mystyle}{
    backgroundcolor=\color{backcolour},
    commentstyle=\color{codegreen},
    keywordstyle=\color{magenta},
    numberstyle=\tiny\color{codegray},
    stringstyle=\color{codepurple},
    basicstyle=\ttfamily\footnotesize,
    breakatwhitespace=false,
    breaklines=true,
    captionpos=b,
    keepspaces=true,
    numbers=left,
    numbersep=5pt,
    showspaces=false,
    showstringspaces=false,
    showtabs=false,
    tabsize=1
}

\lstset{style=mystyle}

\pagestyle{fancy}
\renewcommand{\headrulewidth}{0.4pt}
\lhead{Hyungmo Gu}
\rhead{CSC369 Assignment 1}

\begin{document}
\title{CSC369 Assignment 1 - Hijacking System Calls and Monitoring Process}
\maketitle

\section{Overview}

\begin{itemize}
    \item

    In this assignment, you will achieve the goal of hijacking (intercepting)
    system calls by writing and installing a very basic kernel module to the Linux kernel.

    \bigskip

    Here is what ``hijacking (intercepting) a system call'' means. You will implement
    a new system call named my\_syscall, which will allow you to send commands from
    userspace, to intercept another pre-existing system call (like read, write, o
    pen, etc.). After a system call is intercepted, the intercepted system call
    would log a message first before continuing performing what it was supposed to do.

    \bigskip

    For example, if we call my\_syscall with command REQUEST\_SYSCALL\_INTERCEPT
    and target system call number \_\_NR\_mkdir (which is the macro representing the
    system call mkdir) as parameters, then the mkdir system call would be intercepted;
    then, when another process calls mkdir, mkdir would log some message (e.g.,
    "muhahaha") first, then perform what it was supposed to do (i.e., make a directory).

    \bigskip

    But wait, that's not the whole story yet. Actually we don't want mkdir to log
    a message whenever any process calls it. Instead, we only want mkdir to log
    a message when a certain set of processes (PIDs) are calling mkdir. In other
    words, we want to monitor a set of PIDs for the system call mkdir. Therefore,
    you will need to keep track, for each intercepted system call, of the list of
    monitored PIDs. Our new system call will support two additional commands to add/remove
    PIDs to/from the list.

    \bigskip

    When we want to stop hijacking a system call (let's say mkdir but it can be any
    of the previously hijacked system calls), we can invoke the interceptor
    (my\_syscall), with a REQUEST\_SYSCALL\_RELEASE command as an argument and the
    system call number that we want to release. This will stop intercepting the
    target system call mkdir, and the behaviour of mkdir should go back to normal
    like nothing happened.
\end{itemize}

\section{Checklist}

\begin{itemize}
    \item
\end{itemize}

\section{Goal}

\section{Requirements}

\section{Error Conditions}

\section{General Information}

\end{document}