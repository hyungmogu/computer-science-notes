\documentclass[12pt]{article}
\usepackage[margin=2.5cm]{geometry}
\usepackage{enumerate}
\usepackage{amsfonts}
\usepackage{amsmath}
\usepackage{fancyhdr}
\usepackage{amsmath}
\usepackage{amssymb}
\usepackage{amsthm}
\usepackage{mdframed}
\usepackage{graphicx}
\usepackage{subcaption}
\usepackage{adjustbox}
\usepackage{listings}
\usepackage{xcolor}
\usepackage{booktabs}
\usepackage[utf]{kotex}
\usepackage{hyperref}

\definecolor{codegreen}{rgb}{0,0.6,0}
\definecolor{codegray}{rgb}{0.5,0.5,0.5}
\definecolor{codepurple}{rgb}{0.58,0,0.82}
\definecolor{backcolour}{rgb}{0.95,0.95,0.92}

\lstdefinestyle{mystyle}{
    backgroundcolor=\color{backcolour},
    commentstyle=\color{codegreen},
    keywordstyle=\color{magenta},
    numberstyle=\tiny\color{codegray},
    stringstyle=\color{codepurple},
    basicstyle=\ttfamily\footnotesize,
    breakatwhitespace=false,
    breaklines=true,
    captionpos=b,
    keepspaces=true,
    numbers=left,
    numbersep=5pt,
    showspaces=false,
    showstringspaces=false,
    showtabs=false,
    tabsize=1
}

\lstset{style=mystyle}

\pagestyle{fancy}
\renewcommand{\headrulewidth}{0.4pt}
\lhead{CSC 373}
\rhead{Worksheet 2}

\begin{document}
\title{CSC373 Worksheet 2}
\maketitle

\bigskip

Source: \href{http://pages.cs.wisc.edu/~remzi/OSTEP/cpu-api.pdf}{link}

\bigskip

\section{Homework (Code)}

\begin{enumerate}[1.]
    \item Write a program that calls \texttt{fork()}. Before calling fork(), have the
    main process access a variable (e.g., \texttt{x}) and set its value to something (e.g., \texttt{100}).
    What value is the variable in the child process?
    What happens to the variable when both the child and parent change
    the value of \texttt{x}?

    \item Write a program that opens a file (with the \texttt{open()} system call)
    and then calls \texttt{fork()} to create a new process. Can both the child
    and parent access the file descriptor returned by \texttt{open()}? What
    happens when they are writing to the file concurrently, i.e., at the
    same time?

    \item Write another program using \texttt{fork()}. The child process should
    print “hello”; the parent process should print “goodbye”. You should
    try to ensure that the child process always prints first; can you do
    this without calling \texttt{wait()} in the parent?

    \item Write a program that calls \texttt{fork()} and then calls some form of
    \texttt{exec()} to run the program \texttt{/bin/ls}. See if you can try all of the
    variants of \texttt{exec()}, including (on Linux) \texttt{execl()}, \texttt{execle()},
    \texttt{execlp()}, \texttt{execv()}, \texttt{execvp()}, and \texttt{execvpe()}. Why do
    you think there are so many variants of the same basic call?

    \item Now write a program that uses \texttt{wait()} to wait for the child process
    to finish in the parent. What does \texttt{wait()} return? What happens if
    you use \texttt{wait()} in the child?

    \item Write a slight modification of the previous program, this time using
    \texttt{waitpid()} instead of \texttt{wait()}. When would \texttt{waitpid()}
    be useful?

    \item Write a program that creates a child process, and then in the child
    closes standard output (\texttt{STDOUT\_FILENO}). What happens if the child
    calls \texttt{printf()} to print some output after closing the descriptor?

    \item Write a program that creates two children, and connects the standard output of one to the standard input of the other, using the
    \texttt{pipe()} system call.
\end{enumerate}

\end{document}