\documentclass[12pt]{article}
\usepackage[margin=2.5cm]{geometry}
\usepackage{enumerate}
\usepackage{amsfonts}
\usepackage{amsmath}
\usepackage{fancyhdr}
\usepackage{amsmath}
\usepackage{amssymb}
\usepackage{amsthm}
\usepackage{mdframed}
\usepackage{graphicx}
\usepackage{subcaption}
\usepackage{adjustbox}
\usepackage{listings}
\usepackage{xcolor}
\usepackage{booktabs}
\usepackage[utf]{kotex}
\usepackage{hyperref}

\definecolor{codegreen}{rgb}{0,0.6,0}
\definecolor{codegray}{rgb}{0.5,0.5,0.5}
\definecolor{codepurple}{rgb}{0.58,0,0.82}
\definecolor{backcolour}{rgb}{0.95,0.95,0.92}

\lstdefinestyle{mystyle}{
    backgroundcolor=\color{backcolour},
    commentstyle=\color{codegreen},
    keywordstyle=\color{magenta},
    numberstyle=\tiny\color{codegray},
    stringstyle=\color{codepurple},
    basicstyle=\ttfamily\footnotesize,
    breakatwhitespace=false,
    breaklines=true,
    captionpos=b,
    keepspaces=true,
    numbers=left,
    numbersep=5pt,
    showspaces=false,
    showstringspaces=false,
    showtabs=false,
    tabsize=1
}

\lstset{style=mystyle}

\pagestyle{fancy}
\renewcommand{\headrulewidth}{0.4pt}
\lhead{CSC 369}
\rhead{Worksheet 2}

\begin{document}
\title{CSC 369 Worksheet 2}
\maketitle

\bigskip

Source: \href{http://pages.cs.wisc.edu/~remzi/OSTEP/cpu-api.pdf}{link}

\bigskip

\section{Homework (Simulation)}

\bigskip

\begin{enumerate}[1.]
    \item Run \texttt{./fork.py -s 10} and see which actions are taken. Can you
    predict what the process tree looks like at each step? Use the \texttt{-c}
    flag to check your answers. Try some different random seeds (\texttt{-s})
    or add more actions (\texttt{-a}) to get the hang of it.

    \item One control the simulator gives you is the fork percentage, controlled by the \texttt{-f} flag. The higher it is, the more likely the next
    action is a fork; the lower it is, the more likely the action is an
    exit. Run the simulator with a large number of actions (e.g., \texttt{-a
    100}) and vary the fork percentage from 0.1 to 0.9. What do you
    think the resulting final process trees will look like as the percentage changes? Check your answer with \texttt{-c}.

    \item Now, switch the output by using the \texttt{-t} flag (e.g., run \texttt{./fork.py
    -t}). Given a set of process trees, can you tell which actions were
    taken?

    \item One interesting thing to note is what happens when a child exits;
    what happens to its children in the process tree? To study this, let’s
    create a specific example: \texttt{./fork.py -A a+b,b+c,c+d,c+e,c-}.
    This example has process ’a’ create ’b’, which in turn creates ’c’,
    which then creates ’d’ and ’e’. However, then, ’c’ exits. What do
    you think the process tree should like after the exit? What if you
    use the \texttt{-R} flag? Learn more about what happens to orphaned processes on your own to add more context.

    \item One last flag to explore is the \texttt{-F} flag, which skips intermediate
    steps and only asks to fill in the final process tree. Run \texttt{./fork.py}
    \texttt{-F} and see if you can write down the final tree by looking at the
    series of actions generated. Use different random seeds to try this a
    few times.

    \item Finally, use both \texttt{-t} and \texttt{-F} together. This shows the final process
    tree, but then asks you to fill in the actions that took place. By looking at the tree, can you determine the exact actions that took place?
    In which cases can you tell? In which can’t you tell? Try some different random seeds to delve into this question.
\end{enumerate}

\bigskip

\section{Homework (Code)}

\begin{enumerate}[1.]
    \item Write a program that calls \texttt{fork()}. Before calling fork(), have the
    main process access a variable (e.g., \texttt{x}) and set its value to something (e.g., \texttt{100}).
    What value is the variable in the child process?
    What happens to the variable when both the child and parent change
    the value of \texttt{x}?

    \item Write a program that opens a file (with the \texttt{open()} system call)
    and then calls \texttt{fork()} to create a new process. Can both the child
    and parent access the file descriptor returned by \texttt{open()}? What
    happens when they are writing to the file concurrently, i.e., at the
    same time?

    \item Write another program using \texttt{fork()}. The child process should
    print “hello”; the parent process should print “goodbye”. You should
    try to ensure that the child process always prints first; can you do
    this without calling \texttt{wait()} in the parent?

    \item Write a program that calls \texttt{fork()} and then calls some form of
    \texttt{exec()} to run the program \texttt{/bin/ls}. See if you can try all of the
    variants of \texttt{exec()}, including (on Linux) \texttt{execl()}, \texttt{execle()},
    \texttt{execlp()}, \texttt{execv()}, \texttt{execvp()}, and \texttt{execvpe()}. Why do
    you think there are so many variants of the same basic call?

    \item Now write a program that uses \texttt{wait()} to wait for the child process
    to finish in the parent. What does \texttt{wait()} return? What happens if
    you use \texttt{wait()} in the child?

    \item Write a slight modification of the previous program, this time using
    \texttt{waitpid()} instead of \texttt{wait()}. When would \texttt{waitpid()}
    be useful?

    \item Write a program that creates a child process, and then in the child
    closes standard output (\texttt{STDOUT\_FILENO}). What happens if the child
    calls \texttt{printf()} to print some output after closing the descriptor?

    \item Write a program that creates two children, and connects the standard output of one to the standard input of the other, using the
    \texttt{pipe()} system call.
\end{enumerate}

\end{document}