\documentclass[12pt]{article}
\usepackage[margin=2.5cm]{geometry}
\usepackage{enumerate}
\usepackage{amsfonts}
\usepackage{amsmath}
\usepackage{fancyhdr}
\usepackage{amsmath}
\usepackage{amssymb}
\usepackage{amsthm}
\usepackage{mdframed}
\usepackage{graphicx}
\usepackage{subcaption}
\usepackage{adjustbox}
\usepackage{listings}
\usepackage{xcolor}
\usepackage{courier}
\usepackage[utf]{kotex}
\usepackage{hyperref}
\usepackage{soul}
\usepackage{cancel}


\definecolor{codegreen}{rgb}{0,0.6,0}
\definecolor{codegray}{rgb}{0.5,0.5,0.5}
\definecolor{codepurple}{rgb}{0.58,0,0.82}
\definecolor{backcolour}{rgb}{0.95,0.95,0.92}

\lstdefinestyle{mystyle}{
    backgroundcolor=\color{backcolour},
    commentstyle=\color{codegreen},
    keywordstyle=\color{magenta},
    numberstyle=\tiny\color{codegray},
    stringstyle=\color{codepurple},
    basicstyle=\ttfamily\footnotesize,
    breakatwhitespace=false,
    breaklines=true,
    captionpos=b,
    keepspaces=true,
    numbers=left,
    numbersep=5pt,
    showspaces=false,
    showstringspaces=false,
    showtabs=false,
    tabsize=1
}

\lstset{style=mystyle}

\pagestyle{fancy}
\renewcommand{\headrulewidth}{0.4pt}
\lhead{CSC 369}
\rhead{Notes}

\begin{document}
\title{CSC 369 Notes}

\begin{enumerate}[1.]
    \item \textbf{Secondary Storage Devices}
    \begin{itemize}
        \item Focus will be on hard-drives
    \end{itemize}
    \item \textbf{Disk Components}

    \begin{center}
    \includegraphics[width=0.8\linewidth]{images/notes_1.png}
    \end{center}

    \begin{itemize}
        \item Parts
        \begin{itemize}
            \item \textbf{Platter:}
            \begin{itemize}
                \item Data can be stored in both upper and lower parts of the platter
            \end{itemize}
            \item \textbf{Cyliner:}
            \begin{itemize}
                \item Is a set of tracks that can be read without moving the arm
            \end{itemize}
            \item \textbf{Sector:}
            \begin{itemize}
                \item Size of disk block is multiple of sectors
            \end{itemize}
        \end{itemize}
        \item Disk suface crash
        \begin{center}
        \includegraphics[width=0.4\linewidth]{images/notes_5.png}
        \end{center}

        \begin{itemize}
            \item Occurs when disk arm touching surface
            \item Results in permanent loss of information on the track
        \end{itemize}
    \end{itemize}
    \item \textbf{Disk Performance}
    \begin{itemize}
        \item [\color{red}IMPORTANT\color{black}] We should know the bulk part time of how this works
        \item \textbf{Seek:}
        \begin{itemize}
            \item Is the time it takes to move the disk arm to correct cylinder
            \item Depends on how fast disk arm can move
            \item Typical time: 1-15ms, depending on distance (avg 5-6 ms)
            \item Improves very slowly (7 - 10\% per year)
        \end{itemize}
        \item \textbf{Rotation:}
        \begin{itemize}
            \item Is the time it takes to rotate under the head to get to correct sector
            \item Depends on rotation rate of disk
            \item Average latency of $\frac{1}{2}$ rotation
        \end{itemize}
        \item \textbf{Transfer:}
        \begin{itemize}
            \item Is the time it takes to transfer data from surface to disk controller, electronics
            and sending it back to host
            \item Depends on density
            \item $\sim 100 \text{MB/s}$, average sector transfer time of $\sim 5 \mu s$
            \item Improves rapidly ($\sim 40\%$ per year)
        \end{itemize}
    \end{itemize}
    \item \textbf{Traditional Service Time Component}

    \begin{center}
    \includegraphics[width=0.8\linewidth]{images/notes_2.png}
    \end{center}

    \begin{itemize}
        \item OS tries to minimize the cost of rotational latency, transfer time, and seek time
        \item Improvement attention especially on seek time and rotation latency
    \end{itemize}

    \item \textbf{Some Hardware Optimizations}
    \begin{itemize}
        \item \textbf{Track Skew}

        \begin{center}
        \includegraphics[width=0.8\linewidth]{images/notes_3.png}
        \end{center}

        \begin{itemize}
            \item Has to do with numbering on tracks
            \item Is to reduce rotational latency
        \end{itemize}
        \item \textbf{Zones}

        \begin{center}
        \includegraphics[width=0.8\linewidth]{images/notes_4.png}
        \end{center}

        \item \begin{itemize}
            \item Is to make sure data is stored with same density
            \item Is done to maximize the capacity of hard drive
            \item Outer tracks $\to$ holds more sectors
        \end{itemize}
        \item \textbf{Cache}
        \begin{itemize}
            \item Is also called \textbf{Track Buffer}
            \item Is a small memory chip embedded in hard drive ($8 - 16 \text{MB}$)
            \item Is aware of disk geometry
            \item May cache whole track
            \item Boosts future reads on the same track
        \end{itemize}
    \end{itemize}
    \item \textbf{Disk and the OS}
    \begin{itemize}
        \item The OS provides different levels of disk access to different clients
        \begin{itemize}
            \item Physical disk (e.g surface, cylinder, sector)
            \item [\color{red}IMPORTANT\color{black}] Logical disk (disk block \#) \color{red}$\leftarrow$ what we will do for the first assignment\color{black}
            \item Logical file (e.g file block, record, or byte \#)
        \end{itemize}
    \end{itemize}
\end{enumerate}

\bigskip

\begin{itemize}
    \item \textbf{Enhancing Disk Performance}
    \begin{itemize}
        \item File system needs to be aware of disk charactersistics for performance
        \begin{itemize}
            \item \textbf{Allocation Algorithm} $\to$ enhances performance
            \begin{itemize}
                \item e.g Extent-based allocation, indexed based allocation, linked-based allocation
            \end{itemize}
            \item \textbf{Request Scheduling} $\to$ reduce seek time
            \begin{itemize}
                \item e.g. FCFS, SSTF, SCAN, C-SCAN
            \end{itemize}
        \end{itemize}
        \item Disk characteristics yields to goals:
        \begin{itemize}
            \item \textbf{Amortization}
            \begin{itemize}
                \item Compensates positioning delay
                \item Grabs lots of useful data while at it
                \item Performance improvement upto factor of 10
            \end{itemize}
            \item \textbf{Closeness}
            \begin{itemize}
                \item Done by putting things close to each other
                \item Performance benefit in factors of 2
            \end{itemize}
        \end{itemize}
    \end{itemize}
    \item \textbf{Allocation Strategies}
    \begin{itemize}
        \item Disk perform best if seeks are reduced and large transfers are used
        \begin{itemize}
            \item Done by allocating data close together
            \item Reason why significant improvement in seek time and transmission time over the years
        \end{itemize}
    \end{itemize}
    \item \textbf{Original Unix File System}
    \begin{itemize}
        \item Is simple and straightforward
        \item Is slow (poor use of disk bandwidth)
        \item Has 2 placement problems
        \begin{enumerate}[1.]
            \item Fragmentation
            \begin{center}
            \includegraphics[width=0.8\linewidth]{images/notes_6.png}
            \end{center}

            \begin{itemize}
                \item Causes more seeking
            \end{itemize}
            \item The travel of back and forth between inode and data blocks

            \begin{center}
            \includegraphics[width=0.8\linewidth]{images/notes_7.png}
            \end{center}

            \begin{itemize}
                \item More seeking time
            \end{itemize}
        \end{enumerate}
    \end{itemize}

    \item \textbf{Fast File System}

    \begin{itemize}
        \item Is a disk aware file system
        \item Addressed placement problems using \textbf{cylinder groups}

        \begin{center}
        \includegraphics[width=0.8\linewidth]{images/notes_8.png}
        \end{center}

        \begin{itemize}
            \item Steps
            \begin{itemize}
                \item Data blocks in the same file allocated in same cylinder group
                \item Files in the same directory allocated in same cylinder group
                \item Inodes for files allocated in same cylinder group as file data blocks
            \end{itemize}
            \item Allocation in cylinder groups provide closeness $\to$ less long seeks
            \item Has Free space requirements
            \begin{itemize}
                \item
            \end{itemize}
        \end{itemize}
    \end{itemize}
\end{itemize}


\end{document}
