\documentclass[12pt]{article}
\usepackage[margin=2.5cm]{geometry}
\usepackage{enumerate}
\usepackage{amsfonts}
\usepackage{amsmath}
\usepackage{fancyhdr}
\usepackage{amsmath}
\usepackage{amssymb}
\usepackage{amsthm}
\usepackage{mdframed}
\usepackage{graphicx}
\usepackage{subcaption}
\usepackage{adjustbox}
\usepackage{listings}
\usepackage{xcolor}
\usepackage{courier}
\usepackage[utf]{kotex}
\usepackage{hyperref}
\usepackage{soul}
\usepackage{cancel}


\definecolor{codegreen}{rgb}{0,0.6,0}
\definecolor{codegray}{rgb}{0.5,0.5,0.5}
\definecolor{codepurple}{rgb}{0.58,0,0.82}
\definecolor{backcolour}{rgb}{0.95,0.95,0.92}

\lstdefinestyle{mystyle}{
    backgroundcolor=\color{backcolour},
    commentstyle=\color{codegreen},
    keywordstyle=\color{magenta},
    numberstyle=\tiny\color{codegray},
    stringstyle=\color{codepurple},
    basicstyle=\ttfamily\footnotesize,
    breakatwhitespace=false,
    breaklines=true,
    captionpos=b,
    keepspaces=true,
    numbers=left,
    numbersep=5pt,
    showspaces=false,
    showstringspaces=false,
    showtabs=false,
    tabsize=1
}

\lstset{style=mystyle}

\pagestyle{fancy}
\renewcommand{\headrulewidth}{0.4pt}
\lhead{CSC 369}
\rhead{Reading Notes}

\begin{document}
\title{CSC 369 Reading Notes}

\section{Process API}

\begin{mdframed}
\underline{\textbf{Vocabulary}}

\bigskip

\begin{enumerate}[1.]
    \item \textbf{Process Identifier (PID)}
    \begin{itemize}
        \item Is an unique identifier for an active process
    \end{itemize}
    \item \textbf{CPU Scheduler}
    \begin{itemize}
        \item Is a policy which determines which process to run at a given point in time
    \end{itemize}
    \item \textbf{Concurrency}
    \begin{itemize}
        \item Is the ability of a program to run out of order without affecting the final outcome
    \end{itemize}
    \item \textbf{Deterministic Execution}
    \begin{itemize}
        \item Means path of execution \underline{is} fully determined by the specification of computation
        \item Is guaranteed to procduce the same outcome, given the same input
    \end{itemize}
    \item \textbf{Non-determinism}
    \begin{itemize}
        \item Means path of execution \underline{isn't} fully determined by the specification of computation
        \item Same input can produce different outcomes
    \end{itemize}
    \item \textbf{Multi-threaded Programs}
    \begin{itemize}
        \item Is synonymous to \textbf{Multitasking}
        \item Is a program that processes multiples threads at one time
    \end{itemize}
    \item \textbf{Signal}
    \begin{itemize}
        \item Is events triggered by the CPU and software running on it
        \item Triggers corresponding handler on \underline{per process basis} when invoked
    \end{itemize}
\end{enumerate}

\end{mdframed}

\subsection{fork() System Call}
\begin{itemize}
    \item Creates a new process
    \item Is an almost exact copy of the calling process
    \item \textbf{Parent} is the creator
    \begin{itemize}
        \item Runs from \texttt{main()} (beginning of program)
    \end{itemize}
    \item \textbf{Child} is the newly created process
    \begin{itemize}
        \item Runs from \texttt{fork()} (where \texttt{fork()} occurs)
    \end{itemize}
\end{itemize}

\subsection{wait() System Call}
\begin{itemize}
    \item Forces parent to wait for a child process to finish its process
\end{itemize}

\subsection{exec() System Call}
\begin{itemize}
    \item \underline{Does not} create a new process
    \item Transforms currently running program into a different running program
    \item Current running program is overwritten
    \item Code segment, heap, and stack are re-initialized

    \bigskip

    \underline{\textbf{Example}}

    \bigskip

    \texttt{(pid: 123) p3.c} --- \texttt{exec()} ---$>$ \texttt{(pid: 123) ls -al}

\end{itemize}

\end{document}
