\documentclass[12pt]{article}
\usepackage[margin=2.5cm]{geometry}
\usepackage{enumerate}
\usepackage{amsfonts}
\usepackage{amsmath}
\usepackage{fancyhdr}
\usepackage{amsmath}
\usepackage{amssymb}
\usepackage{amsthm}
\usepackage{mdframed}
\usepackage{graphicx}
\usepackage{subcaption}
\usepackage{adjustbox}
\usepackage{listings}
\usepackage{xcolor}
\usepackage{booktabs}
\usepackage[utf]{kotex}
\usepackage{soul}
\usepackage{hyperref}

\definecolor{codegreen}{rgb}{0,0.6,0}
\definecolor{codegray}{rgb}{0.5,0.5,0.5}
\definecolor{codepurple}{rgb}{0.58,0,0.82}
\definecolor{backcolour}{rgb}{0.95,0.95,0.92}

\lstdefinestyle{mystyle}{
    backgroundcolor=\color{backcolour},
    commentstyle=\color{codegreen},
    keywordstyle=\color{magenta},
    numberstyle=\tiny\color{codegray},
    stringstyle=\color{codepurple},
    basicstyle=\ttfamily\footnotesize,
    breakatwhitespace=false,
    breaklines=true,
    captionpos=b,
    keepspaces=true,
    numbers=left,
    numbersep=5pt,
    showspaces=false,
    showstringspaces=false,
    showtabs=false,
    tabsize=1
}

\lstset{style=mystyle}

\pagestyle{fancy}
\renewcommand{\headrulewidth}{0.4pt}
\rhead{CSC369 Assignment 2}

\begin{document}
\title{CSC369 Assignment 2 - Page Tables and Replacement Algorithms}
\maketitle

\bigskip

\section{Introduction}

\bigskip

For this assignment, we're going back to the realm of user mode programming.
Specifically, you will have to simulate the operation of page tables and page
replacement. As I keep saying: the way to gain a solid understanding of the theory
is by applying it in practice.

\bigskip

\noindent You have two tasks in this assignment, which will be based on a \hl{virtual memory}
simulator. The first task is to implement \hl{virtual-to-physical address translation}
and hl{demand paging} using a \hl{two-level page table}. The second task is to implement
four different page replacement algorithms: \hl{FIFO}, \hl{Clock}, \hl{exact LRU}, and \hl{OPT}.

\bigskip

\noindent Before you start work, you should complete the set of readings about memory, if
you haven't done so already: \href{http://pages.cs.wisc.edu/~remzi/OSTEP/vm-paging.pdf}{link}

\bigskip

\section{Requirements}

\bigskip

\subsection{Setup}

\bigskip

\noindent You will find the starter code HERE. It is your responsibility this time to add
the code in your repository and make sure that you submit all the necessary files!

\noindent Note that you may be generating some large trace files and must NOT commit any of
the trace files that you generate to your repository or you will run into serious
problems with disk quota. Most of the trace programs should be familiar to you from
the exercise in week 6. We have added a blocked version of matrix multiply, \textit{blocked.c}
which should exhibit fewer page faults under at least some of the page replacement
algorithms. The Makefile shows you exactly how to compile and run the traces. Note
that it takes quite a while to run the trace collection.

\bigskip

\noindent \textit{Compile the trace programs and generate the traces.}

\bigskip

\noindent You may have noticed while doing the Exercise that the traces generated
by Valgrind are enormous since they contain every memory reference from the entire
execution. We have provided a program, \textit{fastslim.py} to reduce the traces by removing
repeated references to the same page that occur within a small window of each other
while preserving the important characteristics for \hl{virtual memory} simulation.
(For example, a sequence of references to pages A and B such as "ABABABABAB...AB"
are reduced to just "AB".) The runit script pipes the output of valgrind through
this program to create the reduced trace. If you wish, you can experiment with
fastslim.py to try omitting the instruction references from the trace or using a
smaller or larger window (\textit{fastslim.py --help}). You may also want to create traces
from other programs, and you will definitely want to create small manual traces
for testing.

\subsection{Task 1 - Address Translation and Paging}

\bigskip

\subsection{Task 2}

\bigskip

\subsection{Write up}

\end{document}