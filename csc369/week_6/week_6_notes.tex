\documentclass[12pt]{article}
\usepackage[margin=2.5cm]{geometry}
\usepackage{enumerate}
\usepackage{amsfonts}
\usepackage{amsmath}
\usepackage{fancyhdr}
\usepackage{amsmath}
\usepackage{amssymb}
\usepackage{amsthm}
\usepackage{mdframed}
\usepackage{graphicx}
\usepackage{subcaption}
\usepackage{adjustbox}
\usepackage{listings}
\usepackage{xcolor}
\usepackage{booktabs}
\usepackage[utf]{kotex}
\usepackage{hyperref}

\definecolor{codegreen}{rgb}{0,0.6,0}
\definecolor{codegray}{rgb}{0.5,0.5,0.5}
\definecolor{codepurple}{rgb}{0.58,0,0.82}
\definecolor{backcolour}{rgb}{0.95,0.95,0.92}

\lstdefinestyle{mystyle}{
    backgroundcolor=\color{backcolour},
    commentstyle=\color{codegreen},
    keywordstyle=\color{magenta},
    numberstyle=\tiny\color{codegray},
    stringstyle=\color{codepurple},
    basicstyle=\ttfamily\footnotesize,
    breakatwhitespace=false,
    breaklines=true,
    captionpos=b,
    keepspaces=true,
    numbers=left,
    numbersep=5pt,
    showspaces=false,
    showstringspaces=false,
    showtabs=false,
    tabsize=1
}

\lstset{style=mystyle}

\pagestyle{fancy}
\renewcommand{\headrulewidth}{0.4pt}
\lhead{Hyungmo Gu}
\rhead{CSC369 Week 6 Notes}

\begin{document}
\title{CSC369 Week 6 Notes}
\author{Hyungmo Gu}
\maketitle

\bigskip

\section{Virtual Memory \& Page Replacement}

\bigskip

\begin{itemize}
    % \item Recap
    % \item Page Lookups Overview
    % \item TLBs
    % \item Summary so far: Paging
    % \item How much space does a page table take up?
    % \item Managing Page Tables
    % \item Motivation: Two-Level Page Tables
    % \item Multilevel Page Tables
    \item Two-Level Page Tables
    % \item Two-Level Paging Example
    % \item Pentium Address Translation
    \item Inverted Page Tables (Read the book)
    % \item Efficient Translations
    % \item Page Allocation \& Eviction
    % \item Recap: Paging
    \item Page Faults
    % \item Policy Decision
    \item Demand Paging
    \item Prepaging (aka Prefetching)
    % \item Policy Decisions
    % \item Placement Policy
    % \item Policy Decisions
    % \item Evictng the Best Page
    \item Belady's Algorithm
    % \item What are possible Replacement Algorithms?
    \item Page Table Entries(PTE)
    \item Not-Recently-Used (NRU)
    \item First-In First-Out (FIFO)
    % \item Example of Belady's Anomaly
    \item Second-Chance
    % \item Implementing Second-Chance (clock)
    % \item Modeling Clock
    \item Least Recently Used (LRU)
    % \item Implementing Exact LRU
    % \item Modelling Exact LRU
    % \item Approximating LRU
    \item Counting-based Replacement
    % \item What are Possible Replacement Algorithms?
    % \item Fixed vs Variable Space
    % \item Working Set Model
    % \item Working Set Size
    % \item Working Set Problems
    \item Page Fault Frequench(PFF)
    \item Thrashing
    \item Windows XP Paging Policy
    \item Linux Paging
\end{itemize}

\end{document}