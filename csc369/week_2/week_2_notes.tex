\documentclass[12pt]{article}
\usepackage[margin=2.5cm]{geometry}
\usepackage{enumerate}
\usepackage{amsfonts}
\usepackage{amsmath}
\usepackage{fancyhdr}
\usepackage{amsmath}
\usepackage{amssymb}
\usepackage{amsthm}
\usepackage{mdframed}
\usepackage{graphicx}
\usepackage{subcaption}
\usepackage{adjustbox}
\usepackage{listings}
\usepackage{xcolor}
\usepackage{booktabs}
\usepackage[utf]{kotex}
\usepackage{hyperref}

\definecolor{codegreen}{rgb}{0,0.6,0}
\definecolor{codegray}{rgb}{0.5,0.5,0.5}
\definecolor{codepurple}{rgb}{0.58,0,0.82}
\definecolor{backcolour}{rgb}{0.95,0.95,0.92}

\lstdefinestyle{mystyle}{
    backgroundcolor=\color{backcolour},
    commentstyle=\color{codegreen},
    keywordstyle=\color{magenta},
    numberstyle=\tiny\color{codegray},
    stringstyle=\color{codepurple},
    basicstyle=\ttfamily\footnotesize,
    breakatwhitespace=false,
    breaklines=true,
    captionpos=b,
    keepspaces=true,
    numbers=left,
    numbersep=5pt,
    showspaces=false,
    showstringspaces=false,
    showtabs=false,
    tabsize=1
}

\lstset{style=mystyle}

\pagestyle{fancy}
\renewcommand{\headrulewidth}{0.4pt}
\lhead{Hyungmo Gu}
\rhead{CSC369 Week 2 Notes}

\begin{document}
\title{CSC369 Week 2 Notes}
\author{Hyungmo Gu}
\maketitle

\section{System Calls}

\bigskip

\begin{itemize}
    \item Bootstraping
    \begin{itemize}
        \item Bootstraping

        \begin{center}
        \includegraphics[width=\linewidth]{images/week_2_notes_1_3.png}
        \end{center}


        \begin{itemize}
            \item executes \textbf{Bootstrap Program}
            \begin{itemize}
                \item is the first code that runs when the computer system is started
            \end{itemize}
            \item Entire operating system depnds on the bootstrap program to work
            correctly
            \item Locates and loads kernel (code of operating system) onto RAM
            \begin{itemize}
                \item kernel = code of the operating system
                \item kernel is in HDD
            \end{itemize}
            \item Bootstrap program is in ROM
        \end{itemize}

        \item ROM
        \begin{itemize}
            \item is called \textbf{read-only-memory}
            \item Is also called \textbf{BIOS chip} (Basic Input/Output System)
            \item is non-volatile
            \item is stored in motherboard

            \begin{center}
            \includegraphics[width=0.8\linewidth]{images/week_2_notes_1_2.png}
            \end{center}
        \end{itemize}
    \end{itemize}

    \item Operating System Startup

    \begin{center}
    \includegraphics[width=\linewidth]{images/week_2_notes_1_4.png}
    \end{center}

    \begin{itemize}
        \item Initializes OS
        \begin{itemize}
            \item Initialize internal data structures
            \item Create first process
            \item Switch mode to user and start runing first process
            \item Wait for something to happen
        \end{itemize}
    \end{itemize}

    \item Requesting OS Services
    \begin{itemize}
        \item Some services offered by OS are:
        \begin{itemize}
            \item Program execution
            \begin{itemize}
                \item Loading program to memory and executing program
            \end{itemize}
            \item I/O operations
            \begin{itemize}
                \item Keyboard, mouse, speaker
            \end{itemize}
            \item File system manipulation
            \begin{itemize}
                \item Reading and writing files and directories
            \end{itemize}

            \item Error Detection
            \begin{itemize}
                \item Error that pops when printer ink is empty
            \end{itemize}

            \begin{center}
            \includegraphics[width=0.7\linewidth]{images/week_2_notes_1_6.png}
            \end{center}
        \end{itemize}

        \item Operating system and user programs are isolated

        \item How do they communicate?
    \end{itemize}
    \item Boundary Crossings

    \bigskip

    \begin{itemize}
        \item Boundary
        \begin{itemize}
            \item Is the line between user applications and kernel
            \item Data is difficult to move back and forth between this line
        \end{itemize}
        \item Boundary Corssings
        \begin{itemize}
            \item Is the communication that occurs between a program and kernel

            \begin{center}
            \includegraphics[width=\linewidth]{images/week_2_notes_1_7.png}
            \end{center}

            \item Communication occurs by sending data from one program into kernel,
            and then back
        \end{itemize}
        \item More can be found \href{https://docs.huihoo.com/darwin/kernel-programming-guide/boundaries/chapter_14_section_1.html}{here}
    \end{itemize}

    \item System Calls for Process Management

    \begin{itemize}
        \item Major system calls

        \begin{tabular}{|l|l|}
            \hline
            \textbf{Call} & \textbf{Description}\\
            \hline
            pid = fork() & Create a child process identical to the parent\\
            \hline
            pid = waitpid(pid, \&statloc, options) & Wait for a child to terminate\\
            \hline
            s = execve(name argv, environp) & Replace a process' core image\\
            \hline
            exit(status) & Terminate process execution and return status\\
            \hline
        \end{tabular}
    \end{itemize}

    \bigskip

    \begin{mdframed}
        \begin{itemize}
            \item Wait, System Calls?

            \bigskip

            \textbf{System Calls} are interrupt signals sent by software

            \begin{center}
            \includegraphics[width=\linewidth]{images/week_2_notes_1_8.png}
            \end{center}

            \begin{itemize}
                \item Is a programmatic way of a program requesting for service to
                kernel of operating system
                \item It's like `Hey OS, could you do $y$? It's really important'
                \item i.e. Fetching a file in hard-disk drive
            \end{itemize}

        \end{itemize}
    \end{mdframed}

    \item System Calls for File Management

    \begin{itemize}
        \item Major system calls

        \begin{tabular}{|l|l|}
            \hline
            \textbf{Call} & \textbf{Description}\\
            \hline
            fd = open(file, how, ...) & Open a file for reading, writing, or both\\
            \hline
            s = close(fd) & Close an open file\\
            \hline
            n = read(fd, buffer, nbytes) & Read data from a file into a buffer\\
            \hline
            n = write(fd, buffer, nbytes) & Write data from a buffer into a file\\
            \hline
            position = lseek(fd, offset, whence) & Move the file pointer\\
            \hline
            s = stat(name, \&buf) & get a file's status information\\
            \hline
        \end{tabular}
    \end{itemize}

    \item System Call Interface
    \begin{itemize}
        \item Interface
        \begin{itemize}
            \item Is a point where two systems, subjects, organizations, etc.
            meet and interact. (Definition)
        \end{itemize}
        \item System Call Interface
        \begin{itemize}
            \item Is the point where user mode and kernel mode meet and interact

            \begin{figure}[h!]
            \includegraphics[width=\linewidth]{images/week_2_notes_1_9.png}
            \caption{Now, there really is a party going on here :)}
            \end{figure}

            \item Most of the system call interface are hidden from the programmer
            by API
        \end{itemize}
        \item User Mode
        \begin{itemize}
            \item Cannot directly access memory and other hardwares
            \item Is safe
            \item Crash $\to$ doesn't halt entire system
        \end{itemize}
        \item Kernel Mode
        \begin{itemize}
            \item Does have access to memory and other hardwares
            \item Is a previleged mode
            \item Is not safe
            \item Is fragile
            \item Crash $\to$ halts entire system
            \begin{itemize}
                \item i.e. The Blue Screen of Death \textgreater :)
            \end{itemize}

            \begin{center}
            \includegraphics[width=\linewidth]{images/week_2_notes_1_12.png}
            \end{center}
        \end{itemize}

    \end{itemize}
    \item System Call Operation

    \begin{itemize}
        \item Register is a local storage device present in CPU
        \item Register may include the address of the memory location instead of real data
        \item CPU takes data that has to be processed from register

        \begin{center}
        \includegraphics[width=\linewidth]{images/week_2_notes_1_11.png}
        \end{center}

    \end{itemize}

\end{itemize}

\bigskip

\section{Intro to Synchronization}

\bigskip

\begin{itemize}
    \item Cooperating Process
    \begin{itemize}
        \item Independent Process
        \begin{itemize}
            \item Cannot affect or be affected by other processes
            \item i.e. Running \textit{hello\_world.c} program
        \end{itemize}
        \item Cooperating Process
        \begin{itemize}
            \item is not independent
            \item must be able to communicate and synchronize actions
            \item i.e. One child process outputs `abc', another child process outputs "CBA",
            and parent combines together to outputs `CBAabc' :)

            \begin{center}
            \includegraphics[width=0.8\linewidth]{images/week_2_notes_2_1.png}
            \end{center}

        \end{itemize}
    \end{itemize}
    \item Interprocess Communication
    \begin{itemize}
        \item Cooperating processes need to exchange information using
        either
        \begin{itemize}
            \item Shared memory (e.g. fork())
            \item Message sharing
        \end{itemize}

        \item Message passing models
        \begin{itemize}
            \item Send(P, msg) - Send msg to process P
            \item Receive(Q, msg) - Receive msg from process Q
        \end{itemize}
    \end{itemize}
    \item Motivating Example (importance of communication in cooperating process)
    \begin{itemize}
        \item ATM machine
        \begin{itemize}
            \item Suppose the bank has following functions

            \bigskip

            \begin{center}
            \includegraphics[width=0.8\linewidth]{images/week_2_notes_2_2.png}
            \end{center}

            \item And suppose Moe deposits \$100 and my love withdraws \$100 on two
            different ATM machines using the same account.


            \item Without communication, procceses become interleaved
            \item Then,

            \begin{center}
            \includegraphics[width=\linewidth]{images/week_2_notes_2_3.png}
            \end{center}
        \end{itemize}
    \end{itemize}
    \item What Went Wrong
    \begin{itemize}
        \item Two concurrent threads manipulated a \textit{shared resource} (the account)
        without synchronization.
        \item \textit{Race condition} occured
        \begin{itemize}
            \item Outcome depends on the order in which accesses taking place
        \end{itemize}
        \item Fix $\to$ Mutual Exclusion
        \begin{itemize}
            \item Allow resource be manipulated only one thread at a time
        \end{itemize}

        \bigskip

        \begin{center}
        \includegraphics[width=0.4\linewidth]{images/week_2_notes_2_4.png}
        \end{center}
    \end{itemize}

    \item The Plan - Mutual Exclusion
    \begin{itemize}
        \item Is a program object that prevents simultaneous access to a shared
        \item Is used to avoid the \textit{race condition}
        resource
        \item So, for this problem
        \begin{itemize}
            \item Given

            \begin{itemize}
                \item A set of $n$ threads, $T_0, T_1, ... , T_n$
                \item A set of resources shared between threads
                \item A segment of code which acceses the shared resource, called
                the \textit{critical section}

                \begin{center}
                \includegraphics[width=0.7\linewidth]{images/week_2_notes_2_5.png}
                \end{center}
            \end{itemize}

            \item We want to ensure that

            \begin{itemize}
                \item Only one thread at a time can execute in the critical section
                \item All other threads are forced to wait on entry
                \item When a thread leaves the CS, another can enter


                \begin{center}
                \includegraphics[width=0.5\linewidth]{images/week_2_notes_2_6.png}
                \end{center}
            \end{itemize}
        \end{itemize}
    \end{itemize}

    \item Aside: What program data is shared between threads?
    \begin{itemize}
        \item Shared
        \begin{enumerate}[1.]
            \item Local Variables
            \begin{itemize}
                \item Are stored in their own private stack
            \end{itemize}
        \end{enumerate}
        \item Not Shared
        \begin{enumerate}[1.]
            \item Global Variables
            \begin{itemize}
                \item Are stored in Static Data segment, accessible by any thread

                \begin{center}
                \includegraphics[width=0.8\linewidth]{images/week_2_notes_2_7.png}
                \end{center}

            \end{itemize}
            \item Dynamic Objects and other heap objects
            \begin{itemize}
                \item Allocated from heap with malloc/free or new/delete

                \begin{center}
                \includegraphics[width=0.8\linewidth]{images/week_2_notes_2_8.png}
                \end{center}
            \end{itemize}
        \end{enumerate}
    \end{itemize}

    \item Critical Section Requirements

    \begin{enumerate}[1.]
        \item Mutual Exclusion
        \begin{itemize}
            \item If one thread is in the CS, then no other is
        \end{itemize}
        \item Progress
        \begin{itemize}
            \item If no thread is in the CS, and some threads want to enter CS,
            it should be able to enter in definite time
        \end{itemize}
        \item Bounded Waiting (No Starvation)
        \begin{itemize}
            \item Means no process should wait for a resource for infinite amount of time.
        \end{itemize}
        \item Performance
        \begin{itemize}
            \item The overhead of entering and exiting the CS is small with respect
            to the work being done with it
        \end{itemize}
    \end{enumerate}

    \item 2-Thread Solutions: 1st Try

    \begin{center}
    \includegraphics[width=0.8\linewidth]{images/week_2_notes_2_10.png}
    \end{center}

    \begin{itemize}


        \item Let the threads share an integer variable \textit{turn}
        initialized to 0 (or 1)
        \item If \textit{turn = i}, thread $T_i$ is allowed into its CS

        \bigskip

        \begin{itemize}
            \item [GOOD] Mutual Exclusion is satisfied
            \begin{itemize}
                \item CS can be entered only one thread at a time
            \end{itemize}
            \item [BAD] Progress is not satisfied
            \begin{itemize}
                \item A thread may not be able to enter CS, even if another is in
                the code section
            \end{itemize}
        \end{itemize}
    \end{itemize}

    \begin{center}
    \includegraphics[width=\linewidth]{images/week_2_notes_2_11.png}
    \end{center}

    \item 2-Thread Solutions: 2nd Try


    \begin{center}
    \includegraphics[width=\linewidth]{images/week_2_notes_2_12.png}
    \end{center}

    \item Peterson's Algorithm

    \begin{itemize}
        \item Is a solution to critical section problem
        \item Is developed by Gary L. Peterson in 1981
        \item Allows critical section to be processed one thread at a time
        \begin{itemize}
            \item By rejecting other threads that tries to access the
            critical section at the same time
        \end{itemize}
    \end{itemize}

\end{itemize}

\end{document}