\documentclass[12pt]{article}
\usepackage[margin=2.5cm]{geometry}
\usepackage{enumerate}
\usepackage{amsfonts}
\usepackage{amsmath}
\usepackage{fancyhdr}
\usepackage{amsmath}
\usepackage{amssymb}
\usepackage{amsthm}
\usepackage{mdframed}
\usepackage{graphicx}
\usepackage{subcaption}
\usepackage{adjustbox}
\usepackage{listings}
\usepackage{xcolor}
\usepackage{booktabs}
\usepackage[utf]{kotex}
\usepackage{hyperref}

\definecolor{codegreen}{rgb}{0,0.6,0}
\definecolor{codegray}{rgb}{0.5,0.5,0.5}
\definecolor{codepurple}{rgb}{0.58,0,0.82}
\definecolor{backcolour}{rgb}{0.95,0.95,0.92}

\lstdefinestyle{mystyle}{
    backgroundcolor=\color{backcolour},
    commentstyle=\color{codegreen},
    keywordstyle=\color{magenta},
    numberstyle=\tiny\color{codegray},
    stringstyle=\color{codepurple},
    basicstyle=\ttfamily\footnotesize,
    breakatwhitespace=false,
    breaklines=true,
    captionpos=b,
    keepspaces=true,
    numbers=left,
    numbersep=5pt,
    showspaces=false,
    showstringspaces=false,
    showtabs=false,
    tabsize=1
}

\lstset{style=mystyle}

\pagestyle{fancy}
\renewcommand{\headrulewidth}{0.4pt}
\lhead{CSC 369}
\rhead{Worksheet 5}

\begin{document}
\title{CSC 369 Worksheet 5}
\maketitle

\bigskip

Source: \href{http://pages.cs.wisc.edu/~remzi/Classes/537/Spring2018/Book/cpu-sched-mlfq.pdf}{link}

\bigskip

\begin{enumerate}[1.]
    \item Run a few randomly-\ problems with just two jobs and
    two queues; compute the MLFQ execution trace for each. Make
    your life easier by limiting the length of each job and turning off I/Os.
    \item How would you run the scheduler to reproduce each of the examples in the chapter?
    \item How would you configure the scheduler parameters to behave just
    like a round-robin scheduler?
    \item Craft a workload with two jobs and scheduler parameters so that
    one job takes advantage of the older Rules 4a and 4b (turned on
    with the \texttt{-S} flag) to game the scheduler and obtain 99\% of the CPU
    over a particular time interval.
    \item Given a system with a quantum length of 10 ms in its highest queue,
    how often would you have to boost jobs back to the highest priority
    level (with the \texttt{-B} flag) in order to guarantee that a single longrunning (and potentially-starving) job gets at least 5\% of the CPU?
    \item One question that arises in scheduling is which end of a queue to
    add a job that just finished I/O; the \texttt{-I} flag changes this behavior
    for this scheduling simulator. Play around with some workloads
    and see if you can see the effect of this flag.
\end{enumerate}

\end{document}