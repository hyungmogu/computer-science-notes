\documentclass[12pt]{article}
\usepackage[margin=2.5cm]{geometry}
\usepackage{enumerate}
\usepackage{amsfonts}
\usepackage{amsmath}
\usepackage{fancyhdr}
\usepackage{amsmath}
\usepackage{amssymb}
\usepackage{amsthm}
\usepackage{mdframed}
\usepackage{graphicx}
\usepackage{subcaption}
\usepackage{adjustbox}
\usepackage{listings}
\usepackage{xcolor}
\usepackage{booktabs}
\usepackage[utf]{kotex}
\usepackage{hyperref}

\definecolor{codegreen}{rgb}{0,0.6,0}
\definecolor{codegray}{rgb}{0.5,0.5,0.5}
\definecolor{codepurple}{rgb}{0.58,0,0.82}
\definecolor{backcolour}{rgb}{0.95,0.95,0.92}

\lstdefinestyle{mystyle}{
    backgroundcolor=\color{backcolour},
    commentstyle=\color{codegreen},
    keywordstyle=\color{magenta},
    numberstyle=\tiny\color{codegray},
    stringstyle=\color{codepurple},
    basicstyle=\ttfamily\footnotesize,
    breakatwhitespace=false,
    breaklines=true,
    captionpos=b,
    keepspaces=true,
    numbers=left,
    numbersep=5pt,
    showspaces=false,
    showstringspaces=false,
    showtabs=false,
    tabsize=1
}

\lstset{style=mystyle}

\pagestyle{fancy}
\renewcommand{\headrulewidth}{0.4pt}
\lhead{CSC 369}
\rhead{Worksheet 5 Solution}

\begin{document}
\title{CSC 369 Worksheet 5 Solution}
\maketitle

\bigskip

\begin{enumerate}[1.]
    \item

    \bigskip

    I need to run randomly-generated problems with two jobs and two queues using
    file \texttt{mlfq.py} with I/O turned off, and compute the MLFQ execution trace for each.

    \bigskip

    Using the command \texttt{./mlfq.py -s 1 -m 10 -n 2 -j 2 -M 0}, we have

    \bigskip

    \begin{center}
    \includegraphics[width=0.6\linewidth]{images/worksheet_5_solution_3.png}
    \end{center}

    with

    \begin{itemize}
        \item allotments for queue 1 is 1
        \item quantum length for queue 1 is 10
        \item allotments for queue 0 is 1
        \item quantum length for queue 0 is 10
        \item no priority boost
    \end{itemize},

    the exeuction trace is:

    \bigskip

\begin{lstlisting}
    [time 0] Job begins by job 0
    [time 0] Job begins by job 1
    [time 0] Run job 0 at priority 1 [Ticks 9, Allotment 1, Time 1 (of 2)]
    [time 1] Run job 0 at priority 1 [Ticks 8, Allotment 1, Time 0 (of 2)]
    [time 2] Finished JOB 0
    [time 2] Run job 1 at priority 1 [Ticks 9, Allotment 1, Time 6 (of 7)]
    [time 3] Run job 1 at priority 1 [Ticks 8, Allotment 1, Time 5 (of 7)]
    [time 4] Run job 1 at priority 1 [Ticks 7, Allotment 1, Time 4 (of 7)]
    [time 5] Run job 1 at priority 1 [Ticks 6, Allotment 1, Time 3 (of 7)]
    [time 6] Run job 1 at priority 1 [Ticks 5, Allotment 1, Time 2 (of 7)]
    [time 7] Run job 1 at priority 1 [Ticks 4, Allotment 1, Time 1 (of 7)]
    [time 8] Run job 1 at priority 1 [Ticks 3, Allotment 1, Time 0 (of 7)]
    [time 9] Finished JOB 1
\end{lstlisting}

    \bigskip

    \underline{\textbf{Notes}}

    \begin{itemize}
        \item Learned that when alloted time is up, the next job starts immediately

        (\texttt{./mlfq.py -s 20 -m 20 -n 2 -j 2 -M 0 -c})

        \begin{center}
        \includegraphics[width=0.9\linewidth]{images/worksheet_5_solution_4.png}
        \end{center}

        \item Learned that when all jobs are at the bottom, without priority boost, jobs finishes by round robin

        (\texttt{./mlfq.py -s 20 -m 20 -n 2 -j 2 -M 0 -c})

        \begin{center}
        \includegraphics[width=0.8\linewidth]{images/worksheet_5_solution_5.png}
        \end{center}

        \item Learned that notification and subsequent job execution happen at the same time.
        \item The reason why round robin doesn't occur despite $\text{Priority(A)} = \text{Priority(B)}$
        is because allotment of queue is 1 (i.e. only one job can be in a queue)
        \item \textbf{allotment} means the amount of something allocated to a person/object (i.e. the size of queue)
        \item \texttt{-m 10} sets the maximum runtime of a job to 10
        \item \texttt{-M 0} turns off I/O in \texttt{mlfq.py}
        \item \texttt{-n 2} sets number of queues to 2
        \item \texttt{-j 2} sets number of jobs to 2

        \item \textbf{Multi-level Feeback Queue (MLFQ):}

        \begin{itemize}
            \item Is one of the most well-known approaches to scheduling
            \item Does two things:

            \begin{enumerate}[a)]
                \item Optimizes turnaround time
                \item Minimizes response time
            \end{enumerate}

            \item Uses \textbf{priority level} and \textbf{Queues} to achieve it's goal
        \end{itemize}

        \item \textbf{MLFQ Basic Rules:}
        \begin{itemize}
            \item Jobs on same queue $\to$ Same priority
            \item \textbf{Rule 1:} If $\text{Priority(A)} > \text{Priority(B)}$, A runs (B doesn't)
            \item \textbf{Rule 2:} If $\text{Priority(A)} = \text{Priority(B)}$, A \& B run in RR
        \end{itemize}

        \bigskip

        \begin{center}
        \includegraphics[width=0.8\linewidth]{images/worksheet_5_solution_1.png}
        \end{center}

        \item \textbf{Attemp \#1: How to Change Priority}

        \begin{itemize}
            \item \textbf{Rule 3:} When a job enters the system, it is placed at the \underline{highest}
            priority (the topmost queue)
            \item \textbf{Rule 4a:} If a job uses up an entire time slice while running , its' priority is
            \underline{reduced} (i.e. it moves down on queue).
            \item \textbf{Rule 4b:} If a job gives up the CPU before the time slice is up, it stays
            at the \underline{same} priority level (e.g I/O Operation)

            \begin{itemize}
                \item Means that the shifting down of priority level only depends on CPU time
            \end{itemize}

            \bigskip

            \underline{\textbf{Example (Along Came a Short Job):}}

            \bigskip

            \begin{enumerate}[1)]
                \item A job $A$ enters system
                \item Job is placed on highest Queue $Q_2$
                \item After time-slice (e.g. 10 ms) in $Q_2$, $A$ is placed on lower queue $Q_1$
                \item After time-slice in $Q_1$, $A$ is placed in lowest priority queue $Q_0$
            \end{enumerate}

            \begin{center}
            \includegraphics[width=0.8\linewidth]{images/worksheet_5_solution_2.png}
            \end{center}
        \end{itemize}

        \item \textbf{Attemp \#2: The Priority Boost}

        \begin{itemize}
            \item \textbf{Rule 5:} After some time period $S$, move all the jobs in the system
            to the topmost queue.

            \begin{itemize}
                \item This is to prevent starvation (i.e. a job never being run)
            \end{itemize}
        \end{itemize}

        \item \textbf{Attempt \#3: Better Accounting (Fix of Attempt \# 1)}

        \begin{itemize}
            \item Is to prevent programmers from gaming (i.e tricking) the CPU so
            all programs get a fair share of allotment time
            \item \textbf{Rule 4:} Once a job uses up its time allotment at a given level
            (regardless of how many times it has given up the CPU), its priority is reduced
            (it moves down one queue).
        \end{itemize}
    \end{itemize}

    \item

    I need to run the scheduler (\texttt{mlfq.py}) to reproduce each of the examples
    in the chapter.

    \bigskip

    \begin{itemize}
        \item Example 1: A Single Long-Running Job

        \bigskip

        Here, the example has

        \begin{itemize}
            \item 3 queues
            \item 1 job
            \item \texttt{10ms} as quantum length for queue 1
            \item \texttt{10ms} as quantum length for queue 2
            \item \texttt{10ms} as quantum length for queue 3
            \item \texttt{200ms} as run time for job 1
            \item no priority boost
        \end{itemize}

        \bigskip

        Combining together we have

        \texttt{./mlfq.py -l 0,200,0 -n 3 -j 1 -c}

        \item Example 2: Along Came A Short Job

        \bigskip

        Here, the example has

        \begin{itemize}
            \item 3 queues
            \item 2 jobs
            \item \texttt{10ms} as quantum length for queue 1
            \item \texttt{10ms} as quantum length for queue 2
            \item \texttt{10ms} as quantum length for queue 3
            \item \texttt{180ms} as run time for job 1
            \item \texttt{20ms} as run time for job 2
            \item 0ms as the starting time for job 1
            \item 100ms as the starting time for job 2
            \item no I/O operations for job 1
            \item no I/O operations for job 2
            \item no priority boost
        \end{itemize}

        \bigskip

        Combining together we have

        \texttt{./mlfq.py -l 0,180,0:100,20,0 -n 3 -j 2 -c}

        \item Example 3: What About I/O

        \bigskip

        Here, the example has

        \begin{itemize}
            \item 3 queues
            \item 2 jobs
            \item \texttt{10ms} as quantum length for queue 1
            \item \texttt{10ms} as quantum length for queue 2
            \item \texttt{10ms} as quantum length for queue 3
            \item \texttt{200ms} as run time for job 1
            \item \texttt{0ms} as the starting time for job 1
            \item \texttt{50ms} as the starting time for job 2
            \item \texttt{10ms} as the I/O time for job 2
            \item \texttt{10ms} as the frequency of I/O request
            \item no I/O operations for job 1
            \item no priority boost
        \end{itemize}

        \bigskip


    \end{itemize}


\end{enumerate}

\end{document}