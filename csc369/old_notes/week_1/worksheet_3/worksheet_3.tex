\documentclass[12pt]{article}
\usepackage[margin=2.5cm]{geometry}
\usepackage{enumerate}
\usepackage{amsfonts}
\usepackage{amsmath}
\usepackage{fancyhdr}
\usepackage{amsmath}
\usepackage{amssymb}
\usepackage{amsthm}
\usepackage{mdframed}
\usepackage{graphicx}
\usepackage{subcaption}
\usepackage{adjustbox}
\usepackage{listings}
\usepackage{xcolor}
\usepackage{booktabs}
\usepackage[utf]{kotex}
\usepackage{hyperref}

\definecolor{codegreen}{rgb}{0,0.6,0}
\definecolor{codegray}{rgb}{0.5,0.5,0.5}
\definecolor{codepurple}{rgb}{0.58,0,0.82}
\definecolor{backcolour}{rgb}{0.95,0.95,0.92}

\lstdefinestyle{mystyle}{
    backgroundcolor=\color{backcolour},
    commentstyle=\color{codegreen},
    keywordstyle=\color{magenta},
    numberstyle=\tiny\color{codegray},
    stringstyle=\color{codepurple},
    basicstyle=\ttfamily\footnotesize,
    breakatwhitespace=false,
    breaklines=true,
    captionpos=b,
    keepspaces=true,
    numbers=left,
    numbersep=5pt,
    showspaces=false,
    showstringspaces=false,
    showtabs=false,
    tabsize=1
}

\lstset{style=mystyle}

\pagestyle{fancy}
\renewcommand{\headrulewidth}{0.4pt}
\lhead{CSC 369}
\rhead{Worksheet 3}

\begin{document}
\title{CSC 369 Worksheet 3}
\maketitle

\bigskip

Source: \href{http://pages.cs.wisc.edu/~remzi/Classes/537/Spring2018/Book/cpu-mechanisms.pdf}{link}

\bigskip

\section{Homework (Measurement)}

In this homework, you’ll measure the costs of a system call and context
switch. Measuring the cost of a system call is relatively easy. For example,
you could repeatedly call a simple system call (e.g., performing a 0-byte
read), and time how long it takes; dividing the time by the number of
iterations gives you an estimate of the cost of a system call.

\bigskip

One thing you’ll have to take into account is the precision and accuracy of your timer.
A typical timer that you can use is \texttt{gettimeofday()}; read the man page for details.
What you’ll see there is that \texttt{gettimeofday()}
returns the time in microseconds since 1970; however, this does not mean
that the timer is precise to the microsecond. Measure back-to-back calls
to \texttt{gettimeofday()} to learn something about how precise the timer really is; this will tell you how many iterations of your null system-call
test you’ll have to run in order to get a good measurement result. If
\texttt{gettimeofday()} is not precise enough for you, you might look into
using the rdtsc instruction available on x86 machines.

\bigskip

Measuring the cost of a context switch is a little trickier. The lmbench
benchmark does so by running two processes on a single CPU, and setting up two UNIX pipes between them; a pipe is just one of many ways
processes in a UNIX system can communicate with one another. The first
process then issues a write to the first pipe, and waits for a read on the
second; upon seeing the first process waiting for something to read from
the second pipe, the OS puts the first process in the blocked state, and
switches to the other process, which reads from the first pipe and then
writes to the second. When the second process tries to read from the first
pipe again, it blocks, and thus the back-and-forth cycle of communication
continues. By measuring the cost of communicating like this repeatedly,
lmbench can make a good estimate of the cost of a context switch. You
can try to re-create something similar here, using pipes, or perhaps some
other communication mechanism such as UNIX sockets.

\bigskip

One difficulty in measuring context-switch cost arises in systems with
more than one CPU; what you need to do on such a system is ensure that
your context-switching processes are located on the same processor.
Fortunately, most operating systems have calls to bind a process to a
particular processor; on Linux, for example, the \texttt{sched\_setaffinity()} call
is what you’re looking for. By ensuring both processes are on the same
processor, you are making sure to measure the cost of the OS stopping
one process and restoring another on the same CPU.

\end{document}