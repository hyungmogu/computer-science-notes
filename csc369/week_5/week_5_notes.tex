\documentclass[12pt]{article}
\usepackage[margin=2.5cm]{geometry}
\usepackage{enumerate}
\usepackage{amsfonts}
\usepackage{amsmath}
\usepackage{fancyhdr}
\usepackage{amsmath}
\usepackage{amssymb}
\usepackage{amsthm}
\usepackage{mdframed}
\usepackage{graphicx}
\usepackage{subcaption}
\usepackage{adjustbox}
\usepackage{listings}
\usepackage{xcolor}
\usepackage{booktabs}
\usepackage[utf]{kotex}
\usepackage{hyperref}

\definecolor{codegreen}{rgb}{0,0.6,0}
\definecolor{codegray}{rgb}{0.5,0.5,0.5}
\definecolor{codepurple}{rgb}{0.58,0,0.82}
\definecolor{backcolour}{rgb}{0.95,0.95,0.92}

\lstdefinestyle{mystyle}{
    backgroundcolor=\color{backcolour},
    commentstyle=\color{codegreen},
    keywordstyle=\color{magenta},
    numberstyle=\tiny\color{codegray},
    stringstyle=\color{codepurple},
    basicstyle=\ttfamily\footnotesize,
    breakatwhitespace=false,
    breaklines=true,
    captionpos=b,
    keepspaces=true,
    numbers=left,
    numbersep=5pt,
    showspaces=false,
    showstringspaces=false,
    showtabs=false,
    tabsize=1
}

\lstset{style=mystyle}

\pagestyle{fancy}
\renewcommand{\headrulewidth}{0.4pt}
\lhead{Hyungmo Gu}
\rhead{CSC369 Week 5 Notes}

\begin{document}
\title{CSC369 Week 5 Notes}
\author{Hyungmo Gu}
\maketitle

\bigskip

\section{Memory Management}

\begin{itemize}
    % \item Recap
    % \item What does real systems do?
    \item Physical Memory vs Virtual Memory $^{[1]}$
    \begin{itemize}
        \item Physical Memory
        \begin{itemize}
            \item Is RAM :)!!
            \item Is the first memory used when computer requires memory such as
            loading application or OS
        \end{itemize}
        \item Virtual Memory
        \begin{itemize}
            \item Is stored on hard drive
            \item Is used when RAM is filled
            \item Is slower than RAM
        \end{itemize}
    \end{itemize}

    \underline{\textbf{Refernces:}}

    \bigskip

    \begin{enumerate}[1)]
        \item Tech Walla: What Is the Difference Between Virtual Memory \& Physical Memory?, \href{https://www.techwalla.com/articles/difference-virtual-memory-physical-memory_}{link}
    \end{enumerate}

    \item Memory Management
    \begin{itemize}
        \item Is the process of controlling and coordinating computer memory,
        by assigning portions known as \textbf{blocks} to various programs $^{[1]}$
    \end{itemize}

    \underline{\textbf{Refernces:}}

    \bigskip

    \begin{enumerate}[1)]
        \item Guru 99: Memory Management in OS: Contiguous, Swapping, Fragmentation \& Physical Memory?, \href{https://www.guru99.com/os-memory-management.html#1}{link}
    \end{enumerate}
    % \item Requirements
    % \item Meeting the Requirements
    % \item Address Binding
    % \item When are Addresses Bound?
    % \item Load-Time Binding Example
    % \item A better plan
    % \item Memory Management
    % \item Address Translation: Logical and Physical Addresses
    % \item How to Allocate Physical Memory?
    \item Fixed Partitioning
    % \item Placement with Fixed Partitions
    % \item Placement Example (Queue per Partition)
    \item Dynamic Partitioning
    % \item More Dynamic Partitioning
    % \item Heap Management
    % \item Tracking Memory Allocation
    % \item Freeing Blocks
    % \item Placement Algorithms
    % \item Comparing Placement Algorithms
    % \item Problems with Paging
    \item Paging
    % \item Example of Paging
    \item Address Translation
    % \item Address Translation: Partitioning Schemes
    % \item Hardware Relocation
    % \item Address Translation for Paging
    % \item Support for Paging
    % \item Example Address Translation
    % \item Page Table Entres (PTE)
    % \item Page Lookups Overview
    \item TLBS
    % \item Managing TLBS
    % \item Summary So far: Paging
    % \item How Much Space Does a Page Table Take Up?
    % \item Managing Page Tables
    % \item Motivation: Two-level page Tables
    % \item Pentium Address Translation
    % \item 64-Bit Address Spaces
    % \item Inverted Page Tables
    % \item Efficient Translations
\end{itemize}


\end{document}