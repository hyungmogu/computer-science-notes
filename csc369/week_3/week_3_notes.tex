\documentclass[12pt]{article}
\usepackage[margin=2.5cm]{geometry}
\usepackage{enumerate}
\usepackage{amsfonts}
\usepackage{amsmath}
\usepackage{fancyhdr}
\usepackage{amsmath}
\usepackage{amssymb}
\usepackage{amsthm}
\usepackage{mdframed}
\usepackage{graphicx}
\usepackage{subcaption}
\usepackage{adjustbox}
\usepackage{listings}
\usepackage{xcolor}
\usepackage{booktabs}
\usepackage[utf]{kotex}
\usepackage{hyperref}

\definecolor{codegreen}{rgb}{0,0.6,0}
\definecolor{codegray}{rgb}{0.5,0.5,0.5}
\definecolor{codepurple}{rgb}{0.58,0,0.82}
\definecolor{backcolour}{rgb}{0.95,0.95,0.92}

\lstdefinestyle{mystyle}{
    backgroundcolor=\color{backcolour},
    commentstyle=\color{codegreen},
    keywordstyle=\color{magenta},
    numberstyle=\tiny\color{codegray},
    stringstyle=\color{codepurple},
    basicstyle=\ttfamily\footnotesize,
    breakatwhitespace=false,
    breaklines=true,
    captionpos=b,
    keepspaces=true,
    numbers=left,
    numbersep=5pt,
    showspaces=false,
    showstringspaces=false,
    showtabs=false,
    tabsize=1
}

\lstset{style=mystyle}

\pagestyle{fancy}
\renewcommand{\headrulewidth}{0.4pt}
\lhead{Hyungmo Gu}
\rhead{CSC369 Week 3 Notes}

\begin{document}
\title{CSC369 Week 3 Notes}
\author{Hyungmo Gu}
\maketitle

\bigskip

\section{Synchronization}

\bigskip
\begin{itemize}
    \item Producer and Consumer Problem
    \begin{itemize}
        \item Is also known as \textbf{bound-and-buffer} problem
        \item Achieves synchronization
        \item Has two types of processes
        \begin{enumerate}[1.]
            \item \textbf{Producer}
            \begin{itemize}
                \item Produces data
                \item Puts data into buffer
            \end{itemize}
            \item \textbf{Consumer}
            \begin{itemize}
                \item Consumes data
                \item Removes data from buffer, one piece at a time
            \end{itemize}
        \end{enumerate}
        \item It's like kimchi factory, or delicious cookie factory :)

        \begin{center}
        \includegraphics[width=\linewidth]{images/week_3_notes_1_1.png}
        \end{center}
    \end{itemize}

    \item Semaphore
    \begin{itemize}
        \item Developed by Dijkstra in 1962.
        \item is a signal
        \begin{itemize}
            \item Uses a non-negative integer variable that is shared between threads
            [\textit{Note: Need to come back later}]
            \item Has two ``\textbf{atomic}'' operations
            \begin{enumerate}[1.]
                \item \textbf{Wait} (Also called P, or decrement)
                \item \textbf{Signal} (Also called V, or increment)
            \end{enumerate}
        \end{itemize}
        \item Is easy to understand
        \item Is difficult to use
        \begin{itemize}
            \item One tiny mistake and everything comes to a halt
        \end{itemize}
    \end{itemize}

    \item Types of Semaphores
    \begin{enumerate}[1.]
        \item Counting Semaphore
        \begin{itemize}
            \item \textit{count = N} $\Rightarrow$ Max number of resources
            \item \textit{count} $\uparrow$ when resource added
            \item \textit{count} $\downarrow$ when resource used
            \item \textit{count = 0} $\Rightarrow$ No resources available $\Rightarrow$ \textbf{Wait} until \textit{count \textgreater 0}
        \end{itemize}

        \begin{center}
        \includegraphics[width=0.8\linewidth]{images/week_3_notes_1_3.png}
        \includegraphics[width=0.8\linewidth]{images/week_3_notes_1_4.png}
        \end{center}

        \item Binary Semaphore
        \begin{itemize}
            \item \textit{count = 1} $\Rightarrow$ Unlocked / Available
            \begin{itemize}
                \item A thread can go in
            \end{itemize}
            \item \textit{count = 0} $\Rightarrow$ Locked / Unavailable $\Rightarrow$ \textbf{Wait} until
            \textit{count \textgreater 0}
            \begin{itemize}
                \item Other threads must wait
            \end{itemize}
            \item It's like the security at airport, or the portable bathroom from
            week 1 notes
        \end{itemize}

        \begin{center}
        \includegraphics[width=0.8\linewidth]{images/week_3_notes_1_5.png}
        \end{center}
    \end{enumerate}
    \item Using Binary Semaphores
    \item Atomicity of wait() and signal()
    \item Read-write problem
    \item Reader's operation
    \item Writer's operation
    \item Reader's and Writer's Operation
    \item Notes on Readers/Writers
    \item Monitors
    \begin{itemize}
        \item Is solution to semaphore
        \begin{itemize}
            \item Is easier to implement
            \item Creates less bugs
        \end{itemize}
        \item Still not a good solution
        \begin{itemize}
            \item Usable only in few programming languages
            \begin{itemize}
                \item C not supported
                \item Java supported
            \end{itemize}
            \item Is designed for single or multiple CPUS with access to a common
            memory
            \item Not designed for the age of internet
            \begin{itemize}
                \item Fails with distributed system with each having private memory
                connected via internet
                \item Can't exchange information between machines
            \end{itemize}
        \end{itemize}
    \end{itemize}
\end{itemize}

\bigskip

\section{Intro to Scheduling}

\bigskip

\begin{itemize}
    \item State Queues
    \item Proccessor Scheduling
    \item What Happens on Dispatch / Context Switch
    \item Process Life Cycle
    \item What is Proceess Scheduling
    \begin{itemize}
        \item Is a process at which allows one process to use the CPU while
        another is on hold, to make full use of CPU $^{[1]}$
        \item This is key to multi-programming
    \end{itemize}

    \bigskip

    \underline{\textbf{References}}

    \bigskip

    \begin{enumerate}[1)]
        \item Study Tonight: What is CPU Scheduling?, \href{https://www.studytonight.com/operating-system/cpu-scheduling}{link}
        \item University of Illinois: CPU Scheduling, \href{https://www.cs.uic.edu/~jbell/CourseNotes/OperatingSystems/6_CPU_Scheduling.html}{link}
    \end{enumerate}

    \item When to Schedule
    \item Types of Scheduling
    \begin{itemize}
        \item Non-preemtive Scheduling
        \begin{itemize}
            \item Once the the CPU has been allocated to a process, the CPU
            keeps the process until it releases either by terminating or by switching
            to the waiting state $^{[1]}$

            \begin{center}
            \includegraphics[width=0.8\linewidth]{images/week_3_notes_2_1.png}
            \end{center}

            \item e.g Windows 3.1
            \item Suitable for batch scheduling
        \end{itemize}
        \item Preemptive Scheduling
        \begin{itemize}
            \item Usually assigns tasks with prioirities
            \item Can interrupt for higher priority task
            \item Resumes existing task once priority task completes execution
            \item Needed in interactive or real time systems
            \item Feels like juggling
        \end{itemize}
    \end{itemize}

    \bigskip

    \underline{\textbf{References}}

    \bigskip

    \begin{enumerate}[1)]
        \item Study Tonight: What is CPU Scheduling?, \href{https://www.studytonight.com/operating-system/cpu-scheduling}{link}
    \end{enumerate}
\end{itemize}

\end{document}