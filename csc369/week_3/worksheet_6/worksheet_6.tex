\documentclass[12pt]{article}
\usepackage[margin=2.5cm]{geometry}
\usepackage{enumerate}
\usepackage{amsfonts}
\usepackage{amsmath}
\usepackage{fancyhdr}
\usepackage{amsmath}
\usepackage{amssymb}
\usepackage{amsthm}
\usepackage{mdframed}
\usepackage{graphicx}
\usepackage{subcaption}
\usepackage{adjustbox}
\usepackage{listings}
\usepackage{xcolor}
\usepackage{booktabs}
\usepackage[utf]{kotex}
\usepackage{hyperref}

\definecolor{codegreen}{rgb}{0,0.6,0}
\definecolor{codegray}{rgb}{0.5,0.5,0.5}
\definecolor{codepurple}{rgb}{0.58,0,0.82}
\definecolor{backcolour}{rgb}{0.95,0.95,0.92}

\lstdefinestyle{mystyle}{
    backgroundcolor=\color{backcolour},
    commentstyle=\color{codegreen},
    keywordstyle=\color{magenta},
    numberstyle=\tiny\color{codegray},
    stringstyle=\color{codepurple},
    basicstyle=\ttfamily\footnotesize,
    breakatwhitespace=false,
    breaklines=true,
    captionpos=b,
    keepspaces=true,
    numbers=left,
    numbersep=5pt,
    showspaces=false,
    showstringspaces=false,
    showtabs=false,
    tabsize=1
}

\lstset{style=mystyle}

\pagestyle{fancy}
\renewcommand{\headrulewidth}{0.4pt}
\lhead{CSC 369}
\rhead{Worksheet 6}

\begin{document}
\title{CSC 369 Worksheet 6}
\maketitle

\bigskip

Source: \href{http://pages.cs.wisc.edu/~remzi/OSTEP/cpu-intro.pdf}{link}

\bigskip

\section{Homework (Code)}

\begin{enumerate}[1.]
    \item The first Linux tool you should check out is the very simple tool
    \texttt{free}. First, type \texttt{man free} and read its entire manual page; it’s
    short, don’t worry!

    \item Now, run \texttt{free}, perhaps using some of the arguments that might
    be useful (e.g., \texttt{-m}, to display memory totals in megabytes). How
    much memory is in your system? How much is \texttt{free}? Do these
    numbers match your intuition?

    \item Next, create a little program that uses a certain amount of memory,
    called \texttt{memory-user.c}. This program should take one commandline argument: the number of megabytes of memory it will use.
    When run, it should allocate an array, and constantly stream through
    the array, touching each entry. The program should do this indefinitely, or, perhaps, for a certain amount of time also specified at the
    command line.

    \item Now, while running your \texttt{memory-user} program, also (in a different terminal window, but on the same machine) run the free
    tool. How do the memory usage totals change when your program
    is running? How about when you kill the \texttt{memory-user} program?
    Do the numbers match your expectations? Try this for different
    amounts of memory usage. What happens when you use really
    large amounts of memory?

    \item Let’s try one more tool, known as \texttt{pmap}. Spend some time, and read
    the \texttt{pmap} manual page in detail.

    \item To use \texttt{pmap}, you have to know the process ID of the process you’re
    interested in. Thus, first run ps auxw to see a list of all processes;
    then, pick an interesting one, such as a browser. You can also use
    your \texttt{memory-user} program in this case (indeed, you can even
    have that program call \texttt{getpid()} and print out its PID for your
    convenience).

    \item Now run \texttt{pmap} on some of these processes, using various flags (like
    -X) to reveal many details about the process. What do you see?
    How many different entities make up a modern address space, as
    opposed to our simple conception of code/stack/heap?

    \item Finally, let’s run \texttt{pmap} on your \texttt{memory-user} program, with different amounts of used memory. What do you see here? Does the
    output from \texttt{pmap} match your expectations?
\end{enumerate}

\end{document}