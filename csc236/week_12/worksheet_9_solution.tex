\documentclass[12pt]{article}
\usepackage[margin=2.5cm]{geometry}
\usepackage{enumerate}
\usepackage{amsfonts}
\usepackage{amsmath}
\usepackage{fancyhdr}
\usepackage{amsmath}
\usepackage{amssymb}
\usepackage{amsthm}
\usepackage{mdframed}
\usepackage{graphicx}
\usepackage{subcaption}
\usepackage{adjustbox}
\usepackage{listings}
\usepackage{xcolor}
\usepackage{booktabs}
\usepackage[utf]{kotex}

\definecolor{codegreen}{rgb}{0,0.6,0}
\definecolor{codegray}{rgb}{0.5,0.5,0.5}
\definecolor{codepurple}{rgb}{0.58,0,0.82}
\definecolor{backcolour}{rgb}{0.95,0.95,0.92}

\lstdefinestyle{mystyle}{
    backgroundcolor=\color{backcolour},
    commentstyle=\color{codegreen},
    keywordstyle=\color{magenta},
    numberstyle=\tiny\color{codegray},
    stringstyle=\color{codepurple},
    basicstyle=\ttfamily\footnotesize,
    breakatwhitespace=false,
    breaklines=true,
    captionpos=b,
    keepspaces=true,
    numbers=left,
    numbersep=5pt,
    showspaces=false,
    showstringspaces=false,
    showtabs=false,
    tabsize=1
}

\lstset{style=mystyle}

\begin{document}
\title{CSC236 Worksheet 9 Solution}
\author{Hyungmo Gu}
\maketitle

\section*{Question 1}

\bigskip

\begin{enumerate}[a.]
    \item

    \setcounter{equation}{0}

    I need to evalulate the reg. expressions for

    \begin{align*}
        L = \{x \in \Sigma \mid x \text{has even number of 1s or an odd number of 0s}\}
    \end{align*}

    \bigskip

    I will do so in parts.

    \bigskip

    \underline{\textbf{Part 1 (Finding reg. expressions for even number of 1s):}}

    \bigskip

    In this part, I will find the reg. expressions for even number of 1's.

    \bigskip

    I will do so by finding patterns in series of small examples.

    \bigskip

    Starting with $L = \{x \in \Sigma \mid x\:\text{has 0 number of 1s}\}$,
    it's reg. expressions is

    \bigskip

    \begin{align}
        0^*
    \end{align}

    \bigskip

    Now for $L = \{x \in \Sigma \mid x\:\text{has 2 number of 1s}\}$,
    it's reg. expressions is

    \bigskip

    \begin{align}
        0^*10^*10^*
    \end{align}


    \bigskip

    Now for $L = \{x \in \Sigma \mid x\:\text{has 4 number of 1s}\}$,
    it's reg. expressions is

    \begin{align}
        0^*10^*10^*10^*10^*
    \end{align}

    \bigskip

    From above, I see a pattern that

    \begin{align}
        (0^*10^*1)(0^*10^*1)0^*
    \end{align}

    \bigskip

    Using the pattern, I can conclude that the regular expression for
    even number of 1s is

    \begin{align}
        (0^*10^*1)^*0^*
    \end{align}


    \underline{\textbf{Part 2 (Finding reg. expressions for odd number of 0s):}}

    \bigskip

    In this part, I will find the reg. expressions for odd number of 0's.

    \bigskip

    I will do so by finding patterns in series of small examples.

    \bigskip

    Starting with $L = \{x \in \Sigma \mid x\:\text{has 1 number of 0s}\}$,
    it's reg. expressions is

    \begin{align}
        1^*01^*
    \end{align}

    \bigskip

    Now for  $L = \{x \in \Sigma \mid x\:\text{has 3 number of 0s}\}$,
    it's reg. expressions is

    \bigskip

    \begin{align}
        1^*01^*01^*01^*
    \end{align}

    \bigskip

    Now for $L = \{x \in \Sigma \mid x\:\text{has 5 number of 0s}\}$,
    it's reg. expressions is

    \bigskip

    \begin{align}
        1^*01^*01^*01^*01^*01^*
    \end{align}

    \bigskip

    From above, I see a pattern that

    \begin{align}
        1^*(01^*)(01^*)(01^*)(01^*)(01^*)
    \end{align}

    \bigskip

    Using the pattern, I can conclude that the regular expression for
    odd number of 0s is

    \begin{align}
        1^*(01^*)^*
    \end{align}

    \bigskip

    Thus, by combining the two parts with union, we have

    \begin{align}
    (0^*10^*1)^*0^* + 1^*(01^*)^*
    \end{align}

    % \underline{\textbf{Rough Works:}}

    % \bigskip

    % \begin{enumerate}[1.]
    %     \item Find regular expression for even number of 1's

    %     \bigskip

    %     In this part, I will find the reg. expressions for even number of 1's.

    %     \bigskip

    %     I will do so by finding patterns in series of small examples.

    %     \bigskip

    %     \begin{itemize}
    %         \item Find reg. expressions for $L = \{x \in \Sigma \mid x\:\text{has 0 number of 1s}\}$

    %         \bigskip

    %         Starting with $L = \{x \in \Sigma \mid x\:\text{has 0 number of 1s}\}$,
    %         it's reg. expressions is

    %         \begin{mdframed}
    %         \begin{align}
    %             0^*
    %         \end{align}
    %         \end{mdframed}

    %         \item Find reg. expressions for $L = \{x \in \Sigma \mid x\:\text{has 2 number of 1s}\}$

    %         \bigskip

    %         Now for $L = \{x \in \Sigma \mid x\:\text{has 2 number of 1s}\}$,
    %         it's reg. expressions is

    %         \begin{mdframed}
    %             \begin{align}
    %                 0^*10^*10^*
    %             \end{align}
    %         \end{mdframed}

    %         \item Find reg. expressions for $L = \{x \in \Sigma \mid x\:\text{has 4 number of 1s}\}$

    %         \bigskip

    %         Now for $L = \{x \in \Sigma \mid x\:\text{has 4 number of 1s}\}$,
    %         it's reg. expressions is

    %         \begin{mdframed}
    %             \begin{align}
    %                 0^*10^*10^*10^*10^*
    %             \end{align}
    %         \end{mdframed}

    %         \item Hey I see a pattern!!
    %         \begin{mdframed}
    %             From above, I see a pattern that

    %             \begin{align}
    %                 (0^*10^*1)(0^*10^*1)0^*
    %             \end{align}
    %         \end{mdframed}
    %         \item Conclude :)

    %         \begin{mdframed}
    %             Using the pattern, I can conclude that the regular expression for
    %             even number of 1s is

    %             \begin{align}
    %                 (0^*10^*1)^*0^*
    %             \end{align}
    %         \end{mdframed}
    %     \end{itemize}
    %     \item Find regular expression for odd number of 0's

    %     \bigskip

    %     In this part, I will find the reg. expressions for odd number of 0's.

    %     \bigskip

    %     I will do so by finding patterns in series of small examples.

    %     \bigskip

    %     \begin{itemize}
    %         \item Find reg. expressions for $L = \{x \in \Sigma \mid x\:\text{has 1 number of 0s}\}$

    %         \bigskip

    %         Starting with $L = \{x \in \Sigma \mid x\:\text{has 1 number of 0s}\}$,
    %         it's reg. expressions is

    %         \begin{mdframed}
    %             \begin{align}
    %                 1^*01^*
    %             \end{align}
    %         \end{mdframed}

    %         \item Find reg. expressions for $L = \{x \in \Sigma \mid x\:\text{has 3 number of 0s}\}$

    %         \bigskip

    %         Now for  $L = \{x \in \Sigma \mid x\:\text{has 3 number of 0s}\}$,
    %         it's reg. expressions is

    %         \bigskip

    %         \begin{mdframed}
    %             \begin{align}
    %                 1^*01^*01^*01^*
    %             \end{align}
    %         \end{mdframed}

    %         \item Find reg. expressions for $L = \{x \in \Sigma \mid x\:\text{has 5 number of 0s}\}$

    %         \bigskip

    %         Now for $L = \{x \in \Sigma \mid x\:\text{has 5 number of 0s}\}$,
    %         it's reg. expressions is

    %         \bigskip

    %         \begin{mdframed}
    %             \begin{align}
    %                 1^*01^*01^*01^*01^*01^*
    %             \end{align}
    %         \end{mdframed}

    %         \item Hey I see a pattern!!
    %         \begin{mdframed}
    %             From above, I see a pattern that

    %             \begin{align}
    %                 1^*(01^*)^*
    %             \end{align}
    %         \end{mdframed}
    %         \item Conclude :)

    %         \begin{mdframed}
    %             Using the pattern, I can conclude that the regular expression for
    %             odd number of 0s is

    %             \begin{align}
    %                 1^*(01^*)^*
    %             \end{align}
    %         \end{mdframed}
    %     \end{itemize}
    %     \item Combine 1 and 2 using +

    %     \begin{mdframed}
    %         So, by combining the two with union, we have

    %         \begin{align}
    %         (0^*10^*1)^*0^* + 1^*(01^*)^*
    %         \end{align}

    %     \end{mdframed}
    % \end{enumerate}

    \bigskip

    \underline{\textbf{Notes:}}

    \bigskip

    \begin{itemize}
        \item Regular Expression
        \begin{itemize}
            \item Quick Guide

            \begin{align}
                (0+1)((01)^*0)
            \end{align}

            \bigskip

            The expression implies that

            \bigskip

            \begin{enumerate}[1.]
                \item Starts with 0 \textbf{or} 1
                \begin{itemize}
                    \item indicated by (0 + 1)
                \end{itemize}
                \item Are then followed by \textbf{one or more repeatitions} of 01
                \begin{itemize}
                    \item indicated by $(01)^*$
                \end{itemize}
                \item Ends with 0
                \begin{itemize}
                    \item indicated by the final 0
                \end{itemize}
            \end{enumerate}
        \end{itemize}

        \item Examples
        \begin{enumerate}[1.]
            \item $L = \{w \in \{a,b\}^* \mid w\:\text{has an $a$} \}$

            \bigskip

            \begin{mdframed}
                \underline{\textbf{Answer:}}

                \begin{align}
                (a+b)^*a(a+b)^*
                \end{align}

                \bigskip

                \begin{itemize}
                    \item Means there is one or more repeatitions of $a$ or $b$ at front
                    \item Means there is $a$ in the middle
                    \item Means there is zero or more repeatitions of $a$ or $b$ at end
                \end{itemize}
            \end{mdframed}

            \item $L = \{w \in \{a,b\}^* \mid w\:\text{has at lest two $a$s}\}$

            \bigskip

            \begin{mdframed}
                \underline{\textbf{Answer:}}

                \begin{align}
                (a+b)^*a(a+b)^*a(a+b)^*
                \end{align}
            \end{mdframed}

            \item $L = \{w \in \{a,b\}^* \mid \vert w \vert \geq 2\}$

            \bigskip

            \begin{mdframed}
                \underline{\textbf{Answer:}}

                \begin{align}
                (0+1)(0+1)(0+1)^*
                \end{align}

                \bigskip

                In this example,

                \bigskip

                \begin{itemize}
                    \item Two characters are created (indicated by $(0+1)(0+1)$)
                    \item And more :D!! (indicated by$(0+1)^*$)
                \end{itemize}
            \end{mdframed}
        \end{enumerate}
    \end{itemize}

    \item

    \setcounter{equation}{0}

    I need to find the reg. expressions for $L = \{x \in \Sigma \mid x\:\text{ has
    at least one 1 and at least one 0}\}$. That is, regex expressions for $\{x \in
    \Sigma \mid x\:\text{ has at least one 1 followed by at least one 0}\}$ plus
    $\{x \in \Sigma \mid x\:\text{ has at least one 0 followed by at least one 1}\}$.

    \bigskip

    First, I need to find reg. expressions for

    \begin{align}
    \{x \in \Sigma \mid x\:\text{ has at least one 1 followed by at least one 0}\}
    \end{align}

    \bigskip

    Since the reg expressions for $x$ with at least one 1 is $(0 + 1)^*1(0 + 1)^*$
    and the reg expressions for $x$ with at least one 0 is $(0 + 1)^*0(0 + 1)^*$,
    we have

    \begin{align}
        (0+1)^*1(0+1)^*0(0+1)^*
    \end{align}

    \bigskip

    Second, I need to find reg. expressions for

    \begin{align}
    \{x \in \Sigma \mid x\:\text{ has at least one 0 followed by at least one 1}\}
    \end{align}

    \bigskip

    Using the facts provided above, we have

    \begin{align}
        (0+1)^*0(0+1)^*1(0+1)^*
    \end{align}

    \bigskip

    Thus, by combining the two, we can conclude

    \begin{align}
        (0+1)^*1(0+1)^*0(0+1)^* + (0+1)^*0(0+1)^*1(0+1)^*
    \end{align}

    % \underline{\textbf{Rough Works}}

    % \bigskip

    % I need to find the reg. expressions for $L = \{x \in \Sigma \mid x\:\text{ has at least one 1 and at least one 0}\}$.
    % That is, regex expressions for $\{x \in \Sigma \mid x\:\text{ has at least one 1 followed by at least one 0}\}$ plus
    % $\{x \in \Sigma \mid x\:\text{ has at least one 0 followed by at least one 1}\}$

    % \begin{enumerate}[1.]
    %     \item Find reg. expressions for $L = \{x \in \Sigma \mid x\:\text{ has at least one 1 followed by at least one 0}\}$

    %     \bigskip

    %     First, I need to find reg. expressions for

    %     \begin{align}
    %     \{x \in \Sigma \mid x\:\text{ has at least one 1 followed by at least one 0}\}
    %     \end{align}

    %     \bigskip

    %     \begin{mdframed}
    %     Since the reg expressions for $x$ with at least one 1 is $(0 + 1)^*1(0 + 1)^*$
    %     and the reg expressions for $x$ with at least one 0 is $(0 + 1)^*0(0 + 1)^*$,
    %     we have

    %     \begin{align}
    %         (0+1)^*1(0+1)^*0(0+1)^*
    %     \end{align}
    %     \end{mdframed}

    %     \item Find reg. expressions for $L = \{x \in \Sigma \mid x\:\text{ has at least one 0 followed by at least one 1}\}$

    %     \bigskip

    %     Second, I need to find reg. expressions for

    %     \begin{align}
    %     \{x \in \Sigma \mid x\:\text{ has at least one 0 followed by at least one 1}\}
    %     \end{align}

    %     \bigskip

    %     \begin{mdframed}
    %     Using the facts provided above, we have

    %     \begin{align}
    %         (0+1)^*0(0+1)^*1(0+1)^*
    %     \end{align}
    %     \end{mdframed}

    %     \item Conclude

    %     \begin{mdframed}
    %     Thus, by combining the two, we can conclude

    %     \begin{align}
    %         (0+1)^*1(0+1)^*0(0+1)^* + (0+1)^*0(0+1)^*1(0+1)^*
    %     \end{align}
    %     \end{mdframed}

    % \end{enumerate}

    \item

    \setcounter{equation}{0}

    I need to find reg. expressions for

    \begin{align*}
    \{x \in \Sigma \mid \text{every 1 in $x$ is immediately preceded and followed by a 0}\}
    \end{align*}

    \bigskip

    An example expresion of the above is

    \bigskip

    \begin{align}
        0^*0100^*0100^*0100^*
    \end{align}

    \bigskip

    From above, we can see the following pattern

    \bigskip

    \begin{align}
        (0^*010)(0^*010)(0^*010)0^*
    \end{align}

    \bigskip

    Thus, we have

    \begin{align}
    (0^*010)^*0^*
    \end{align}
\end{enumerate}

\end{document}