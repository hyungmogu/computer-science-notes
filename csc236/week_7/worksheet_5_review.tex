\documentclass[12pt]{article}
\usepackage[margin=2.5cm]{geometry}
\usepackage{enumerate}
\usepackage{amsfonts}
\usepackage{amsmath}
\usepackage{fancyhdr}
\usepackage{amsmath}
\usepackage{amssymb}
\usepackage{amsthm}
\usepackage{mdframed}
\usepackage{graphicx}
\usepackage{subcaption}
\usepackage{adjustbox}
\usepackage{listings}
\usepackage{xcolor}
\usepackage{booktabs}
\usepackage[utf]{kotex}

\definecolor{codegreen}{rgb}{0,0.6,0}
\definecolor{codegray}{rgb}{0.5,0.5,0.5}
\definecolor{codepurple}{rgb}{0.58,0,0.82}
\definecolor{backcolour}{rgb}{0.95,0.95,0.92}

\lstdefinestyle{mystyle}{
    backgroundcolor=\color{backcolour},
    commentstyle=\color{codegreen},
    keywordstyle=\color{magenta},
    numberstyle=\tiny\color{codegray},
    stringstyle=\color{codepurple},
    basicstyle=\ttfamily\footnotesize,
    breakatwhitespace=false,
    breaklines=true,
    captionpos=b,
    keepspaces=true,
    numbers=left,
    numbersep=5pt,
    showspaces=false,
    showstringspaces=false,
    showtabs=false,
    tabsize=1
}

\lstset{style=mystyle}

\begin{document}
\title{CSC236 Worksheet 5 Review}
\author{Hyungmo Gu}
\maketitle

\section*{Question 1}
\begin{enumerate}[a.]
    \item

    \begin{mdframed}
        \underline{\textbf{Rough Work:}}

        \bigskip

        Define $P(k): R(3^k) = k3^k$. Note that when $n = 3^k$, this is equivalent
        to $R(n) = n\log_3 n$. I will use simple induction to prove $P(k)$.

        \bigskip

        \begin{enumerate}[1.]
            \item Base Case ($k = 0$)

            \begin{mdframed}
            Let $k = 0$.

            \bigskip

            Then,

            \begin{align}
                R(3^k) &= 0 & [\text{By def., since $n = 3^0 = 1$}]\\
                &= 0 \cdot 3^0\\
                &= k \cdot 3^k
            \end{align}

            \bigskip

            Thus, $P(k)$ is verified in this step.

            \end{mdframed}

            \item Inductive Step
        \end{enumerate}
    \end{mdframed}
\end{enumerate}

\end{document}