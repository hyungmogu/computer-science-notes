\documentclass[12pt]{article}
\usepackage[margin=2.5cm]{geometry}
\usepackage{enumerate}
\usepackage{amsfonts}
\usepackage{amsmath}
\usepackage{fancyhdr}
\usepackage{amsmath}
\usepackage{amssymb}
\usepackage{amsthm}
\usepackage{mdframed}
\usepackage{graphicx}
\usepackage{subcaption}
\usepackage{adjustbox}
\usepackage{listings}
\usepackage{xcolor}
\usepackage{booktabs}
\usepackage[utf]{kotex}

\definecolor{codegreen}{rgb}{0,0.6,0}
\definecolor{codegray}{rgb}{0.5,0.5,0.5}
\definecolor{codepurple}{rgb}{0.58,0,0.82}
\definecolor{backcolour}{rgb}{0.95,0.95,0.92}

\lstdefinestyle{mystyle}{
    backgroundcolor=\color{backcolour},
    commentstyle=\color{codegreen},
    keywordstyle=\color{magenta},
    numberstyle=\tiny\color{codegray},
    stringstyle=\color{codepurple},
    basicstyle=\ttfamily\footnotesize,
    breakatwhitespace=false,
    breaklines=true,
    captionpos=b,
    keepspaces=true,
    numbers=left,
    numbersep=5pt,
    showspaces=false,
    showstringspaces=false,
    showtabs=false,
    tabsize=1
}

\lstset{style=mystyle}

\begin{document}
\title{CSC236 Worksheet 5 Review}
\author{Hyungmo Gu}
\maketitle

\section*{Question 1}
\begin{enumerate}[a.]
    \item

    \begin{proof}
        Define $P(k): R(3^k) = k3^k$. Note that when $n = 3^k$, this is equivalent
        to $R(n) = n\log_3 n$. I will use simple induction to prove $P(k)$.

        \bigskip

        \underline{\textbf{Base Case ($k = 0$):}}

        \bigskip

        Let $k = 0$.

        \bigskip

        Then,

        \begin{align}
            R(3^k) &= 0 & [\text{By def., since $n = 3^0 = 1$}]\\
            &= 0 \cdot 3^0\\
            &= k \cdot 3^k
        \end{align}

        \bigskip

        Thus, $P(k)$ is verified in this step.

        \bigskip

        \underline{\textbf{Inductive Step:}}

        \bigskip

        Let $k \in \mathbb{N}$. Assume $P(k)$. That is, $R(3^k) = k \cdot 3^k$.
        I need to prove $P(k+1)$ follows. That is, $R(3^{k+1}) = (k+1) \cdot 3^{k+1}$

        \bigskip

        Starting from $R(3^{k+1})$, we have

        \begin{align}
            R^(3^{k+1}) &= 3^{k+1} + 3R(\lceil 3^{k+1}/3 \rceil) & [\text{By def., since $0<k+1$, and $1<3^{k+1}$}]\\
            &= 3^{k+1} + 3R(\lceil 3^k \rceil)\\
            &= 3^{k+1} + 3R(3^k) & [\text{Since $\lceil 3^k \rceil = 3^k$}]\\
            &= 3^{k+1} + 3 (k \cdot 3^k) & [\text{By I.H}]\\
            &= 3^{k+1} + (k \cdot 3^{k+1})\\
            &= (k + 1) \cdot 3^{k+1}
        \end{align}
    \end{proof}

    % \bigskip

    % \begin{mdframed}
    %     \underline{\textbf{Rough Work:}}

    %     \bigskip

    %     Define $P(k): R(3^k) = k3^k$. Note that when $n = 3^k$, this is equivalent
    %     to $R(n) = n\log_3 n$. I will use simple induction to prove $P(k)$.

    %     \bigskip

    %     \begin{enumerate}[1.]
    %         \item Base Case ($k = 0$)

    %         \begin{mdframed}
    %         \underline{\textbf{Base Case ($k = 0$):}}

    %         \bigskip

    %         Let $k = 0$.

    %         \bigskip

    %         Then,

    %         \begin{align}
    %             R(3^k) &= 0 & [\text{By def., since $n = 3^0 = 1$}]\\
    %             &= 0 \cdot 3^0\\
    %             &= k \cdot 3^k
    %         \end{align}

    %         \bigskip

    %         Thus, $P(k)$ is verified in this step.

    %         \end{mdframed}

    %         \item Inductive Step

    %         \begin{mdframed}
    %         \underline{\textbf{Inductive Step:}}

    %         \bigskip

    %         Let $k \in \mathbb{N}$. Assume $P(k)$. That is, $R(3^k) = k \cdot 3^k$.
    %         I need to prove $P(k+1)$ follows. That is, $R(3^{k+1}) = (k+1) \cdot 3^{k+1}$

    %         \bigskip

    %         Starting from $R(3^{k+1})$, we have

    %         \begin{align}
    %             R^(3^{k+1}) &= 3^{k+1} + 3R(\lceil 3^{k+1}/3 \rceil) & [\text{By def., since $0<k+1$, and $1<3^{k+1}$}]\\
    %             &= 3^{k+1} + 3R(\lceil 3^k \rceil)\\
    %             &= 3^{k+1} + 3R(3^k) & [\text{Since $\lceil 3^k \rceil = 3^k$}]\\
    %             &= 3^{k+1} + 3 (k \cdot 3^k) & [\text{By I.H}]\\
    %             &= 3^{k+1} + (k \cdot 3^{k+1})\\
    %             &= (k + 1) \cdot 3^{k+1}
    %         \end{align}

    %         \end{mdframed}
    %     \end{enumerate}
    % \end{mdframed}

    \item

    \bigskip

    For convenience, define $P(n): \bigwedge\limits_{i=1}^{i=n} R(i) \leq R(n)$.

    \bigskip

    I will use complete induction to prove that $\forall n \in \mathbb{N},
    0 < n \Rightarrow P(n)$.

    \bigskip

    \underline{\textbf{Inductive Step:}}

    \bigskip

    Let $n \in \mathbb{N} \setminus \{0\}$. Assume $\bigwedge\limits_{i=1}^{i=n-1} P(i)$.
    I will show that $P(n)$ follows.

    \bigskip

    \underline{\textbf{Base Case ($n = 1$):}}

    \bigskip

    Let $n = 1$.

    \bigskip

    Then, $\bigwedge\limits_{i=1}^{i=n} R(i) = R(n)$.

    \bigskip

    Thus, $P(n)$ follows in this step.

    \bigskip

    \underline{\textbf{Base Case ($n = 2$):}}

    \bigskip

    Let $n = 2$.

    \bigskip

    In this step, I need to prove $P(n)$ follows. That is, $R(1) \leq R(2)$ and
    $R(2) \leq R(2)$.

    \bigskip

    I will do so in parts.

    \bigskip

    \textbf{Part 1 (Proving $R(1) \leq R(2)$):}

    \bigskip

    The definition tells us $R(1) = 1$ and $R(2) = 2 +
    3R(\lceil 2/3 \rceil) = 2 + 3R(1) = 2$.

    \bigskip

    Since $R(1) = 1 < R(2) = 2$, we can conclude $R(1) \leq R(2)$ holds.

    \bigskip

    \textbf{Part 2 (Proving $R(2) \leq R(2)$):}

    \bigskip

    Since $R(2) = R(2)$, $R(2) \leq R(2)$ holds.

    \bigskip

    \underline{\textbf{Case ($n > 2$):}}

    \bigskip

    Since $n > 2$, $1 \leq n-1 < n$. So, by induction hypothesis, $P(n-1)$
    holds. Then, by transitivity of $\leq$, it is suffice to prove $P(n)$
    by showing $R(n-1) \leq R(n)$.

    \bigskip

    Starting with $R(n-1)$, we have

    \begin{align}
    R(n-1) &= n-1 + 3R(\lceil (n-1)/3 \rceil) & [\text{By def., since $n > 2$ and $n - 1 > 1$}]\\
    &\leq n + 3R(\lceil (n-1)/3 \rceil)\\
    &\leq n + 3R(\lceil n/3 \rceil) & [\text{By I.H, since $1 \leq \lceil (n-1)/3 \rceil < \lceil n/3 \rceil < n$}]\\
    &= R(n)
    \end{align}

    \bigskip

    Thus, $P(n)$ follows from $\bigwedge\limits_{i=1}^{i=n-1} P(i)$ in this step.


    % \bigskip

    % \begin{mdframed}
    %     \underline{\textbf{Rough Work:}}

    %     \bigskip

    %     For convenience, define $P(n): \bigwedge\limits_{i=1}^{i=n} R(i) \leq R(n)$.

    %     \bigskip

    %     I will use complete induction to prove that $\forall n \in \mathbb{N},
    %     0 < n \Rightarrow P(n)$.

    %     \bigskip

    %     \begin{enumerate}[1.]
    %         \item Inductive Step

    %         \begin{mdframed}
    %         \underline{\textbf{Inductive Step:}}

    %         \bigskip

    %         Let $n \in \mathbb{N} \setminus \{0\}$. Assume $\bigwedge\limits_{i=1}^{i=n-1} P(i)$.
    %         I will show that $P(n)$ follows.
    %         \end{mdframed}

    %         \item Base Case ($n = 1$)

    %         \begin{mdframed}
    %         \underline{\textbf{Base Case ($n = 1$):}}

    %         \bigskip

    %         Let $n = 1$.

    %         \bigskip

    %         Then, $\bigwedge\limits_{i=1}^{i=n} R(i) = R(n)$.

    %         \bigskip

    %         Thus, $P(n)$ follows in this step.

    %         \end{mdframed}

    %         \item Base Case ($n = 2$)

    %         \begin{mdframed}
    %         \underline{\textbf{Base Case ($n = 2$):}}

    %         \bigskip

    %         Let $n = 2$.

    %         \bigskip

    %         In this step, I need to prove $P(n)$ follows. That is, $R(1) \leq R(2)$ and
    %         $R(2) \leq R(2)$.

    %         \bigskip

    %         I will do so in parts.

    %         \bigskip

    %         \textbf{Part 1 (Proving $R(1) \leq R(2)$):}

    %         \bigskip

    %         The definition tells us $R(1) = 1$ and $R(2) = 2 +
    %         3R(\lceil 2/3 \rceil) = 2 + 3R(1) = 2$.

    %         \bigskip

    %         Since $R(1) = 1 < R(2) = 2$, we can conclude $R(1) \leq R(2)$ holds.

    %         \bigskip

    %         \textbf{Part 2 (Proving $R(2) \leq R(2)$):}

    %         \bigskip

    %         Since $R(2) = R(2)$, $R(2) \leq R(2)$ holds.

    %         \end{mdframed}

    %         \item Case ($n > 2$)
    %         \begin{mdframed}
    %         \underline{\textbf{Case ($n > 2$):}}

    %         \bigskip

    %         Since $n > 2$, $1 \leq n-1 < n$. So, by induction hypothesis, $P(n-1)$
    %         holds. Then, by transitivity of $\leq$, it is suffice to prove $P(n)$
    %         by showing $R(n-1) \leq R(n)$.

    %         \bigskip

    %         Starting with $R(n-1)$, we have

    %         \begin{align}
    %         R(n-1) &= n-1 + 3R(\lceil (n-1)/3 \rceil) & [\text{By def., since $n > 2$ and $n - 1 > 1$}]\\
    %         &\leq n + 3R(\lceil (n-1)/3 \rceil)\\
    %         &\leq n + 3R(\lceil n/3 \rceil) & [\text{By I.H, since $1 \leq \lceil (n-1)/3 \rceil < \lceil n/3 \rceil < n$}]\\
    %         &= R(n)
    %         \end{align}

    %         \bigskip

    %         Thus, $P(n)$ follows from $\bigwedge\limits_{i=1}^{i=n-1} P(i)$ in this step.

    %         \end{mdframed}
    %     \end{enumerate}
    % \end{mdframed}
\end{enumerate}

\end{document}