\documentclass[12pt]{article}
\usepackage[margin=2.5cm]{geometry}
\usepackage{enumerate}
\usepackage{amsfonts}
\usepackage{amsmath}
\usepackage{fancyhdr}
\usepackage{amsmath}
\usepackage{amssymb}
\usepackage{amsthm}
\usepackage{mdframed}
\usepackage{graphicx}
\usepackage{subcaption}
\usepackage{adjustbox}
\usepackage{listings}
\usepackage{xcolor}
\usepackage{booktabs}
\usepackage[utf]{kotex}

\definecolor{codegreen}{rgb}{0,0.6,0}
\definecolor{codegray}{rgb}{0.5,0.5,0.5}
\definecolor{codepurple}{rgb}{0.58,0,0.82}
\definecolor{backcolour}{rgb}{0.95,0.95,0.92}

\lstdefinestyle{mystyle}{
    backgroundcolor=\color{backcolour},
    commentstyle=\color{codegreen},
    keywordstyle=\color{magenta},
    numberstyle=\tiny\color{codegray},
    stringstyle=\color{codepurple},
    basicstyle=\ttfamily\footnotesize,
    breakatwhitespace=false,
    breaklines=true,
    captionpos=b,
    keepspaces=true,
    numbers=left,
    numbersep=5pt,
    showspaces=false,
    showstringspaces=false,
    showtabs=false,
    tabsize=1
}

\lstset{style=mystyle}

\begin{document}
\title{CSC236 Worksheet 5 Solution}
\author{Hyungmo Gu}
\maketitle

\section*{Question 1}
\begin{enumerate}[a.]
    \item

    \bigskip

    \begin{mdframed}
        \underline{\textbf{Rough Works:}}

        \bigskip

        For convenience, define $H(k): R(3^k) = 3^kk$. Notice that when $n = 3^k$,
        $3^kk$ is the same as $n\log_3 n$. I will use simple induction to prove
        $\forall k \in \mathbb{N},\:H(k)$.

        \bigskip

        \begin{enumerate}[1.]
            \item Base Case ($k = 0$)

            \begin{mdframed}
            Let $k = 0$.

            \bigskip

            Then,

            \begin{align}
                3^k \cdot k &= 3^0 \cdot 0\\
                &= 0\\
                &= R(n) & [\text{By def.}]
            \end{align}

            Thus, $H(0)$ is verified.

            \end{mdframed}

            \item Inductive Step

            \begin{mdframed}
            Let $k \in \mathbb{N}$. Assume $H(k)$. That is $R(3^k) = 3^kk$.

            \bigskip

            I will show that $H(k+1)$ follows. That is, $R(3^{k+1}) = (k+1)3^{k+1}$.

            \bigskip

            The definition of $R(3^{k+1})$ tells us,

            \begin{align}
                R(3^{k+1}) &= 3^{k+1} + 3R(\lceil 3^{k+1}/3 \rceil)\\
                &= 3^{k+1} + 3R(3^k)
            \end{align}

            \bigskip

            Then, using this fact, we can conclude

            \begin{align}
                R(3^{k+1}) &= 3^{k+1} + 3 \cdot 3^kk & [\text{By I.H}]\\
                &= 3^{k+1} + 3^{k+1}k\\
                &= 3^{k+1}(k + 1)
            \end{align}

            \end{mdframed}
        \end{enumerate}

    \end{mdframed}

\end{enumerate}


\end{document}