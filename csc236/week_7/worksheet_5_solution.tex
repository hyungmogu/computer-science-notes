\documentclass[12pt]{article}
\usepackage[margin=2.5cm]{geometry}
\usepackage{enumerate}
\usepackage{amsfonts}
\usepackage{amsmath}
\usepackage{fancyhdr}
\usepackage{amsmath}
\usepackage{amssymb}
\usepackage{amsthm}
\usepackage{mdframed}
\usepackage{graphicx}
\usepackage{subcaption}
\usepackage{adjustbox}
\usepackage{listings}
\usepackage{xcolor}
\usepackage{booktabs}
\usepackage[utf]{kotex}

\definecolor{codegreen}{rgb}{0,0.6,0}
\definecolor{codegray}{rgb}{0.5,0.5,0.5}
\definecolor{codepurple}{rgb}{0.58,0,0.82}
\definecolor{backcolour}{rgb}{0.95,0.95,0.92}

\lstdefinestyle{mystyle}{
    backgroundcolor=\color{backcolour},
    commentstyle=\color{codegreen},
    keywordstyle=\color{magenta},
    numberstyle=\tiny\color{codegray},
    stringstyle=\color{codepurple},
    basicstyle=\ttfamily\footnotesize,
    breakatwhitespace=false,
    breaklines=true,
    captionpos=b,
    keepspaces=true,
    numbers=left,
    numbersep=5pt,
    showspaces=false,
    showstringspaces=false,
    showtabs=false,
    tabsize=1
}

\lstset{style=mystyle}

\begin{document}
\title{CSC236 Worksheet 5 Solution}
\author{Hyungmo Gu}
\maketitle

\section*{Question 1}
\begin{enumerate}[a.]
    \item

    \begin{proof}
        For convenience, define $H(k): R(3^k) = 3^kk$. Notice that when $n = 3^k$,
        $3^kk$ is the same as $n\log_3 n$. I will use simple induction to prove
        $\forall k \in \mathbb{N},\:H(k)$.

        \bigskip

        \underline{\textbf{Base Case ($k = 0$):}}

        \bigskip

        Let $k = 0$.

        \bigskip

        Then,

        \begin{align}
            3^k \cdot k &= 3^0 \cdot 0\\
            &= 0\\
            &= R(n) & [\text{By def.}]
        \end{align}

        Thus, $H(0)$ is verified.

        \bigskip

        \underline{\textbf{Inductive Step:}}

        \bigskip

        Let $k \in \mathbb{N}$. Assume $H(k)$. That is $R(3^k) = 3^kk$.

        \bigskip

        I will show that $H(k+1)$ follows. That is, $R(3^{k+1}) = (k+1)3^{k+1}$.

        \bigskip

        The definition of $R(3^{k+1})$ tells us,

        \begin{align}
            R(3^{k+1}) &= 3^{k+1} + 3R(\lceil 3^{k+1}/3 \rceil)\\
            &= 3^{k+1} + 3R(3^k)
        \end{align}

        \bigskip

        Then, using this fact, we can conclude

        \begin{align}
            R(3^{k+1}) &= 3^{k+1} + 3 \cdot 3^kk & [\text{By I.H}]\\
            &= 3^{k+1} + 3^{k+1}k\\
            &= 3^{k+1}(k + 1)
        \end{align}
    \end{proof}

    \bigskip

    \begin{mdframed}
        \underline{\textbf{Correct Solution:}}

        \bigskip

        For convenience, define $H(k): R(3^k) = 3^kk$. Notice that when $n = 3^k$,
        $3^kk$ is the same as $n\log_3 n$. I will use simple induction to prove
        $\forall k \in \mathbb{N},\:H(k)$.

        \bigskip

        \underline{\textbf{Base Case ($k = 0$):}}

        \bigskip

        Let $k = 0$.

        \bigskip

        Then,

        \begin{align}
            3^k \cdot k &= 3^0 \cdot 0\\
            &= 0\\
            &= R(n) & [\text{By def.}]
        \end{align}

        Thus, $H(0)$ is verified.

        \bigskip

        \underline{\textbf{Inductive Step:}}

        \bigskip

        Let $k \in \mathbb{N}$. Assume $H(k)$. That is $R(3^k) = 3^kk$.

        \bigskip

        I will show that $H(k+1)$ follows. That is, $R(3^{k+1}) = (k+1)3^{k+1}$.

        \bigskip

        \color{red}Since $k+1 > 0$, $3^{k+1} > 1$.

        \bigskip

        So\color{black}\:the definition of $R(3^{k+1})$ tells us,

        \begin{align}
            R(3^{k+1}) &= 3^{k+1} + 3R(\lceil 3^{k+1}/3 \rceil)\\
            &= 3^{k+1} + 3R(3^k)
        \end{align}

        \bigskip

        Then, using this fact, we can conclude

        \begin{align}
            R(3^{k+1}) &= 3^{k+1} + 3 \cdot 3^kk & [\text{By I.H}]\\
            &= 3^{k+1} + 3^{k+1}k\\
            &= 3^{k+1}(k + 1)
        \end{align}
    \end{mdframed}

    \bigskip

    \underline{\textbf{Notes:}}

    \bigskip

    \begin{itemize}
        \item Noticed that professor used the phrase `Notice that when $n = 3^k$,
        $3^kk$ is the same as $n\log_3 n$.' to express $n$ interms of $3^k$.
        \item I feel I should review this problem to make sure I understood.
    \end{itemize}

    % \begin{mdframed}
    %     \underline{\textbf{Rough Works:}}

    %     \bigskip

    %     For convenience, define $H(k): R(3^k) = 3^kk$. Notice that when $n = 3^k$,
    %     $3^kk$ is the same as $n\log_3 n$. I will use simple induction to prove $\forall k \in \mathbb{N},\:H(k)$.

    %     \bigskip

    %     \begin{enumerate}[1.]
    %         \item Base Case ($k = 0$)

    %         \begin{mdframed}
    %         \underline{\textbf{Base Case ($k = 0$):}}

    %         \bigskip

    %         Let $k = 0$.

    %         \bigskip

    %         Then,

    %         \begin{align}
    %             3^k \cdot k &= 3^0 \cdot 0\\
    %             &= 0\\
    %             &= R(n) & [\text{By def.}]
    %         \end{align}

    %         Thus, $H(0)$ is verified.

    %         \end{mdframed}

    %         \item Inductive Step

    %         \begin{mdframed}
    %         \underline{\textbf{Inductive Step:}}

    %         \bigskip

    %         Let $k \in \mathbb{N}$. Assume $H(k)$. That is $R(3^k) = 3^kk$.

    %         \bigskip

    %         I will show that $H(k+1)$ follows. That is, $R(3^{k+1}) = (k+1)3^{k+1}$.

    %         \bigskip

    %         The definition of $R(3^{k+1})$ tells us,

    %         \begin{align}
    %             R(3^{k+1}) &= 3^{k+1} + 3R(\lceil 3^{k+1}/3 \rceil)\\
    %             &= 3^{k+1} + 3R(3^k)
    %         \end{align}

    %         \bigskip

    %         Then, using this fact, we can conclude

    %         \begin{align}
    %             R(3^{k+1}) &= 3^{k+1} + 3 \cdot 3^kk & [\text{By I.H}]\\
    %             &= 3^{k+1} + 3^{k+1}k\\
    %             &= 3^{k+1}(k + 1)
    %         \end{align}

    %         \end{mdframed}
    %     \end{enumerate}

    % \end{mdframed}

    \item

    \bigskip

    \begin{mdframed}

    \underline{\textbf{Rough Works:}}

    \bigskip

    Define $P(k)$ as

    \begin{center}
        $P(k):$ $\bigwedge\limits_{m=1}^{m=k} R(m) \leq R(k)$
    \end{center}

    \bigskip

    I will prove $\forall n \in \mathbb{N}^+, P(n)$ using complete induction.

    \bigskip

    \begin{enumerate}[1.]
        \item Inductive Step

        \begin{mdframed}
        Let $n \in \mathbb{N}$. Assume $n \geq 1$. Assume $H(n):\bigwedge\limits_{i=1}^{n-1} P(i)$.
        I will prove that $P(n)$ follows.

        \end{mdframed}

        \item Base Case ($n = 1$)

        \begin{mdframed}
        Let $n = 1$.

        \bigskip

        Then, $\bigwedge\limits_{m=1}^{m=1} R(m) = R(1)$.

        \bigskip

        So, $P(1)$ follows.

        \end{mdframed}

        \item Base Case ($n = 2$)

        \begin{mdframed}
        Let $n = 2$.

        \bigskip

        I need to show $P(2)$ holds. That is, $R(1) \leq R(2)$ and $R(2) \leq R(2)$.

        \bigskip

        I will do so in parts.

        \bigskip

        \textbf{Part 1 ($R(1) \leq R(2)$):}

        \bigskip

        In this part, $R(1) = 0$ and $R(2) = 2 + R(\lceil 2/3 \rceil) = 2 + R(1) = 2$.

        \bigskip

        Since $0 \leq 2$, we can conclude $R(1) \leq R(2)$.

        \bigskip

        \textbf{Part 2 ($R(2) \leq R(2)$):}

        \bigskip

        In this part, $R(2) = R(2)$, so we can conclude $R(2) \leq R(2)$.

        \end{mdframed}

        \item Case ($n > 2$)

        \begin{itemize}
            \item Show that only $R(n-1) \leq R(n)$ needs proving

            \begin{mdframed}
            Since $n > 2$, $1 \leq n - 1 < n$, so we know $P(n-1)$ holds.

            \bigskip

            Then, it is suffice to prove $R(n-1) \leq R(n)$.

            \end{mdframed}

            \item Show $R(n)$ holds, starting from $R(n-1)$

            \begin{mdframed}
            Starting from $R(n-1)$, we have

            \begin{align}
                R(n-1) &= (n-1) + R(\lceil (n-1)/3 \rceil) & [\text{By def.}]\\
                &\leq n + R(\lceil (n-1)/3 \rceil) \\
                &\leq n + R(\lceil n/3 \rceil) & [\text{By I.H, Since $1 \leq \lceil (n-1)/3 \rceil \leq \lceil n/3 \rceil < n$}]\\
                &= R(n)
            \end{align}

            \bigskip

            Thus, $P(n)$ holds.
            \end{mdframed}
        \end{itemize}

    \end{enumerate}

    \end{mdframed}

\end{enumerate}


\end{document}