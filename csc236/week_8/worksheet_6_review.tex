\documentclass[12pt]{article}
\usepackage[margin=2.5cm]{geometry}
\usepackage{enumerate}
\usepackage{amsfonts}
\usepackage{amsmath}
\usepackage{fancyhdr}
\usepackage{amsmath}
\usepackage{amssymb}
\usepackage{amsthm}
\usepackage{mdframed}
\usepackage{graphicx}
\usepackage{subcaption}
\usepackage{adjustbox}
\usepackage{listings}
\usepackage{xcolor}
\usepackage{booktabs}
\usepackage[utf]{kotex}

\definecolor{codegreen}{rgb}{0,0.6,0}
\definecolor{codegray}{rgb}{0.5,0.5,0.5}
\definecolor{codepurple}{rgb}{0.58,0,0.82}
\definecolor{backcolour}{rgb}{0.95,0.95,0.92}

\lstdefinestyle{mystyle}{
    backgroundcolor=\color{backcolour},
    commentstyle=\color{codegreen},
    keywordstyle=\color{magenta},
    numberstyle=\tiny\color{codegray},
    stringstyle=\color{codepurple},
    basicstyle=\ttfamily\footnotesize,
    breakatwhitespace=false,
    breaklines=true,
    captionpos=b,
    keepspaces=true,
    numbers=left,
    numbersep=5pt,
    showspaces=false,
    showstringspaces=false,
    showtabs=false,
    tabsize=1
}

\lstset{style=mystyle}

\begin{document}
\title{CSC236 Worksheet 6 Review}
\author{Hyungmo Gu}
\maketitle

\section*{Question 1}
\begin{itemize}
    \item
    \begin{mdframed}
        \underline{\textbf{Rough Work:}}

        \bigskip

        Assume that $\forall k \in \mathbb{N}$, $R(3^k)=k3^k$.

        \bigskip

        I need to prove $R \in \Theta(n \lg n)$. That is, $R \in \mathcal{O}(n \lg n)$
        and $R \in \Omega(n \lg n)$.

        \bigskip

        I will do so in parts.

        \begin{enumerate}[1.]
            \item Part 1 (Proving $R \in \mathcal{O}(n \lg n)$)

            \begin{mdframed}

            Let $n \in \mathbb{N}$. Define $n^* = 3^{\lceil \log_3 n \rceil}$.
            Then, we have

            \begin{align}
                \lceil \log_3 n \rceil - 1 < \log_3 n \leq \lceil \log_3 n \rceil
            \end{align}

            \bigskip

            I will also use the fact proved in week 7 tutorial exercise that
            $R$ is non-decreasing.

            \bigskip

            Let $d = 6$. Then, $d \in \mathbb{R}^+$. Let $B = 3$. Then, $B \in \mathbb{R}^+$.
            Assume $n \geq B$.

            \bigskip

            I need to show $R(n) \leq dn\lg n$.

            \bigskip

            Starting from $R(n)$, we have

            \begin{align}
                R(n) &\leq R(n^*) & [\text{Since $n \leq n^*$ and $R$ is non-decreasing}]\\
                &= n^* \log_3 n^* & [\text{By replacing $3^k$ for $n^*$}]\\
                &\leq 3n \log_3 3n & [\text{Since $n \leq n^* \Rightarrow 3n \leq 3n^*$}]\\
                &= 3n(\log_3 n + 1)\\
                &= 3n(\log_3 n + \log_3 n) & [\text{Since $n \leq B = 3 \Rightarrow \log_3 n \leq 1$}]\\
                &\leq 6n\log_3 n\\
                &\leq dn\log_3 n & [\text{Since $d = 6$}]\\
                &\leq dn\lg n
            \end{align}

            \end{mdframed}

            \item Part 2 (Proving $R \in \Omega(n \lg n)$)

            \begin{mdframed}
            Let $n \in \mathbb{N}$. Define $n^* = 3^{\lceil \log_3 n \rceil}$.
            Then, we have

            \begin{align}
                \lceil \log_3 n \rceil - 1 < \log_3 n \leq \lceil \log_3 n \rceil
            \end{align}

            \bigskip

            I will also use the fact proved in week 7 tutorial exercise that
            $R$ is non-decreasing.

            \bigskip

            Let $d = 1/(6\lg 3)$. Then, $d \in \mathbb{R}^+$. Let $B = 9$. Then, $B \in \mathbb{R}^+$.
            Assume $n \geq B$.

            \bigskip

            I need to show $R(n) \geq dn\lg n$.

            \bigskip

            Starting from $R(n)$, we have

            \begin{align}
                R(n)&\geq R(n^/3) & [\text{Since $n^*/3 < n$ and $R$ is non-decreasing}]\\
                &= (n^*/3)\log_3 (n^*/3) & [\text{By replacing $3^k$ for $n^*$}]\\
                &= (n/3)\log_3 (n/3) & [\text{Since $n < n^* \Rightarrow (n/3) \leq (n^*/3)$}]\\
                &= (n/3)(\log_3 n - 1)\\
                &\geq (\log_3 n - (\log_3 n)/2)& [\text{Since $n \geq B = 9 \Rightarrow (\log_3 n)/2 \geq 1$}]\\
                &= (n\log_3 n)/6\\
                &= (n\lg n)/(6\lg 3)\\
                &= dn\lg n
            \end{align}

            \end{mdframed}
        \end{enumerate}

    \end{mdframed}

\end{itemize}

\end{document}