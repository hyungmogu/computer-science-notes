\documentclass[12pt]{article}
\usepackage[margin=2.5cm]{geometry}
\usepackage{enumerate}
\usepackage{amsfonts}
\usepackage{amsmath}
\usepackage{fancyhdr}
\usepackage{amsmath}
\usepackage{amssymb}
\usepackage{amsthm}
\usepackage{mdframed}
\usepackage{graphicx}
\usepackage{subcaption}
\usepackage{adjustbox}
\usepackage{listings}
\usepackage{xcolor}
\usepackage{booktabs}
\usepackage[utf]{kotex}

\definecolor{codegreen}{rgb}{0,0.6,0}
\definecolor{codegray}{rgb}{0.5,0.5,0.5}
\definecolor{codepurple}{rgb}{0.58,0,0.82}
\definecolor{backcolour}{rgb}{0.95,0.95,0.92}

\lstdefinestyle{mystyle}{
    backgroundcolor=\color{backcolour},
    commentstyle=\color{codegreen},
    keywordstyle=\color{magenta},
    numberstyle=\tiny\color{codegray},
    stringstyle=\color{codepurple},
    basicstyle=\ttfamily\footnotesize,
    breakatwhitespace=false,
    breaklines=true,
    captionpos=b,
    keepspaces=true,
    numbers=left,
    numbersep=5pt,
    showspaces=false,
    showstringspaces=false,
    showtabs=false,
    tabsize=1
}

\lstset{style=mystyle}

\begin{document}
\title{CSC236 Worksheet 6 Solution}
\author{Hyungmo Gu}
\maketitle

\section*{Question 1}
\begin{itemize}
    \item

    \begin{proof}

    Assume that for all $k \in \mathbb{N}$, $R(3^k) = k3^k$.

    \bigskip

    I need to prove $R \in \mathcal{O}(n \lg n)$ and $R \in \Omega(n \lg n)$.

    \bigskip

    I will do so in parts.

    \bigskip

    \underline{\textbf{Part 1 (Proving $R \in \mathcal{O}(n \lg n)$):}}

    \bigskip

    Define $n^* = 3^{\lceil \log_3 n \rceil}$. Then, we have,

    \begin{align}
        \lceil \log_3 n \rceil - 1 < \log_3 n \leq \lceil \log_3 n \rceil \Rightarrow n^*/3 < n \leq n^*
    \end{align}

    I will also use the assumption (proved last week) that $R$ is non-decreasing.

    \bigskip

    Let $d = 6$. Then $d \in \mathbb{R}^+$. Let $B = 3$. Then $B \in \mathbb{N}^+$. Let
    $n$ be an arbitrary natural number no smaller than $B$. Then,

    \bigskip

    \begin{align}
    R(n) &\leq R(n^*)  & [\text{Since $n < n^*$, and $R$ is non-decreasing}]\\
    &= n^* \log_3 n^* & [\text{By assumption, and replacing $n^*$ for $3^k$}]\\
    &\leq 3n \log_3 3n  & [\text{Since $n \leq n^* \Rightarrow 3n \leq 3n^*$}]\\
    &\leq 3n(\log_3 n + 1)\\
    &\leq 3n(\log_3 n + \log_3 n) & [\text{Since $n \geq 3 \Rightarrow \log_3 n \geq 1$}]\\
    &= 6n \log_3 n\\
    &\leq (6n \lg n)/\lg 3 & [\text{By change of basis to $\lg$}]\\
    &< 6n \lg n\\
    &= dn \lg n & [\text{Since $d = 6$}]
    \end{align}

    \bigskip

    So $R \in \mathcal{O}(n \lg n)$, since $\log_3 n$ differs from $\lg n$ by a constant factor.

    \bigskip

    \underline{\textbf{Part 2 (Proving $R \in \Omega(n \lg n)$):}}

    \bigskip

    Define $n^* = 3^{\lceil \log_3 n \rceil}$. Then, we have,

    \begin{align}
        \lceil \log_3 n \rceil - 1 < \log_3 n \leq \lceil \log_3 n \rceil \Rightarrow n^*/3 < n \leq n^*
    \end{align}

    I will also use the assumption (proved last week) that $R$ is non-decreasing.

    \bigskip

    Let $d = 1/(6\lg 3)$. Then $d \in \mathbb{R}^+$. Let $B = 9$. Then $B \in \mathbb{N}^+$. Let
    $n$ be an arbitrary natural number no smaller than $B$. Then,

    \begin{align}
    R(n) &\geq R(n^*/3) & [\text{Since $n^*/3 < n$, and $R$ is non-decreasing}]\\
    &= (n^*/3) \cdot \log_3 (n^*/3) & [\text{By assumption, and replacing $n^*$ for $3^k$}]\\
    &\geq (n/3) \cdot \log_3 (n/3) & [\text{Since $n^* \leq n \Rightarrow n^*/3 \leq n/3$}]\\
    &= (n/3) \cdot (\log_3 n - 1)\\
    &\geq (n/3) \cdot (\log_3 n - (\log_3 n)/2) & [\text{Since $n \geq 9 \Rightarrow (\log_3 n)/2 \geq 1$}]\\
    &= (n/6) \cdot \log_3 n\\
    &= (n/6) \cdot (\lg n/\lg 3)\\
    &= (n/(6\lg 3)) \cdot \lg n\\
    &= dn \cdot \lg n & [\text{Since $d = 1/(6\lg 3)$}]
    \end{align}

    So, $R \in \Omega(n \lg n)$.

    \end{proof}

    \bigskip

    \begin{mdframed}
    \underline{\textbf{Correct Solution:}}

    \bigskip

    Assume that for all $k \in \mathbb{N}$, $R(3^k) = k3^k$.

    \bigskip

    I need to prove $R \in \mathcal{O}(n \lg n)$ and $R \in \Omega(n \lg n)$.

    \bigskip

    I will do so in parts.

    \bigskip

    \underline{\textbf{Part 1 (Proving $R \in \mathcal{O}(n \lg n)$):}}

    \bigskip

    Define $n^* = 3^{\lceil \log_3 n \rceil}$. Then, we have,
    \setcounter{equation}{0}
    \begin{align}
        \lceil \log_3 n \rceil - 1 < \log_3 n \leq \lceil \log_3 n \rceil \Rightarrow n^*/3 < n \leq n^*
    \end{align}

    I will also use the assumption (proved last week) that $R$ is non-decreasing.

    \bigskip

    Let $d = 6$. Then $d \in \mathbb{R}^+$. Let $B = 3$. Then $B \in \mathbb{N}^+$. Let
    $n$ be an arbitrary natural number no smaller than $B$. Then,

    \bigskip

    \begin{align}
    R(n) &\leq R(n^*)  & [\text{Since $n < n^*$, and $R$ is non-decreasing}]\\
    &= n^* \log_3 n^* & [\text{By assumption, and replacing $n^*$ for $3^k$}]\\
    &\leq 3n \log_3 3n  & [\text{Since $n \leq n^* \Rightarrow 3n \leq 3n^*$}]\\
    &\color{red}=\color{black} 3n(\log_3 n + 1)\\
    &\leq 3n(\log_3 n + \log_3 n) & [\text{Since $n \geq 3 \Rightarrow \log_3 n \geq 1$}]\\
    &= 6n \log_3 n\\
    &\color{red}=\color{black} (6n \lg n)/\lg 3 & [\text{By change of basis to $\lg$}]\\
    &< 6n \lg n\\
    &= dn \lg n & [\text{Since $d = 6$}]
    \end{align}

    \bigskip

    So $R \in \mathcal{O}(n \lg n)$, since $\log_3 n$ differs from $\lg n$ by a constant factor.

    \bigskip

    \underline{\textbf{Part 2 (Proving $R \in \Omega(n \lg n)$):}}

    \bigskip

    Define $n^* = 3^{\lceil \log_3 n \rceil}$. Then, we have,

    \begin{align}
        \lceil \log_3 n \rceil - 1 < \log_3 n \leq \lceil \log_3 n \rceil \Rightarrow n^*/3 < n \leq n^*
    \end{align}

    I will also use the assumption (proved last week) that $R$ is non-decreasing.

    \bigskip

    Let $d = 1/(6\lg 3)$. Then $d \in \mathbb{R}^+$. Let $B = 9$. Then $B \in \mathbb{N}^+$. Let
    $n$ be an arbitrary natural number no smaller than $B$. Then,

    \begin{align}
    R(n) &\geq R(n^*/3) & [\text{Since $n^*/3 < n$, and $R$ is non-decreasing}]\\
    &= (n^*/3) \cdot \log_3 (n^*/3) & [\text{By assumption, and replacing $n^*$ for $3^k$}]\\
    &\geq (n/3) \cdot \log_3 (n/3) & [\text{Since $n^* \leq n \Rightarrow n^*/3 \leq n/3$}]\\
    &= (n/3) \cdot (\log_3 n - 1)\\
    &\geq (n/3) \cdot (\log_3 n - (\log_3 n)/2) & [\text{Since $n \geq 9 \Rightarrow (\log_3 n)/2 \geq 1$}]\\
    &= (n/6) \cdot \log_3 n\\
    &= (n/6) \cdot (\lg n/\lg 3)\\
    &= (n/(6\lg 3)) \cdot \lg n\\
    &= dn \cdot \lg n & [\text{Since $d = 1/(6\lg 3)$}]
    \end{align}

    So, $R \in \Omega(n \lg n)$\color{red}, since $\log_3 n$ differs from $\lg n$ by a constant factor\color{black}.
\end{mdframed}

\end{itemize}

% \bigskip

% \begin{mdframed}
%     \underline{\textbf{Rough Work:}}

%     \bigskip

%     Assume that for all $k \in \mathbb{N}$, $R(3^k) = k3^k$.

%     \bigskip

%     I need to prove $R \in \mathcal{O}(n \lg n)$ and $R \in \Omega(n \lg n)$.

%     \bigskip

%     I will do so in parts.

%     \begin{enumerate}[1.]
%         \item Prove that $R \in \mathcal{O}(n \lg n)$.

%         \bigskip

%         Define $n^* = 3^{\lceil \log_3 n \rceil}$. Then, we have,

%         \begin{align}
%             \lceil \log_3 n \rceil - 1 < \log_3 n \leq \lceil \log_3 n \rceil \Rightarrow n^*/3 < n \leq n^*
%         \end{align}

%         I will also use the assumption (proved last week) that $R$ is non-decreasing.

%         \bigskip

%         Let $d = 6$. Then $d \in \mathbb{R}^+$. Let $B = 3$. Then $B \in \mathbb{N}^+$. Let
%         $n$ be an arbitrary natural number no smaller than $B$. Then,

%         \begin{mdframed}
%         \begin{align}
%         R(n) &\leq R(n^*)  & [\text{Since $n < n^*$, and $R$ is non-decreasing}]\\
%         &= n^* \log_3 n^* & [\text{By assumption, and replacing $n^*$ for $3^k$}]\\
%         &\leq 3n \log_3 3n  & [\text{Since $n \leq n^* \Rightarrow 3n \leq 3n^*$}]\\
%         &\leq 3n(\log_3 n + 1)\\
%         &\leq 3n(\log_3 n + \log_3 n) & [\text{Since $n \geq 3 \Rightarrow \log_3 n \geq 1$}]\\
%         &= 6n \log_3 n\\
%         &\leq (6n \lg n)/\lg 3 & [\text{By change of basis to $\lg$}]\\
%         &< 6n \lg n\\
%         &= dn \lg n & [\text{Since $d = 6$}]
%         \end{align}

%         \end{mdframed}

%         So $R \in \mathcal{O}(n \lg n)$, since $\log_3 n$ differs from $\lg n$ by a constant factor.

%         \item Prove that $R \in \Omega(n \log n)$

%         \begin{mdframed}

%         Define $n^* = 3^{\lceil \log_3 n \rceil}$. Then, we have,

%         \begin{align}
%             \lceil \log_3 n \rceil - 1 < \log_3 n \leq \lceil \log_3 n \rceil \Rightarrow n^*/3 < n \leq n^*
%         \end{align}

%         I will also use the assumption (proved last week) that $R$ is non-decreasing.

%         \bigskip

%         Let $d = 1/(6\lg 3)$. Then $d \in \mathbb{R}^+$. Let $B = 9$. Then $B \in \mathbb{N}^+$. Let
%         $n$ be an arbitrary natural number no smaller than $B$. Then,

%         \begin{align}
%         R(n) &\geq R(n^*/3) & [\text{Since $n^*/3 < n$, and $R$ is non-decreasing}]\\
%         &= (n^*/3) \cdot \log_3 (n^*/3) & [\text{By assumption, and replacing $n^*$ for $3^k$}]\\
%         &\geq (n/3) \cdot \log_3 (n/3) & [\text{Since $n^* \leq n \Rightarrow n^*/3 \leq n/3$}]\\
%         &= (n/3) \cdot (\log_3 n - 1)\\
%         &\geq (n/3) \cdot (\log_3 n - (\log_3 n)/2) & [\text{Since $n \geq 9 \Rightarrow (\log_3 n)/2 \geq 1$}]\\
%         &= (n/6) \cdot \log_3 n\\
%         &= (n/6) \cdot (\lg n/\lg 3)\\
%         &= (n/(6\lg 3)) \cdot \lg n\\
%         &= dn \cdot \lg n & [\text{Since $d = 1/(6\lg 3)$}]
%         \end{align}

%         So, $R \in \Omega(n \lg n)$.

%         \end{mdframed}
%     \end{enumerate}

% \end{mdframed}

\bigskip


\bigskip

\underline{\textbf{Notes:}}

\bigskip

\begin{itemize}
    \item Learned that if there is trouble going from $\log_3n - 1$ to $dn\lg n$,
    a good approach is to increase the value of B.
    \item Noticed that professor used `Let $d = \underline{\hspace{1cm}}$. Then $d \in \mathbb{R}^+$'
    to define variable's value as well as its type.
    \item
    $g \in \Theta(f):\: g \in \mathcal{O}(f) \land g \in \Omega(f)$

    or

    $g \in \Theta(f):\:\exists c_1,c_2,n_1 \in \mathbb{R}^{+}, \forall n \in \mathbb{N}, n \geq n_1
    \Rightarrow c_1g(n) \leq f(n) \leq c_2g(n)$, where $f,g:\:\mathbb{N} \to \mathbb{R}^{\geq 0}$

    \item
    $g \in \Omega(f):\:\exists c,n_o \in \mathbb{R}^{+},\:\forall n \in
    \mathbb{N},\:n \geq n_0 \Rightarrow g(n) \geq cf(n)$, where $f,g:\mathbb{N} \to \mathbb{R}^{\geq 0}$

    \item

    $g \in \mathcal{O}(f):\:\exists c,n_o \in \mathbb{R}^{+},\:\forall n \in
    \mathbb{N},\:n \geq n_0 \Rightarrow g(n) \leq cf(n)$, where $f,g:\mathbb{N} \to \mathbb{R}^{\geq 0}$
\end{itemize}

\bigskip

\section*{Question 2}

\begin{itemize}
    \item

    \begin{proof}
    Let $n \in \mathbb{N}$ and bits $b_0,\cdots,b_k \in \{0,1\}$ be such that
    $n = \sum\limits_{i = 0}^{i = k} 2^ib_i$. I will use identities:
    \setcounter{equation}{0}
    \begin{align}
        \lfloor n/2 \rfloor &= \sum\limits_{i=1}^{i=k} 2^ib_i\\
        b_0 &= n \mod 2
    \end{align}

    \bigskip

    Define $P(n):$ ``If $n$ is a natural number, then \textit{decimal\_to\_binary($n$)}''
    terminates and returns binary string representing $n$ with no leading zeros,
    except if $n$ is 0.

    \bigskip

    I will use complete induction to prove $\forall n \in \mathbb{N}, P(n)$.

    \bigskip

    \underline{\textbf{Inductive Step}}

    \bigskip

    Let $n \in \mathbb{N}$. Assume $H(n):\bigwedge\limits_{i=0}^{i=n-1} P(i)$. I will
    show $P(n)$ follows.

    \bigskip

    \underline{\textbf{Base Case ($n < 2$):}}

    \bigskip

    Let $n < 2$.

    \bigskip

    Then, the if part of \textit{decimal\_to\_binary($n$)} executes, and the
    program\\ terminates by outputing the input value.

    \bigskip

    Since, binary rep of $n=0$ is 0, and binary rep of $n=1$ is 1, $P(n)$ follows
    in this step.

    \bigskip

    \underline{\textbf{Case ($n \geq 2$):}}

    \bigskip

    Let $n \geq 2$.

    \bigskip

    Then, since $n \geq 2$, and $0 \leq n\:\%\:2 \leq n // 2 < n$, the
    induction hypothesis tells us $P(n//2)$ and $P(n\:\%\:2)$ holds.

    \bigskip

    Then, since $n // 2$ is $\lfloor n/2 \rfloor = \sum\limits_{i=1}^{i=k} 2^ib_i$,
    and $n\:\%\:2$ is $n \mod 2 = b_0$, we have

    \begin{align}
        b_0 + \lfloor n/2 \rfloor &= b_0 + \sum\limits_{i=1}^{i=k} 2^ib_i\\
        &= 2^0 \cdot b_0 + \sum\limits_{i=1}^{i=k} 2^ib_i\\
        &= \sum\limits_{i=0}^{i=k} 2^ib_i\\
        &= n
    \end{align}

    Thus, $P(n)$ follows from $H(n)$ in this step.

    \end{proof}

    % \begin{mdframed}
    %     \underline{\textbf{Rough Work:}}

    %     \bigskip

    %     Let $n \in \mathbb{N}$ and bits $b_0,\cdots,b_k \in \{0,1\}$ be such that
    %     $n = \sum\limits_{i = 0}^{i = k} 2^ib_i$. I will use identities:

    %     \begin{align}
    %         \lfloor n/2 \rfloor &= \sum\limits_{i=1}^{i=k} 2^ib_i\\
    %         b_0 &= n \mod 2
    %     \end{align}

    %     \bigskip

    %     Define $P(n):$ ``If $n$ is a natural number, then \textit{decimal\_to\_binary($n$)}''
    %     terminates and returns binary string representing $n$ with no leading zeros,
    %     except if $n$ is 0.

    %     \bigskip

    %     I will use complete induction to prove $\forall n \in \mathbb{N}, P(n)$.

    %     \bigskip

    %     \begin{enumerate}[1.]
    %         \item Inductive Step
    %         \begin{mdframed}
    %         \underline{\textbf{Inductive Step}}

    %         \bigskip

    %         Let $n \in \mathbb{N}$. Assume $H(n):\bigwedge\limits_{i=0}^{i=n-1} P(i)$. I will
    %         show $P(n)$ follows.
    %         \end{mdframed}
    %         \item Base Case ($n < 2$)

    %         \begin{mdframed}
    %         \underline{\textbf{Base Case ($n < 2$):}}

    %         \bigskip

    %         Let $n < 2$.

    %         \bigskip

    %         Then, the if part of \textit{decimal\_to\_binary($n$)} executes, and the
    %         program\\ terminates by outputing the input value.

    %         \bigskip

    %         Since, binary rep of $n=0$ is 0, and binary rep of $n=1$ is 1, $P(n)$ follows
    %         in this step.

    %         \end{mdframed}

    %         \item Case ($n \geq 2$)

    %         \begin{mdframed}
    %         \underline{\textbf{Case ($n \geq 2$):}}

    %         \bigskip

    %         Let $n \geq 2$.

    %         \bigskip

    %         Then, since $n \geq 2$, and $0 \leq n\:\%\:2 \leq n // 2 < n$, the
    %         induction hypothesis tells us $P(n//2)$ and $P(n\:\%\:2)$ holds.

    %         \bigskip

    %         Then, since $n // 2$ is $\lfloor n/2 \rfloor = \sum\limits_{i=1}^{i=k} 2^ib_i$,
    %         and $n\:\%\:2$ is $n \mod 2 = b_0$, we have

    %         \begin{align}
    %             b_0 + \lfloor n/2 \rfloor &= b_0 + \sum\limits_{i=1}^{i=k} 2^ib_i\\
    %             &= 2^0 \cdot b_0 + \sum\limits_{i=1}^{i=k} 2^ib_i\\
    %             &= \sum\limits_{i=0}^{i=k} 2^ib_i\\
    %             &= n
    %         \end{align}

    %         Thus, $P(n)$ follows from $H(n)$ in this step.

    %         \end{mdframed}
    %     \end{enumerate}
    % \end{mdframed}

\end{itemize}

\bigskip

\underline{\textbf{Notes:}}

\bigskip

\begin{itemize}
    \item \textbf{Correct:} A program is correct if it produces output on every acceptable input
    \item \textbf{Precondition} and \textbf{Postcondition} are assertions involving some of the variables of the program
    \begin{itemize}
        \item \textbf{Precondition} states what must be true \textit{before} program starts execution
        \item \textbf{Postcondition} states what must be true when the program \textit{ends}
    \end{itemize}
\end{itemize}

\end{document}