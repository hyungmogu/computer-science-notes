\documentclass[12pt]{article}
\usepackage[margin=2.5cm]{geometry}
\usepackage{enumerate}
\usepackage{amsfonts}
\usepackage{amsmath}
\usepackage{fancyhdr}
\usepackage{amsmath}
\usepackage{amssymb}
\usepackage{amsthm}
\usepackage{mdframed}
\usepackage{graphicx}
\usepackage{subcaption}
\usepackage{adjustbox}
\usepackage{listings}
\usepackage{xcolor}
\usepackage{booktabs}
\usepackage[utf]{kotex}

\definecolor{codegreen}{rgb}{0,0.6,0}
\definecolor{codegray}{rgb}{0.5,0.5,0.5}
\definecolor{codepurple}{rgb}{0.58,0,0.82}
\definecolor{backcolour}{rgb}{0.95,0.95,0.92}

\lstdefinestyle{mystyle}{
    backgroundcolor=\color{backcolour},
    commentstyle=\color{codegreen},
    keywordstyle=\color{magenta},
    numberstyle=\tiny\color{codegray},
    stringstyle=\color{codepurple},
    basicstyle=\ttfamily\footnotesize,
    breakatwhitespace=false,
    breaklines=true,
    captionpos=b,
    keepspaces=true,
    numbers=left,
    numbersep=5pt,
    showspaces=false,
    showstringspaces=false,
    showtabs=false,
    tabsize=1
}

\lstset{style=mystyle}

\begin{document}
\title{CSC236 Worksheet 4 Solution}
\author{Hyungmo Gu}
\maketitle

\section*{Question 1}
\begin{itemize}
    \item

    Let $n \in \mathbb{N}$ and assume that $\exists k \in \mathbb{N}^+$, $n = 3^k$,
    so $k = \log_3 n$.

    \bigskip

    Then, since $n = 3^k$ and $3 \mid n$, we have $\lceil n/3 \rceil = n/3$.

    \bigskip

    Then,

    \begin{align}
        T(n) &= 2n + T(n/3) & [\text{By def.}]\\
        &= 2(n/3) + (2(n/3) +  T(n/3^2)) & [\text{By subtituting n/3 for n in def.}]\\
        &= 2^2(n/3) + T(n/3^2)\\
        &= 2^3(n/3^2) + T(n/3^3) & [\text{By subtituting n/3 for n in def.}]\\
        &\vdots\\
        &= 2^k(n/3^{k-1}) + T(n/3^k) & [\text{After $k$ applications}]\\
        &= 2^{\log_3 n}(n/3^{log_3 n - 1}) + T(n/3^{\log_3 n}) & [\text{By replacing $k = \log_3 n$}]\\
        &= 2^{\log_3 n}(n(3)/n) + T(n/n)\\
        &= 3 \cdot 2^{\log_3 n} + T(1)\\
        &= 3 \cdot 2^{\log_3 n} + 2
    \end{align}
\end{itemize}

\bigskip

\begin{mdframed}
    \underline{\textbf{Correct Solution:}}

    \bigskip

    Let $n \in \mathbb{N}$ and assume that $\exists k \in \mathbb{N}^+$, $n = 3^k$,
    so $k = \log_3 n$.

    \bigskip

    Then, since $n = 3^k$ and $3 \mid n$, we have $\lceil n/3 \rceil = n/3$.

    \bigskip

    Then,

    \color{red}
    \begin{align}
        T(n) &= 2n + T(n/3) & [\text{By def.}]\\
        &= 2n + 2(n/3) +  T(n/3^2) & [\text{By subtituting n/3 for n in def.}]\\
        &= 2n + 2(n/3) + 2(n/3^2) + T(n/3^3) & [\text{By subtituting n/3 for n in def.}]\\
        &\vdots\\
        &= 2\sum\limits_{i=0}^{k-1} n/3^i + T(n/3^k)\\
        &= 2 \cdot 3^k \Bigl( \frac{1 - \bigl(1/3\bigr)^k}{1 - 1/3} \Bigr) + T(n/3^k) & [\text{By using geometric series}]\\
        &= 2 \cdot 3^k \cdot 3/2 \Bigl( 1 - \bigl(1/3\bigr)^k \Bigr) + T(n/n)\\
        &= 3(3^k - 1) + T(1)\\
        &= 3^{k+1} - 1
    \end{align}
    \color{black}

\end{mdframed}

\bigskip

\underline{\textbf{Notes:}}

\begin{itemize}
    \item \textbf{Repeated Subtitution:}

    \begin{itemize}
        \item Is a technique used to find a closed form formula
        \item \textbf{closed form formula} is a simple formula that allows evaluation
        of $T(n)$ without the need to evaluate, say $T(n/2)$

        \bigskip

        i.e. from

        \begin{align}
            T(n) =
            \begin{cases}
            c & \text{if $n = 1$}\\
            2T(n/2) + dn & \text{if $n > 1$}
            \end{cases}
        \end{align}

        to

        \bigskip

        $T(n) = cn + dn \log_2 n$
    \end{itemize}

    \bigskip

    \begin{mdframed}
        \underline{\textbf{Example:}}

        \bigskip

        Consider the recurrence
        \setcounter{equation}{0}
        \begin{align}
            T(n) =
            \begin{cases}
            c & \text{if $n = 1$}\\
            2T(n/2) + dn & \text{if $n > 1$}
            \end{cases}
        \end{align}

        Find closed form formula for $T(n)$, where $n$ is an arbitrary power of 2.
        That is $\exists k \in \mathbb{N}, n = 2^k$.

        \bigskip

        Let $n \in \mathbb{N}$ and assume that $\exists k \in \mathbb{N}^+$, $n = 2^k$, so $k = \log_2 n$.

        \bigskip

        Then,

        \begin{align}
            T(n) &= 2T(n/2) + dn & [\text{By 1}]\\
            &= 2\Bigl(2T(n/2^2) + dn/2\Bigr) + dn & [\text{By subtituting $n/2$ for $n$ in 1}]\\
            &= 2^2T(n/2^2) + 2dn\\
            &= 2^2\Bigl(2T(n/2^3) + dn/2^2\Bigr) + 2dn & [\text{By subtituting $n/2^2$ for $n$ in 1}]\\
            &= 2^3T(n/2^3) + 3dn & [\text{By subtituting $n/2^2$ for $n$ in 1}]\\
            &\vdots\\
            &= 2^kT(n/2^k) + kdn & [\text{After $k$ applications}]\\
            &= 2^{\log_2 n}T(n/2^{\log_2 n}) + (\log_2 n)dn & [\text{By replacing $k = \log_2 n$}]\\
            &= nT(1) + (\log_2 n)dn\\
            &= cn + (\log_2 n)dn
        \end{align}

        \bigskip

    \end{mdframed}

\end{itemize}

\section*{Question 2}

\end{document}