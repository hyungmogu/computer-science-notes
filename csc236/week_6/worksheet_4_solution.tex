\documentclass[12pt]{article}
\usepackage[margin=2.5cm]{geometry}
\usepackage{enumerate}
\usepackage{amsfonts}
\usepackage{amsmath}
\usepackage{fancyhdr}
\usepackage{amsmath}
\usepackage{amssymb}
\usepackage{amsthm}
\usepackage{mdframed}
\usepackage{graphicx}
\usepackage{subcaption}
\usepackage{adjustbox}
\usepackage{listings}
\usepackage{xcolor}
\usepackage{booktabs}
\usepackage[utf]{kotex}

\definecolor{codegreen}{rgb}{0,0.6,0}
\definecolor{codegray}{rgb}{0.5,0.5,0.5}
\definecolor{codepurple}{rgb}{0.58,0,0.82}
\definecolor{backcolour}{rgb}{0.95,0.95,0.92}

\lstdefinestyle{mystyle}{
    backgroundcolor=\color{backcolour},
    commentstyle=\color{codegreen},
    keywordstyle=\color{magenta},
    numberstyle=\tiny\color{codegray},
    stringstyle=\color{codepurple},
    basicstyle=\ttfamily\footnotesize,
    breakatwhitespace=false,
    breaklines=true,
    captionpos=b,
    keepspaces=true,
    numbers=left,
    numbersep=5pt,
    showspaces=false,
    showstringspaces=false,
    showtabs=false,
    tabsize=1
}

\lstset{style=mystyle}

\begin{document}
\title{CSC236 Worksheet 4 Solution}
\author{Hyungmo Gu}
\maketitle

\section*{Question 1}
\begin{itemize}
    \item
\end{itemize}

\bigskip

\underline{\textbf{Notes:}}

\begin{itemize}
    \item \textbf{Repeated Subtitution:}

    \begin{itemize}
        \item Is a technique used to find a closed form formula
        \item \textbf{closed form formula} is a simple formula that allows evaluation
        of $T(n)$ without the need to evaluate $T([n/3])$
    \end{itemize}

    \bigskip

    \begin{mdframed}
        \underline{\textbf{Example:}}

        \bigskip

        Consider the recurrence

        \begin{align}
            T(n) =
            \begin{cases}
            c & \text{if $n = 1$}\\
            2T([n/2]) + dn & \text{if $n > 1$}
            \end{cases}
        \end{align}

        Find closed form formula for $T(n)$, where $n$ is an arbitrary power of 2.
        That is $\exists k \in \mathbb{N}, n = 2^k$.

        \bigskip

        \begin{align}
            T(n) &= 2T(n/2) + dn & [\text{By 1}]\\
            &= 2\Bigl(2T(n/2^2) + dn/2\Bigr) + dn & [\text{By subtituting $n/2$ for $n$ in 1}]\\
            &= 2^2T(n/2^2) + 2dn\\
            &= 2^2\Bigl(2T(n/2^3) + dn/2^2\Bigr) + 2dn & [\text{By subtituting $n/2^2$ for $n$ in 1}]\\
            &= 2^3T(n/2^3) + 3dn & [\text{By subtituting $n/2^2$ for $n$ in 1}]\\
            &\vdots\\
            &= 2^iT(n/2^i) + idn & [\text{After $i$ applications}]
        \end{align}
    \end{mdframed}

\end{itemize}

\section*{Question 2}

\end{document}