\documentclass[12pt]{article}
\usepackage[margin=2.5cm]{geometry}
\usepackage{enumerate}
\usepackage{amsfonts}
\usepackage{amsmath}
\usepackage{fancyhdr}
\usepackage{amsmath}
\usepackage{amssymb}
\usepackage{amsthm}
\usepackage{mdframed}
\usepackage{graphicx}
\usepackage{subcaption}
\usepackage{adjustbox}
\usepackage{listings}
\usepackage{xcolor}
\usepackage{booktabs}
\usepackage[utf]{kotex}

\definecolor{codegreen}{rgb}{0,0.6,0}
\definecolor{codegray}{rgb}{0.5,0.5,0.5}
\definecolor{codepurple}{rgb}{0.58,0,0.82}
\definecolor{backcolour}{rgb}{0.95,0.95,0.92}

\lstdefinestyle{mystyle}{
    backgroundcolor=\color{backcolour},
    commentstyle=\color{codegreen},
    keywordstyle=\color{magenta},
    numberstyle=\tiny\color{codegray},
    stringstyle=\color{codepurple},
    basicstyle=\ttfamily\footnotesize,
    breakatwhitespace=false,
    breaklines=true,
    captionpos=b,
    keepspaces=true,
    numbers=left,
    numbersep=5pt,
    showspaces=false,
    showstringspaces=false,
    showtabs=false,
    tabsize=1
}

\lstset{style=mystyle}

\begin{document}
\title{CSC236 Midterm 2 Version 1 Solution}
\author{Hyungmo Gu}
\maketitle

\section*{Question 1}
\begin{itemize}
    \item

    Let $n, q \in \mathbb{N}$. Let $r \in \{0,1\}$

    \bigskip

    Assume $n > 2$, and $n = 2q + r$.

    \bigskip

    I need to find a closed form for $T(2q + r)$, using repeated subtitution.

    \bigskip

    Starting from $T(n)$, we have

    \begin{align}
        T(n) &= n + T(n-2) & [\text{By def. since $n > 2$}]\\
        T(2q+r) &= 2q+r + T(2q+r-2) & [\text{By replacing $n$ for $2q + r$}]\\
        &= 2q+r + T(2(q-1)+r)\\
        &\vdots\\
        &= \sum\limits_{i=0}^{i=q-1} (2(q-i) + r) + T(r) & [\text{After $q-1$ repeatitions}]\\
        &= 2\sum\limits_{i=0}^{i=q-1} (q-i) + \sum\limits_{i=0}^{i=q-1} r + T(r)\\
        &= 2\sum\limits_{i=0}^{i=q-1} (q-i) + \sum\limits_{i=0}^{i=q-1} r & [\text{Since $T(r) = 0$}]\\
        &= 2\sum\limits_{i'=1}^{i=q} i' + \sum\limits_{i=0}^{i=q-1} r\\
        &= 2\sum\limits_{i'=1}^{i'=q} i' + \sum\limits_{i=0}^{i=q-1} r\\
        &= 2\sum\limits_{i'=1}^{i'=q} i' + \sum\limits_{i=0}^{i=q-1} r\\
        &= 2(q(q+1))/2 + \sum\limits_{i=0}^{i=q-1} r &[\text{By using $\sum\limits_{i=1}^{i=n} i =(n(n+1))/2$}]\\
        &= q(q+1) + rq\\
        &= q(q+1+r)
    \end{align}

    % \begin{mdframed}
    % \underline{\textbf{Rough Works:}}
    % \bigskip

    % Let $n, q \in \mathbb{N}$. Let $r \in \{0,1\}$

    % \bigskip

    % Assume $n > 2$, and $n = 2q + r$.

    % \bigskip

    % I need to find a closed form for $T(2q + r)$, using repeated subtitution.

    % \begin{enumerate}[1.]
    %     \item Find $T(2q + r)$ in closed form

    %     \begin{mdframed}
    %     Starting from $T(n)$, we have

    %     \begin{align}
    %         T(n) &= n + T(n-2) & [\text{By def. since $n > 2$}]\\
    %         T(2q+r) &= 2q+r + T(2q+r-2) & [\text{By replacing $n$ for $2q + r$}]\\
    %         &= 2q+r + T(2(q-1)+r)\\
    %         &\vdots\\
    %         &= \sum\limits_{i=0}^{i=q-1} (2(q-i) + r) + T(r) & [\text{After $q-1$ repeatitions}]\\
    %         &= 2\sum\limits_{i=0}^{i=q-1} (q-i) + \sum\limits_{i=0}^{i=q-1} r + T(r)\\
    %         &= 2\sum\limits_{i=0}^{i=q-1} (q-i) + \sum\limits_{i=0}^{i=q-1} r & [\text{Since $T(r) = 0$}]\\
    %         &= 2\sum\limits_{i'=1}^{i=q} i' + \sum\limits_{i=0}^{i=q-1} r\\
    %         &= 2\sum\limits_{i'=1}^{i'=q} i' + \sum\limits_{i=0}^{i=q-1} r\\
    %         &= 2\sum\limits_{i'=1}^{i'=q} i' + \sum\limits_{i=0}^{i=q-1} r\\
    %         &= 2(q(q+1))/2 + \sum\limits_{i=0}^{i=q-1} r &[\text{By using $\sum\limits_{i=1}^{i=n} i =(n(n+1))/2$}]\\
    %         &= q(q+1) + rq\\
    %         &= q(q+1+r)
    %     \end{align}
    %     \end{mdframed}
    % \end{enumerate}

    % \end{mdframed}

    \item

    \begin{proof}
    \setcounter{equation}{0}

    For convenience, define $H(q): q(q + r + 1) = T(2q+r)$.

    \bigskip

    I will use simple induction to prove $\forall q \in \mathbb{N}$, $H(q)$.

    \bigskip

    \underline{\textbf{Base Case ($q = 0$):}}

    \bigskip

    Let $q = 0$.

    \bigskip

    Then,

    \begin{align}
        q(q + r + 1) &= 0\\
        &= T(2\cdot 0 + r) & [\text{By def.}]\\
        &= T(2q + r)
    \end{align}

    \bigskip

    Thus, $T(2q+r)$ verifies in this step.

    \bigskip

    \underline{\textbf{Inductive Step:}}

    \bigskip

    Let $q \in \mathbb{N}$. Assume $H(q)$.

    \bigskip

    I need to show $H(q+1)$ follows. That is, $(q+1)\Bigl[ (q+1) + r + 1 \Bigr] = T(2(q+1) + r)$.

    \bigskip

    Starting with $(q+1)\Bigl[ (q+1) + r + 1 \Bigr]$, we have

    \bigskip


    \begin{align}
        (q+1)\Bigl[ (q+1) + r + 1 \Bigr] &= (q+1)(q+1) + (q+1)r + (q+1)\\
        &= q^2 + 2q + 1 + (qr+r) + (q+1)\\
        &= (q^2 + qr + q) + (2q + r + 2)\\
        &= q(q + r + 1) + (2(q+1) + r)\\
        &= T(2q+r) + 2(q+1) + r & [\text{By I.H}]\\
        &= T(2(q+1) + r) & [\text{By def.}]
    \end{align}

    \bigskip

    Thus, $H(q+1)$ follows from $H(q)$ in this step.

    \end{proof}

    \bigskip

    % \begin{mdframed}
    %     \underline{\textbf{Rough Works:}}

    %     \bigskip
    %     \setcounter{equation}{0}

    %     For convenience, define $H(q): q(q + r + 1) = T(2q+r)$.

    %     \bigskip

    %     I will use simple induction to prove $\forall q \in \mathbb{N}$, $H(q)$.

    %     \bigskip

    %     \begin{enumerate}[1.]
    %         \item Base Case ($q = 0$)

    %         \begin{mdframed}
    %         \underline{\textbf{Base Case ($q = 0$):}}

    %         \bigskip

    %         Let $q = 0$.

    %         \bigskip

    %         Then,

    %         \begin{align}
    %             q(q + r + 1) &= 0\\
    %             &= T(2\cdot 0 + r) & [\textbf{By def.}]\\
    %             &= T(2q + r)
    %         \end{align}

    %         \bigskip

    %         Thus, $T(2q+r)$ verifies in this step.
    %         \end{mdframed}

    %         \item Inductive Step

    %         \begin{mdframed}
    %         \underline{\textbf{Inductive Step:}}

    %         \bigskip

    %         Let $q \in \mathbb{N}$. Assume $H(q)$.

    %         \bigskip

    %         I need to show $H(q+1)$ follows. That is, $(q+1)\Bigl[ (q+1) + r + 1 \Bigr] = T(2(q+1) + r)$.

    %         \bigskip

    %         Starting with $(q+1)\Bigl[ (q+1) + r + 1 \Bigr]$, we have

    %         \bigskip


    %         \begin{align}
    %             (q+1)\Bigl[ (q+1) + r + 1 \Bigr] &= (q+1)(q+1) + (q+1)r + (q+1)\\
    %             &= q^2 + 2q + 1 + (qr+r) + (q+1)\\
    %             &= (q^2 + qr + q) + (2q + r + 2)\\
    %             &= q(q + r + 1) + (2(q+1) + r)\\
    %             &= T(2q+r) + 2(q+1) + r & [\text{By I.H}]\\
    %             &= T(2(q+1) + r) & [\text{By def.}]
    %         \end{align}

    %         \bigskip

    %         Thus, $H(q+1)$ follows from $H(q)$ in this step.

    %         \end{mdframed}

    %     \end{enumerate}

    % \end{mdframed}

    \item
    \setcounter{equation}{0}

    \begin{proof}
        Define for convenience

        \begin{align}
            H(n): \bigwedge\limits_{i=0}^{n-1} T(n) - T(i) \geq 0
        \end{align}

        \bigskip

        I will use complete induction to prove that $\forall n \in \mathbb{N}, H(n)$.

        \bigskip

        \underline{\textbf{Inductive Step:}}

        \bigskip

        Let $n \in \mathbb{N}$. Assume $\bigwedge\limits_{i=0}^{n-1}H(i)$. I will
        show $H(n)$ follows.

        \bigskip

        \underline{\textbf{Base Case($n < 2$):}}

        \bigskip

        Assume $n < 2$.

        \bigskip

        Then, all $T(n)$ and $T(n-1)$ are 0 by definition.

        \bigskip

        So, $T(n) - T(n-1) \geq 0$.

        \bigskip

        Thus, $C(n)$ follows in this step.

        \bigskip

        \underline{\textbf{Case ($n \geq 2$):}}

        \bigskip

        Assume $n \geq 2$.

        \bigskip

        Then,

        \begin{align}
        \begin{split}
        T(n) - T(n-1) &= n + T(n-2)\\
        &- \Bigl[(n-1) + T(n-3) \Bigr]
        \end{split} & [\text{By def.}]\\
        &= 1 + T(n-2) - T(n-3)\\
        &\geq 1 & [\text{By I.H, since $0 \leq n-2 < n$}]\\
        &> 0
        \end{align}

        \bigskip

        Thus, $C(n)$ follows from $H(n)$ in this step.
    \end{proof}

    % \begin{mdframed}
    % \underline{\textbf{Rough Works:}}

    % \bigskip
    % \setcounter{equation}{0}

    % Define for convenience

    % \begin{align}
    %     C(n): \bigwedge\limits_{i=0}^{n-1} T(n) - T(i) \geq 0
    % \end{align}

    % \bigskip

    % I will use complete induction to prove that $\forall n \in \mathbb{N}, C(n)$.

    % \begin{enumerate}[1.]
    %     \item Inductive Step

    %     \begin{mdframed}
    %     \underline{\textbf{Inductive Step:}}

    %     \bigskip

    %     Let $n \in \mathbb{N}$. Assume $H(n):\bigwedge\limits_{i=0}^{n-1}C(i)$. I will
    %     show $C(n)$ follows.
    %     \end{mdframed}

    %     \item Base Case($n < 2$)

    %     \bigskip

    %     \begin{mdframed}
    %     \underline{\textbf{Base Case($n < 2$):}}

    %     \bigskip

    %     Assume $n < 2$.

    %     \bigskip

    %     Then, all $T(n)$ and $T(n-1)$ are 0 by definition.

    %     \bigskip

    %     So, $T(n) - T(n-1) \geq 0$.

    %     \bigskip

    %     Thus, $C(n)$ follows in this step.

    %     \end{mdframed}

    %     \item Case ($n \geq 2$)

    %     \begin{mdframed}
    %     \underline{\textbf{Case ($n \geq 2$):}}

    %     \bigskip

    %     Assume $n \geq 2$.

    %    \bigskip

    %    Then,

    %     \begin{align}
    %     T(n) - T(n-1) &= n + T(n-2) - \Bigl[(n-1) + T(n-3) \Bigr] & [\text{By def.}]\\
    %     &= 1 + T(n-2) - T(n-3)\\
    %     &\geq 1 & [\text{By I.H, since $0 \leq n-2 < n$}]\\
    %     &> 0
    %     \end{align}

    %     \bigskip

    %     Thus, $C(n)$ follows from $H(n)$ in this step.

    %     \end{mdframed}
    % \end{enumerate}
    % \end{mdframed}
\end{itemize}

\end{document}