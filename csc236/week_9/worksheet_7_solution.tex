\documentclass[12pt]{article}
\usepackage[margin=2.5cm]{geometry}
\usepackage{enumerate}
\usepackage{amsfonts}
\usepackage{amsmath}
\usepackage{fancyhdr}
\usepackage{amsmath}
\usepackage{amssymb}
\usepackage{amsthm}
\usepackage{mdframed}
\usepackage{graphicx}
\usepackage{subcaption}
\usepackage{adjustbox}
\usepackage{listings}
\usepackage{xcolor}
\usepackage{booktabs}
\usepackage[utf]{kotex}

\definecolor{codegreen}{rgb}{0,0.6,0}
\definecolor{codegray}{rgb}{0.5,0.5,0.5}
\definecolor{codepurple}{rgb}{0.58,0,0.82}
\definecolor{backcolour}{rgb}{0.95,0.95,0.92}

\lstdefinestyle{mystyle}{
    backgroundcolor=\color{backcolour},
    commentstyle=\color{codegreen},
    keywordstyle=\color{magenta},
    numberstyle=\tiny\color{codegray},
    stringstyle=\color{codepurple},
    basicstyle=\ttfamily\footnotesize,
    breakatwhitespace=false,
    breaklines=true,
    captionpos=b,
    keepspaces=true,
    numbers=left,
    numbersep=5pt,
    showspaces=false,
    showstringspaces=false,
    showtabs=false,
    tabsize=1
}

\lstset{style=mystyle}

\begin{document}
\title{CSC236 Worksheet 7 Solution}
\author{Hyungmo Gu}
\maketitle

\section*{Question 1}
\begin{itemize}
    \item

    \begin{mdframed}
    \underline{\textbf{Rough Works:}}

    \bigskip

    \begin{itemize}
        \item Find the value of $k$. And conclude the non-recursive cost of function.

        \bigskip

        First, I need to find the value of $k$.

        \begin{mdframed}
        The definition tells us $k$ is the non-recursive cost.

        \bigskip

        Since the non-recursive part of call occurs when $len(s) < 2$ and it
        returns the input as output, it has cost of 1.
        \end{mdframed}

        \item Find the value of $b$.

        \bigskip

        Second, we need to find the value of $b$.

        \bigskip

        \begin{mdframed}
        The definition tells us $b$ is the number of almost-equal parts the input
        is divided into.

        \bigskip

        Since the input $s$ is divided into three roughly equal parts, we can conclude $b = 3$.
        \end{mdframed}

        \item Find the value of $a$.

        \bigskip

        Third, we need to find the value of $a$.

        \bigskip

        \begin{mdframed}
        The definition tells us $a$ is the number of recursive calls.

        \bigskip

        Since the recursive calls in this problem are $r(s_1)$, $r(s_2)$ and $r(s_3)$,
        there are three of them, so $a = 3$.
        \end{mdframed}

        \item Find the value of $f$.

        \bigskip

        Fourth, we need to find the value of $f$.

        \bigskip

        \begin{mdframed}

        \end{mdframed}

        \item Use master's theorem to evaluate asymptotic time complexity of function r.
        \item Compare its time complexity to using loop
    \end{itemize}
    \end{mdframed}

\end{itemize}

\bigskip

\underline{\textbf{Notes:}}

\bigskip

\begin{itemize}
    \item

    \underline{\textbf{Divide and Conquer:}} Partitions problem into $b$ roughly
    equal subproblems, solve, and recombine:

    \bigskip

    \begin{align}
        T(n) &= \begin{cases}
        k & \text{if $n \leq B$}\\
        a_1T(\lceil n/b \rceil) + a_2T(\lfloor n/b \rfloor) + f(n) & \text{if $n > B$}
        \end{cases}
    \end{align}

    \begin{align}
        T(n) &= \begin{cases}
        k & \text{if $n \leq B$}\\
        aT(n/b) + f(n) & \text{if $n > B$}
        \end{cases}
    \end{align}

    \bigskip

    where $b,k > 0$, $a_1,a_2 \geq 0$, and $a = a_1 + a_2 > 0$. $f(n)$ is the
    cost of slptting and recombining.

    \bigskip

    \begin{mdframed}
    \underline{\textbf{Note:}}

    \bigskip

    $k:$ non-recursive cost, when $n < b$

    $b:$ number of almost-equal parts we divide problem into

    $a_1:$ number of recursive calls to ceiling

    $a_2:$ number of recursive calls to floor

    $a:$ number of recursive calls

    $f:$ cost of splittig and later recombining (should be $n^d$ for master theorem)
    \end{mdframed}

    \item

    \underline{\textbf{Divide and Conquer Master Theorem:}}

    \bigskip

    If $f \in \Theta(n^d)$, then

    \begin{align}
        T(n) &\in \begin{cases}
        \Theta(n^d) & \text{if $a \leq b^d$}\\
        \Theta(n^d\log_b n) & \text{if $a = b^d$}\\
        \Theta(n^{\log_b a}) & \text{if $a > b^d$}
        \end{cases}
    \end{align}

    \item The master theorem is for master method.
    \item The master method provides a cookbook method for solving recurrences
    of the form

    \begin{align}
        T(n) &= aT(n/b) + f(n)
    \end{align}

    \bigskip

    where $a \geq 1$ and $b > 1$.
\end{itemize}

\end{document}