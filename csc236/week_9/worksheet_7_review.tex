\documentclass[12pt]{article}
\usepackage[margin=2.5cm]{geometry}
\usepackage{enumerate}
\usepackage{amsfonts}
\usepackage{amsmath}
\usepackage{fancyhdr}
\usepackage{amsmath}
\usepackage{amssymb}
\usepackage{amsthm}
\usepackage{mdframed}
\usepackage{graphicx}
\usepackage{subcaption}
\usepackage{adjustbox}
\usepackage{listings}
\usepackage{xcolor}
\usepackage{booktabs}
\usepackage[utf]{kotex}

\definecolor{codegreen}{rgb}{0,0.6,0}
\definecolor{codegray}{rgb}{0.5,0.5,0.5}
\definecolor{codepurple}{rgb}{0.58,0,0.82}
\definecolor{backcolour}{rgb}{0.95,0.95,0.92}

\lstdefinestyle{mystyle}{
    backgroundcolor=\color{backcolour},
    commentstyle=\color{codegreen},
    keywordstyle=\color{magenta},
    numberstyle=\tiny\color{codegray},
    stringstyle=\color{codepurple},
    basicstyle=\ttfamily\footnotesize,
    breakatwhitespace=false,
    breaklines=true,
    captionpos=b,
    keepspaces=true,
    numbers=left,
    numbersep=5pt,
    showspaces=false,
    showstringspaces=false,
    showtabs=false,
    tabsize=1
}

\lstset{style=mystyle}

\begin{document}
\title{CSC236 Worksheet 7 Review}
\author{Hyungmo Gu}
\maketitle

\section*{Question 1}
\begin{itemize}
    \item

    Let $n$ be the length of input $s$. Then $n \in \mathbb{N}$.

    \bigskip

    The algorithm divides problem into 3 (roughly) equal parts, i.e. $s_1$, $s_2$,
    $s_3$, calls the function recursively 3 times on those parts, i.e. $r(s_1)$,
    $r(s_2)$, $r(s_3)$, divides the problem in constant time, and combines the
    result in time proportional to $\text{len($s_3$)} + \text{len($s_2$)} + \text{len($s_1$)} = n$.

    \bigskip

    Thus, $b = 3$, $a = 3$, $f = n$.

    \bigskip

    Since $a = b = b^1$, the master's theorem tells us the time complexity of
    function $r$ is $\Theta(n\log_3 n)$.

    \bigskip

    The time complexity of copying the string elements in reverse order using loop
    is $\Theta(n)$.

    \bigskip

    So, in comparison to the divide and conquer method, this is faster.
\end{itemize}

\end{document}