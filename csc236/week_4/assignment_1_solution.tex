\documentclass[12pt]{article}
\usepackage[margin=2.5cm]{geometry}
\usepackage{enumerate}
\usepackage{amsfonts}
\usepackage{amsmath}
\usepackage{fancyhdr}
\usepackage{amsmath}
\usepackage{amssymb}
\usepackage{amsthm}
\usepackage{mdframed}
\usepackage{graphicx}
\usepackage{subcaption}
\usepackage{adjustbox}
\usepackage{listings}
\usepackage{xcolor}
\usepackage{booktabs}
\usepackage[utf]{kotex}

\definecolor{codegreen}{rgb}{0,0.6,0}
\definecolor{codegray}{rgb}{0.5,0.5,0.5}
\definecolor{codepurple}{rgb}{0.58,0,0.82}
\definecolor{backcolour}{rgb}{0.95,0.95,0.92}

\lstdefinestyle{mystyle}{
    backgroundcolor=\color{backcolour},
    commentstyle=\color{codegreen},
    keywordstyle=\color{magenta},
    numberstyle=\tiny\color{codegray},
    stringstyle=\color{codepurple},
    basicstyle=\ttfamily\footnotesize,
    breakatwhitespace=false,
    breaklines=true,
    captionpos=b,
    keepspaces=true,
    numbers=left,
    numbersep=5pt,
    showspaces=false,
    showstringspaces=false,
    showtabs=false,
    tabsize=1
}

\lstset{style=mystyle}

\begin{document}
\title{CSC236 Assignment 1}
\author{Hyungmo Gu}
\maketitle

\section*{Question 1}
\begin{enumerate}[a.]
    \item

    Yes. We can prove $P(235)$ follows from $P(234)$.

    \bigskip

    \begin{proof}
    Let $b$ be the bipartite graph with 235 vertices where
    117 vertices are in one partition and 118 vertices in
    the other partition (Note this is the configuration where maximum number of edges form).

    \bigskip

    The bipartite graph with 117 vertices on both sides of partition has
    $\frac{234^2}{4}$ edges, and the assumption tells us this is the maximum number of edges the
    bipartite graph could form.

    \bigskip

    Since we know $b$ has 117 more edges than the bipartite graph with 117 vertices on
    both sides, using these facts, we can conclude the upper bound number of edges for the bipartite
    graph with 235 vertices is

    \begin{align}
        \frac{234^2}{4} + 117 &= \frac{234^2}{4} + \frac{4 \cdot 117}{4}\\
        &= \frac{234^2 + 2 \cdot 234}{4}\\
        &\leq \frac{234^2 + 2 \cdot 234 + 1}{4}\\
        &= \frac{(234+1)^2}{4}\\
        &= \frac{(235)^2}{4}
    \end{align}

    \end{proof}

    \bigskip

    \begin{mdframed}
        \underline{\textbf{Attempt \#2:}}

        \bigskip

        Assume $P(234)$. That is, every bipartite graph on 234 vertices has no more
        than $\frac{234^2}{4}$ edges.

        \bigskip

        We need to prove $P(235)$ follows. That is, every bipartite graph on 235
        vertices has no more than $\frac{235^2}{4}$ edges.

        \bigskip

        Let $b$ be the bipartite graph with 235 vertices. Let $b'$ be the bipartite
        graph with a vertex removed from the larger of two partitions in $b$ along with its edges.

        \bigskip

        Since we know the maximum number of edges occur in $b'$ when there are
        117 vertices in both sides of the partitions, and since we know the edges
        of removed vertex forms edges with partition with bigger number of vertices,
        we can conclude the removed vertex forms at most 117 edges.

        \bigskip

        Since the assumption tells us $b'$ has at most $\frac{234^2}{4}$ edges,
        we can conclude the upper bound number of edges for the bipartite
        graph with 235 vertices is


        \begin{align}
            \frac{234^2}{4} + 117 &= \frac{234^2}{4} + \frac{4 \cdot 117}{4}\\
            &= \frac{234^2 + 2 \cdot 234}{4}\\
            &\leq \frac{234^2 + 2 \cdot 234 + 1}{4}\\
            &= \frac{(234+1)^2}{4}\\
            &= \frac{(235)^2}{4}
        \end{align}

        \bigskip

        So $P(235)$ follows.

    \end{mdframed}

    \bigskip

    \underline{\textbf{Notes:}}

    \bigskip

    \begin{itemize}
        \item I have a stuck feeling as to how I can formulate this kind of proof.
        \item 5+ hours spent on this problem
        \item I feel I wronged the proof by using existential quantifier
        \item Noticed professor generalized his statement instead of using bipartite
        with $x$ number of vertices in one partition and $y$ vertices in other partition
        example

        \begin{center}
        \includegraphics[width=0.8\linewidth]{images/assignment_1_q1a_note.png}
        \end{center}
    \end{itemize}

    % \bigskip

    % \begin{mdframed}
    %     \underline{\textbf{Rough Work:}}

    %     Assume $P(234)$. That is, every bipartite graph on 234 vertices has no more
    %     than $\frac{234^2}{4}$ edges.

    %     \bigskip

    %     We need to prove $P(235)$ follows. That is, every bipartite graph on 235
    %     vertices has no more than $\frac{235^2}{4}$ edges.

    %     \begin{enumerate}[1.]
    %         \item Find the configuration where bipartite graph on 235 vertices form most number of edges in
    %         terms of bipartite graph on 234 vertices.

    %         \begin{mdframed}
    %         Let $b$ be the bipartite graph with 235 vertices where
    %         117 vertices are in one partition and 118 vertices in
    %         the other partition (Note this is the configuration where maximum number of edges form).

    %         \end{mdframed}

    %         \item Show that $\frac{235^2}{4}$ is the most number of edges bipartite graph on 235
    %         vertices could form.

    %         \begin{mdframed}
    %         The bipartite graph with 117 vertices on both sides of partition has
    %         $\frac{234^2}{4}$ edges, and the assumption tells us this is the maximum number of edges the
    %         bipartite graph could form.

    %         \bigskip

    %         Since we know $b$ has 117 more edges than the bipartite graph with 117 vertices on
    %         both sides, using these facts, we can conclude the upper bound number of edges for the bipartite
    %         graph with 235 vertices is

    %         \begin{align}
    %             \frac{234^2}{4} + 117 &= \frac{234^2}{4} + \frac{4 \cdot 117}{4}\\
    %             &= \frac{234^2 + 2 \cdot 234}{4}\\
    %             &\leq \frac{234^2 + 2 \cdot 234 + 1}{4}\\
    %             &= \frac{(234+1)^2}{4}\\
    %             &= \frac{(235)^2}{4}
    %         \end{align}

    %         \bigskip
    %         \end{mdframed}
    %     \end{enumerate}

    % \end{mdframed}

    \item

    No. $P(236)$ doesn't follow from $P(235)$.

    \begin{mdframed}
        \underline{\textbf{Rough Work:}}

        \bigskip

        Let $b$ be an arbitrary bipartite graph with 236 vertices. Let $b'$ be an
        arbitrary bipartite graph with 235 vertices. Assume $P(235)$. That is,
        $b'$ has no more than $\frac{235^2}{4}$ edges.

        \bigskip

        We need to show $P(236)$ doesn't follow. That is, there is $b$ with more
        than $\frac{236^2}{4}$ edges.

        \bigskip

        Let \underline{\hspace{3cm}}.

        \begin{enumerate}[1.]
            \item Show this choice of $b$ has more than $\frac{236^2}{4}$ edges.
        \end{enumerate}

    \end{mdframed}
\end{enumerate}

\section*{Question 2}

\section*{Question 3}

\section*{Question 4}

\end{document}