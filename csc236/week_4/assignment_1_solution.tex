\documentclass[12pt]{article}
\usepackage[margin=2.5cm]{geometry}
\usepackage{enumerate}
\usepackage{amsfonts}
\usepackage{amsmath}
\usepackage{fancyhdr}
\usepackage{amsmath}
\usepackage{amssymb}
\usepackage{amsthm}
\usepackage{mdframed}
\usepackage{graphicx}
\usepackage{subcaption}
\usepackage{adjustbox}
\usepackage{listings}
\usepackage{xcolor}
\usepackage{booktabs}
\usepackage[utf]{kotex}

\definecolor{codegreen}{rgb}{0,0.6,0}
\definecolor{codegray}{rgb}{0.5,0.5,0.5}
\definecolor{codepurple}{rgb}{0.58,0,0.82}
\definecolor{backcolour}{rgb}{0.95,0.95,0.92}

\lstdefinestyle{mystyle}{
    backgroundcolor=\color{backcolour},
    commentstyle=\color{codegreen},
    keywordstyle=\color{magenta},
    numberstyle=\tiny\color{codegray},
    stringstyle=\color{codepurple},
    basicstyle=\ttfamily\footnotesize,
    breakatwhitespace=false,
    breaklines=true,
    captionpos=b,
    keepspaces=true,
    numbers=left,
    numbersep=5pt,
    showspaces=false,
    showstringspaces=false,
    showtabs=false,
    tabsize=1
}

\lstset{style=mystyle}

\begin{document}
\title{CSC236 Assignment 1}
\author{Hyungmo Gu}
\maketitle

\section*{Question 1}
\begin{enumerate}[a.]
    \item

    Yes. We can prove $P(235)$ follows from $P(234)$.

    \bigskip

    \begin{proof}
    Let $b$ be the bipartite graph with 235 vertices where
    117 vertices are in one partition and 118 vertices in
    the other partition (Note this is the configuration where maximum number of edges form).

    \bigskip

    The bipartite graph with 117 vertices on both sides of partition has
    $\frac{234^2}{4}$ edges, and the assumption tells us this is the maximum number of edges the
    bipartite graph could form.

    \bigskip

    Since we know $b$ has 117 more edges than the bipartite graph with 117 vertices on
    both sides, using these facts, we can conclude the upper bound number of edges for the bipartite
    graph with 235 vertices is

    \begin{align}
        \frac{234^2}{4} + 117 &= \frac{234^2}{4} + \frac{4 \cdot 117}{4}\\
        &= \frac{234^2 + 2 \cdot 234}{4}\\
        &\leq \frac{234^2 + 2 \cdot 234 + 1}{4}\\
        &= \frac{(234+1)^2}{4}\\
        &= \frac{(235)^2}{4}
    \end{align}

    \end{proof}

    % \bigskip

    % \begin{mdframed}
    %     \underline{\textbf{Rough Work:}}

    %     Assume $P(234)$. That is, every bipartite graph on 234 vertices has no more
    %     than $\frac{234^2}{4}$ edges.

    %     \bigskip

    %     We need to prove $P(235)$ follows. That is, every bipartite graph on 235
    %     vertices has no more than $\frac{235^2}{4}$ edges.

    %     \begin{enumerate}[1.]
    %         \item Find the configuration where bipartite graph on 235 vertices form most number of edges in
    %         terms of bipartite graph on 234 vertices.

    %         \begin{mdframed}
    %         Let $b$ be the bipartite graph with 235 vertices where
    %         117 vertices are in one partition and 118 vertices in
    %         the other partition (Note this is the configuration where maximum number of edges form).

    %         \end{mdframed}

    %         \item Show that $\frac{235^2}{4}$ is the most number of edges bipartite graph on 235
    %         vertices could form.

    %         \begin{mdframed}
    %         The bipartite graph with 117 vertices on both sides of partition has
    %         $\frac{234^2}{4}$ edges, and the assumption tells us this is the maximum number of edges the
    %         bipartite graph could form.

    %         \bigskip

    %         Since we know $b$ has 117 more edges than the bipartite graph with 117 vertices on
    %         both sides, using these facts, we can conclude the upper bound number of edges for the bipartite
    %         graph with 235 vertices is

    %         \begin{align}
    %             \frac{234^2}{4} + 117 &= \frac{234^2}{4} + \frac{4 \cdot 117}{4}\\
    %             &= \frac{234^2 + 2 \cdot 234}{4}\\
    %             &\leq \frac{234^2 + 2 \cdot 234 + 1}{4}\\
    %             &= \frac{(234+1)^2}{4}\\
    %             &= \frac{(235)^2}{4}
    %         \end{align}

    %         \bigskip
    %         \end{mdframed}
    %     \end{enumerate}

    % \end{mdframed}
\end{enumerate}

\section*{Question 2}

\section*{Question 3}

\section*{Question 4}

\end{document}