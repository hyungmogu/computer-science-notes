\documentclass[12pt]{article}
\usepackage[margin=2.5cm]{geometry}
\usepackage{enumerate}
\usepackage{amsfonts}
\usepackage{amsmath}
\usepackage{fancyhdr}
\usepackage{amsmath}
\usepackage{amssymb}
\usepackage{amsthm}
\usepackage{mdframed}
\usepackage{graphicx}
\usepackage{subcaption}
\usepackage{adjustbox}
\usepackage{listings}
\usepackage{xcolor}
\usepackage{booktabs}
\usepackage[utf]{kotex}

\definecolor{codegreen}{rgb}{0,0.6,0}
\definecolor{codegray}{rgb}{0.5,0.5,0.5}
\definecolor{codepurple}{rgb}{0.58,0,0.82}
\definecolor{backcolour}{rgb}{0.95,0.95,0.92}

\lstdefinestyle{mystyle}{
    backgroundcolor=\color{backcolour},
    commentstyle=\color{codegreen},
    keywordstyle=\color{magenta},
    numberstyle=\tiny\color{codegray},
    stringstyle=\color{codepurple},
    basicstyle=\ttfamily\footnotesize,
    breakatwhitespace=false,
    breaklines=true,
    captionpos=b,
    keepspaces=true,
    numbers=left,
    numbersep=5pt,
    showspaces=false,
    showstringspaces=false,
    showtabs=false,
    tabsize=1
}

\lstset{style=mystyle}

\begin{document}
\title{CSC236 Worksheet 3}
\author{Hyungmo Gu}
\maketitle

\section*{Question 1}

\bigskip

\begin{mdframed}
    \underline{\textbf{Rough Work:}}

    \bigskip

    \textbf{Predicate Logic:} $\forall A \subseteq \mathbb{N},\:A \neq \emptyset \Rightarrow
    (\exists a \in A, \: \forall x \in A,\: a \leq x)$

    \bigskip

    Given the statement to prove

    \bigskip

    \begin{center}
        $P(x,y,z):$ There are no positive integers $x,y,z$ such that $x^3 + 3y^3 = 9z^3$
    \end{center}

    \bigskip

    I will prove $P(x,y,z)$ using proof by contradiction.

    \bigskip

    Assume $\exists x,y,z \in \mathbb{N}^{+}$, $x^3 + 3y^3 = 9z^3$.

    \begin{enumerate}[1.]

        \item State that there is smallest element $x_0 \in X$ with $y_0,z_0 \in \mathbb{N}^+$
        satisfying $x^3 + 3y^3 = 9z^3$, using well-ordering principle

        \bigskip

        First, we need to show there is smallest element $x_0 \in X$ with $y_0,z_0 \in \mathbb{N}^+$
        satisfying $x^3 + 3y^3 = 9z^3$, using well-ordering principle.

        \bigskip

        \begin{mdframed}
        First, we need to show there is smallest element $x_0 \in X$ with $y_0,z_0 \in \mathbb{N}^+$
        satisfying $x^3 + 3y^3 = 9z^3$, using well-ordering principle.

        \bigskip

        The header tells us there are elements $x,y,z \in \mathbb{N}^+$, satisfying
        $x^3 + 3y^3 = 9z^3$.

        \bigskip

        Then, we can write the set $X = \{x \mid x \in \mathbb{N}^+,\:\exists y,z \in
        \mathbb{N}^+,\:x^3 + 3y^3 = 9z^3\}$ is not empty.

        \bigskip

        Then, using principle of well-ordering, we can write that there is smallest
        positive natural number $x_0 \in X$ along with $y_0,z_0 \in \mathbb{N}^+$ satisfying
        $x^3 + 3y^3 = 9z^3$.
        \end{mdframed}

        \item Show that $x_1^3 = 9z_1^3 - 3y_1^3$ is satisfied, given $x_0 > x_1$

        \bigskip

        Second, we need to show that $x_1^3 = 9z_1^3 - 3y_1^3$ is satisfied, given
        $x_0 > x_1$.

        \bigskip

        We will do so in parts.

        \bigskip

        \begin{itemize}
            \item First, show that $x_0 = 3 \cdot x_1$, using $x_0^3 + 3y_0^3 = 9z_0^3$ and the
            fact if a prime number $p$ divides a perfect cube $n^3$, then $p$ also divides $n$.

            \begin{mdframed}
            \underline{\textbf{Part 1 (Showing $x_0 = 3 \cdot x_1$):}}

            \bigskip

            The assumption tells us

            \begin{align}
                x_0^3 + 3y_0^3 &= 9z_0^3\\
                x_0^3 &= 9z_0^3 - 3y_0^3
            \end{align}

            \bigskip

            Since $3 \mid 9z_0^3 - 3y_0^3$, we can write that $3 \mid x_0^3$.

            \bigskip

            Then, since 3 is a prime number, by using the hint provided in question 3
            of assignment 1, we can write there is $x_1 \in \mathbb{Z}$, $x_0 = 3 \cdot x_1$.

            \bigskip

            Then, because we know $x_0, 3 \in \mathbb{N}^+$, we can conclude
            $x_1 \in \mathbb{N}^+$.

            \end{mdframed}

            \item Second, show that $y_0 = 3 \cdot y_1$, using $x_0^3 = 3^3 x_1^3 = 9z_0^3 - 3y_0^3$

            \begin{mdframed}
            \underline{\textbf{Part 2 (Showing $y_0 = 3 \cdot y_1$):}}

            \bigskip

            The assumption tells us

            \begin{align}
                x_0^3 + 3y_0^3 &= 9z_0^3\\
                3y_0^3 &= 9z_0^3 - x_0^3
            \end{align}

            \bigskip

            Then, using the fact from part 1 that $x_0 = 3 \cdot x_1$, we can calculate

            \begin{align}
                3y_0^3 &= 9z_0^3 - 3^3x_1^3\\
                y_0^3 &= 3z_0^3 - 3^2x_1^3
            \end{align}

            \bigskip

            Since $3 \mid 3z_0^3 - 3^2x_1^3$, we can write that $3 \mid y_0^3$.

            \bigskip

            Then, since 3 is a prime number, by using the hint provided in question 3
            of assignment 1, we can write there is $y_1 \in \mathbb{Z}$, $y_0 = 3 \cdot y_1$.

            \bigskip

            Then, because we know $y_0, 3 \in \mathbb{N}^+$, we can conclude
            $y_1 \in \mathbb{N}^+$.

            \end{mdframed}


            \item Third, show that $z_0 = 3 \cdot z_1$, using $x_0^3 = 3^3 x_1^3 = 9z_0^3 - 3y_0^3 =  9z_0^3 - 3^4y_1^3$
            \item Finally, show $x_1^3 = 9z_1^3 - 3y_1^3$
        \end{itemize}

        \bigskip

        \begin{mdframed}
        Second, we need to show that $x_1^3 = 9z_1^3 - 3y_1^3$ is satisfied, given
        $x_0 > x_1$.

        \bigskip

        We will do so in parts.

        \bigskip


        \end{mdframed}

        \item Conclude that this contracts the original claim that $x_0$ is the smallest
        non-negative value satisfying $x^3 + 3y^3 = 9z^3$.

        \begin{mdframed}

        \end{mdframed}
    \end{enumerate}

\end{mdframed}

\bigskip

\underline{\textbf{Notes:}}

\begin{itemize}
    \item \underline{\textbf{Proof By Contradiction:}} $\neg P \Rightarrow \neg Q \land Q$ (Assuming
    we are proving $P \Rightarrow Q$)
    \item \underline{\textbf{Principle of Well-Ordering:}} Any nonempty subset $A$
    of $\mathbb{N}$ contains a minimum element; i.e. for any $A \subseteq \mathbb{N}$
    such that $A \neq \emptyset$, there is some $a \in A$ such that for all $a' \in A$, $a \leq a'$.

    \item examples of well-ordered sets
    \begin{enumerate}[1.]
        \item $\mathbb{N} \cup \{0\}$
        \item $\mathbb{N} \cup \{1,2\}$
        \item $\{n \in \mathbb{N}: n > 5\}$
    \end{enumerate}
    \item examples of non-well-ordered sets
    \begin{enumerate}[1.]
        \item $\mathbb{R}$ and the open interval $(0,2)$
        \item $\mathbb{Z}$
    \end{enumerate}
\end{itemize}

\section*{Question 2}

\section*{Question 3}

\end{document}