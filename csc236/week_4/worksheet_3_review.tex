\documentclass[12pt]{article}
\usepackage[margin=2.5cm]{geometry}
\usepackage{enumerate}
\usepackage{amsfonts}
\usepackage{amsmath}
\usepackage{fancyhdr}
\usepackage{amsmath}
\usepackage{amssymb}
\usepackage{amsthm}
\usepackage{mdframed}
\usepackage{graphicx}
\usepackage{subcaption}
\usepackage{adjustbox}
\usepackage{listings}
\usepackage{xcolor}
\usepackage{booktabs}
\usepackage[utf]{kotex}

\definecolor{codegreen}{rgb}{0,0.6,0}
\definecolor{codegray}{rgb}{0.5,0.5,0.5}
\definecolor{codepurple}{rgb}{0.58,0,0.82}
\definecolor{backcolour}{rgb}{0.95,0.95,0.92}

\lstdefinestyle{mystyle}{
    backgroundcolor=\color{backcolour},
    commentstyle=\color{codegreen},
    keywordstyle=\color{magenta},
    numberstyle=\tiny\color{codegray},
    stringstyle=\color{codepurple},
    basicstyle=\ttfamily\footnotesize,
    breakatwhitespace=false,
    breaklines=true,
    captionpos=b,
    keepspaces=true,
    numbers=left,
    numbersep=5pt,
    showspaces=false,
    showstringspaces=false,
    showtabs=false,
    tabsize=1
}

\lstset{style=mystyle}

\begin{document}
\title{CSC236 Worksheet 3 Review}
\author{Hyungmo Gu}
\maketitle

\section*{Question 2}
\begin{itemize}
    \item

    \begin{proof}
    Define $P(e):S_1(e) = 3(s_2(e) - 1)$

    \bigskip

    I will use structural induction to prove $\forall e \in \varepsilon,\:P(e)$.

    \bigskip

    \underline{\textbf{Basis:}}

    \bigskip

    Let $\{x,y,z\} \in \varepsilon$.

    \bigskip

    In this step, there are following cases to consider: $e = x$, $e = y$, and $e = z$.

    \bigskip

    In each of the cases, we have $s_1(e) = 0$ and $s_2(e) = 1$.

    \bigskip

    Thus,

    \begin{align}
        s_1(e) = 0 &= 3(0)\\
        &= 3(1-1)\\
        &= 3(s_2(e) - 1)
    \end{align}

    \bigskip

    So, , $P(e)$ holds.

    \bigskip

    \underline{\textbf{Inductive Step:}}

    \bigskip

    Let $e_1, e_2 \in \varepsilon$. Assume $H(e):$ $P(e_1)$ and $P(e_2)$.
    That is, $s_1(e_1) = 3(s_2(e_1) - 1)$ and $s_2(e_2) = 3(s_2(e_2) - 1)$.

    \bigskip

    I need to prove all possible combinations of $e_1$ and $e_2$ satisfy
    the statement. That is $P((e_1 + e_2))$ and $P((e_1 - e_2))$.

    \bigskip

    In each of the combination, the total number of variables of $e$
    is the sum of the number of variables in $e_1$ and $e_2$, and the
    total number of parenthesis and operators in $e$ is the sum of
    operators and parenthesis in $e_1$ and $e_2$ plus 3.

    \bigskip

    Then, using these facts, we have

    \bigskip

    \begin{align}
        s_2(e) &= s_2(e_1) + s_2(e_2)\\
        s_1(e) &= s_1(e_1) + s_1(e_2) + 3
    \end{align}

    \bigskip

    Thus, we can calculate

    \begin{align}
        s_1(e) &= s_1(e_1) + s_1(e_2) + 3 & [\text{By 5}]\\
        &= 3(s_2(e_1) - 1) + 3(s_2(e_2) - 1) + 3 & [\text{By I.H}]\\
        &= 3(s_2(e_1) + s_2(e_2)) - 6 + 3 & [\text{By I.H}]\\
        &= 3s_2(e) - 3 & [\text{By 4}]\\
        &= 3(s_2(e) - 1)
    \end{align}

    \end{proof}

    \bigskip

    \begin{mdframed}
        \underline{\textbf{Correct Solution:}}

        \bigskip

        Define $P(e):S_1(e) = 3(s_2(e) - 1)$

        \bigskip

        I will use structural induction to prove $\forall e \in \varepsilon,\:P(e)$.

        \bigskip

        \underline{\textbf{Basis:}}

        \bigskip

        Let \color{red}$e \in \{x,y,z\}$\color{black}.

        \bigskip

        Then, $s_1(e) = 0$ and $s_2(e) = 1$.

        \bigskip

        Thus, \color{red}we have,\color{black}

        \begin{align}
            s_1(e) = 0 &= 3(0)\\
            &= 3(1-1)\\
            &= 3(s_2(e) - 1)
        \end{align}

        \bigskip

        So, $P(e)$ \color{red}follows in this step\color{black}.

        \bigskip

        \underline{\textbf{Inductive Step:}}

        \bigskip

        Let $e_1, e_2 \in \varepsilon$. Assume $H(e):$ $P(e_1)$ and $P(e_2)$.
        That is, $s_1(e_1) = 3(s_2(e_1) - 1)$ and $s_2(e_2) = 3(s_2(e_2) - 1)$.
        \bigskip

        \color{red}I will prove $P(e)$ holds for any $e$ that can be constructed
        from $e_1$ and $e_2$. There are two cases:  $e = e_1 + e_2$ and $e = e_1 - e_2$.

        \bigskip

        In each cases, we have\color{black}

        \bigskip

        \begin{align}
            s_2(e) &= s_2(e_1) + s_2(e_2)\\
            s_1(e) &= s_1(e_1) + s_1(e_2) + 3
        \end{align}

        \bigskip

        Thus, we can calculate

        \begin{align}
            s_1(e) &= s_1(e_1) + s_1(e_2) + 3 & [\text{By 5}]\\
            &= 3(s_2(e_1) - 1) + 3(s_2(e_2) - 1) + 3 & [\text{By I.H}]\\
            &= 3(s_2(e_1) + s_2(e_2)) - 6 + 3 & [\text{By I.H}]\\
            &= 3s_2(e) - 3 & [\text{By 4}]\\
            &= 3(s_2(e) - 1)
        \end{align}

        \bigskip

        \color{red}So, $P(e)$ follows from $H(e)$ in this step.\color{black}
    \end{mdframed}
    % \bigskip

    % \begin{mdframed}
    %     \underline{\textbf{Rough Work:}}

    %     \bigskip

    %     Define $P(e):S_1(e) = 3(s_2(e) - 1)$

    %     \bigskip

    %     I will use structural induction to prove $\forall e \in \varepsilon,\:P(e)$.

    %     \bigskip

    %     \begin{enumerate}[1.]
    %         \item Basis

    %         \begin{mdframed}
    %         \underline{\textbf{Basis:}}

    %         \bigskip

    %         Let $\{x,y,z\} \in \varepsilon$.

    %         \bigskip

    %         In this step, there are following cases to consider: $e = x$, $e = y$, and $e = z$.

    %         \bigskip

    %         In each of the cases, we have $s_1(e) = 0$ and $s_2(e) = 1$.

    %         \bigskip

    %         Thus,

    %         \begin{align}
    %             s_1(e) = 0 &= 3(0)\\
    %             &= 3(1-1)\\
    %             &= 3(s_2(e) - 1)
    %         \end{align}

    %         \bigskip

    %         So, , $P(e)$ holds.
    %         \end{mdframed}
    %         \item Inductive Step

    %         \begin{mdframed}

    %             \underline{\textbf{Inductive Step:}}

    %             \bigskip

    %             Let $e_1, e_2 \in \varepsilon$. Assume $H(e):$ $P(e_1)$ and $P(e_2)$.
    %             That is, $s_1(e_1) = 3(s_2(e_1) - 1)$ and $s_2(e_2) = 3(s_2(e_2) - 1)$.

    %             \bigskip

    %             I need to prove all possible combinations of $e_1$ and $e_2$ satisfy
    %             the statement. That is $P((e_1 + e_2))$ and $P((e_1 - e_2))$.

    %             \bigskip

    %             In each of the combination, the total number of variables of $e$
    %             is the sum of the number of variables in $e_1$ and $e_2$, and the
    %             total number of parenthesis and operators in $e$ is the sum of
    %             operators and parenthesis in $e_1$ and $e_2$ plus 3.

    %             \bigskip

    %             Then, using these facts, we have

    %             \bigskip

    %             \begin{align}
    %                 s_2(e) &= s_2(e_1) + s_2(e_2)\\
    %                 s_1(e) &= s_1(e_1) + s_1(e_2) + 3
    %             \end{align}

    %             \bigskip

    %             Thus, we can calculate

    %             \begin{align}
    %                 s_1(e) &= s_1(e_1) + s_1(e_2) + 3 & [\text{By 5}]\\
    %                 &= 3(s_2(e_1) - 1) + 3(s_2(e_2) - 1) + 3 & [\text{By I.H}]\\
    %                 &= 3(s_2(e_1) + s_2(e_2)) - 6 + 3 & [\text{By I.H}]\\
    %                 &= 3s_2(e) - 3 & [\text{By 4}]\\
    %                 &= 3(s_2(e) - 1)
    %             \end{align}

    %         \end{mdframed}
    %     \end{enumerate}

    % \end{mdframed}
\end{itemize}

\end{document}