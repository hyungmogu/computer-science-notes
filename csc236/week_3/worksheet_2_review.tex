\documentclass[12pt]{article}
\usepackage[margin=2.5cm]{geometry}
\usepackage{enumerate}
\usepackage{amsfonts}
\usepackage{amsmath}
\usepackage{fancyhdr}
\usepackage{amsmath}
\usepackage{amssymb}
\usepackage{amsthm}
\usepackage{mdframed}
\usepackage{graphicx}
\usepackage{subcaption}
\usepackage{adjustbox}
\usepackage{listings}
\usepackage{xcolor}
\usepackage{booktabs}
\usepackage[utf]{kotex}

\definecolor{codegreen}{rgb}{0,0.6,0}
\definecolor{codegray}{rgb}{0.5,0.5,0.5}
\definecolor{codepurple}{rgb}{0.58,0,0.82}
\definecolor{backcolour}{rgb}{0.95,0.95,0.92}

\lstdefinestyle{mystyle}{
    backgroundcolor=\color{backcolour},
    commentstyle=\color{codegreen},
    keywordstyle=\color{magenta},
    numberstyle=\tiny\color{codegray},
    stringstyle=\color{codepurple},
    basicstyle=\ttfamily\footnotesize,
    breakatwhitespace=false,
    breaklines=true,
    captionpos=b,
    keepspaces=true,
    numbers=left,
    numbersep=5pt,
    showspaces=false,
    showstringspaces=false,
    showtabs=false,
    tabsize=1
}

\lstset{style=mystyle}

\begin{document}
\title{CSC236 Worksheet 2 Review}
\author{Hyungmo Gu}
\maketitle

\section*{Question 3}

\begin{itemize}
    \item

    \begin{proof}
        For convenience, define $P(n): f(n) \leq 3^n$. I will use complete induction to
        prove that $\forall n \in \mathbb{N}$, $P(n)$.

        \bigskip

        \underline{\textbf{Inductive Step:}}

        \bigskip

        Let $n \in \mathbb{N}$. Assume $H(n): \bigwedge_{i=0}^{n-1} P(i)$. I will
        show $P(n)$ follows. That is $f(n) \leq 3^n$.

        \bigskip

        \underline{\textbf{Base Case ($n = 0$):}}

        \bigskip

        Let $n = 0$.

        \bigskip

        Then,

        \begin{align}
            f(n) &= 1 & [\text{By def.}]\\
            &= 3^0\\
            &\leq 3^0\\
            &= 3^n
        \end{align}

        \bigskip

        Thus, $P(n)$ follows.

        \bigskip

        \underline{\textbf{Base Case ($n = 1$):}}

        \bigskip

        Let $n = 1$.

        \bigskip

        Then,

        \begin{align}
            f(n) &= 3 & [\text{By def.}]\\
            &= 3^1\\
            &\leq 3^1\\
            &= 3^n
        \end{align}

        \bigskip

        Thus, $P(n)$ follows.

        \bigskip

        \underline{\textbf{Case ($n > 1$):}}

        \bigskip

        Let $n \in \mathbb{N} \setminus \{0\}$.

        \bigskip

        Then, we have

        \begin{align}
            f(n) &= 2(f(n-2) + f(n-1)) + 1 & [\text{By def., since $1 < n$}]\\
            &\leq 2(3^{n-2} + 3^{n-1}) + 1 & [\text{By I.H, since $1 \leq n-2 < n-1 < n$}]\\
            &= 2 \cdot 3^{n-2}(1 + 3) + 1\\
            &= 8 \cdot 3^{n-2} + 1 \\
            &\leq 8 \cdot 3^{n-2} + 3^{n-2} & [\text{Since $1 < n$ and $0 \leq 3^{n-2}$}]\\
            &= 9 \cdot 3^{n-2}\\
            &= 3^n
        \end{align}

        \bigskip

        Thus, $P(n)$ follows.
    \end{proof}
\end{itemize}

\bigskip

\begin{mdframed}
    \underline{\textbf{Correct Solution:}}

    \bigskip

    For convenience, define $P(n): f(n) \leq 3^n$. I will use complete induction to
    prove that $\forall n \in \mathbb{N}$, $P(n)$.

    \bigskip

    \underline{\textbf{Inductive Step:}}

    \bigskip

    Let $n \in \mathbb{N}$. Assume $H(n): \bigwedge_{i=0}^{n-1} P(i)$. I will
    show $P(n)$ follows. That is $f(n) \leq 3^n$.

    \bigskip

    \underline{\textbf{Base Case ($n = 0$):}}

    \bigskip

    Let $n = 0$.

    \bigskip

    Then,

    \begin{align}
        f(n) &= 1 & [\text{By def.}]\\
        &= 3^0\\
        &\leq 3^0\\
        &= 3^n
    \end{align}

    \bigskip

    Thus, $P(n)$ follows \color{red}in this case\color{black}.

    \bigskip

    \underline{\textbf{Base Case ($n = 1$):}}

    \bigskip

    Let $n = 1$.

    \bigskip

    Then,

    \begin{align}
        f(n) &= 3 & [\text{By def.}]\\
        &= 3^1\\
        &\leq 3^1\\
        &= 3^n
    \end{align}

    \bigskip

    Thus, $P(n)$ follows \color{red}in this case\color{black}.

    \bigskip

    \underline{\textbf{Case ($n > 1$):}}

    \bigskip

    Let $n \in \mathbb{N} \setminus \{0\}$.

    \bigskip

    Then, we have

    \begin{align}
        f(n) &= 2(f(n-2) + f(n-1)) + 1 & [\text{By def., since $1 < n$}]\\
        &\leq 2(3^{n-2} + 3^{n-1}) + 1 & [\text{By I.H, since $1 \leq n-2 < n-1 < n$}]\\
        &= 2 \cdot 3^{n-2}(1 + 3) + 1\\
        &= 8 \cdot 3^{n-2} + 1 \\
        &\leq 8 \cdot 3^{n-2} + 3^{n-2} & [\text{Since $1 < n$ and $\color{red}1\color{black} \leq 3^{n-2}$}]\\
        &= 9 \cdot 3^{n-2}\\
        &= 3^n
    \end{align}

    \bigskip

    Thus, $P(n)$ follows \color{red}from $H(n)$ in this case\color{black}.

\end{mdframed}

\bigskip

\underline{\textbf{Notes:}}

\bigskip

\begin{itemize}
    \item Learned $n \in \mathbb{N} \setminus \{0,\cdots,k\}$ is used to express
    $n > k$, where$n \in \mathbb{N}$.
    \item Noticed professor wrote `$\cdots$ in this case.' at the end of each case.
\end{itemize}

% \bigskip

% \begin{mdframed}
%     \underline{\textbf{Rough Works:}}

%     \bigskip

%     For convenience, define $P(n): f(n) \leq 3^n$. I will use complete induction to
%     prove that $\forall n \in \mathbb{N}$, $P(n)$.

%     \bigskip

%     \begin{itemize}
%         \item Inductive Step

%         \begin{mdframed}
%         \underline{\textbf{Inductive Step:}}

%         \bigskip

%         Let $n \in \mathbb{N}$. Assume $H(n): \bigwedge_{i=0}^{n-1} P(i)$. I will
%         show $P(n)$ follows. That is $f(n) \leq 3^n$.
%         \end{mdframed}

%         \item Base Case ($n = 0$)

%         \begin{mdframed}
%         \underline{\textbf{Base Case ($n = 0$):}}

%         \bigskip

%         Let $n = 0$.

%         \bigskip

%         Then,

%         \begin{align}
%             f(n) &= 1 & [\text{By def.}]\\
%             &= 3^0\\
%             &\leq 3^0\\
%             &= 3^n
%         \end{align}

%         \bigskip

%         Thus, $P(n)$ follows.

%         \end{mdframed}

%         \item Base Case ($n = 1$)

%         \begin{mdframed}

%         \underline{\textbf{Base Case ($n = 1$):}}

%         \bigskip

%         Let $n = 1$.

%         \bigskip

%         Then,

%         \begin{align}
%             f(n) &= 3 & [\text{By def.}]\\
%             &= 3^1\\
%             &\leq 3^1\\
%             &= 3^n
%         \end{align}

%         \bigskip

%         Thus, $P(n)$ follows.

%         \end{mdframed}
%         \item Case ($n > 1$)

%         \begin{mdframed}
%             \underline{\textbf{Case ($n > 1$):}}

%             \bigskip

%             Let $n \in \mathbb{N} \setminus \{0\}$.

%             \bigskip

%             Then, we have

%             \begin{align}
%                 f(n) &= 2(f(n-2) + f(n-1)) + 1 & [\text{By def., since $1 < n$}]\\
%                 &\leq 2(3^{n-2} + 3^{n-1}) + 1 & [\text{By I.H, since $1 \leq n-2 < n-1 < n$}]\\
%                 &= 2 \cdot 3^{n-2}(1 + 3) + 1\\
%                 &= 8 \cdot 3^{n-2} + 1 \\
%                 &\leq 8 \cdot 3^{n-2} + 3^{n-2} & [\text{Since $1 < n$ and $0 \leq 3^{n-2}$}]\\
%                 &= 9 \cdot 3^{n-2}\\
%                 &= 3^n
%             \end{align}

%             \bigskip

%             Thus, $P(n)$ follows.
%         \end{mdframed}
%     \end{itemize}

% \end{mdframed}

\section*{Question 2}

\bigskip

\begin{itemize}
    \item

    \bigskip
    \begin{proof}
    \setcounter{equation}{0}
    Define $P(n):$ Postage of exactly $n$ cents can be made using only 3-cent
    and 4-cent stamps

    \bigskip

    I will use complete induction to prove that $\forall n \in \mathbb{N}$, $n \geq 6 \Rightarrow P(n)$.

    \bigskip

    \underline{\textbf{Base Case ($n = 7$):}}

    \bigskip

    Let $n = 7$.

    \bigskip

    Since $n = 7$ can be made using 1 3-cent stamp and 1 4-cent stamp, $P(n)$
    follows in this step.

    \bigskip

    \underline{\textbf{Base Case ($n = 8$):}}

    \bigskip

    Let $n = 8$.

    \bigskip

    Since $n = 8$ can be made using 2 4-cent stamps, $P(n)$
    follows in this step.

    \bigskip

    \underline{\textbf{Base Case ($n = 9$):}}

    \bigskip

    Let $n = 9$.

    \bigskip

    Since $n = 9$ can be made using 3 3-cent stamps, $P(n)$
    follows in this step.

    \bigskip

    \underline{\textbf{Case ($n < 9$):}}

    \bigskip

    I need to show $\exists d,e \in \mathbb{N}$, $n = d \cdot 3 + e \cdot 4$.

    \bigskip

    Since $n > 9$, $6 \leq n - 4 < n$, so $P(n-4)$ is true. That is,
    postage of $n-4$ cents can be made using 4-cents stamps and 3-cents stamps.
    In other words, $\exists d',e' \in \mathbb{N}$, $n-4 = d' \cdot 3 + e' \cdot 4$.

    \bigskip

    Thus, we have

    \begin{align}
        n-4+4 &= d' \cdot 3 + e' \cdot 4 + 4\\
        n &= d' \cdot 3 + (e'+1) \cdot 4
    \end{align}

    \bigskip

    So, by choosing $d = d'$ and $e = e' + 1$, $P(n)$ follows from $H(n)$ in this step.

    \end{proof}

    \bigskip

    % \begin{mdframed}
    % \underline{\textbf{Rough Work:}}

    % \bigskip

    % Define $P(n):$ Postage of exactly $n$ cents can be made using only 3-cent
    % and 4-cent stamps

    % \bigskip

    % I will use complete induction to prove that $\forall n \in \mathbb{N}$, $n \geq 6 \Rightarrow P(n)$.

    % \bigskip

    % \begin{enumerate}[1.]
    %     \item Inductive Step

    %     \begin{mdframed}
    %     \underline{\textbf{Inductive Step:}}

    %     \bigskip

    %     Let $n \in \mathbb{N}$. Assume $H(n): \bigwedge\limits_{i=0}^{n-1} P(i)$. I will show
    %     $P(n)$ follows.
    %     \end{mdframed}

    %     \item Base Case ($n = 6$)

    %     \begin{mdframed}
    %     \underline{\textbf{Base Case ($n = 6$):}}

    %     \bigskip

    %     Let $n = 6$.

    %     \bigskip

    %     Since $n = 6$ can be made using 2 3-cent stamps, $P(n)$ follows in this step.
    %     \end{mdframed}

    %     \item Base Case ($n = 7$)

    %     \begin{mdframed}
    %     \underline{\textbf{Base Case ($n = 7$):}}

    %     \bigskip

    %     Let $n = 7$.

    %     \bigskip

    %     Since $n = 7$ can be made using 1 3-cent stamp and 1 4-cent stamp, $P(n)$
    %     follows in this step.

    %     \end{mdframed}

    %     \item Base Case ($n = 8$)

    %     \begin{mdframed}
    %     \underline{\textbf{Base Case ($n = 8$):}}

    %     \bigskip

    %     Let $n = 8$.

    %     \bigskip

    %     Since $n = 8$ can be made using 2 4-cent stamps, $P(n)$
    %     follows in this step.

    %     \end{mdframed}

    %     \item Base Case ($n = 9$)

    %     \begin{mdframed}
    %     \underline{\textbf{Base Case ($n = 9$):}}

    %     \bigskip

    %     Let $n = 9$.

    %     \bigskip

    %     Since $n = 9$ can be made using 3 3-cent stamps, $P(n)$
    %     follows in this step.

    %     \end{mdframed}

    %     \item Case ($n < 9$)

    %     \begin{mdframed}
    %     \underline{\textbf{Case ($n < 9$):}}

    %     \bigskip

    %     I need to show $\exists d,e \in \mathbb{N}$, $n = d \cdot 3 + e \cdot 4$.

    %     \bigskip

    %     Since $n > 9$, $6 \leq n - 4 < n$, so $P(n-4)$ is true. That is,
    %     postage of $n-4$ cents can be made using 4-cents stamps and 3-cents stamps.
    %     In other words, $\exists d',e' \in \mathbb{N}$, $n-4 = d' \cdot 3 + e' \cdot 4$.

    %     \bigskip

    %     Thus, we have

    %     \begin{align}
    %         n-4+4 &= d' \cdot 3 + e' \cdot 4 + 4\\
    %         n &= d' \cdot 3 + (e'+1) \cdot 4
    %     \end{align}

    %     \bigskip

    %     So, by choosing $d = d'$ and $e = e' + 1$, $P(n)$ follows from $H(n)$ in this step.
    %     \end{mdframed}
    % \end{enumerate}
    % \end{mdframed}
\end{itemize}



\end{document}