\documentclass[12pt]{article}
\usepackage[margin=2.5cm]{geometry}
\usepackage{enumerate}
\usepackage{amsfonts}
\usepackage{amsmath}
\usepackage{fancyhdr}
\usepackage{amsmath}
\usepackage{amssymb}
\usepackage{amsthm}
\usepackage{mdframed}
\usepackage{graphicx}
\usepackage{subcaption}
\usepackage{adjustbox}
\usepackage{listings}
\usepackage{xcolor}
\usepackage{booktabs}
\usepackage[utf]{kotex}

\definecolor{codegreen}{rgb}{0,0.6,0}
\definecolor{codegray}{rgb}{0.5,0.5,0.5}
\definecolor{codepurple}{rgb}{0.58,0,0.82}
\definecolor{backcolour}{rgb}{0.95,0.95,0.92}

\lstdefinestyle{mystyle}{
    backgroundcolor=\color{backcolour},
    commentstyle=\color{codegreen},
    keywordstyle=\color{magenta},
    numberstyle=\tiny\color{codegray},
    stringstyle=\color{codepurple},
    basicstyle=\ttfamily\footnotesize,
    breakatwhitespace=false,
    breaklines=true,
    captionpos=b,
    keepspaces=true,
    numbers=left,
    numbersep=5pt,
    showspaces=false,
    showstringspaces=false,
    showtabs=false,
    tabsize=1
}

\lstset{style=mystyle}

\begin{document}
\title{CSC236 Worksheet 2 Solution}
\author{Hyungmo Gu}
\maketitle

\section*{Question 1}
\begin{itemize}
    \item

    \bigskip

    \underline{\textbf{Statement:}} Any full binary tree with at least 1 node has
    more leaves than internal nodes.

    \bigskip

    \begin{mdframed}
        \underline{\textbf{Rough Work:}}

        \bigskip

        Let $n$ be the total number of nodes in a full binary tree.

        \bigskip

        We will prove the statement by complete induction on $n$.

        \bigskip

        \begin{enumerate}[1.]
            \item Base Case ($n = 1$)
            \item Base Case ($n = 2$)
            \item Base Case ($n = 3$)
            \item Inductive Step
        \end{enumerate}

    \end{mdframed}

    \bigskip

    \underline{\textbf{Notes:}}

    \bigskip

    \begin{itemize}
        \item Complete Induction
        \begin{itemize}
            \item \underline{\textbf{Statement:}} $\forall i \in \mathbb{N},\:\forall n \in \mathbb{N},\:n < i \Rightarrow A(n) \Rightarrow \forall i \in \mathbb{N},\:A(i)$
            \item \underline{\textbf{Statement Alt.:}} $\Bigl(\forall n \in \mathbb{N},\:\Bigl[ \ \bigwedge\limits_{k = 0}^{k=n-1} P(k) \Bigr] \Rightarrow P(n) \Bigr) \Rightarrow \forall n \in \mathbb{N}, P(n)$
            \item

            \begin{mdframed}
                \textbf{Simple Example 1:}

                \bigskip

                \underline{\textbf{Statement:}} $\forall n \in \mathbb{N},\:n \geq 0 \Rightarrow 10 \mid (n^5 - n)$

                \bigskip

                We will prove the statement by strong induction on $n$.

                \begin{enumerate}[1.]
                    \item Base Case ($n = 0$)

                    \begin{mdframed}

                    Let $n = 0$.

                    \bigskip

                    We need to prove $10 \mid (n^5 - n)$ is true when $n = 0$. That is,
                    there exists $k \in \mathbb{Z}$ such that $(n^5 - n) = 10k$.

                    \bigskip

                    Let $k = 0$.

                    \bigskip

                    Starting from the left hand side, using the fact $n = 0$,
                    we can write

                    \begin{align}
                        (n^5 - n) = 0
                    \end{align}

                    \bigskip

                    Then, because we know $10k = 0$, we can conclude

                    \begin{align}
                        (n^5 - n) = 10k
                    \end{align}

                    \end{mdframed}

                    \item Base Case ($n = 1$)

                    \begin{mdframed}

                    Let $n = 1$.

                    \bigskip

                    We need to prove $10 \mid (n^5 - n)$ is true when $n = 1$. That is,
                    there exists $k \in \mathbb{Z}$ such that $(n^5 - n) = 10k$.

                    \bigskip

                    Let $k = 0$.

                    \bigskip

                    Starting from the left hand side, using the fact $n = 0$,
                    we can write

                    \begin{align}
                        (n^5 - n) &= 1 - 1\\
                        &= 0
                    \end{align}

                    \bigskip

                    Then, because we know $10k = 0$, we can conclude

                    \begin{align}
                        (n^5 - n) = 10k
                    \end{align}

                    \end{mdframed}

                    \item Inductive Step

                    \begin{mdframed}

                    Assume $k \geq 1$. Assume that for all natural number $i$ satisfying $0 \leq i \leq k$, $10 \mid (i^5 - i)$.
                    That is, $\exists d \in \mathbb{Z},\:(i^5 - i) = 10d$.

                    \bigskip

                    We need to prove $\exists \tilde{d} \in \mathbb{Z}$ such that
                    $((k+1)^5 - (k+1)) = 10 \tilde{d}$.

                    \bigskip

                    Let $\tilde{d} = c + (k-1)^4 + 4 \cdot (k-1)^3 + 8 \cdot (k-1)^2 +
                    8 \cdot(k-1) + 3$.

                    \bigskip

                    Starting from $((k+1)^5 - (k+1))$, using binominal theorem, we can write,

                    \begin{align}
                        (k+1)^5 - (k+1) &= \Bigl[ (k-1) + 2 \Bigr]^5 - \Bigl[ (k-1) + 2 \Bigr]\\
                        &= \sum\limits_{b=0}^5 \binom{5}{b} (k-1)^{5-b} \cdot 2^b\\
                        \begin{split}
                        &= (k-1)^5 + 10 \cdot (k-1)^4 + 40 \cdot (k-1)^3 + \\
                        & 80 \cdot (k-1)^2 + 80 \cdot(k-1) + 32 - \Bigl[ (k-1) + 2 \Bigr] \\
                        \end{split}\\
                        \begin{split}
                        &= \Bigl[ (k-1)^5 - (k-1) \Bigr] + 10 \cdot (k-1)^4 + \\
                        & 40 \cdot (k-1)^3 + 80 \cdot (k-1)^2 + 80 \cdot(k-1) + 30
                        \end{split}
                    \end{align}

                    (The reason why $k-1$ is chosen instead of $k-2$ and $k-3$ is
                    because of the last term $2^5 = 32$, i.e $32 -2 = 30$)

                    \bigskip

                    Then, because we know $0 \leq k - 1 \leq k$ and $10 \mid (k-1)^5 - (k-1)$
                    from the header, we can write $\exists c \in \mathbb{Z}$ such that
                    $(k-1)^5 - (k-1) = 10c$, and

                    \begin{align}
                        \begin{split}
                        (k+1)^5 - (k+1) &= 10c + 10 \cdot (k-1)^4 + 40 \cdot (k-1)^3 + \\
                        & 80 \cdot (k-1)^2 + 80 \cdot(k-1) + 30
                        \end{split}\\
                        \begin{split}
                        (k+1)^5 - (k+1) &= 10 \cdot \Bigl[c +\cdot (k-1)^4 + 4 \cdot (k-1)^3 + \\
                        & 8 \cdot (k-1)^2 + 8 \cdot(k-1) + 3\Bigr]
                        \end{split}\\
                    \end{align}

                    \bigskip

                    Then, because we know $\tilde{d} = c + (k-1)^4 + 4 \cdot (k-1)^3 + 8 \cdot (k-1)^2 +
                    8 \cdot(k-1) + 3$ from the header, we can conclude

                    \begin{align}
                        (k+1)^5 - (k+1) &= 10 \tilde{d}
                    \end{align}

                    \end{mdframed}
                \end{enumerate}
            \end{mdframed}
        \end{itemize}
    \end{itemize}
\end{itemize}

\section*{Question 2}

\section*{Question 3}

\end{document}