\documentclass[12pt]{article}
\usepackage[margin=2.5cm]{geometry}
\usepackage{enumerate}
\usepackage{amsfonts}
\usepackage{amsmath}
\usepackage{fancyhdr}
\usepackage{amsmath}
\usepackage{amssymb}
\usepackage{amsthm}
\usepackage{mdframed}
\usepackage{graphicx}
\usepackage{subcaption}
\usepackage{adjustbox}
\usepackage{listings}
\usepackage{xcolor}
\usepackage{booktabs}
\usepackage[utf]{kotex}

\definecolor{codegreen}{rgb}{0,0.6,0}
\definecolor{codegray}{rgb}{0.5,0.5,0.5}
\definecolor{codepurple}{rgb}{0.58,0,0.82}
\definecolor{backcolour}{rgb}{0.95,0.95,0.92}

\lstdefinestyle{mystyle}{
    backgroundcolor=\color{backcolour},
    commentstyle=\color{codegreen},
    keywordstyle=\color{magenta},
    numberstyle=\tiny\color{codegray},
    stringstyle=\color{codepurple},
    basicstyle=\ttfamily\footnotesize,
    breakatwhitespace=false,
    breaklines=true,
    captionpos=b,
    keepspaces=true,
    numbers=left,
    numbersep=5pt,
    showspaces=false,
    showstringspaces=false,
    showtabs=false,
    tabsize=1
}

\lstset{style=mystyle}

\begin{document}
\title{CSC236 Term Test 1 Version 2 Review}
\author{Hyungmo Gu}
\maketitle

\section*{Question 1}
\begin{itemize}
    \item

    \begin{proof}
        \setcounter{equation}{0}

        Define $P(n):$ $f(n)=3^n$.

        \bigskip

        I will use complete induction to prove that $\forall n \in \mathbb{N}, P(n)$.

        \bigskip

        \underline{\textbf{Base Case ($n = 0$):}}

        \bigskip

        Let $n = 0$.

        \bigskip

        Then,

        \begin{align}
            f(n) &= 1 & [\text{By def.}]\\
            &\leq 3^0\\
            &= 3^n
        \end{align}

        \bigskip

        Thus, $P(n)$ follows in this step.

        \bigskip

        \underline{\textbf{Base Case ($n = 1$):}}

        \bigskip

        Let $n = 1$.

        \bigskip

        Then,

        \begin{align}
            f(n) &= 1 & [\text{By def., since $n = 1$}]\\
            &\leq 3^1\\
            &= 3^n
        \end{align}

        \bigskip

        Thus, $P(n)$ follows in this step.

        \bigskip

        \underline{\textbf{Base Case ($n = 2$):}}

        \bigskip

        Let $n = 2$.

        \bigskip

        Then,

        \begin{align}
            f(n) &= 9 & [\text{By def., since $n = 2$}]\\
            &\leq 3^2\\
            &= 3^n
        \end{align}

        \bigskip

        Thus, $P(n)$ follows in this step.

        \bigskip

        \underline{\textbf{Base Case ($n = 3$):}}

        \bigskip

        Let $n = 3$.

        \bigskip

        Then,

        \begin{align}
            \begin{split}
            f(n) &= f(n-1) + 3f(n-2) +\\
            &9f(n-3)
            \end{split} & [\text{By def., since $n = 2$}]\\
            &= 9 + 3 \cdot 3 + 9 \cdot 1 & [\text{By def., since $n-1 = 2$, $n-2=1$, $n-3=0$}]\\
            &= 3^2 + 3^2 + 3^2\\
            &= 3^3\\
            &= 3^n\\
            &\leq 3^n
        \end{align}

        \bigskip

        Thus, $P(n)$ follows in this step.

        \underline{\textbf{Case ($n > 3$):}}

        \bigskip

        Let $n > 3$.

        \bigskip

        Then, since $0 \leq n-3 < n-2 < n-1 < n$, $P(n-3)$, $P(n-2)$, $P(n-1)$
        holds by induction hypothesis. That is, $P(n-3) \leq 3^{n-3}$, $P(n-2) \leq 3^{n-2}$,
        $P(n-1) \leq 3^{n-1}$.

        \bigskip

        Thus,

        \begin{align}
            f(n) &= f(n-1) + 3f(n-2) + 9f(n-3) & [\text{By def., since $n > 2$}]\\
            &\leq 3^{n-1} + 3 \cdot 3^{n-2} + 9 \cdot 3^{n-3} & [\text{By header}]\\
            &= 3^{n-1} + 3^{n-1} + 3^{n-1}\\
            &= 3^n
        \end{align}

        \bigskip

        So, $P(n)$ follows from $H(n)$ in this step.

    \end{proof}

    % \bigskip

    % \begin{mdframed}
    % \underline{\textbf{Rough Works:}}

    % \bigskip
    % \setcounter{equation}{0}

    % Define $P(n):$ $f(n)=3^n$.

    % \bigskip

    % I will use complete induction to prove that $\forall n \in \mathbb{N}, P(n)$.

    % \bigskip

    % \begin{enumerate}[1.]
    %     \item Inductive Step

    %     \begin{mdframed}
    %     \underline{\textbf{Inductive Step:}}

    %     \bigskip

    %     Let $n \in \mathbb{N}$. Assume $H(n):\bigwedge\limits_{i=0}^{n-1}P(i)$.
    %     I will show $P(n)$ follows.
    %     \end{mdframed}

    %     \item Base Case ($n = 0$)

    %     \begin{mdframed}
    %     \underline{\textbf{Base Case ($n = 0$):}}

    %     \bigskip

    %     Let $n = 0$.

    %     \bigskip

    %     Then,

    %     \begin{align}
    %         f(n) &= 1 & [\text{By def.}]\\
    %         &\leq 3^0\\
    %         &= 3^n
    %     \end{align}

    %     \bigskip

    %     Thus, $P(n)$ follows in this step.
    %     \end{mdframed}

    %     \item Base Case ($n = 1$)

    %     \begin{mdframed}
    %     \underline{\textbf{Base Case ($n = 1$):}}

    %     \bigskip

    %     Let $n = 1$.

    %     \bigskip

    %     Then,

    %     \begin{align}
    %         f(n) &= 1 & [\text{By def., since $n = 1$}]\\
    %         &\leq 3^1\\
    %         &= 3^n
    %     \end{align}

    %     \bigskip

    %     Thus, $P(n)$ follows in this step.
    %     \end{mdframed}

    %     \item Base Case ($n = 2$)

    %     \begin{mdframed}
    %     \underline{\textbf{Base Case ($n = 2$):}}

    %     \bigskip

    %     Let $n = 2$.

    %     \bigskip

    %     Then,

    %     \begin{align}
    %         f(n) &= 9 & [\text{By def., since $n = 2$}]\\
    %         &\leq 3^2\\
    %         &= 3^n
    %     \end{align}

    %     \bigskip

    %     Thus, $P(n)$ follows in this step.
    %     \end{mdframed}

    %     \item Base Case ($n = 3$)

    %     \begin{mdframed}
    %     \underline{\textbf{Base Case ($n = 3$):}}

    %     \bigskip

    %     Let $n = 3$.

    %     \bigskip

    %     Then,

    %     \begin{align}
    %         \begin{split}
    %         f(n) &= f(n-1) + 3f(n-2) +\\
    %         &9f(n-3)
    %         \end{split} & [\text{By def., since $n = 2$}]\\
    %         &= 9 + 3 \cdot 3 + 9 \cdot 1 & [\text{By def., since $n-1 = 2$, $n-2=1$, $n-3=0$}]\\
    %         &= 3^2 + 3^2 + 3^2\\
    %         &= 3^3\\
    %         &= 3^n\\
    %         &\leq 3^n
    %     \end{align}

    %     \bigskip

    %     Thus, $P(n)$ follows in this step.
    %     \end{mdframed}

    %     \item Case ($n > 3$)

    %     \begin{mdframed}
    %     \underline{\textbf{Case ($n > 3$):}}

    %     \bigskip

    %     Let $n > 3$.

    %     \bigskip

    %     Then, since $0 \leq n-3 < n-2 < n-1 < n$, $P(n-3)$, $P(n-2)$, $P(n-1)$
    %     holds by induction hypothesis. That is, $P(n-3) \leq 3^{n-3}$, $P(n-2) \leq 3^{n-2}$,
    %     $P(n-1) \leq 3^{n-1}$.

    %     \bigskip

    %     Thus,

    %     \begin{align}
    %         f(n) &= f(n-1) + 3f(n-2) + 9f(n-3) & [\text{By def., since $n > 2$}]\\
    %         &\leq 3^{n-1} + 3 \cdot 3^{n-2} + 9 \cdot 3^{n-3} & [\text{By header}]\\
    %         &= 3^{n-1} + 3^{n-1} + 3^{n-1}\\
    %         &= 3^n
    %     \end{align}

    %     \bigskip

    %     So, $P(n)$ follows from $H(n)$ in this step.
    %     \end{mdframed}
    % \end{enumerate}
    % \end{mdframed}
\end{itemize}

\bigskip

\section*{Question 2}

\bigskip

\begin{itemize}
    \item

    \begin{mdframed}
    \underline{\textbf{Rough Works:}}

    \bigskip

    Define for convenience

    \begin{center}
        $P(x,y,z,w):$ There are no positive integers $x,y,z,w$ such that $x^4 + 3y^4 + 9z^4 = 27w^4$.
    \end{center}

    \bigskip

    I will prove $P(x,y,z,w)$ by contradiction.

    \bigskip

    Assume $\neg P(x,y,z,w)$. That is, $\exists x,y,z,w \in \mathbb{N}^+$,
    $x^4 + 3y^4 + 9z^4 = 27w^4$.

    \bigskip

    Then, $X = \{x \in \mathbb{N}^+ \mid \exists y,z,w \in \mathbb{N}^+, x^4+3y^4+9z^4 = 27w^4\}$
    is not empty.

    \bigskip

    Then, by the principle of well-ordering, $X$ has smallest element.

    \bigskip

    Let $x_0 \in X$ be its smallest element, and let $y_0,z_0,w_0 \in \mathbb{N}^+,
    x_0^4+3y_0^4+9z_0^4 = 27w_0^4$.

    \bigskip

    Then,


    \end{mdframed}
\end{itemize}

\end{document}