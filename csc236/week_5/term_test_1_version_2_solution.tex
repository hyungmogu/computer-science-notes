\documentclass[12pt]{article}
\usepackage[margin=2.5cm]{geometry}
\usepackage{enumerate}
\usepackage{amsfonts}
\usepackage{amsmath}
\usepackage{fancyhdr}
\usepackage{amsmath}
\usepackage{amssymb}
\usepackage{amsthm}
\usepackage{mdframed}
\usepackage{graphicx}
\usepackage{subcaption}
\usepackage{adjustbox}
\usepackage{listings}
\usepackage{xcolor}
\usepackage{booktabs}
\usepackage[utf]{kotex}

\definecolor{codegreen}{rgb}{0,0.6,0}
\definecolor{codegray}{rgb}{0.5,0.5,0.5}
\definecolor{codepurple}{rgb}{0.58,0,0.82}
\definecolor{backcolour}{rgb}{0.95,0.95,0.92}

\lstdefinestyle{mystyle}{
    backgroundcolor=\color{backcolour},
    commentstyle=\color{codegreen},
    keywordstyle=\color{magenta},
    numberstyle=\tiny\color{codegray},
    stringstyle=\color{codepurple},
    basicstyle=\ttfamily\footnotesize,
    breakatwhitespace=false,
    breaklines=true,
    captionpos=b,
    keepspaces=true,
    numbers=left,
    numbersep=5pt,
    showspaces=false,
    showstringspaces=false,
    showtabs=false,
    tabsize=1
}

\lstset{style=mystyle}

\begin{document}
\title{CSC236 Term Test 1 Version 2 Solution}
\author{Hyungmo Gu}
\maketitle

\section*{Question 1}
\begin{itemize}
    \item

    aa
\end{itemize}

\bigskip

\begin{mdframed}
    \underline{\textbf{Rough Work:}}

    \bigskip

    Define $P(n):f(n) = 3^n$.

    \bigskip

    I will use complete induction to prove that $\forall n \in \mathbb{N}, n > 2 \Rightarrow \:C(n)$.

    \begin{enumerate}[1.]
        \item Inductive Step

        \begin{mdframed}
        \underline{\textbf{Inductive Step:}}

        \bigskip

        Let $n \in \mathbb{N}$. Assume $n > 2$. Assume
        $H(n):\bigwedge\limits_{i=0}^{n-1} P(i)$. I will prove $C(n)$ follows.
        That is, $f(n) = 3^n$.
        \end{mdframed}

        \item Base Case ($n = 0$)

        \begin{mdframed}
        \underline{\textbf{Base Case ($n = 0$):}}

        \bigskip

        Let $n = 0$.

        \bigskip

        Then, the definition of $f(n)$ tells us $f(n) = 1$.

        \bigskip

        Then, we have

        \begin{align}
            f(n) &= 3^0\\
            &= 3^n
        \end{align}

        \bigskip

        Thus, $P(n)$ follows.
        \end{mdframed}

        \item Base Case ($n = 1$)

        \begin{mdframed}
        \underline{\textbf{Base Case ($n = 1$):}}

        \bigskip

        Let $n = 1$.

        \bigskip

        Then, the definition of $f(n)$ tells us $f(n) = 3$.

        \bigskip

        Then, we have

        \begin{align}
            f(n) &= 3^1\\
            &= 3^n
        \end{align}

        \bigskip

        Thus, $P(n)$ follows.

        \end{mdframed}

        \item Base Case ($n = 2$)

        \begin{mdframed}
        \underline{\textbf{Base Case ($n = 2$):}}

        \bigskip

        Let $n = 2$.

        \bigskip

        Then, the definition of $f(n)$ tells us $f(n) = 9$.

        \bigskip

        Then, we have

        \begin{align}
            f(n) &= 3^2\\
            &= 3^n
        \end{align}

        \bigskip

        Thus, $P(n)$ follows.

        \end{mdframed}

        \item Case ($n > 2$)
    \end{enumerate}

\end{mdframed}

\section*{Question 2}

\section*{Question 3}

\end{document}