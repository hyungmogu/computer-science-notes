\documentclass[12pt]{article}
\usepackage[margin=2.5cm]{geometry}
\usepackage{enumerate}
\usepackage{amsfonts}
\usepackage{amsmath}
\usepackage{fancyhdr}
\usepackage{amsmath}
\usepackage{amssymb}
\usepackage{amsthm}
\usepackage{mdframed}
\usepackage{graphicx}
\usepackage{subcaption}
\usepackage{adjustbox}
\usepackage{listings}
\usepackage{xcolor}
\usepackage{booktabs}
\usepackage[utf]{kotex}
\usepackage{hyperref}
\usepackage{accents}

\definecolor{codegreen}{rgb}{0,0.6,0}
\definecolor{codegray}{rgb}{0.5,0.5,0.5}
\definecolor{codepurple}{rgb}{0.58,0,0.82}
\definecolor{backcolour}{rgb}{0.95,0.95,0.92}

\lstdefinestyle{mystyle}{
    backgroundcolor=\color{backcolour},
    commentstyle=\color{codegreen},
    keywordstyle=\color{magenta},
    numberstyle=\tiny\color{codegray},
    stringstyle=\color{codepurple},
    basicstyle=\ttfamily\footnotesize,
    breakatwhitespace=false,
    breaklines=true,
    captionpos=b,
    keepspaces=true,
    numbers=left,
    numbersep=5pt,
    showspaces=false,
    showstringspaces=false,
    showtabs=false,
    tabsize=1
}

\lstset{style=mystyle}

\pagestyle{fancy}
\renewcommand{\headrulewidth}{0.4pt}
\lhead{CSC 343}
\rhead{Worksheet 5}

\begin{document}
\title{CSC343 Worksheet 5}
\maketitle

\begin{enumerate}[1.]
    \item \textbf{Exercise 7.1.1:} Our running example movie database of Section
    2.2.8 has keys defined for all its relations.

    \begin{lstlisting}[language=SQL]
    Movies(title, year, length, genre, studioName, producerC#)
    Starsln(movieTitle, movieYear, starName)
    MovieStar(name, address, gender, birthdate)
    MovieExec(name, address, cert#, netWorth)
    Studio(name, address, presC#)
    \end{lstlisting}

    \bigskip

    Declare the following referential integrity constraints for the movie database as
    in Exercise 7.1.1.

    \begin{enumerate}[a)]
        \item The producer of a movie must be someone mentioned in MovieExec. Modifications to MovieExec that violate this constraint are rejected.
        \item Repeat (a), but violations result in the producerC\# in Movie being set to NULL.
        \item Repeat (a), but violations result in the deletion or update of the offending Movie tuple.
        \item A movie that appears in Stars ln must also appear in Movie. Handle violations by rejecting the modification.
        \item A star appearing in Stars ln must also appear in MovieStar. Handle violations by deleting violating tuples.
    \end{enumerate}

    \item \textbf{Exercise 7.1.2:} We would like to declare the constraint that every movie in
    the relation Movie must appear with at least one star in StarsIn. Can we do
    so with a foreign-key constraint? Why or why not?

    \item \textbf{Exercise 7.1.3:} Suggest suitable keys and foreign keys for the relations of the
    PC database:

    \begin{lstlisting}[language=SQL]
    Product(maker, model, type)
    PC(model, speed, ram, hd, price)
    Laptop(model, speed, ram, hd, screen, price)
    Printer(model, color, type, price)
    \end{lstlisting}

    \bigskip

    of Exercise 2.4.1. Modify your SQL schema from Exercise 2.3.1 to include
    declarations of these keys.

    \item \textbf{Exercise 7.1.4:} Suggest suitable keys for the relations of the battleships
    database

    \bigskip

    \begin{lstlisting}[language=SQL]
    Classes(class, type, country, numGuns, bore, displacement)
    Ships(name, class, launched)
    Battles(name, date)
    Outcomes(ship, battle, result)
    \end{lstlisting}

    \bigskip

    of Exercise 2.4.3. Modify your SQL schema from Exercise 2.3.2 to include
    declarations of these keys.

    \item \textbf{Exercise 7.1.5:} Exercise 7.1.5: Write the following referential
    integrity constraints for the battleships database as in Exercise 7.1.4. Use
    your assumptions about keys from that exercise, and handle all violations by
    setting the referencing attribute value to NULL

    \begin{enumerate}[a)]
        \item Every class mentioned in \textbf{Ships} must be mentioned in \textbf{Classes}.
        \item Every battle mentioned in \textbf{Outcomes} must be mentioned in \textbf{Battles}.
        \item Every ship mentioned in \textbf{Outcomes} must be mentioned in \textbf{Ships}.
    \end{enumerate}

\end{enumerate}

\end{document}