\documentclass[12pt]{article}
\usepackage[margin=2.5cm]{geometry}
\usepackage{enumerate}
\usepackage{amsfonts}
\usepackage{amsmath}
\usepackage{fancyhdr}
\usepackage{amsmath}
\usepackage{amssymb}
\usepackage{amsthm}
\usepackage{mdframed}
\usepackage{graphicx}
\usepackage{subcaption}
\usepackage{adjustbox}
\usepackage{listings}
\usepackage{xcolor}
\usepackage{booktabs}
\usepackage[utf]{kotex}
\usepackage{hyperref}

\definecolor{codegreen}{rgb}{0,0.6,0}
\definecolor{codegray}{rgb}{0.5,0.5,0.5}
\definecolor{codepurple}{rgb}{0.58,0,0.82}
\definecolor{backcolour}{rgb}{0.95,0.95,0.92}

\lstdefinestyle{mystyle}{
    backgroundcolor=\color{backcolour},
    commentstyle=\color{codegreen},
    keywordstyle=\color{magenta},
    numberstyle=\tiny\color{codegray},
    stringstyle=\color{codepurple},
    basicstyle=\ttfamily\footnotesize,
    breakatwhitespace=false,
    breaklines=true,
    captionpos=b,
    keepspaces=true,
    numbers=left,
    numbersep=5pt,
    showspaces=false,
    showstringspaces=false,
    showtabs=false,
    tabsize=1
}

\lstset{style=mystyle}

\pagestyle{fancy}
\renewcommand{\headrulewidth}{0.4pt}
\lhead{CSC 343}
\rhead{Worksheet 2}

\begin{document}
\title{CSC343 Worksheet 2}
\maketitle

\noindent \textbf{Note:} This is student designed study guide to make learnings easier.
This does not reflect the course material. Please take it only as a reference.

\begin{enumerate}[1.]
    \item \textbf{Exercise 2.4.1:} This exercise builds upon the products schema of
    Exercise 2.3.1. Recall that the database schema consists of four relations, whose
    schemas are:

    \begin{lstlisting}
    Product(maker, model, type)
    PC(model, speed, ram, hd, price)
    Laptop(model, speed, ram, hd, screen, price)
    Printer(model, color, type, price)
    \end{lstlisting}

    \bigskip

    Some sample data for the relation Product is shown in Fig. 2.20. Sample data
    for the outer three relations is shown in Fig. 2.21. Manufactureres and model
    numbers have been ``Sanitized'', but the data is typical of products on sale
    at the beginning of 2007.

    \bigskip

    Write expressions of relational algebra to answer the following queries. You
    may use the linear notation of Section 2.4.13 if you wish. For the data of Figs. 2.20
    and 2.21, show the result of your query. However, your answer should work for
    arbitrary data, not just the data of these figures

    \bigskip

    \begin{enumerate}[a)]
        \item What PC models have a speed of at least 3.00?
        \item Which manufacturers make laptops with a hard disk of at least 100GB?
        \item Find the model number and price of all products (of any type) made by manufacturer B..
        \item Find the model nubmers of all color laser printers
        \item Find those maufactueres that sell Laptops, but not PC's.
        \item Find those hard-disk sizes that occur in two or more PC's.
    \end{enumerate}

    \begin{center}
    \includegraphics[width=0.75\linewidth]{images/worksheet_2_5.png}
    \includegraphics[width=0.75\linewidth]{images/worksheet_2_6.png}
    \includegraphics[width=0.75\linewidth]{images/worksheet_2_7.png}
    \includegraphics[width=0.75\linewidth]{images/worksheet_2_8.png}
    \end{center}

    \item \textbf{Exercise 2.4.2:} Draw expression trees for each of your expressions
    of Exercise 2.4.1

    \item \textbf{Exercise 2.4.3:} This exercise builds upon Exercise 2.3.2 concerning
    World War II capital ships. Recall it involves the following relations:

    \begin{lstlisting}
    Classes(class, type, country, numGuns, bore, displacement)
    Ships(name, class, launched)
    Battles(name, date)
    Outcomes(ship, battle, result)
    \end{lstlisting}

    Figures 2.22 and 2.23 give some same data for these four relations. Note that unlike
    the data for Exercise 2.4.1, there are some ``dangling tuples'' in this data,
    e.g. ships mentioned in \textit{Outcomes} that are not mentioned in \textit{Ships}.

    \bigskip

    Write expressions of relational algebra to answer the following queries. You
    may use the linear notation of Section 2.4.13 if you wish. For the data of Figs.
    2.22 and 2.23, show the result of your query. However, your answer should work
    for arbitrary data, not just the data of these figures.

    \bigskip

    \begin{enumerate}[a)]
        \item Give the class names and countries of the classes that carried guns
        of at least 16-inch bore.
        \item Find the ships launched prior to 1921.
        \item Find the ships sunk in the battle of the Denmark Strait.
        \item The treaty of Washington in 1921 prohibited capital ships heavier than
        35,000 tons. List the ships that violated the treaty of Washington.
        \item List the name, displacement, and number of guns of the ships engaged
        in the battle of Guadalcanal.
        \item List all the capital ships mentioned in the database. (Remember that all
        these ships may not appear in the \textit{Ships} relation).
        \item Find the class that had only one ship as a member of that class
        \item Find those countries that had both battleships and battlecruisers.
        \item Find those ships that ``lived to fight another day''; they were damanged
        one battle, but later fought in another
    \end{enumerate}

    \begin{center}
    \includegraphics[width=0.75\linewidth]{images/worksheet_2_1.png}
    \includegraphics[width=0.75\linewidth]{images/worksheet_2_2.png}
    \includegraphics[width=0.75\linewidth]{images/worksheet_2_3.png}
    \includegraphics[width=0.75\linewidth]{images/worksheet_2_4.png}
    \end{center}

    \item \textbf{Exercise 2.4.4:} Draw expression trees for each of your expressions of
    Exercise 2.4.3

    \item \textbf{Exercise 2.4.5:} What is the difference between the natural join $R \bowtie S$ and the
    theta-join $R \bowtie_C S$ where the condition $C$ is that $R.A = S.A$ for
    each attribute $A$ appearing in the schemas of both $R$ and $S$

    \item \textbf{Exercise 5.1.1:} Let PC be the relation of FIg. 2.21(A), and suppose
    we compute the projection $\pi_{\text{speed}}(PC)$. What is the value of this
    expression as a set? As a bag? What is the average value of tuples in this projection,
    when treated as a set? As a bag?

    \item \textbf{Exercise 5.1.2} Repeat Exercise 5.1.1 for the projection $\pi_{\text{hd}}(PC)$

    \item \textbf{Exercise 5.1.3} This exercise refers to the ``battleship'' relations
    of Exercise 2.4.3.

    \bigskip

    \begin{enumerate}[a)]
        \item The expression $\pi_{\text{bore}}(Classes)$ yields a single-column
        relations with the bores of the various classes. For the data of Exercise 2.4.3,
        what is the relation as a set? As a bag?
    \end{enumerate}
\end{enumerate}

\bigskip

\underline{\textbf{Reference}}

\bigskip

\begin{enumerate}[1)]
    \item Stanford: CS145 - Introduction to Databases, \href{http://infolab.stanford.edu/~ullman/fcdb/aut07/index.html}{link}
\end{enumerate}

\end{document}