\documentclass[12pt]{article}
\usepackage[margin=2.5cm]{geometry}
\usepackage{enumerate}
\usepackage{amsfonts}
\usepackage{amsmath}
\usepackage{fancyhdr}
\usepackage{amsmath}
\usepackage{amssymb}
\usepackage{amsthm}
\usepackage{mdframed}
\usepackage{graphicx}
\usepackage{subcaption}
\usepackage{adjustbox}
\usepackage{listings}
\usepackage{xcolor}
\usepackage{booktabs}
\usepackage[utf]{kotex}
\usepackage{hyperref}
\usepackage{accents}

\definecolor{codegreen}{rgb}{0,0.6,0}
\definecolor{codegray}{rgb}{0.5,0.5,0.5}
\definecolor{codepurple}{rgb}{0.58,0,0.82}
\definecolor{backcolour}{rgb}{0.95,0.95,0.92}

\lstdefinestyle{mystyle}{
    backgroundcolor=\color{backcolour},
    commentstyle=\color{codegreen},
    keywordstyle=\color{magenta},
    numberstyle=\tiny\color{codegray},
    stringstyle=\color{codepurple},
    basicstyle=\ttfamily\footnotesize,
    breakatwhitespace=false,
    breaklines=true,
    captionpos=b,
    keepspaces=true,
    numbers=left,
    numbersep=5pt,
    showspaces=false,
    showstringspaces=false,
    showtabs=false,
    tabsize=1
}

\lstset{style=mystyle}

\pagestyle{fancy}
\renewcommand{\headrulewidth}{0.4pt}
\lhead{CSC 343}
\rhead{Worksheet 15 (final)}

\begin{document}
\title{CSC343 Worksheet 15 (final)}
\maketitle

\begin{enumerate}[1.]
    \item \textbf{Exercise 4.7.1:} Draw a UML diagram for the problem of Exercise 4.1.1.
    \item \textbf{Exercise 4.7.2:} Modify your diagram from Exercise 4.7.1 in accordance with the requirements of Exercise 4.1.2.
    \item \textbf{Exercise 4.7.3:} Repeat Exercise 4.1.3 using UML.
    \item \textbf{Exercise 4.7.4:} Repeat Exercise 4.1.6 using UML.
    \item \textbf{Exercise 4.7.5:} Repeat Exercise 4.1. 7 using UML. Are your subclasses disjoint or overlapping? Are they complete or partial?
    \item \textbf{Exercise 4.7.6:} Repeat Exercise 4.1.9 using UML.
    \item \textbf{Exercise 4.7.7:} Convert the E/R diagram of Fig. 4.30 to a UML diagram.
    \item \textbf{Exercise 4.7.8:} How would you represent the 3-way relationship of Contracts among movies, stars, and studios (see Fig. 4.4) in UML?
    \item \textbf{Exercise 4.7.9:} Repeat Exercise 4.2.5 using UML.
    \item \textbf{Exercise 4.8.1:} Convert the UML diagram of Fig. 4.43 to relations.
    \item \textbf{Exercise 4.8.2:} Convert the following UML diagrams to relations:

    \begin{enumerate}[a)]
        \item Figure 4.37.
        \item Figure 4.40.
        \item Your solution to Exercise 4.7.1.
        \item Your solution to Exercise 4.7.3.
        \item Your solution to Exercise 4.7.4.
        \item Your solution to Exercise 4.7.6.
    \end{enumerate}

    \item \textbf{Exercise 4.8.3:} How many relations do we create, using the object-oriented
    approach, if we have a three-level hierarchy with three subclasses of each class
    at the first and second levels, and that hierarchy is:

    \begin{enumerate}[a)]
        \item Disjoint and complete at each level.
        \item Disjoint but not complete at each level.
        \item Neither disjoint nor complete.
    \end{enumerate}

    \item \textbf{Exercise 4.9.1:} In Exercise 4.1.1 was the informal description of a bank database. Render this design in ODL, including keys as appropriate.
    \item \textbf{Exercise 4.9.2:} Modify your design of Exercise 4.9.1 in the ways enumerated in Exercise 4.1.2. Describe the changes; do not write a complete, new schema.
    \item \textbf{Exercise 4.9.3:} Render the teams-players-fans database of Exercise 4.1.3 in ODL, including keys, as appropriate. Why does the complication about sets of team colors, which was mentioned in the original exercise, not present a problem in ODL?
    \item \textbf{Exercise 4.9.4:} Suppose we wish to keep a genealogy. We shall have one class, Person. The information we wish to record about persons includes their name (an attribute) and the following relationships: mother, father, and children. Give an ODL design for the Person class. Be sure to indicate the inverses of the relationships that, like mother, father, and children, are also relationships from Person to itself. Is the inverse of the mother relationship the children relationship? Why or why not? Describe each of the relationships and their inverses as sets of pairs.
    \item \textbf{Exercise 4.9.5:} Let us add to the design of Exercise 4.9.4 the attribute education. The value of this attribute is intended to be a collection of the degrees obtained by each person, including the name of the degree (e.g., B.S.), the school, and the date. This collection of structs could be a set, bag, list, or array. Describe the consequences of each of these four choices. What information could be gained or lost by making each choice? Is the information lost likely to be important in practice?
    \item \textbf{Exercise 4.9.6:} In Exercise 4.4.4 we saw two examples of situations where weak entity sets were essential. Render these databases in ODL, including declarations for suitable keys.
    \item \textbf{Exercise 4.9.7:} Give an ODL design for the registrar's database described in Exercise 4.1.9.
    \item Exercise 4.10.1: Convert your ODL designs from the following exercises to relational database schemas.

    \begin{enumerate}[a)]
        \item Exercise 4.9.1.
        \item Exercise 4.9.2 (include all four of the modifications specified by that exercise).
        \item Exercise 4.9.3.
        \item Exercise 4.9.4.
        \item Exercise 4.9.5.
    \end{enumerate}
\end{enumerate}

\end{document}