\documentclass[12pt]{article}
\usepackage[margin=2.5cm]{geometry}
\usepackage{enumerate}
\usepackage{amsfonts}
\usepackage{amsmath}
\usepackage{fancyhdr}
\usepackage{amsmath}
\usepackage{amssymb}
\usepackage{amsthm}
\usepackage{mdframed}
\usepackage{graphicx}
\usepackage{subcaption}
\usepackage{adjustbox}
\usepackage{listings}
\usepackage{xcolor}
\usepackage{booktabs}
\usepackage[utf]{kotex}
\usepackage{hyperref}
\usepackage{accents}

\definecolor{codegreen}{rgb}{0,0.6,0}
\definecolor{codegray}{rgb}{0.5,0.5,0.5}
\definecolor{codepurple}{rgb}{0.58,0,0.82}
\definecolor{backcolour}{rgb}{0.95,0.95,0.92}

\lstdefinestyle{mystyle}{
    backgroundcolor=\color{backcolour},
    commentstyle=\color{codegreen},
    keywordstyle=\color{magenta},
    numberstyle=\tiny\color{codegray},
    stringstyle=\color{codepurple},
    basicstyle=\ttfamily\footnotesize,
    breakatwhitespace=false,
    breaklines=true,
    captionpos=b,
    keepspaces=true,
    numbers=left,
    numbersep=5pt,
    showspaces=false,
    showstringspaces=false,
    showtabs=false,
    tabsize=1
}

\lstset{style=mystyle}

\pagestyle{fancy}
\renewcommand{\headrulewidth}{0.4pt}
\lhead{CSC 343}
\rhead{Worksheet 15 Solution (Final)}

\begin{document}
\title{CSC343 Worksheet 15 Solution (Final)}
\maketitle

\begin{enumerate}[1.]
    \item

    \begin{itemize}
        \item E/R Diagram

        \begin{center}
        \includegraphics[width=0.7\linewidth]{images/worksheet_15_solution_9.png}
        \end{center}

        \item UML

        \begin{center}
        \includegraphics[width=\linewidth]{images/worksheet_15_solution_10.png}
        \end{center}


    \end{itemize}

    \bigskip

    \underline{\textbf{Notes:}}

    \bigskip

    \begin{itemize}
        \item UML
        \begin{itemize}
            \item Was developed originally as a graphical notation for describing software designs in an object-oriented style
            \item Offers the same as E/R model, with the exception of multiway relationship
        \end{itemize}

        \begin{center}
        \includegraphics[width=0.7\linewidth]{images/worksheet_15_solution_1.png}
        \end{center}

        \item UML Class

        \begin{center}
        \includegraphics[width=0.5\linewidth]{images/worksheet_15_solution_2.png}
        \end{center}

        \item Associations

        \begin{center}
        \includegraphics[width=\linewidth]{images/worksheet_15_solution_4.png}
        \end{center}

        \bigskip

        \underline{Multiplicity in UML}

        \bigskip

        \begin{center}
            \begin{tabular}{|c|c|c|}
                \hline
                Multiplicity & Option & Cardinality\\
                \hline
                0..0 & 0 & Collection must be empty\\
                \hline
                0..1 & & No instances or one instance\\
                \hline
                1..1 & 1 & Exactly one instance\\
                \hline
                0..* & * & Zero or more instance\\
                \hline
                5..5 & 5 & Exactly 5 instances\\
                \hline
                $m..n$ & & At least $m$ but no more than $n$ instances\\
                \hline
            \end{tabular}
        \end{center}

        \bigskip

        \underline{Example:}

        \begin{center}
        \includegraphics[width=0.7\linewidth]{images/worksheet_15_solution_3.png}
        \end{center}

        \bigskip

        \underline{References:}

        \bigskip

        \begin{enumerate}[1)]
            \item uml-diagrams, UML Multiplicity and Collections, \href{https://www.uml-diagrams.org/multiplicity.html}{link}
        \end{enumerate}

        \item Referential Integrity
        \begin{itemize}
            \item Means that a value appearing in one context must also appear in another
        \end{itemize}

        \bigskip

        \begin{center}
        \includegraphics[width=\linewidth]{images/worksheet_15_solution_5.png}
        \includegraphics[width=\linewidth]{images/worksheet_15_solution_6.png}
        \end{center}

        \item Self-Assoiations

        \begin{center}
        \includegraphics[width=\linewidth]{images/worksheet_15_solution_7.png}
        \end{center}


        \item Assoiations

        \begin{center}
        \includegraphics[width=\linewidth]{images/worksheet_15_solution_8.png}
        \end{center}

    \end{itemize}

    \item

    \begin{enumerate}[a)]
        \item

        \underline{\textbf{Solution:}}

        \begin{center}
        \includegraphics[width=\linewidth]{images/worksheet_15_solution_11.png}
        \end{center}

        \item

        \underline{\textbf{Solution:}}

        \begin{center}
        \includegraphics[width=\linewidth]{images/worksheet_15_solution_12.png}
        \end{center}
    \end{enumerate}

    \item

    \begin{itemize}
        \item E/R Diagram

        \begin{center}
        \includegraphics[width=\linewidth]{images/worksheet_15_solution_13.png}
        \end{center}

        \item UML

        \begin{center}
        \includegraphics[width=\linewidth]{images/worksheet_15_solution_14.png}
        \end{center}
    \end{itemize}

    \bigskip

    \begin{mdframed}
        \underline{\textbf{Correct Solution:}}

        \bigskip

        \begin{itemize}
            \item E/R Diagram

            \begin{center}
            \includegraphics[width=\linewidth]{images/worksheet_15_solution_13.png}
            \end{center}

            \item UML

            \begin{center}
            \includegraphics[width=\linewidth]{images/worksheet_15_solution_35.png}
            \end{center}
        \end{itemize}

    \end{mdframed}

    \item

    \begin{itemize}
        \item E/R Diagram

        \begin{center}
        \includegraphics[width=\linewidth]{images/worksheet_15_solution_15.png}
        \end{center}

        \item UML

        \begin{center}
        \includegraphics[width=\linewidth]{images/worksheet_15_solution_16.png}
        \end{center}
    \end{itemize}

    \item

    \begin{itemize}
        \item E/R Diagram

        \begin{center}
        \includegraphics[width=\linewidth]{images/worksheet_15_solution_17.png}
        \end{center}

        \item UML

        \begin{center}
        \includegraphics[width=\linewidth]{images/worksheet_15_solution_18.png}
        \end{center}

    \end{itemize}

    \item

    \begin{itemize}
        \item E/R Diagram

        \begin{center}
        \includegraphics[width=\linewidth]{images/worksheet_15_solution_19.png}
        \end{center}

        \item UML

        \begin{center}
        \includegraphics[width=\linewidth]{images/worksheet_15_solution_20.png}
        \end{center}

    \end{itemize}

    \item

    \begin{itemize}
        \item E/R Diagram

        \begin{center}
        \includegraphics[width=\linewidth]{images/worksheet_15_solution_21.png}
        \end{center}

        \item UML

        \begin{center}
        \includegraphics[width=\linewidth]{images/worksheet_15_solution_22.png}
        \end{center}

    \end{itemize}

    \item

    \begin{itemize}

        \item UML

        \begin{center}
        \includegraphics[width=\linewidth]{images/worksheet_15_solution_23.png}
        \end{center}

    \end{itemize}

    \item

    \begin{center}
    \includegraphics[width=\linewidth]{images/worksheet_15_solution_24.png}
    \end{center}

    \begin{enumerate}[a)]
        \item

        \underline{\textbf{Solution:}}

        \begin{itemize}
            \item E/R Model
            \begin{center}
            \includegraphics[width=\linewidth]{images/worksheet_15_solution_25.png}
            \end{center}

            \item UML

            \begin{center}
            \includegraphics[width=\linewidth]{images/worksheet_15_solution_27.png}
            \end{center}
        \end{itemize}

        \begin{mdframed}
            \underline{\textbf{Correct Solution:}}

            \bigskip

            \begin{itemize}
                \item E/R Model
                \begin{center}
                \includegraphics[width=\linewidth]{images/worksheet_15_solution_28.png}
                \end{center}

                \item UML

                \begin{center}
                \includegraphics[width=\linewidth]{images/worksheet_15_solution_29.png}
                \end{center}
            \end{itemize}

        \end{mdframed}

        \bigskip

        \underline{\textbf{Notes:}}

        \bigskip

        \begin{itemize}
            \item An N-ary association is equivalent to one “central” class and N binary associations connecting the central class to the participant classes of the N-ary association


            \begin{center}
            \includegraphics[width=\linewidth]{images/worksheet_15_solution_26.png}
            \end{center}
        \end{itemize}

        \item

        \underline{\textbf{Solution:}}

        \begin{itemize}
            \item E/R Model
            \begin{center}
            \includegraphics[width=\linewidth]{images/worksheet_15_solution_30.png}
            \end{center}

            \item UML

            \begin{center}
            \includegraphics[width=\linewidth]{images/worksheet_15_solution_31.png}
            \end{center}
        \end{itemize}

        \item

        \underline{\textbf{Solution:}}

        \begin{itemize}
            \item E/R Model
            \begin{center}
            \includegraphics[width=\linewidth]{images/worksheet_15_solution_32.png}
            \end{center}

            \item UML

            \begin{center}
            \includegraphics[width=\linewidth]{images/worksheet_15_solution_33.png}
            \end{center}
        \end{itemize}



    \end{enumerate}

    \item

    \begin{itemize}
        \item Bookings

        \bigskip

        Bookings(\underline{SSNo}, \underline{number}, \underline{day}, row, seat)

        \bigskip

        \item Customers

        \bigskip

        Customers(\underline{SSNo}, name, addr, phone)

        \bigskip

        \item Flights

        \bigskip

        Flights(\underline{number}, \underline{day}, aircraft)

        \bigskip
    \end{itemize}

    \bigskip

    \underline{\textbf{Notes:}}

    \begin{itemize}
        \item Weak Entity sets in UML

        \bigskip

        \underline{\textbf{Example:}}

        \bigskip

        \begin{center}
        \includegraphics[width=\linewidth]{images/worksheet_15_solution_34.png}
        \end{center}
    \end{itemize}

    \item

    \begin{enumerate}[a)]
        \item

        \begin{itemize}
            \item Movies

            \bigskip

            \quad Movies(\underline{title}, \underline{year}, length, genre)

            \bigskip

            \item Studios

            \bigskip

            \quad Studios(\underline{name}, address)

            \bigskip

            \item Presidents

            \bigskip

            \quad Presidents (\underline{cert\#}, name, address)

            \bigskip

            \item Owns

            \bigskip

            \quad Owns(\underline{title}, \underline{year}, \underline{name})

            \bigskip

            \item Runs

            \bigskip

            \quad Runs (\underline{cert\#}, \underline{name})

            \bigskip
        \end{itemize}

        \item

        \begin{itemize}
            \item Movies

            \bigskip

            \quad Movies(\underline{title}, \underline{year}, length, genre)

            \bigskip

            \item Muder-Mysteries

            \bigskip

            \quad Muder-Mysteries(\underline{title}, \underline{year}, weapon)

            \bigskip

            \item Cartoons

            \bigskip

            \quad Cartoons\underline{title}, \underline{year})

            \bigskip

            \item Cartoon-Murder-Mysteries

            \bigskip

            \quad Cartoon-Murder-Mysteries(\underline{title}, \underline{year}, weapon)

            \bigskip

            \item Voices (for Cartoons and Cartoon-Murder-Mysteries)

            \bigskip

            \quad Voices (\underline{starName}, \underline{title}, \underline{year})

            \bigskip
        \end{itemize}

        \item


        \begin{itemize}
            \item Customer

            \bigskip

            \quad Customer(\underline{Phone}, \underline{Address}, \underline{SSN}, Name)

            \bigskip

            \item Account

            \bigskip

            \quad Account(\underline{Number}, types, Balance)

            \bigskip

            \item Account Record

            \bigskip

            \quad AccountRecord(\underline{Phone}, \underline{Address}, \underline{SSN}, \underline{Number})

            \bigskip
        \end{itemize}

        \item

        \begin{itemize}
            \item Team

            \bigskip

            \quad Team(\underline{Name})

            \bigskip

            \item Fans

            \bigskip

            \quad Fans(\underline{Name})

            \bigskip

            \item Color(\underline{Name})

            \bigskip

            \quad Color(\underline{Name})

            \bigskip

            \item Players

            \bigskip

            \quad Players(\underline{Name})

            \bigskip

            \item Captain

            \bigskip

            \quad Captain(\underline{Name})

            \bigskip

            \item CaptainOf

            \bigskip

            \quad CaptainOf(\underline{CaptainName}, \underline{TeamName})

            \bigskip

            \item PlayersOf

            \bigskip

            \quad PlayersOf(\underline{PlayerName}, \underline{TeamName})

            \bigskip

            \item FavouriteTeam

            \bigskip

            \quad FavouriteTeam(\underline{FanName}, \underline{TeamName})

            \bigskip

            \item FavouriteColor

            \bigskip

            \quad FavouriteColor(\underline{FanName}, \underline{ColorName})

            \bigskip

            \item FavouritePlayer

            \bigskip

            \quad FavouritePlayer(\underline{FanName}, \underline{PlayerName})

            \bigskip
        \end{itemize}

        \item

        \begin{itemize}
            \item
        \end{itemize}
    \end{enumerate}


\end{enumerate}

\end{document}