\documentclass[12pt]{article}
\usepackage[margin=2.5cm]{geometry}
\usepackage{enumerate}
\usepackage{amsfonts}
\usepackage{amsmath}
\usepackage{fancyhdr}
\usepackage{amsmath}
\usepackage{amssymb}
\usepackage{amsthm}
\usepackage{mdframed}
\usepackage{graphicx}
\usepackage{subcaption}
\usepackage{adjustbox}
\usepackage{listings}
\usepackage{xcolor}
\usepackage{booktabs}
\usepackage[utf]{kotex}
\usepackage{hyperref}
\usepackage{accents}

\definecolor{codegreen}{rgb}{0,0.6,0}
\definecolor{codegray}{rgb}{0.5,0.5,0.5}
\definecolor{codepurple}{rgb}{0.58,0,0.82}
\definecolor{backcolour}{rgb}{0.95,0.95,0.92}

\lstdefinestyle{mystyle}{
    backgroundcolor=\color{backcolour},
    commentstyle=\color{codegreen},
    keywordstyle=\color{magenta},
    numberstyle=\tiny\color{codegray},
    stringstyle=\color{codepurple},
    basicstyle=\ttfamily\footnotesize,
    breakatwhitespace=false,
    breaklines=true,
    captionpos=b,
    keepspaces=true,
    numbers=left,
    numbersep=5pt,
    showspaces=false,
    showstringspaces=false,
    showtabs=false,
    tabsize=1
}

\lstset{style=mystyle}

\pagestyle{fancy}
\renewcommand{\headrulewidth}{0.4pt}
\lhead{CSC 343}
\rhead{Worksheet 4 Solution}

\begin{document}
\title{CSC343 Worksheet 4 Solution}
\maketitle

\begin{enumerate}[1.]
    \item
    \begin{enumerate}[a)]
        \item [(1,0,1),(5,4,9),(1,0,1),(6,4,16),(7,9,16)]
        \item [(1,0),(3,3),(3,4),(4,3),(1,1),(4,3)]
        \item [(0,1),(0,1),(2,3),(2,4),(3,4)]

        \bigskip

        \underline{\textbf{Notes:}}

        \bigskip

        \begin{itemize}
            \item $\tau_L(R)$ sorts tuples in order indicated by $L$.
            \begin{itemize}
                \item e.g.

                \bigskip

                $\tau_{C,B}(R)$ in $R(A,B,C)$ orders the tuples of $R$ by their
                values of $C$, and tuples with the same $C$-value are ordered by their
                $B$ value.
            \end{itemize}
        \end{itemize}

        \item [(0,1),(0,2),(2,4),(2,5),(3,4),(3,4)]
        \item [(0,1),(2,4),(2,5),(3,4),(0,2)]

        \bigskip

        \underline{\textbf{Notes:}}

        \bigskip

        \begin{itemize}
            \item $\delta(R)$ converts a bag into a set
            \begin{itemize}
                \item e.g.

                \bigskip

                Let $R = [(1,2),(3,4),(1,2),(1,2)]$

                \bigskip

                $\delta(R(A,B)) = [(1,2),(3,4)]$
            \end{itemize}
        \end{itemize}

    \end{enumerate}
\end{enumerate}

\end{document}