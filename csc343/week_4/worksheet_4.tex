\documentclass[12pt]{article}
\usepackage[margin=2.5cm]{geometry}
\usepackage{enumerate}
\usepackage{amsfonts}
\usepackage{amsmath}
\usepackage{fancyhdr}
\usepackage{amsmath}
\usepackage{amssymb}
\usepackage{amsthm}
\usepackage{mdframed}
\usepackage{graphicx}
\usepackage{subcaption}
\usepackage{adjustbox}
\usepackage{listings}
\usepackage{xcolor}
\usepackage{booktabs}
\usepackage[utf]{kotex}
\usepackage{hyperref}
\usepackage{accents}

\definecolor{codegreen}{rgb}{0,0.6,0}
\definecolor{codegray}{rgb}{0.5,0.5,0.5}
\definecolor{codepurple}{rgb}{0.58,0,0.82}
\definecolor{backcolour}{rgb}{0.95,0.95,0.92}

\lstdefinestyle{mystyle}{
    backgroundcolor=\color{backcolour},
    commentstyle=\color{codegreen},
    keywordstyle=\color{magenta},
    numberstyle=\tiny\color{codegray},
    stringstyle=\color{codepurple},
    basicstyle=\ttfamily\footnotesize,
    breakatwhitespace=false,
    breaklines=true,
    captionpos=b,
    keepspaces=true,
    numbers=left,
    numbersep=5pt,
    showspaces=false,
    showstringspaces=false,
    showtabs=false,
    tabsize=1
}

\lstset{style=mystyle}

\pagestyle{fancy}
\renewcommand{\headrulewidth}{0.4pt}
\lhead{CSC 343}
\rhead{Worksheet 4}

\begin{document}
\title{CSC343 Worksheet 4}
\maketitle

\bigskip

\begin{enumerate}[1.]
    \item \textbf{Exercise 5.2.1}: Here are two relations

    \bigskip

    $R(A,B)$: [(0,1),(2,3),(0,1,),(2,4),(3,4)]

    \bigskip

    $S(A,B)$: [(0,1), (2,4), (2,5), (3,4), (0,2), (3,4)]

    \bigskip

    Compute the following

    \begin{enumerate}[a)]
        \item $\pi_{A+B,A^2,B^2}$
        \item $\pi_{B+1,C-1}(S)$
        \item $\tau_{B,A}(R)$
        \item $\tau_{B,C}(S)$
        \item $\delta(S)$
        \item $\gamma_{A, SUM(B)}(R)$
        \item $\gamma_{B, AVG(C)}(S)$
        \item $\gamma_A(R)$
        \item $\gamma_{A, MAX(C)}(R \bowtie S)$
        \item $R \accentset{\circ}{\bowtie}_L S$
        \item $R \accentset{\circ}{\bowtie}_R S$
        \item $R \accentset{\circ}{\bowtie} S$
        \item $R \accentset{\circ}{\bowtie}_{R.B < S.B} S$
    \end{enumerate}

    \item \textbf{Exercise 6.4.1:} Write each of the quires in Exercise 2.4.1 in SQL,
    making sure that duplicates are eliminated

    \item \textbf{Exercise 6.4.2:} Write each of the queries in Exercise 2.4.3 in
    SQL, making sure duplicates are eliminated

    \item \textbf{Exercise 6.4.6:} Write the following queires, based on the database
    schema

    \begin{lstlisting}[language=SQL]
    Product(maker, model, type)
    PC(model, speed, ram, hd, price)
    Laptop(model, speed, ram, hd, screen, price)
    Printer(model, color, type, price)
    \end{lstlisting}

    \begin{enumerate}[a)]
        \item Find the avergage speed of PC's
        \item Find the average speed of laptops costing over \$1000
        \item Find the average price of PC's made by manufacturer ``A''
        \item Find the average price of PC's and laptops made by manufacturer ``D''
        \item Find, for each different speed, the average price of a PC
        \item Find for each manufacturer, the average screen size of its laptop
        \item Find the manufacturers that make at least three different models of PC
        \item Find for each manufacturer who sells PC's the maximum price of a PC
        \item Find, for each speed of PC above 2.0, the average price.
    \end{enumerate}

    \item Write the following queires, based on the database schema

    \begin{lstlisting}[language=SQL]
    Classes(class, type, country, numGuns, bore, displacement)
    Ships(name, class, launched)
    Battles(name, date)
    Outcomes(ship, battle, result)
    \end{lstlisting}

    \bigskip

    \begin{enumerate}[a)]
        \item Find the number of battleship classes
        \item Find the average number of guns of battleship classes
        \item Find the average number of guns of battleships. Note the difference
        between b) and c); do we weight a class by the number of ships of that class
        or not?
        \item Find for each class the year in which the first ship of that class was
        launched
        \item Find for each class the number of ships of that class sunk in battle
    \end{enumerate}

    \item \textbf{Exercise 6.4.8:} In Example 5.10, we gave an example of the query:
    ``find, for each star who has appeared in at least three movies, the earliest year
    in they appeared.'' We wrote this query as a $\gamma$ operation. Write it in SQL.

    \item \textbf{Exercise 6.4.9:} The $\gamma$ operator of extended relational algebra
    does not have a feature that corresponds to the \textbf{HAVING} clause of SQL.
    Is it possible to mimic a SQL query with a \textbf{HAVING} clause in relational
    algebra?

    \item \textbf{Exercise 6.5.1:} Write the following database modifications, based on the
    database schema

    \begin{lstlisting}[language=SQL]
    Product(maker, model, type)
    PC(model, speed, ram, hd, price)
    Laptop(model, speed, ram, hd, screen, price)
    Printer(model, color, type, price)
    \end{lstlisting}

    \bigskip

    of Exercise 2.4.1. Describe the effect of the modifications on the data of that
    exercise

    \begin{enumerate}[a)]
        \item Using two INSERT statements, store in the database the fact that PC
        model 1100 is made by manufacturer C, has spped 3.2, RAM 1024, hard disk 180,
        and sells for \$2499
        \item Intert the facts that for every PC there is a laptop with the same
        manufacturer, speed, RAM, and hard disk, a 17-inch screen, a model number
        110 greater, and a price \$500 more.
        \item Delete all PC's with less than 100 gigabytes of hard disk
        \item Delete all laptops made by a manufacturer that doesn't make printers
        \item Manufacturer $A$ buys manufacturer $B$. Change all products made by
        $B$ so they are now made by $A$.
        \item For each PC, double the amount of RAM and add 60 gigabytes to the
        amount of hard disk (Remember that several attributes can be changed by
        one UPDATE statement)
        \item For each laptop made by manufactuerer $B$, add one inch to the screen
        size and subtract \$100 from the price.
    \end{enumerate}

    \item \textbf{Exercise 6.5.2:} Write the following database modifications, based
    on the database schema.

    \bigskip

    \begin{lstlisting}[language=SQL]
    Classes(class, type, country, numGuns, bore, displacement)
    Ships(name, class, launched)
    Battles(name, date)
    Outcomes(ship, battle, result)
    \end{lstlisting}

    of Exercise 2.4.3. Describe the effect of the modifications on the data of that
    exercise.

    \bigskip

    \begin{enumerate}[a)]
        \item The two British battleships of the Nelson class - Nelson and Rodney -
        were both launched in 1927, had 16-inch guns, and a displacement of 34,000
        tones. Insert these facts into the database
        \item Two of the three battleships of the Italian Vittorio Veneto class
        Vittorio Veneto and Italia - were launched in 1940; the third ship of that
        class, Roma, was launched in 1942. Each had nine 15-inch guns and a displacement
        of 41,000 tons. Insert these facdts into the database.
        \item Delete from \textbf{Ships} all ships sunk in battle
        \item Modify the \textbf{Classes} relation so that gun bores are measured
        in centimeters (one inch = 2.5 centimeters) and displacements are measured in
        metric tons (one metric ton = 1.1 tons).
        \item Delete all classes with fewer than three ships
    \end{enumerate}

\end{enumerate}

\end{document}