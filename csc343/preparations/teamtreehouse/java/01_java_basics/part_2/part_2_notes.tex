\documentclass[12pt]{article}
\usepackage[margin=2.5cm]{geometry}
\usepackage{enumerate}
\usepackage{amsfonts}
\usepackage{amsmath}
\usepackage{fancyhdr}
\usepackage{amsmath}
\usepackage{amssymb}
\usepackage{amsthm}
\usepackage{mdframed}
\usepackage{graphicx}
\usepackage{subcaption}
\usepackage{adjustbox}
\usepackage{listings}
\usepackage{xcolor}
\usepackage{booktabs}
\usepackage[utf]{kotex}
\usepackage{hyperref}

\definecolor{codegreen}{rgb}{0,0.6,0}
\definecolor{codegray}{rgb}{0.5,0.5,0.5}
\definecolor{codepurple}{rgb}{0.58,0,0.82}
\definecolor{backcolour}{rgb}{0.95,0.95,0.92}

\lstdefinestyle{mystyle}{
    backgroundcolor=\color{backcolour},
    commentstyle=\color{codegreen},
    keywordstyle=\color{magenta},
    numberstyle=\tiny\color{codegray},
    stringstyle=\color{codepurple},
    basicstyle=\ttfamily\footnotesize,
    breakatwhitespace=false,
    breaklines=true,
    captionpos=b,
    keepspaces=true,
    numbers=left,
    numbersep=5pt,
    showspaces=false,
    showstringspaces=false,
    showtabs=false,
    tabsize=1
}

\lstset{style=mystyle}

\pagestyle{fancy}
\renewcommand{\headrulewidth}{0.4pt}
\lhead{Team Treehouse}
\rhead{Java Basics Part 2 Notes}

\begin{document}
\title{Java Basics Part 2 Notes}
\author{Team Treehouse}
\maketitle

\section{Multiple Strings}

\bigskip

\begin{itemize}
    \item Multiple variables can be put in single formatted Strings.
    \item \textbf{Syntax:} console.printf("\%s ... \%s ... ", VAR1, VAR2, ...)
    \bigskip

    \begin{lstlisting}[language=java]
    import java.io.Console;

    public class Introductions {
        public static void main(String[] args) {
            Console console = System.console();
            String name = console.readLine("Enter name:   "); // <- Let's write 'Moe' here
            String adjective = console.readLine("Enter adjective:   "); // <- and 'glad to be with his love' here :)
            console.printf("%s is very %s\n", name, adjective);
        }
    }
    \end{lstlisting}

\end{itemize}

\section{Exercise 1}

\bigskip

\begin{itemize}
    \item Solution included in \textit{exercise\_1.java}
\end{itemize}

\section{Errors}

\section{Quiz 1}

\bigskip

\begin{enumerate}[1.]
    \item

    You will make mistakes

    \bigskip

    \begin{enumerate}[A.]
        \item True
        \item Yes
    \end{enumerate}

    \bigskip

    \textbf{Answer:} A, B

    \item

    You will learn from your mistakes ;)

    \bigskip

    \begin{enumerate}[A.]
        \item True
        \item Yes
    \end{enumerate}

    \bigskip

    \textbf{Answer:} A, B

    \item The Java compiler will tell you about syntax errors, such as `cannot
    find symbol`.

    \bigskip

    \begin{enumerate}[A.]
        \item True
        \item False
    \end{enumerate}

    \bigskip

    \textbf{Answer:} A

\end{enumerate}

\section{Coding the Prototype}

\bigskip

\begin{itemize}
    \item

    \begin{lstlisting}[language=java]
    import java.io.Console;

    public class Introductions {
        public static void main(String[] args) {
            Console console = System.console();
            String name = console.readLine("Enter name:   ");
            String adjective = console.readLine("Enter an adjective:   ");
            String noun = console.readLine("Enter a noun:   ");
            String adverb = console.readLine("Enter an adverb:   ");
            String verb = console.readLine("Enter a verb ending in -ing:   ");

            console.printf("Your TreeStory:\n------------------\n");
            console.printf("% is a %s %s", name, adjective, noun);
            console.printf("They are always %s %s\n", adverb, verb);
        }
    }
    \end{lstlisting}

\end{itemize}

\section{Exercise 2}

\bigskip

\begin{itemize}
    \item Solution included in \textit{exercise\_2.java}
\end{itemize}

\end{document}