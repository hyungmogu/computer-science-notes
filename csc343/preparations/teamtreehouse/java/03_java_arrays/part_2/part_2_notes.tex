\documentclass[12pt]{article}
\usepackage[margin=2.5cm]{geometry}
\usepackage{enumerate}
\usepackage{amsfonts}
\usepackage{amsmath}
\usepackage{fancyhdr}
\usepackage{amsmath}
\usepackage{amssymb}
\usepackage{amsthm}
\usepackage{mdframed}
\usepackage{graphicx}
\usepackage{subcaption}
\usepackage{adjustbox}
\usepackage{listings}
\usepackage{xcolor}
\usepackage{booktabs}
\usepackage[utf]{kotex}
\usepackage{hyperref}

\definecolor{codegreen}{rgb}{0,0.6,0}
\definecolor{codegray}{rgb}{0.5,0.5,0.5}
\definecolor{codepurple}{rgb}{0.58,0,0.82}
\definecolor{backcolour}{rgb}{0.95,0.95,0.92}

\lstdefinestyle{mystyle}{
    backgroundcolor=\color{backcolour},
    commentstyle=\color{codegreen},
    keywordstyle=\color{magenta},
    numberstyle=\tiny\color{codegray},
    stringstyle=\color{codepurple},
    basicstyle=\ttfamily\footnotesize,
    breakatwhitespace=false,
    breaklines=true,
    captionpos=b,
    keepspaces=true,
    numbers=left,
    numbersep=5pt,
    showspaces=false,
    showstringspaces=false,
    showtabs=false,
    tabsize=1
}

\lstset{style=mystyle}

\pagestyle{fancy}
\renewcommand{\headrulewidth}{0.4pt}
\lhead{Team Treehouse}
\rhead{Java Arrays Part 2 Notes}

\begin{document}
\title{Java Arrays Part 2 Notes}
\author{Team Treehouse}
\maketitle

\section{Enhanced For Loop}

\bigskip

\begin{itemize}
    \item \textbf{Syntax:} \textit{for (DATA\_TYPE ITER\_VAR : ARR\_VAR) \{...\}}
    \begin{itemize}
        \item \textit{:} Means `in', like python
    \end{itemize}

    \bigskip

    \underline{\textbf{Example:}}

    \bigskip

    \begin{lstlisting}[language=Java, caption={lesson\_01/Explore.java}]
    String[] classmates = {"Ben", "Johnny", "Pasan"};
    for (String classmate : classmates) {
        System.out.printf("%s is my class friend\n", classmate);
    }
    \end{lstlisting}

    \underline{\textbf{Notes:}}

    \bigskip

    \begin{itemize}
        \item Files can be compiled and displayed by typing \textit{javac Explore.java \&\& java Explore}
        in terminal
    \end{itemize}
\end{itemize}

\bigskip

\section{Quiz 1}

\bigskip

\begin{enumerate}[1.]
    \item

    The reason you can use an array in an enhanced for loop is because

    \begin{enumerate}[A.]
        \item arrays are special. They are the only type of object you can use in an enhanced for loop.
        \item they are contiguous.
        \item arrays are iterable. Any object that is considered iterable can be used.
    \end{enumerate}

    \bigskip

    \textbf{Answer:} C

    \item

    The code below can be read as:

    \begin{lstlisting}[language=Java]
    String[] flavors = {"vanilla", "chocolate", "strawberry"};
    for (String flavor : flavors) {
        // ...
    }
    \end{lstlisting}

    \bigskip

    \begin{enumerate}[A.]
        \item For each flavor in the flavors array...
        \item Flavors should be reduced to a single value
        \item Once flavor is set, concatenate ...
    \end{enumerate}

    \bigskip

    \textbf{Answer:} A

\end{enumerate}

\bigskip

\section{Ye Olde Unenhanced For Loop}

\bigskip

\begin{itemize}
    \item \textbf{Syntax:} \textit{for (int i = 0; i \textless ARR\_VAR.length; i++) {...}}

    \begin{lstlisting}[language=Java, caption={lesson\_03/Explore.java}]
    String[] classmates = {"Ben", "Johnny", "Pasan"};
    for (int i = 0; i < classmates.length; i++) {
        String classmate = classmate[i];
        System.out.printf("%s is my class friend", classmate);
    }
    \end{lstlisting}
\end{itemize}

\bigskip

\section{Exercise 1}

\bigskip

\begin{itemize}
    \item Solution included in \textit{exercise\_1.java}
\end{itemize}

\bigskip

\section{Multidimensional Arrays}

\bigskip

\begin{itemize}
    \item \textbf{Syntax:} \textit{DATA\_TYPE[][] ARR\_VAR = { {...}, {...}, ..., {...} }}

    \begin{lstlisting}[language=Java, caption={lesson\_05/Explore.java}]
    import java.util.Arrays;

    public class Explore {
        public static void main(String[] args) {
            int[][] scoreBoards = {
                {1,2,4,2,6,5,4},
                {2,3,5,1,1,2,3},
                {4,4,2,1,2,2,1}
            };
            System.out.println(Arrays.toString(scoreBoards[0]));
            System.out.println(scoreBoards[1][2]);
        }
    }
    \end{lstlisting}


    \underline{\textbf{Notes:}}

    \bigskip

    \begin{itemize}
        \item Files can be compiled and displayed by typing \textit{javac Explore.java \&\& java Explore}
        in terminal
        \item \textit{scoreBoards[0]} alone prints its memory location, like C :)
    \end{itemize}
\end{itemize}

\bigskip

\section{Quiz 2}

\bigskip

\begin{enumerate}[1.]
    \item

    Considering the following code:

    \begin{lstlisting}[language=Java]
    String[][] bradys = {
        {"Mike", "Carol", "Alice"},
        {"Bobby", "Peter", "Greg"},
        {"Cindy", "Jan", "Marsha"}
    };
    \end{lstlisting}

    What is the value at bradys[2][2]

    \begin{enumerate}[A.]
        \item Marsha
        \item Greg
        \item Peter
    \end{enumerate}

    \bigskip

    \textbf{Answer:} A

    \item

    Considering the following code:

    \begin{lstlisting}[language=Java]
    String[][] bradys = {
        {"Mike", "Carol", "Alice"},
        {"Bobby", "Peter", "Greg"},
        {"Cindy", "Jan", "Marsha"}
    };
    \end{lstlisting}

    How would you reference Alice

    \bigskip

    \begin{enumerate}[A.]
        \item bradys(0, 2)
        \item bradys[2][0]
        \item bradys[0][2]
        \item bradys[1][3]
    \end{enumerate}

    \bigskip

    \textbf{Answer:} C

\end{enumerate}

\bigskip

\section{Looping Over 2D Arrays}

\bigskip

\begin{itemize}
    \item \textbf{Syntax:}

    \textit{for(int i = 0; i \textless ARR\_VAR.length; i++) \{}

    \textit{    for(int j = 0; j \textless ARR\_VAR[i].length; j++) \{...\}}

    \textit{\}}

    \bigskip

    \underline{\textbf{Example:}}

    \bigskip

    \begin{lstlisting}[language=Java, caption={lesson\_07/Explore.java}]
    int[][] scoreBoards = {
        {1,2,4,2,6,5,4},
        {2,3,5,1,1,2,3},
        {4,4,2,1,2,2,1}
    };

    for (int i = 0; i < scoreBoards.length; i++) {
        for (int j = 0; j < scoreBoards[i].length; j++) {
            System.out.println(scoreBoards[i][j]);
        }
    }
    \end{lstlisting}

    \underline{\textbf{Notes:}}

    \bigskip

    \begin{itemize}
        \item Files can be compiled and displayed by typing \textit{javac Explore.java \&\& java Explore}
        in terminal
    \end{itemize}
\end{itemize}

\bigskip



\bigskip

\section{Quiz 3}

\bigskip

\begin{enumerate}[1.]
    \item

    Please fill in the correct answer in each blank provided below.


    \begin{lstlisting}[language=Java]
    char[][] boggle = {
            {'C', 'A', 'T'},
            {'D', 'R', 'I'},
            {'L', 'O', 'G'}
    };

    System.out.printf("-------------%n");
    for (int ____  = 0; ____ < boggle.length; ____++) {
      for (int ____ = 0; ____ < boggle[____].length; ____++) {
        System.out.printf("| %s ", boggle[____][____]);
       }
       System.out.printf("|%n-------------%n");
    }
    \end{lstlisting}

    \bigskip

    \textbf{Answer:}

    \begin{lstlisting}[language=Java]
    char[][] boggle = {
            {'C', 'A', 'T'},
            {'D', 'R', 'I'},
            {'L', 'O', 'G'}
    };

    System.out.printf("-------------%n");
    for (int i  = 0; i < boggle.length; i++) {
        for (int j = 0; j < boggle[i].length; j++) {
        System.out.printf("| %s ", boggle[i][j]);
        }
        System.out.printf("|%n-------------%n");
    }
    \end{lstlisting}
\end{enumerate}



\end{document}