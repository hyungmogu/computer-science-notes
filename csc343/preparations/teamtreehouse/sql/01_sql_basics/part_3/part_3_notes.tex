\documentclass[12pt]{article}
\usepackage[margin=2.5cm]{geometry}
\usepackage{enumerate}
\usepackage{amsfonts}
\usepackage{amsmath}
\usepackage{fancyhdr}
\usepackage{amsmath}
\usepackage{amssymb}
\usepackage{amsthm}
\usepackage{mdframed}
\usepackage{graphicx}
\usepackage{subcaption}
\usepackage{adjustbox}
\usepackage{listings}
\usepackage{xcolor}
\usepackage{booktabs}
\usepackage[utf]{kotex}
\usepackage{hyperref}

\definecolor{codegreen}{rgb}{0,0.6,0}
\definecolor{codegray}{rgb}{0.5,0.5,0.5}
\definecolor{codepurple}{rgb}{0.58,0,0.82}
\definecolor{backcolour}{rgb}{0.95,0.95,0.92}

\lstdefinestyle{mystyle}{
    backgroundcolor=\color{backcolour},
    commentstyle=\color{codegreen},
    keywordstyle=\color{magenta},
    numberstyle=\tiny\color{codegray},
    stringstyle=\color{codepurple},
    basicstyle=\ttfamily\footnotesize,
    breakatwhitespace=false,
    breaklines=true,
    captionpos=b,
    keepspaces=true,
    numbers=left,
    numbersep=5pt,
    showspaces=false,
    showstringspaces=false,
    showtabs=false,
    tabsize=1
}

\lstset{style=mystyle}

\pagestyle{fancy}
\renewcommand{\headrulewidth}{0.4pt}
\lhead{Team Treehouse}
\rhead{SQL Basics Part 3 Notes}

\begin{document}
\title{SQL Basics Part 3 Notes}
\author{Team Treehouse}
\maketitle

\bigskip

\section{Searching Tables with 'WHERE'}

\bigskip

\begin{itemize}
    \item WHERE clause
    \begin{itemize}
        \item \textbf{Syntax:} SELECT \textit{columns} FROM \textit{table name} WHERE \textit{condition};
        \item \textbf{Syntax (Condition):} \textit{Columns} \textit{Operator} \textit{Value}
    \end{itemize}

    \item Equality Operator
    \begin{itemize}
        \item \textbf{Syntax:} SELECT \textit{columns} FROM \textit{table name} WHERE \textit{column name} = \textit{value};
    \end{itemize}

    \bigskip

    \underline{\textbf{Examples:}}

    \bigskip

    \begin{lstlisting}[language=SQL]
    SELECT * FROM contacts WHERE first_name = "Andrew";


    SELECT first_name, email FROM users WHERE last_name = "Chalkley";


    SELECT name AS "Product Name" FROM products WHERE stock_count = 0;


    SELECT title "Book Title" FROM books WHERE year_published = 1999;
    \end{lstlisting}

    \item Inequality Operator
    \begin{itemize}
        \item \textbf{Syntax:} SELECT \textit{columns} FROM \textit{table name} WHERE \textit{column name} != \textit{value};
    \end{itemize}

    \bigskip

    \underline{\textbf{Examples:}}

    \bigskip

    \begin{lstlisting}[language=SQL]
    SELECT * FROM contacts WHERE first_name != "Kenneth";


    SELECT first_name, email FROM users WHERE last_name != "L:one";


    SELECT name AS "Product Name" FROM products WHERE stock_count != 0;


    SELECT title "Book Title" FROM books WHERE year_published != 2015;
    \end{lstlisting}

    \item Greater than/ Less than Operator
    \begin{itemize}
        \item \textbf{Syntax (less than):} SELECT \textit{columns} FROM \textit{table name} WHERE \textit{column name} $<$ \textit{value};
        \item \textbf{Syntax (greater than):} SELECT \textit{columns} FROM \textit{table name} WHERE \textit{column name} $>$ \textit{value};
    \end{itemize}

    \item Cheat Sheet: \href{https://github.com/treehouse/cheatsheets/blob/master/sql_basics/cheatsheet.md}{Link}
\end{itemize}

\bigskip

\section{Exercise 1}

\bigskip

\begin{itemize}
    \item Solution included in \textit{exercise\_1.sql}
\end{itemize}

\bigskip

\section{Filtering by Comparing Values}

\bigskip

\begin{itemize}
    \item \textbf{Syntax (Less than):} SELECT \textit{columns} FROM \textit{table name} WHERE \textit{column name} $<$ \textit{value};
    \item \textbf{Syntax (Less than or equal):} SELECT \textit{columns} FROM \textit{table name} WHERE \textit{column name} $<=$ \textit{value};
    \item \textbf{Syntax (Greater than):} SELECT \textit{columns} FROM \textit{table name} WHERE \textit{column name} $>$ \textit{value};
    \item \textbf{Syntax (Greater than or equal):} SELECT \textit{columns} FROM \textit{table name} WHERE \textit{column name} $>=$ \textit{value};

    \bigskip

    \underline{\textbf{Example:}}

    \bigskip

    \begin{lstlisting}[language=SQL]
    SELECT first_name, last_name FROM users WHERE date_of_birth < '1998-12-01';


    SELECT title AS "Book Title", author AS Author FROM books WHERE year_released <= 2015;


    SELECT name, description FROM products WHERE price > 9.99;


    SELECT title FROM movies WHERE release_year >= 2000;
    \end{lstlisting}
\end{itemize}

\bigskip

\section{Exercise 2}

\bigskip

\begin{itemize}
    \item Solution included in \textit{exercise\_2.sql}
\end{itemize}

\bigskip

\section{Filtering on More than One Condition}

\bigskip

\begin{itemize}
    \item Is used when filtering with multiple conditions
    \item Can be done using \textit{AND} and/or \textit{OR} operator
    \item \textbf{Syntax (AND):} SELECT \textit{columns} FROM \textit{table name} WHERE <condition 1> AND <condition 2> ...;
    \item \textbf{Syntax (OR):} SELECT \textit{columns} FROM \textit{table name} WHERE <condition 1> OR <condition 2> ...;

    \bigskip

    \underline{\textbf{Examples:}}

    \bigskip

    \begin{lstlisting}[language=SQL]
    SELECT username FROM users WHERE last_name = "Chalkley" AND first_name = "Andrew";


    SELECT * FROM products WHERE category = "Games Consoles" AND price < 400;


    SELECT * FROM movies WHERE title = "The Matrix" OR title = "The Matrix Reloaded" OR title = "The Matrix Revolutions";


    SELECT country FROM countries WHERE population < 1000000 OR population > 100000000;
    \end{lstlisting}
\end{itemize}

\bigskip

\section{Exercise 3}

\bigskip

\begin{itemize}
    \item Solution included in \textit{exercise\_3.sql}
\end{itemize}

\bigskip

\section{Filtering By Dates}

\bigskip

\begin{itemize}
    \item Is done using comparison operators (same as part 3).
    \item \textbf{Syntax (Less than):} SELECT \textit{columns} FROM \textit{table name} WHERE \textit{column name} $<$ \textit{value};
    \item \textbf{Syntax (Less than or equal):} SELECT \textit{columns} FROM \textit{table name} WHERE \textit{column name} $<=$ \textit{value};
    \item \textbf{Syntax (Greater than):} SELECT \textit{columns} FROM \textit{table name} WHERE \textit{column name} $>$ \textit{value};
    \item \textbf{Syntax (Greater than or equal):} SELECT \textit{columns} FROM \textit{table name} WHERE \textit{column name} $>=$ \textit{value};


    \bigskip

    \underline{\textbf{Examples:}}

    \bigskip

    \begin{lstlisting}[language=SQL]
    SELECT first_name, last_name FROM users WHERE date_of_birth < '1998-12-01';


    SELECT title AS "Book Title", author AS Author FROM books WHERE year_released <= 2015;


    SELECT name, description FROM products WHERE price > 9.99;


    SELECT title FROM movies WHERE release_year >= 2000;
    \end{lstlisting}

\end{itemize}

\bigskip

\section{Exercise 4}

\bigskip

\begin{itemize}
    \item Solution included in \textit{exercise\_4.sql}
\end{itemize}

\bigskip

\section{Searching Within a Set of Values}

\bigskip

\begin{itemize}
    \item Returns results with matching sets of values in a columns
    \item Is similar to Python's \textit{x in [Value1, value2, ....]}
    \item \textbf{Syntax:} SELECT \textit{columns} FROM \textit{table name} WHERE \textit{column name} IN (\textit{value 1}, \textit{value 2}, ...);
    \item \textbf{Syntax (Negation):} SELECT \textit{columns} FROM \textit{table name} WHERE \textit{column name} NOT IN (\textit{value 1}, \textit{value 2}, ...);

    \bigskip

    \underline{\textbf{Examples:}}

    \bigskip

    \begin{lstlisting}[language=SQL]
    SELECT name FROM islands WHERE id IN (4, 8, 15, 16, 23, 42);


    SELECT * FROM products WHERE category IN ("eBooks", "Books", "Comics");


    SELECT title FROM courses WHERE topic IN ("JavaScript", "Databases", "CSS");


    SELECT * FROM campaigns WHERE medium IN ("email", "blog", "ppc");


    SELECT * FROM products WHERE category NOT IN ("Electronics");


    SELECT title FROM courses WHERE topic NOT IN ("SQL", "NoSQL");
    \end{lstlisting}
\end{itemize}

\bigskip

\section{Exercise 5}

\bigskip

\begin{itemize}
    \item Solution included in \textit{exercise\_5.sql}
\end{itemize}

\bigskip

\section{Searching Within a Range of Values}

\bigskip

\begin{itemize}
    \item Returns results between \textit{lesser value} and \textit{greater value}
    \item \textbf{Syntax:} SELECT \textit{columns} FROM \textit{table name} WHERE \textit{column name} BETWEEN \textit{lesser value} AND \textit{greater value};

    \bigskip

    \underline{\textbf{Examples:}}

    \bigskip

    \begin{lstlisting}[language=SQL]
    SELECT * FROM movies WHERE release_year BETWEEN 2000 AND 2010;


    SELECT name, description FROM products WHERE price BETWEEN 9.99 AND 19.99;


    SELECT name, appointment_date FROM appointments WHERE appointment_date BETWEEN "2015-01-01" AND "2015-01-07";
    \end{lstlisting}
\end{itemize}

\bigskip

\section{Exercise 6}

\bigskip

\begin{itemize}
    \item Solution included in \textit{exercise\_6.sql}
\end{itemize}

\bigskip

\section{Finding Data that Matches a Pattern}

\bigskip

\begin{itemize}
    \item LIKE operator
    \begin{itemize}
        \item Is used inside of \textit{WHERE} clause to match a pattern
        \item \textbf{Syntax:} SELECT \textit{columns} FROM \textit{table name} WHERE \textit{column name} LIKE \textit{pattern};
        \item Can be used to make search \underline{case insensitive}

    \begin{lstlisting}[language=SQL]
    SELECT title FROM books WHERE title LIKE "Harry Potter";
    // returns items like 'Harry potter', 'harry potter'

    \end{lstlisting}
    \end{itemize}
    \item LIKE operator with wild card \%
    \begin{itemize}
        \item Works to match zero or more unspecified characters
        \item works the same as `\*' in regex
    \end{itemize}


    \begin{lstlisting}[language=SQL]
    SELECT title FROM books WHERE title LIKE "Harry Potter%Fire";
    // returns items like 'Harry Potter and Dragon Fire', 'Harry Potter and Fire', 'Harry Potter Rising Fire'

    SELECT title FROM movies WHERE title LIKE "Alien%";
    // Returns items like 'Alien attack', 'Alien', "Alienate"


    SELECT * FROM contacts WHERE first_name LIKE "%drew";
    // Returns items like 'tigerdrew', 'mountaindrew', 'morning drew', 'andrew'


    SELECT * FROM books WHERE title LIKE "%Brief History%";
    // Returns items like 'Canadian Brief History Channel', 'Brief History'
    \end{lstlisting}
\end{itemize}

\bigskip

\section{Exercise 7}

\bigskip

\begin{itemize}
    \item Solution included in \textit{exercise\_7.sql}
\end{itemize}

\bigskip

\section{Filtering Out or Finding Missing Information}

\bigskip

\begin{itemize}
    \item Using IS NULL
    \begin{itemize}
        \item Is used in WHERE
        \item Retrieve rows with information missing.
        \item \textbf{Syntax:} SELECT \textit{columns} FROM \textit{table name} WHERE \textit{column name} IS NULL;
    \end{itemize}
    \item Using IS NOT NULL
    \begin{itemize}
        \item \textbf{Syntax:} SELECT \textit{columns} FROM \textit{table name} WHERE \textit{column name} IS NOT NULL;
    \end{itemize}

    \bigskip

    \underline{\textbf{Example:}}

    \bigskip

    \begin{lstlisting}[language=SQL]
    SELECT address FROM records WHERE address IS NOT NULL;
    \end{lstlisting}
\end{itemize}

\bigskip

\section{Exercise 8}

\bigskip

\begin{itemize}
    \item Solution included in \textit{exercise\_8.sql}
\end{itemize}

\bigskip

\section{Review \& Practice with SQL Playgrounds}

\bigskip

\section{Quiz 1}

\bigskip

\begin{enumerate}[1.]
    \item

    Which keyword could you use to rewrite this query in a shorter form?

    \bigskip

    \begin{lstlisting}[language=SQL]
    SELECT <columns> FROM <table> WHERE <column 1> = <value 1> OR <column 1> = <value 2> OR <column 1> = <value 3>;
    \end{lstlisting}

    \begin{enumerate}[A.]
        \item ALL
        \item BETWEEN
        \item IN
    \end{enumerate}

    \bigskip

    \textbf{Answer:} C

    \item

    Please fill in the correct answer in each blank provided below.

    \bigskip

    I want to categorize products by price on a website. Cheap is defined by the
    prices from 0.01 and 9.99. Enter the missing keywords.

    \bigskip

    \begin{lstlisting}[language=SQL]
    SELECT name, description FROM products WHERE price ____  0.01 ____ 9.99;
    \end{lstlisting}

    \bigskip

    \textbf{Answer:} BETWEEN, AND

    \item

    Imagine you wanted to retrieve all appointments in for the upcoming week. Monday
    is 7th October 2019 and Friday is 11th October 2019.

    \bigskip

    Which query is the correct one to use?

    \bigskip

    \begin{enumerate}[A.]
        \item SELECT * FROM appointments WHERE day BETWEEN "2019-10-07" AND "2019-10-11";
        \item SELECT * FROM appointments WHERE day > "2019-10-07" AND day < "2019-10-11";
    \end{enumerate}

    \bigskip

    \textbf{Answer:} A

    \item

    What's missing from this query to find all contacts without a phone number?

    \bigskip

    \begin{lstlisting}[language=SQL]
    SELECT * FROM contacts WHERE phone ___________ NULL;
    \end{lstlisting}

    \bigskip

    \begin{enumerate}[A.]
        \item IS NOT
        \item IS
    \end{enumerate}

    \bigskip

    \textbf{Answer:} B

    \item

    Please fill in the correct answer in each blank provided below.

    \bigskip

    \bigskip

    \begin{lstlisting}[language=SQL]
    SELECT <columns> FROM <table> WHERE <column> ____  <value>;
    \end{lstlisting}

    \bigskip

    \textbf{Answer:} =

    \item

    Please fill in the correct answer in each blank provided below.

    \bigskip

    Enter the inequality operator:

    \bigskip

    \begin{lstlisting}[language=SQL]
    SELECT <columns> FROM <table> WHERE <column> ____  <value>;
    \end{lstlisting}

    \bigskip

    \textbf{Answer:} =

    \item

    Which operator is the greater than operator?

    \bigskip

    \begin{enumerate}[A.]
        \item <=
        \item >
        \item >=
        \item <
    \end{enumerate}

    \bigskip

    \textbf{Answer:} B

    \item

    You have a table full of words. You want to find all words ending with the
    `tion' at the end.

    \bigskip

    Which query would most likely display the correct results?

    \bigskip

    \begin{enumerate}[A.]
        \item SELECT word FROM words WHERE word LIKE "\%tion";
        \item SELECT word FROM words WHERE word LIKE "tion\%";
        \item SELECT word FROM words WHERE word LIKE "\%tion\%";
        \item SELECT word FROM words WHERE word LIKE "tion";
    \end{enumerate}

    \bigskip

    \textbf{Answer:} B

    \item

    You have a table full of words. You want to find all words ending with the tion
    at the end.

    \bigskip

    Which query would most likely display the correct results?

    \bigskip

    \begin{enumerate}[A.]
        \item SELECT word FROM words WHERE word LIKE "\%tion";
        \item SELECT word FROM words WHERE word LIKE "tion\%";
        \item SELECT word FROM words WHERE word LIKE "\%tion\%";
        \item SELECT word FROM words WHERE word LIKE "tion";
    \end{enumerate}

    \bigskip

    \textbf{Answer:} A

    \item

    You have a table full of words. You want to find all words ending with the tion
    at the end.

    \bigskip

    Which query would most likely display the correct results?

    \bigskip

    \begin{enumerate}[A.]
        \item SELECT word FROM words WHERE word LIKE "\%tion";
        \item SELECT word FROM words WHERE word LIKE "tion\%";
        \item SELECT word FROM words WHERE word LIKE "\%tion\%";
        \item SELECT word FROM words WHERE word LIKE "tion";
    \end{enumerate}

    \bigskip

    \textbf{Answer:} A

    \item

    Please fill in the correct answer in each blank provided below.

    \bigskip

    \begin{lstlisting}[language=SQL]
    SELECT <columns> FROM <table> ___  <condition>;
    \end{lstlisting}

    \bigskip

    \textbf{Answer:} WHERE

    \item

    Which operator is the less than operator?

    \bigskip

    \begin{enumerate}[A.]
        \item <
        \item >=
        \item >
        \item <=
    \end{enumerate}

    \bigskip

    \textbf{Answer:} A

    \item

    Please fill in the correct answer in each blank provided below

    \bigskip

    Fill in the missing operator. Today is 19th October 2019. I want to find all
    matches happening today and in the future.

    \bigskip

    \begin{lstlisting}[language=SQL]
    SELECT * FROM football_matches WHERE event_date ___ "2019-10-19";
    \end{lstlisting}

    \bigskip

    \textbf{Answer:} >=

    \item

    What's missing from this query to find all contacts with an email address
    present?

    \bigskip

    \begin{lstlisting}[language=SQL]
    SELECT * FROM football_matches WHERE event_date ___ "2019-10-19";
    \end{lstlisting}

    \begin{enumerate}[A.]
        \item IS NOT
        \item NOT
    \end{enumerate}

    \bigskip

    \textbf{Answer:} A

    \item

    What's missing from this query to find all contacts with an email address
    present?

    \bigskip

    \begin{lstlisting}[language=SQL]
    SELECT * FROM football_matches WHERE event_date ___ "2019-10-19";
    \end{lstlisting}

    \bigskip

    \begin{enumerate}[A.]
        \item IS NOT
        \item NOT
    \end{enumerate}

    \bigskip

    \textbf{Answer:} A

    \item

    Please fill in the correct answer in each blank provided below.

    \bigskip

    If I wanted to return rows that match both conditions, which keyword would I use?

    \bigskip

    \begin{lstlisting}[language=SQL]
    SELECT <columns> FROM <table> WHERE <condition 1> _____  <condition 2>;
    \end{lstlisting}

    \bigskip

    \textbf{Answer:} AND

    \item

    Please fill in the correct answer in each blank provided below.

    \bigskip

    If I wanted to return rows that match either conditions, which keyword would I use?

    \bigskip

    \begin{lstlisting}[language=SQL]
    SELECT <columns> FROM <table> WHERE <condition 1> _____  <condition 2>;
    \end{lstlisting}

    \bigskip

    \textbf{Answer:} OR

    \item

    What's the way to represent a missing value in SQL?

    \bigskip

    If I wanted to return rows that match either conditions, which keyword would I use?

    \bigskip

    \begin{lstlisting}[language=SQL]
    SELECT <columns> FROM <table> WHERE <condition 1> _____  <condition 2>;
    \end{lstlisting}

    \bigskip

    \begin{enumerate}[A.]
        \item MISSING
        \item EMPTY
        \item NIL
        \item NULL
    \end{enumerate}

    \bigskip

    \textbf{Answer:} D

\end{enumerate}

\end{document}