\documentclass[12pt]{article}
\usepackage[margin=2.5cm]{geometry}
\usepackage{enumerate}
\usepackage{amsfonts}
\usepackage{amsmath}
\usepackage{fancyhdr}
\usepackage{amsmath}
\usepackage{amssymb}
\usepackage{amsthm}
\usepackage{mdframed}
\usepackage{graphicx}
\usepackage{subcaption}
\usepackage{adjustbox}
\usepackage{listings}
\usepackage{xcolor}
\usepackage{booktabs}
\usepackage[utf]{kotex}
\usepackage{hyperref}

\definecolor{codegreen}{rgb}{0,0.6,0}
\definecolor{codegray}{rgb}{0.5,0.5,0.5}
\definecolor{codepurple}{rgb}{0.58,0,0.82}
\definecolor{backcolour}{rgb}{0.95,0.95,0.92}

\lstdefinestyle{mystyle}{
    backgroundcolor=\color{backcolour},
    commentstyle=\color{codegreen},
    keywordstyle=\color{magenta},
    numberstyle=\tiny\color{codegray},
    stringstyle=\color{codepurple},
    basicstyle=\ttfamily\footnotesize,
    breakatwhitespace=false,
    breaklines=true,
    captionpos=b,
    keepspaces=true,
    numbers=left,
    numbersep=5pt,
    showspaces=false,
    showstringspaces=false,
    showtabs=false,
    tabsize=1
}

\lstset{style=mystyle}

\pagestyle{fancy}
\renewcommand{\headrulewidth}{0.4pt}
\lhead{Team Treehouse}
\rhead{Reporting with SQL Part 1 Notes}

\begin{document}
\title{Reporting with SQL Part 1 Notes}
\author{Team Treehouse}
\maketitle

\bigskip

\section{Overview}

\bigskip

\section{Retrieving Results in a Particular Order}

\bigskip

\begin{itemize}
    \item ORDER BY
    \begin{itemize}
        \item Allows to retrieve items in a particular order
        \item \textbf{Syntax:} SELECT * FROM \textit{table name} ORDER BY \textit{column name} [ASC|DESC];
    \end{itemize}
\end{itemize}

\bigskip

\section{Retrieving Results in a Particular Order}

\bigskip

\begin{itemize}
    \item \textbf{Syntax:} SELECT * FROM \textit{table name} ORDER BY \textit{column name} [ASC|DESC], \textit{column 2 name} [ASC|DESC], ..., \textit{column n name} [ASC|DESC];

    \bigskip

    \underline{\textbf{Example:}}

    \bigskip

    \begin{lstlisting}[language=SQL]
    SELECT * FROM books ORDER BY title ASC;


    SELECT * FROM products WHERE name = "Sonic T-Shirt" ORDER BY stock_count DESC;


    SELECT * FROM users ORDER BY signed_up_on DESC;


    SELECT * FROM countries ORDER BY population DESC;


    SELECT * FROM books ORDER BY    genre ASC,
                                    title ASC;


    SELECT * FROM books ORDER BY    genre ASC,
                                    year_published DESC;


    SELECT * FROM users ORDER BY    last_name ASC,
                                    first_name ASC;
    \end{lstlisting}
\end{itemize}


\bigskip

\section{Exercise 1}

\bigskip

\begin{itemize}
    \item Solution included in \textit{exercise\_1.sql}
\end{itemize}

\bigskip

\section{Retrieving Results in a Particular Order}

\bigskip

\begin{itemize}
    \item LIMIT
    \begin{itemize}
        \item \textbf{Syntax (SQLite, PostgreSQL, MySQL):} SELECT \textit{columns name} FROM \textit{table name} ... LIMIT \textit{\# of rows};
        \item Must be placed at the end
    \end{itemize}
\end{itemize}

\bigskip

\section{Exercise 2}

\bigskip

\begin{itemize}
    \item Solution included in \textit{exercise\_2.sql}
\end{itemize}

\bigskip

\section{Paging Thrugh Results}

\bigskip

\begin{itemize}
    \item OFFSET
    \begin{itemize}
        \item \textbf{Syntax 1 (SQLite, PostgreSQL, MySQL):} SELECT <columns> FROM \textit{table name} LIMIT \textit{\# of rows} OFFSET \textit{skipped rows};
        \item \textbf{Syntax 2 (SQLite, PostgreSQL, MySQL):} SELECT <columns> FROM \textit{table name} LIMIT \textit{skipped rows}, \textit{\# of rows};
        \item Is based on number of rows, and NOT by pages (i.e. LIMIT 10 OFFSET 10 is on page 2)
        \item Is useful when creating multi-page reports, blog archive, or listing search results
    \end{itemize}
\end{itemize}


\bigskip

\section{Exercise 3}

\bigskip

\begin{itemize}
    \item Solution included in \textit{exercise\_3.sql}
\end{itemize}


\bigskip

\section{Practice Session}

\bigskip

\begin{itemize}
    \item OFFSET
    \begin{itemize}
        \item \textbf{Syntax 1 (SQLite, PostgreSQL, MySQL):} SELECT <columns> FROM \textit{table name} LIMIT \textit{\# of rows} OFFSET \textit{skipped rows};
        \item \textbf{Syntax 2 (SQLite, PostgreSQL, MySQL):} SELECT <columns> FROM \textit{table name} LIMIT \textit{skipped rows}, \textit{\# of rows};
        \item Is based on number of rows, and NOT by pages (i.e. LIMIT 10 OFFSET 10 is on page 2)
        \item Is useful when creating multi-page reports, blog archive, or listing search results
    \end{itemize}
\end{itemize}


\bigskip

\section{Quiz 1}

\bigskip

\begin{enumerate}[1.]
    \item

    Please fill in the correct answer in each blank provided below.

    \bigskip

    I would type in \_\_\_  at the end of an`ORDER BY` clause to order in ascending order.

    \bigskip

    \textbf{Answer:} ASC

    \item

    Can you guess what the following query would do?

    \bigskip

    \begin{lstlisting}[language=SQL]
    SELECT * FROM passport_holders WHERE name = "Lauren" LIMIT 0, 50;
    \end{lstlisting}

    \bigskip

    \begin{enumerate}[A.]
        \item Return the first 50 results for the query SELECT \* FROM passport\_holders WHERE name = "Lauren"
        \item Return the last 50 results for the query SELECT \* FROM passport\_holders WHERE name = "Lauren"
    \end{enumerate}

    \textbf{Answer:} A


    \item

    Please fill in the correct answer in each blank provided below.

    \bigskip

    I would type in \_\_\_  at the end of an`ORDER BY` clause to order in ascending order.

    \bigskip

    \textbf{Answer:} DESC

    \item

    Back in the teachers notes I included a reference to how other databases
    LIMIT results. Which of the following is a valid way to order results in
    another relational database system?

    \bigskip

    \begin{enumerate}[A.]
        \item SELECT name FROM people TOP 50;
        \item SELECT TOP 50 name FROM people;
    \end{enumerate}

    \textbf{Answer:} B

    \item

    When using the ORDER BY keywords, what is the default order that the results
    will appear in?

    \bigskip

    For example: SELECT name, population FROM cities ORDER BY name;

    \bigskip

    \begin{enumerate}[A.]
        \item Descending order
        \item Ascending order
        \item The order it was entered in the database
    \end{enumerate}

    \textbf{Answer:} B

\end{enumerate}


\end{document}