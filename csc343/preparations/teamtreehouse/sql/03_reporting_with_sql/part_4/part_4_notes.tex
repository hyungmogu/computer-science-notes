\documentclass[12pt]{article}
\usepackage[margin=2.5cm]{geometry}
\usepackage{enumerate}
\usepackage{amsfonts}
\usepackage{amsmath}
\usepackage{fancyhdr}
\usepackage{amsmath}
\usepackage{amssymb}
\usepackage{amsthm}
\usepackage{mdframed}
\usepackage{graphicx}
\usepackage{subcaption}
\usepackage{adjustbox}
\usepackage{listings}
\usepackage{xcolor}
\usepackage{booktabs}
\usepackage[utf]{kotex}
\usepackage{hyperref}

\definecolor{codegreen}{rgb}{0,0.6,0}
\definecolor{codegray}{rgb}{0.5,0.5,0.5}
\definecolor{codepurple}{rgb}{0.58,0,0.82}
\definecolor{backcolour}{rgb}{0.95,0.95,0.92}

\lstdefinestyle{mystyle}{
    backgroundcolor=\color{backcolour},
    commentstyle=\color{codegreen},
    keywordstyle=\color{magenta},
    numberstyle=\tiny\color{codegray},
    stringstyle=\color{codepurple},
    basicstyle=\ttfamily\footnotesize,
    breakatwhitespace=false,
    breaklines=true,
    captionpos=b,
    keepspaces=true,
    numbers=left,
    numbersep=5pt,
    showspaces=false,
    showstringspaces=false,
    showtabs=false,
    tabsize=1
}

\lstset{style=mystyle}

\pagestyle{fancy}
\renewcommand{\headrulewidth}{0.4pt}
\lhead{Team Treehouse}
\rhead{Reporting with SQL Part 4 Notes}

\begin{document}
\title{Reporting with SQL Part 4 Notes}
\author{Team Treehouse}
\maketitle

\bigskip

\section{Differences Between Databases}

\bigskip

\section{Creating Up-to-the-minute Reports}

\bigskip

\begin{itemize}
    \item Writing Today's date
    \begin{itemize}
        \item SQLite
        \begin{itemize}
            \item \textbf{Syntax (Current Date):} DATE("now")
            \item \textbf{Syntax (Current Time):} TIME("now")
            \item \textbf{Syntax (Current Date Time):} DATETIME("NOW")
        \end{itemize}
        \item MySQL
        \item \begin{itemize}
            \item \textbf{Syntax (Current Date):} CURDATE()
            \item \textbf{Syntax (Current Time):} CURTIME()
            \item \textbf{Syntax (Current Date Time):} NOW()
        \end{itemize}
        \item Oracle and PostgreSQL
        \item \begin{itemize}
            \item \textbf{Syntax (Current Date):} CURRENT\_DATE
            \item \textbf{Syntax (Current Time):} CURRENT\_TIME
            \item \textbf{Syntax (Current Date Time):} CURRENT\_TIMESTAMP
        \end{itemize}
    \end{itemize}

    \bigskip

    \underline{\textbf{Exmaple:}}

    \bigskip

    \begin{lstlisting}[language=SQL]
    SELECT * FROM orders WHERE status = "placed" AND ordered_on = DATE("now");
    \end{lstlisting}

\end{itemize}

\bigskip

\section{Exercise 1}

\bigskip

\begin{itemize}
    \item Solution included in \textit{exercise\_1.sql}
\end{itemize}

\bigskip

\section{Calculating Dates}

\bigskip

\begin{itemize}
    \item \textbf{Syntax 1:} DATE(\textit{time string})
    \begin{itemize}
        \item NOTE: the only recognized time string format is `YYYY-MM-DD'
    \end{itemize}
    \item \textbf{Syntax 2:} DATE(\textit{time string}, \textit{modifier}, ...)

    \bigskip

    \underline{\textbf{Example:}}

    \bigskip

    \begin{lstlisting}[language=SQL]
    DATE("2016-02-01", "-7 days") # 2016-01-25


    DATE("2016-02-01", "+7 days") # 2016-02-08


    DATE("2016-02-01", "+7 months") # 2016-09-01


    DATE("2016-02-01", "+7 years") # 2023-02-01


    SELECT COUNT(*) FROM orders WHERE ordered_on
        BETWEEN DATE("now", "-7 days")
        AND DATE("now", "-1 day");
    \end{lstlisting}

\end{itemize}

\bigskip

\section{Exercise 2}

\bigskip

\begin{itemize}
    \item Solution included in \textit{exercise\_2.sql}
\end{itemize}

\bigskip

\section{Formatting Dates For Reporting}

\bigskip

\begin{itemize}
    \item From datetime string to date string
    \begin{itemize}
        \item \textbf{Syntax:} DATE(\textit{datetime string}) $\to$ \textit{date string}
    \end{itemize}
    \item From datetime string to time string
    \begin{itemize}
        \item \textbf{Syntax:} TIME(\textit{datetime string}) $\to$ \textit{date string}
    \end{itemize}

    \bigskip

    \underline{\textbf{Example:}}

    \bigskip

    \begin{lstlisting}[language=SQL]
    DATE("2015-04-01 23:12:01") # "2015-04-01"


    TIME("2015-04-01 23:12:01") # "23:12:01"
    \end{lstlisting}


    \item STRFTIME()
    \begin{itemize}
        \item Formats date string / time string / datetime string into human readible format
        \item \textbf{Syntax:} STRFTIME(\textit{format string}, \textit{time string}, \textit{modifier})
    \end{itemize}

    \bigskip

    \underline{\textbf{Example:}}

    \bigskip

    \begin{lstlisting}[language=SQL]
    STRFTIME("%d/%m/%Y", "2015-04-01 23:12:01") # 01/04/2015


    STRFTIME("%d/%m/%Y", "2015-04-01 23:12:01", "+1 year") # 01/04/2016


    SELECT *, STRFTIME("%d/%m/%Y", ordered_on) AS UK_date FROM orders;
    \end{lstlisting}
\end{itemize}


\bigskip

\section{Exercise 3}

\bigskip

\begin{itemize}
    \item Solution included in \textit{exercise\_3.sql}
\end{itemize}

\bigskip

\section{Practice Session}



\bigskip

\section{Quiz 1}

\bigskip

\begin{enumerate}[1.]
    \item

    Counting all values in a specific column will count all rows including empty or NULL values.

    \bigskip

    \begin{enumerate}[A.]
        \item True
        \item False
    \end{enumerate}

    \bigskip

    \textbf{Answer:} B

    \item

    LENGTH() is described as a:

    \bigskip

    \begin{enumerate}[A.]
        \item Operator
        \item Keyword
        \item Function
    \end{enumerate}

    \bigskip

    \textbf{Answer:} C


    \item

    What function would I use to get the average of numeric values in a column?


    \bigskip

    \begin{enumerate}[A.]
        \item AVG();
        \item AVERAGE();

    \end{enumerate}

    \bigskip

    \textbf{Answer:} A

    \item

    Please fill in the correct answer in each blank provided below.

    \bigskip

    The \_\_\_  keyword is used after the ORDER BY clause to order dates from the most recent to the furthest back in time.

    \bigskip

    \textbf{Answer:} DESC

    \item

    What will this function return?


    \bigskip

    \begin{lstlisting}[language=SQL]
    TIME("2016-10-11 23:59:00", "+2 minutes")
    \end{lstlisting}

    \bigskip

    \begin{enumerate}[A.]
        \item "2016-10-12 00:01:00"
        \item "2016-10-12"
        \item "00:01:00"
    \end{enumerate}

    \bigskip

    \textbf{Answer:} C

    \item

    Please fill in the correct answer in each blank provided below.

    \bigskip

    The \_\_\_  function is used to make strings uppercase.

    \bigskip

    \textbf{Answer:} UPPER

    \item

    Which of the following is the correct way to use the STRFTIME() function?

    \bigskip

    \begin{enumerate}[A.]
        \item STRFTIME("3118-12-01", "\%Y-\%m-\%d", "+3 years")
        \item STRFTIME("\%Y-\%m-\%d", "3118-12-01")
        \item STRFTIME("3118-12-01", "+3 years", "\%Y-\%m-\%d")

    \end{enumerate}

    \bigskip

    \textbf{Answer:} B

    \item

    What will the following function return?

    \bigskip

    \begin{enumerate}[A.]
        \item "2016-02-10"
        \item "2017-02-10"
        \item "2017-02-10 22:21:23"
    \end{enumerate}

    \bigskip

    \textbf{Answer:} B

    \item

    Please fill in the correct answer in each blank provided below.

    \bigskip

    \_\_\_  BY is used to aggregate rows together.

    \bigskip

    \textbf{Answer:} GROUP

    \item

    Please fill in the correct answer in each blank provided below.

    \bigskip

    \_\_\_  is the operator you'd use to add two pieces of text together in SQLite/SQL Playground.

    \bigskip

    \textbf{Answer:} \textbar \textbar

    \item

    What will this date and modifier produce?

    \bigskip

    \begin{lstlisting}[language=SQL]
    DATE("now", "+1 day")
    \end{lstlisting}

    \begin{enumerate}[A.]
        \item Yesterday
        \item Today
        \item Tomorrow
    \end{enumerate}

    \bigskip

    \textbf{Answer:} C

    \item

    If today was the 8th July 2018, how would I write that date in SQLite/SQL Playground?

    \bigskip

    \begin{enumerate}[A.]
        \item "2018-08-07"
        \item "2018-07-08"
        \item "07-08-2018"
        \item "08-07-2018"
    \end{enumerate}

    \bigskip

    \textbf{Answer:} B

    \item

    Please fill in the correct answer in each blank provided below.

    \bigskip

    \_\_\_  is a keyword that can be used in conjunction with LIMIT to page through results.

    \bigskip

    \textbf{Answer:} OFFSET


    \item

    What is the correct way of using the REPLACE() function for replacing "javascript" with "JavaScript" ?

    \bigskip

    \begin{enumerate}[A.]
        \item SELECT REPLACE("javascript", "JavaScript", "I can't capitalize javascript correctly!");
        \item SELECT REPLACE("I can't capitalize javascript correctly!", "javascript", "JavaScript");
        \item SELECT REPLACE("JavaScript", "javascript", "I can't capitalize javascript correctly!");
        \item SELECT REPLACE("I can't capitalize javascript correctly!", "JavaScript", "javascript");
    \end{enumerate}

    \bigskip

    \textbf{Answer:} B

    \item

    The - is a(n):

    \bigskip

    \begin{enumerate}[A.]
        \item Operator
        \item Keyword
        \item Function

    \end{enumerate}

    \bigskip

    \textbf{Answer:} A



\end{enumerate}






\end{document}