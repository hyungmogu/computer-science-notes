\documentclass[12pt]{article}
\usepackage[margin=2.5cm]{geometry}
\usepackage{enumerate}
\usepackage{amsfonts}
\usepackage{amsmath}
\usepackage{fancyhdr}
\usepackage{amsmath}
\usepackage{amssymb}
\usepackage{amsthm}
\usepackage{mdframed}
\usepackage{graphicx}
\usepackage{subcaption}
\usepackage{adjustbox}
\usepackage{listings}
\usepackage{xcolor}
\usepackage{booktabs}
\usepackage[utf]{kotex}
\usepackage{hyperref}

\definecolor{codegreen}{rgb}{0,0.6,0}
\definecolor{codegray}{rgb}{0.5,0.5,0.5}
\definecolor{codepurple}{rgb}{0.58,0,0.82}
\definecolor{backcolour}{rgb}{0.95,0.95,0.92}

\lstdefinestyle{mystyle}{
    backgroundcolor=\color{backcolour},
    commentstyle=\color{codegreen},
    keywordstyle=\color{magenta},
    numberstyle=\tiny\color{codegray},
    stringstyle=\color{codepurple},
    basicstyle=\ttfamily\footnotesize,
    breakatwhitespace=false,
    breaklines=true,
    captionpos=b,
    keepspaces=true,
    numbers=left,
    numbersep=5pt,
    showspaces=false,
    showstringspaces=false,
    showtabs=false,
    tabsize=1
}

\lstset{style=mystyle}

\pagestyle{fancy}
\renewcommand{\headrulewidth}{0.4pt}
\lhead{Team Treehouse}
\rhead{Querying Relational Databases Part 3 Notes}

\begin{document}
\title{Querying Relational Databases Part 3 Notes}
\author{Team Treehouse}
\maketitle

\bigskip

\section{One to Many Relationships}

\bigskip

\begin{itemize}
    \item Is most common type of relationship
    \item Is case where there is one table with many foreignkeys and not the other
    \item Foreign Key is always the many side

    \begin{center}
    \includegraphics[width=0.8\linewidth]{images/part_3_notes_1.png}
    \end{center}
\end{itemize}

\bigskip

\section{Many to Many Relationships}

\bigskip

\begin{itemize}
    \item mean that a record in one table can relate to many other records in
    another table, and one record from the second table can also relate back
    to many records in the first table
    \item Creates a junction table between two tables
    \begin{itemize}
        \item is a combination of two primary keys
        \item junction table is not a third table (there are still 2 tables)
    \end{itemize}

    \begin{center}
    \includegraphics[width=0.8\linewidth]{images/part_3_notes_2.png}
    \end{center}
\end{itemize}

\bigskip

\section{One to One Relationships}

\bigskip

\begin{itemize}
    \item Means that a row in one table can only relate to one row in the table
    on the other side of their relationship and vice versa
    \item Is least common relationship
    \item Is for boosting performance with infrequently used columns
    \item Is for working with 3rd party databases
    \begin{itemize}
        \item to extend the existing database
    \end{itemize}

    \begin{center}
    \includegraphics[width=0.8\linewidth]{images/part_3_notes_3.png}
    \end{center}
\end{itemize}

\bigskip

\section{Modeling Table Relationships}

\bigskip

\begin{itemize}
    \item Notations
    \begin{itemize}
        \item This is really cool!!
    \end{itemize}

    \begin{center}
    \includegraphics[width=0.8\linewidth]{images/part_3_notes_4.png}
    \includegraphics[width=0.8\linewidth]{images/part_3_notes_5.png}
    \includegraphics[width=0.8\linewidth]{images/part_3_notes_6.png}
    \includegraphics[width=0.8\linewidth]{images/part_3_notes_7.png}
    \end{center}
\end{itemize}

\bigskip

\section{Quiz 1}

\bigskip


\begin{enumerate}[1.]
    \item

    Which relationship type is the most common in a relational database?

    \bigskip

    \begin{enumerate}[A.]
        \item Many to Many
        \item One to Many
        \item One to One
        \item One to Ten
    \end{enumerate}

    \bigskip

    \textbf{Answer:} B

    \item

    In the diagram shown, this notation style is called:

    \bigskip

    \begin{center}
    \includegraphics[width=0.4\linewidth]{images/part_3_notes_8.png}
    \end{center}

    \begin{enumerate}[A.]
        \item Chen Notation
        \item Stick Figure Notation
        \item UML Notation
        \item Crow's Foot Notation
    \end{enumerate}

    \bigskip

    \textbf{Answer:} D

    \item

    What do we do to handle a Many to Many relationship between tables during
    the design process?

    \bigskip

    \begin{enumerate}[A.]
        \item Add a Foreign Key to the table on the left
        \item Add a Foreign Key to the table on the right
        \item Add a third table between the original two tables
        \item Link the two tables with an extra index
    \end{enumerate}

    \bigskip

    \textbf{Answer:} C

    \item

    In the diagram shown, what is the relationship between the two tables?

    \bigskip

    \begin{center}
    \includegraphics[width=0.4\linewidth]{images/part_3_notes_8.png}
    \end{center}

    \begin{enumerate}[A.]
        \item Many to Many
        \item One to Many
        \item One to One
    \end{enumerate}

    \bigskip

    \textbf{Answer:} B

    \item

    Which of the following is NOT a type of table relationship?

    \bigskip

    \begin{enumerate}[A.]
        \item One to Many
        \item Many to Many
        \item One to One
        \item Slim to None
    \end{enumerate}

    \bigskip

    \textbf{Answer:} D

\end{enumerate}

\end{document}