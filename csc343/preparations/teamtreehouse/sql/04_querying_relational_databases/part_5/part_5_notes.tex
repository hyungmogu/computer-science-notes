\documentclass[12pt]{article}
\usepackage[margin=2.5cm]{geometry}
\usepackage{enumerate}
\usepackage{amsfonts}
\usepackage{amsmath}
\usepackage{fancyhdr}
\usepackage{amsmath}
\usepackage{amssymb}
\usepackage{amsthm}
\usepackage{mdframed}
\usepackage{graphicx}
\usepackage{subcaption}
\usepackage{adjustbox}
\usepackage{listings}
\usepackage{xcolor}
\usepackage{booktabs}
\usepackage[utf]{kotex}
\usepackage{hyperref}

\definecolor{codegreen}{rgb}{0,0.6,0}
\definecolor{codegray}{rgb}{0.5,0.5,0.5}
\definecolor{codepurple}{rgb}{0.58,0,0.82}
\definecolor{backcolour}{rgb}{0.95,0.95,0.92}

\lstdefinestyle{mystyle}{
    backgroundcolor=\color{backcolour},
    commentstyle=\color{codegreen},
    keywordstyle=\color{magenta},
    numberstyle=\tiny\color{codegray},
    stringstyle=\color{codepurple},
    basicstyle=\ttfamily\footnotesize,
    breakatwhitespace=false,
    breaklines=true,
    captionpos=b,
    keepspaces=true,
    numbers=left,
    numbersep=5pt,
    showspaces=false,
    showstringspaces=false,
    showtabs=false,
    tabsize=1
}

\lstset{style=mystyle}

\pagestyle{fancy}
\renewcommand{\headrulewidth}{0.4pt}
\lhead{Team Treehouse}
\rhead{Querying Relational Databases Part 5 Notes}

\begin{document}
\title{Querying Relational Databases Part 5 Notes}
\author{Team Treehouse}
\maketitle

\bigskip

\section{What are Set Operations?}

\bigskip

\begin{itemize}
    \item Combine or limit results using two or more datasets
    \item has 4 set operations
    \begin{itemize}
        \item UNION / UNION ALL
        \item INTERSET
        \item EXCEPT
    \end{itemize}
\end{itemize}

\bigskip

\section{Union Operations}

\bigskip

\begin{itemize}
    \item Stacks data vertically

    \begin{center}
    \includegraphics[width=0.8\linewidth]{images/part_5_notes_2.png}
    \end{center}

    \item has to have matching number of columns
    \item \textbf{Syntax:} \textit{query 1} UNION \textit{query 2}

    \bigskip

    \underline{\textbf{Example:}}

    \bigskip

    \begin{lstlisting}[language=SQL]
    SELECT MakeID, MakeName FROM Make UNION SELECT ForeignMakeID, MakeName FROM ForeignMake;
    \end{lstlisting}

    \bigskip

    \begin{center}
    \includegraphics[width=0.8\linewidth]{images/part_5_notes_1.png}
    \end{center}

    \bigskip

    \underline{\textbf{Example 2:}}

    \bigskip

    \begin{lstlisting}[language=SQL]
    SELECT MakeID, MakeName FROM Make
        WHERE MakeName < "D"
    UNION
    SELECT ForeignMakeID, MakeName FROM ForeignMake
        WHERE MakeName < "D"
        ORDER BY MakeName;
    \end{lstlisting}

\end{itemize}

\bigskip

\section{Union All Operations}

\bigskip

\begin{itemize}
    \item Is the same as union but does not eliminate duplicates
    \item \textbf{Syntax:} \textit{query 1} UNION ALL \textit{query 2}
\end{itemize}

\bigskip

\section{Intersect}

\bigskip

\begin{center}
\includegraphics[width=0.8\linewidth]{images/part_5_notes_3.png}
\end{center}


\begin{itemize}
    \item Only returns results that exist in both
    \item Intersection is based on supplied columns
    \begin{itemize}
        \item multiple columns $\to$ intersection is based on intersecting values in those columns
    \end{itemize}
    \item \textbf{Syntax:} \textit{query 1} INTERSECT \textit{query 2}

    \bigskip

    \underline{\textbf{Example:}}

    \bigskip

    \begin{lstlisting}[language=SQL]
    SELECT MakeName FROM Make
        INTERSECT
    SELECT MakeName FROM  ForeignMake ORDER BY MakeName DESC;
    \end{lstlisting}

    \bigskip

    \begin{center}
    \includegraphics[width=\linewidth]{images/part_5_notes_4.png}
    \end{center}

    \bigskip

    \underline{\textbf{Example 2:}}

    \bigskip

    \begin{lstlisting}[language=SQL]
    SELECT MakeID MakeName FROM Make
        INTERSET
    SELECT ForeignMakeID, MakeName FROM  ForeignMake ORDER BY MakeName DESC; // <- Returns empty result
    \end{lstlisting}

    \begin{center}
    \includegraphics[width=\linewidth]{images/part_5_notes_5.png}
    \end{center}
\end{itemize}

\bigskip

\section{Except Operations}

\bigskip

\begin{center}
\includegraphics[width=0.8\linewidth]{images/part_5_notes_6.png}
\end{center}


\begin{itemize}
    \item \textbf{Syntax:} \textit{Query 1} EXCEPT \textit{Query 2}
    \item SQL accounts for all columns considered
    \item Except uses the same format as INTERSET but outputs only the records
    that are not in the latter table

    \bigskip

    \underline{\textbf{Example:}}

    \bigskip

    \begin{lstlisting}[language=SQL]
    SELECT ForeignMakeID, MakeName FROM ForeignMake EXCEPT SELECT MakeID, MakeName FROM Make; // shows only forien made goods
    \end{lstlisting}
\end{itemize}

\bigskip

\section{Set Operations Reivew}

\bigskip

\begin{enumerate}[1.]
    \item

    Which Set Operation is used to find and return values that exist in two
    different data sets?

    \begin{enumerate}[A.]
        \item EXCEPT
        \item UNION
        \item INTERSECT
        \item UNION ALL
    \end{enumerate}

    \bigskip

    \textbf{Answer:} C

    \item

    Which Operator eliminates duplicates while combining multiple data sets into
    one result set?

    \begin{enumerate}[A.]
        \item EXCEPT
        \item UNION ALL
        \item MERGE
        \item UNION
    \end{enumerate}

    \bigskip

    \textbf{Answer:} D

    \item

    It is valid to have fewer columns in the query that comes after the UNION
    operation.

    \begin{enumerate}[A.]
        \item True
        \item False
    \end{enumerate}

    \bigskip

    \textbf{Answer:} B

    \item

    Which of the following is NOT a SQL Set Operation?

    \begin{enumerate}[A.]
        \item UNION
        \item INTERSECT
        \item EXCEPT
        \item CONJOIN
    \end{enumerate}

    \bigskip

    \textbf{Answer:} D

    \item

    Which Set Operation is used to return only the results that are NOT in another
    table?

    \begin{enumerate}[A.]
        \item UNION ALL
        \item INTERSECT
        \item EXCEPT
        \item UNION
    \end{enumerate}

    \bigskip

    \textbf{Answer:} C
\end{enumerate}

\bigskip

\section{Review and Practice}

\bigskip

\section{Exercise 1}

\bigskip

\begin{itemize}
    \item Solution included in \textit{exercise\_1.sql}
\end{itemize}

\end{document}