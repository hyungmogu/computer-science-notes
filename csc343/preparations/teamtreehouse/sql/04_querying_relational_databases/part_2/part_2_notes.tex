\documentclass[12pt]{article}
\usepackage[margin=2.5cm]{geometry}
\usepackage{enumerate}
\usepackage{amsfonts}
\usepackage{amsmath}
\usepackage{fancyhdr}
\usepackage{amsmath}
\usepackage{amssymb}
\usepackage{amsthm}
\usepackage{mdframed}
\usepackage{graphicx}
\usepackage{subcaption}
\usepackage{adjustbox}
\usepackage{listings}
\usepackage{xcolor}
\usepackage{booktabs}
\usepackage[utf]{kotex}
\usepackage{hyperref}

\definecolor{codegreen}{rgb}{0,0.6,0}
\definecolor{codegray}{rgb}{0.5,0.5,0.5}
\definecolor{codepurple}{rgb}{0.58,0,0.82}
\definecolor{backcolour}{rgb}{0.95,0.95,0.92}

\lstdefinestyle{mystyle}{
    backgroundcolor=\color{backcolour},
    commentstyle=\color{codegreen},
    keywordstyle=\color{magenta},
    numberstyle=\tiny\color{codegray},
    stringstyle=\color{codepurple},
    basicstyle=\ttfamily\footnotesize,
    breakatwhitespace=false,
    breaklines=true,
    captionpos=b,
    keepspaces=true,
    numbers=left,
    numbersep=5pt,
    showspaces=false,
    showstringspaces=false,
    showtabs=false,
    tabsize=1
}

\lstset{style=mystyle}

\pagestyle{fancy}
\renewcommand{\headrulewidth}{0.4pt}
\lhead{Team Treehouse}
\rhead{Querying Relational Databases Part 2 Notes}

\begin{document}
\title{Querying Relational Databases Part 2 Notes}
\author{Team Treehouse}
\maketitle

\bigskip

\section{Unique Keys}

\bigskip

\begin{itemize}
    \item Is configured so that no value can be repeated
    \item Can be null (if schema permits)
    \item Can have multiple unique keys per table
    \begin{itemize}
        \item e.g. unique emails column, and uniqe ssn column in one table
    \end{itemize}
    \item Can be modified to a new value
    \begin{itemize}
        \item As long as it is not conflicted with others
    \end{itemize}
\end{itemize}

\bigskip

\section{PRIMARY Keys}

\bigskip

\begin{itemize}
    \item Never be null
    \item One primary key per table
    \item Cannot be modified to a new value

    \begin{center}
    \includegraphics[width=0.7\linewidth]{images/part_2_notes_1.png}
    \end{center}
\end{itemize}

\bigskip

\section{Quiz 1}

\bigskip

\begin{enumerate}[1.]
    \item

    What happens when a Unique Constraint is violated in a database system?

    \bigskip

    \begin{enumerate}[A.]
        \item An email alert is sent to the Database Administrator.
        \item The database locks up until the Database Administrator does a reboot.
        \item The database does not allow the data to be written to the table and an error is returned.
        \item The database allows the data to be written anyway.
    \end{enumerate}

    \bigskip

    \textbf{Answer:} C

    \item

    What data type does a primary key have to be?

    \bigskip

    \begin{enumerate}[A.]
        \item Integer
        \item Text
        \item Either, as long as the value guarantees uniqueness
    \end{enumerate}

    \bigskip

    \textbf{Answer:} C

    \item

    Which of the following is NOT something a database key can do?

    \bigskip

    \begin{enumerate}[A.]
        \item Ensure a value does not repeat within a given column.
        \item Guarantee a table does not return data when queried unless a specific password is supplied.
        \item Guarantee an entire row is unique within a table.
        \item Act as a pointer or a link back to another table.
    \end{enumerate}

    \bigskip

    \textbf{Answer:} B

    \item

    A Primary Key will allow one NULL value, but no more than that.

    \bigskip

    \begin{enumerate}[A.]
        \item True
        \item False
    \end{enumerate}

    \bigskip

    \textbf{Answer:} B

\end{enumerate}

\bigskip

\section{Foreign Keys}

\bigskip

\begin{itemize}
    \item value cannot be added to table with foreign key unless the value exists
    in the table with primary key
    \begin{itemize}
        \item is called \textbf{referential integrity}
    \end{itemize}

    \bigskip

    \begin{center}
    \includegraphics[width=0.8\linewidth]{images/part_2_notes_2.png}
    \end{center}
\end{itemize}

\bigskip

\section{Quiz 2}

\bigskip

\begin{enumerate}[1.]
    \item

    \bigskip

    Assuming 1, 2 and 3 are all values in the primary key table, this is a valid
    Foreign Key column.

    \begin{center}
    \includegraphics[width=0.4\linewidth]{images/part_2_notes_3.png}
    \end{center}

    \bigskip

    \begin{enumerate}[A.]
        \item True
        \item False
    \end{enumerate}

    \bigskip

    \textbf{Answer:} A


\end{enumerate}



\end{document}