\documentclass[12pt]{article}
\usepackage[margin=2.5cm]{geometry}
\usepackage{enumerate}
\usepackage{amsfonts}
\usepackage{amsmath}
\usepackage{fancyhdr}
\usepackage{amsmath}
\usepackage{amssymb}
\usepackage{amsthm}
\usepackage{mdframed}
\usepackage{graphicx}
\usepackage{subcaption}
\usepackage{adjustbox}
\usepackage{listings}
\usepackage{xcolor}
\usepackage{booktabs}
\usepackage[utf]{kotex}
\usepackage{hyperref}

\definecolor{codegreen}{rgb}{0,0.6,0}
\definecolor{codegray}{rgb}{0.5,0.5,0.5}
\definecolor{codepurple}{rgb}{0.58,0,0.82}
\definecolor{backcolour}{rgb}{0.95,0.95,0.92}

\lstdefinestyle{mystyle}{
    backgroundcolor=\color{backcolour},
    commentstyle=\color{codegreen},
    keywordstyle=\color{magenta},
    numberstyle=\tiny\color{codegray},
    stringstyle=\color{codepurple},
    basicstyle=\ttfamily\footnotesize,
    breakatwhitespace=false,
    breaklines=true,
    captionpos=b,
    keepspaces=true,
    numbers=left,
    numbersep=5pt,
    showspaces=false,
    showstringspaces=false,
    showtabs=false,
    tabsize=1
}

\lstset{style=mystyle}

\pagestyle{fancy}
\renewcommand{\headrulewidth}{0.4pt}
\lhead{Team Treehouse}
\rhead{SQL Basics Part 3 Notes}

\begin{document}
\title{SQL Basics Part 3 Notes}
\author{Team Treehouse}
\maketitle

\bigskip

\section{Searching Tables with 'WHERE'}

\bigskip

\begin{itemize}
    \item WHERE clause
    \begin{itemize}
        \item \textbf{Syntax:} SELECT \textit{columns} FROM \textit{table name} WHERE \textit{condition};
        \item \textbf{Syntax (Condition):} \textit{Columns} \textit{Operator} \textit{Value}
    \end{itemize}

    \item Equality Operator
    \begin{itemize}
        \item \textbf{Syntax:} SELECT \textit{columns} FROM \textit{table name} WHERE \textit{column name} = \textit{value};
    \end{itemize}

    \bigskip

    \underline{\textbf{Examples:}}

    \bigskip

    \begin{lstlisting}[language=SQL]
    SELECT * FROM contacts WHERE first_name = "Andrew";


    SELECT first_name, email FROM users WHERE last_name = "Chalkley";


    SELECT name AS "Product Name" FROM products WHERE stock_count = 0;


    SELECT title "Book Title" FROM books WHERE year_published = 1999;
    \end{lstlisting}

    \item Inequality Operator
    \begin{itemize}
        \item \textbf{Syntax:} SELECT \textit{columns} FROM \textit{table name} WHERE \textit{column name} != \textit{value};
    \end{itemize}

    \bigskip

    \underline{\textbf{Examples:}}

    \bigskip

    \begin{lstlisting}[language=SQL]
    SELECT * FROM contacts WHERE first_name != "Kenneth";


    SELECT first_name, email FROM users WHERE last_name != "L:one";


    SELECT name AS "Product Name" FROM products WHERE stock_count != 0;


    SELECT title "Book Title" FROM books WHERE year_published != 2015;
    \end{lstlisting}

    \item Greater than/ Less than Operator
    \begin{itemize}
        \item \textbf{Syntax (less than):} SELECT \textit{columns} FROM \textit{table name} WHERE \textit{column name} $<$ \textit{value};
        \item \textbf{Syntax (greater than):} SELECT \textit{columns} FROM \textit{table name} WHERE \textit{column name} $>$ \textit{value};
    \end{itemize}

    \item Cheat Sheet: \href{https://github.com/treehouse/cheatsheets/blob/master/sql_basics/cheatsheet.md}{Link}
\end{itemize}

\bigskip

\section{Exercise 1}

\bigskip

\begin{itemize}
    \item Solution included in \textit{exercise\_1.sql}
\end{itemize}

\bigskip

\section{Filtering by Comparing Values}

\bigskip

\begin{itemize}
    \item \textbf{Syntax (Less than):} SELECT \textit{columns} FROM \textit{table name} WHERE \textit{column name} $<$ \textit{value};
    \item \textbf{Syntax (Less than or equal):} SELECT \textit{columns} FROM \textit{table name} WHERE \textit{column name} $<=$ \textit{value};
    \item \textbf{Syntax (Greater than):} SELECT \textit{columns} FROM \textit{table name} WHERE \textit{column name} $>$ \textit{value};
    \item \textbf{Syntax (Greater than or equal):} SELECT \textit{columns} FROM \textit{table name} WHERE \textit{column name} $>=$ \textit{value};

    \bigskip

    \underline{\textbf{Example:}}

    \bigskip

    \begin{lstlisting}[language=SQL]
    SELECT first_name, last_name FROM users WHERE date_of_birth < '1998-12-01';


    SELECT title AS "Book Title", author AS Author FROM books WHERE year_released <= 2015;


    SELECT name, description FROM products WHERE price > 9.99;


    SELECT title FROM movies WHERE release_year >= 2000;
    \end{lstlisting}
\end{itemize}

\bigskip

\section{Exercise 2}

\bigskip

\begin{itemize}
    \item Solution included in \textit{exercise\_2.sql}
\end{itemize}

\bigskip

\section{Filtering on More than One Condition}

\bigskip

\begin{itemize}
    \item Is used when filtering with multiple conditions
    \item Can be done using \textit{AND} and/or \textit{OR} operator
    \item \textbf{Syntax (AND):} SELECT \textit{columns} FROM \textit{table name} WHERE <condition 1> AND <condition 2> ...;
    \item \textbf{Syntax (OR):} SELECT \textit{columns} FROM \textit{table name} WHERE <condition 1> OR <condition 2> ...;

    \bigskip

    \underline{\textbf{Examples:}}

    \bigskip

    \begin{lstlisting}[language=SQL]
    SELECT username FROM users WHERE last_name = "Chalkley" AND first_name = "Andrew";


    SELECT * FROM products WHERE category = "Games Consoles" AND price < 400;


    SELECT * FROM movies WHERE title = "The Matrix" OR title = "The Matrix Reloaded" OR title = "The Matrix Revolutions";


    SELECT country FROM countries WHERE population < 1000000 OR population > 100000000;
    \end{lstlisting}
\end{itemize}

\bigskip

\section{Exercise 3}

\bigskip

\begin{itemize}
    \item Solution included in \textit{exercise\_3.sql}
\end{itemize}

\bigskip

\section{Filtering By Dates}

\bigskip

\begin{itemize}
    \item Is done using comparison operators (same as part 3).
    \item \textbf{Syntax (Less than):} SELECT \textit{columns} FROM \textit{table name} WHERE \textit{column name} $<$ \textit{value};
    \item \textbf{Syntax (Less than or equal):} SELECT \textit{columns} FROM \textit{table name} WHERE \textit{column name} $<=$ \textit{value};
    \item \textbf{Syntax (Greater than):} SELECT \textit{columns} FROM \textit{table name} WHERE \textit{column name} $>$ \textit{value};
    \item \textbf{Syntax (Greater than or equal):} SELECT \textit{columns} FROM \textit{table name} WHERE \textit{column name} $>=$ \textit{value};


    \bigskip

    \underline{\textbf{Examples:}}

    \bigskip

    \begin{lstlisting}[language=SQL]
    SELECT first_name, last_name FROM users WHERE date_of_birth < '1998-12-01';


    SELECT title AS "Book Title", author AS Author FROM books WHERE year_released <= 2015;


    SELECT name, description FROM products WHERE price > 9.99;


    SELECT title FROM movies WHERE release_year >= 2000;
    \end{lstlisting}

\end{itemize}

\bigskip

\section{Exercise 4}

\bigskip

\begin{itemize}
    \item Solution included in \textit{exercise\_4.sql}
\end{itemize}

\bigskip

\section{Searching Within a Set of Values}

\bigskip

\begin{itemize}
    \item Returns results with matching sets of values in a columns
    \item Is similar to Python's \textit{x in [Value1, value2, ....]}
    \item \textbf{Syntax:} SELECT \textit{columns} FROM \textit{table name} WHERE \textit{column name} IN (\textit{value 1}, \textit{value 2}, ...);
    \item \textbf{Syntax (Negation):} SELECT \textit{columns} FROM \textit{table name} WHERE \textit{column name} NOT IN (\textit{value 1}, \textit{value 2}, ...);

    \bigskip

    \underline{\textbf{Examples:}}

    \bigskip

    \begin{lstlisting}[language=SQL]
    SELECT name FROM islands WHERE id IN (4, 8, 15, 16, 23, 42);


    SELECT * FROM products WHERE category IN ("eBooks", "Books", "Comics");


    SELECT title FROM courses WHERE topic IN ("JavaScript", "Databases", "CSS");


    SELECT * FROM campaigns WHERE medium IN ("email", "blog", "ppc");


    SELECT * FROM products WHERE category NOT IN ("Electronics");


    SELECT title FROM courses WHERE topic NOT IN ("SQL", "NoSQL");
    \end{lstlisting}
\end{itemize}

\bigskip

\section{Exercise 5}

\bigskip

\begin{itemize}
    \item Solution included in \textit{exercise\_5.sql}
\end{itemize}

\bigskip

\section{Searching Within a Range of Values}

\bigskip

\begin{itemize}
    \item Returns results between \textit{lesser value} and \textit{greater value}
    \item \textbf{Syntax:} SELECT \textit{columns} FROM \textit{table name} WHERE \textit{column name} BETWEEN \textit{lesser value} AND \textit{greater value};

    \bigskip

    \underline{\textbf{Examples:}}

    \bigskip

    \begin{lstlisting}[language=SQL]
    SELECT * FROM movies WHERE release_year BETWEEN 2000 AND 2010;


    SELECT name, description FROM products WHERE price BETWEEN 9.99 AND 19.99;


    SELECT name, appointment_date FROM appointments WHERE appointment_date BETWEEN "2015-01-01" AND "2015-01-07";
    \end{lstlisting}
\end{itemize}

\bigskip

\section{Exercise 6}

\bigskip

\begin{itemize}
    \item Solution included in \textit{exercise\_6.sql}
\end{itemize}

\bigskip

\section{Finding Data that Matches a Pattern}

\bigskip

\begin{itemize}
    \item LIKE operator
    \begin{itemize}
        \item Is used inside of \textit{WHERE} clause to match a pattern
        \item \textbf{Syntax:} SELECT \textit{columns} FROM \textit{table name} WHERE \textit{column name} LIKE \textit{pattern};
        \item Can be used to make search \underline{case insensitive}

    \begin{lstlisting}[language=SQL]
    SELECT title FROM books WHERE title LIKE "Harry Potter";
    // returns items like 'Harry potter', 'harry potter'

    \end{lstlisting}
    \end{itemize}
    \item LIKE operator with wild card \%
    \begin{itemize}
        \item Works to match zero or more unspecified characters
        \item works the same as `\*' in regex
    \end{itemize}


    \begin{lstlisting}[language=SQL]
    SELECT title FROM books WHERE title LIKE "Harry Potter%Fire";
    // returns items like 'Harry Potter and Dragon Fire', 'Harry Potter and Fire', 'Harry Potter Rising Fire'

    SELECT title FROM movies WHERE title LIKE "Alien%";
    // Returns items like 'Alien attack', 'Alien', "Alienate"


    SELECT * FROM contacts WHERE first_name LIKE "%drew";
    // Returns items like 'tigerdrew', 'mountaindrew', 'morning drew', 'andrew'


    SELECT * FROM books WHERE title LIKE "%Brief History%";
    // Returns items like 'Canadian Brief History Channel', 'Brief History'
    \end{lstlisting}
\end{itemize}

\bigskip

\section{Exercise 7}

\bigskip

\begin{itemize}
    \item Solution included in \textit{exercise\_7.sql}
\end{itemize}

\bigskip


\end{document}