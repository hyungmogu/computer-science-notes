\documentclass[12pt]{article}
\usepackage[margin=2.5cm]{geometry}
\usepackage{enumerate}
\usepackage{amsfonts}
\usepackage{amsmath}
\usepackage{fancyhdr}
\usepackage{amsmath}
\usepackage{amssymb}
\usepackage{amsthm}
\usepackage{mdframed}
\usepackage{graphicx}
\usepackage{subcaption}
\usepackage{adjustbox}
\usepackage{listings}
\usepackage{xcolor}
\usepackage{booktabs}
\usepackage[utf]{kotex}
\usepackage{hyperref}

\definecolor{codegreen}{rgb}{0,0.6,0}
\definecolor{codegray}{rgb}{0.5,0.5,0.5}
\definecolor{codepurple}{rgb}{0.58,0,0.82}
\definecolor{backcolour}{rgb}{0.95,0.95,0.92}

\lstdefinestyle{mystyle}{
    backgroundcolor=\color{backcolour},
    commentstyle=\color{codegreen},
    keywordstyle=\color{magenta},
    numberstyle=\tiny\color{codegray},
    stringstyle=\color{codepurple},
    basicstyle=\ttfamily\footnotesize,
    breakatwhitespace=false,
    breaklines=true,
    captionpos=b,
    keepspaces=true,
    numbers=left,
    numbersep=5pt,
    showspaces=false,
    showstringspaces=false,
    showtabs=false,
    tabsize=1
}

\lstset{style=mystyle}

\pagestyle{fancy}
\renewcommand{\headrulewidth}{0.4pt}
\lhead{Team Treehouse}
\rhead{SQL Basics Part 2 Notes}

\begin{document}
\title{SQL Basics Part 2 Notes}
\author{Team Treehouse}
\maketitle

\bigskip

\section{Tools We'll be Using}

\bigskip

\section{Your First SQL Statement}

\begin{itemize}
    \item SELECT
    \begin{itemize}
        \item \textbf{Syntax:} SELECT * from \textit{TABLE\_NAME};
        \begin{itemize}
            \item * means `all columns'
        \end{itemize}

        \bigskip

        \underline{\textbf{Examples:}}

    \begin{lstlisting}[language=SQL]
    SELECT * FROM books;
    SELECT * FROM products;
    SELECT * FROM users;
    SELECT * FROM countries;
    \end{lstlisting}
    \end{itemize}
\end{itemize}

\bigskip

\section{Exercise 1}

\bigskip

\begin{itemize}
    \item Solution included in \textit{exercise\_1.sql}
\end{itemize}

\bigskip

\section{Retrieving Specific Columns of Information}

\bigskip

\begin{itemize}
    \item SELECT with columns
    \begin{itemize}
        \item \textbf{Syntax:} SELECT \textit{COLUMN\_NAME1, COLUMN\_NAME2, ...} FROM \textit{TABLE\_NAME};
    \end{itemize}

    \bigskip

    \underline{\textbf{Examples:}}

    \bigskip

    \begin{lstlisting}[language=SQL]
    SELECT first_name, last_name FROM customers;
    SELECT name, description, price FROM products;
    SELECT title, author, isbn, year_released FROM books;
    SELECT name, species, legs FROM pets;
    \end{lstlisting}
\end{itemize}

\bigskip

\section{Exercise 2}

\bigskip

\begin{itemize}
    \item Solution included in \textit{exercise\_2.sql}
\end{itemize}

\bigskip

\section{Categorizing Your Output with 'AS'}

\bigskip

\begin{itemize}
    \item \textbf{Syntax:} SELECT \textit{COLUMN\_NAME} \textbf{AS} \textit{ALIAS} FROM \textit{TABLE\_NAME};
    \item `AS' is used to relabel a column to another

    \bigskip

    \underline{\textbf{Examples:}}

    \bigskip

    \begin{lstlisting}[language=SQL]
    SELECT username AS Username, first_name AS "First Name" FROM users;
    SELECT title AS Title, year AS "Year Released" FROM movies;
    SELECT name AS Name, description AS Description, price AS "Current Price" FROM products;
    SELECT name Name, description Description, price "Current Price" FROM products;
    \end{lstlisting}
\end{itemize}


\bigskip

\section{Exercise 3}

\bigskip

\begin{itemize}
    \item Solution included in \textit{exercise\_3.sql}
\end{itemize}


\bigskip

\section{Quiz 1}

\bigskip


\begin{enumerate}[1.]
    \item

    How do you separate column names in an SQL statement?

    \bigskip

    \begin{enumerate}[A.]
        \item With a period
        \item With a question mark
        \item With a comma
        \item With an asterisk
    \end{enumerate}

    \bigskip

    \textbf{Answer:} C

    \item

    A keyword is:

    \bigskip

    \begin{enumerate}[A.]
        \item A password for accessing databases.
        \item A command issued to a computer.
    \end{enumerate}

    \bigskip

    \textbf{Answer:} B

    \item

    What's missing where the question marks are in this statement?

    \bigskip

    \begin{lstlisting}[language=SQL]
    SELECT name? age? dob FROM customers;
    \end{lstlisting}

    \bigskip

    \begin{enumerate}[A.]
        \item A comma
        \item An asterisk
        \item A semi-colon
        \item A period
    \end{enumerate}

    \item

    Please fill in the correct answer in each blank provided below.

    \bigskip

    What's the missing character to bring back all columns?

    \bigskip

    \begin{lstlisting}[language=SQL]
    SELECT \_\_\_  FROM solar_systems;
    \end{lstlisting}

    \bigskip

    \textbf{Answer:} *

    \item

    If I wanted to select all columns in an SQL statement what special character
    would I use?

    \bigskip

    \begin{enumerate}[A.]
        \item ~
        \item \_
        \item \%
        \item \*
    \end{enumerate}

    \bigskip

    \textbf{Answer:} D

    \item

    Syntax is best described as:

    \bigskip

    \begin{enumerate}[A.]
        \item The set of grammatical rules and structure of programming code.
        \item A tool to use to run your programming code.
    \end{enumerate}

    \bigskip

    \textbf{Answer:} A

    \item

    Which results would you expect this SQL query to generate?

    \bigskip

    \begin{lstlisting}[language=Java]
    SELECT email, phone FROM contacts;
    \end{lstlisting}

    \bigskip

    \begin{enumerate}[A.]

        \item
        4075551234	andrew@example.com

        \bigskip

        4075554321	lauren@example.com

        \item
        Andrew	andrew@example.com

        \bigskip

        Lauren	lauren@example.com

        \item
        andrew@example.com	Andrew

        \bigskip

        lauren@example.com	Lauren

        \item
        andrew@example.com	4075551234

        \bigskip

        lauren@example.com	4075554321
    \end{enumerate}

    \bigskip

    \textbf{Answer:} D

    \item

    What keyword would I use to change the value of the column header presented
    in the results of a query?

    \bigskip

    \begin{enumerate}[A.]
        \item RENAME
        \item SELECT
        \item WHERE
        \item AS
    \end{enumerate}

    \bigskip

    \textbf{Answer:} D

\end{enumerate}




\end{document}