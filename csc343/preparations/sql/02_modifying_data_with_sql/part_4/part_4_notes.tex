\documentclass[12pt]{article}
\usepackage[margin=2.5cm]{geometry}
\usepackage{enumerate}
\usepackage{amsfonts}
\usepackage{amsmath}
\usepackage{fancyhdr}
\usepackage{amsmath}
\usepackage{amssymb}
\usepackage{amsthm}
\usepackage{mdframed}
\usepackage{graphicx}
\usepackage{subcaption}
\usepackage{adjustbox}
\usepackage{listings}
\usepackage{xcolor}
\usepackage{booktabs}
\usepackage[utf]{kotex}
\usepackage{hyperref}

\definecolor{codegreen}{rgb}{0,0.6,0}
\definecolor{codegray}{rgb}{0.5,0.5,0.5}
\definecolor{codepurple}{rgb}{0.58,0,0.82}
\definecolor{backcolour}{rgb}{0.95,0.95,0.92}

\lstdefinestyle{mystyle}{
    backgroundcolor=\color{backcolour},
    commentstyle=\color{codegreen},
    keywordstyle=\color{magenta},
    numberstyle=\tiny\color{codegray},
    stringstyle=\color{codepurple},
    basicstyle=\ttfamily\footnotesize,
    breakatwhitespace=false,
    breaklines=true,
    captionpos=b,
    keepspaces=true,
    numbers=left,
    numbersep=5pt,
    showspaces=false,
    showstringspaces=false,
    showtabs=false,
    tabsize=1
}

\lstset{style=mystyle}

\pagestyle{fancy}
\renewcommand{\headrulewidth}{0.4pt}
\lhead{Team Treehouse}
\rhead{Modifying Data with SQL Part 4 Notes}

\begin{document}
\title{Modifying Data with SQL Part 4 Notes}
\author{Team Treehouse}
\maketitle

\bigskip

\section{Intro to Transactions}

\bigskip

\begin{itemize}
    \item Transaction
    \begin{itemize}
        \item \textbf{Syntax:} BEGIN TRANSACTION; or BEGIN;
        \begin{itemize}
            \item Turns off autocommit and begins a transaction
        \end{itemize}
        \item \textbf{Syntax:} COMMIT;
        \begin{itemize}
            \item Sallves all results of the statements at the begininng of transaction to disk
        \end{itemize}
        \item Makes a batch of process atomic
        \begin{itemize}
            \item All are processed before being committed
            \item If power loss occurs during processing, no data is saved to database
        \end{itemize}
    \end{itemize}
\end{itemize}

\bigskip

\section{Rolling Back from Transactions}

\bigskip

\begin{itemize}
    \item \textbf{Syntax:} ROLLBACK;
    \begin{itemize}
        \item Resets the state of database to before the beginning of transaction
        \item Is useful when an error / power outage occurs in the middle of transaction
        \item Or spelling mistakes
    \end{itemize}
\end{itemize}


\bigskip

\section{Quiz 1}

\bigskip

\begin{enumerate}[1.]
    \item

    Which of these is the correct way to start a transaction?

    \bigskip

    \begin{enumerate}[A.]
        \item TRANSACTION BEGIN;
        \item BEGIN;
        \item START;
    \end{enumerate}

    \bigskip

    \textbf{Answer:} B

    \item

    The BEGIN; or the BEGIN TRANSACTION; statement does what?

    \bigskip

    \begin{enumerate}[A.]
        \item Reverses any changes since the before the statement.
        \item Turns autocommit mode on.
        \item Turns autocommit mode off.
    \end{enumerate}

    \bigskip

    \textbf{Answer:} C

    \item

    What following statement reverts changes since a transaction began

    \bigskip

    \begin{enumerate}[A.]
        \item UNDO;
        \item COMMIT;
        \item REVERT;
        \item ROLLBACK;
    \end{enumerate}

    \bigskip

    \textbf{Answer:} D

    \item

    If my computer crashed in the middle of a transaction, what state would my database be in?

    \bigskip

    \begin{enumerate}[A.]
        \item It would have some of my statements persisted.
        \item It would be in the same state before the transaction.
    \end{enumerate}

    \bigskip

    \textbf{Answer:} B

    \item

    What does the COMMIT; statement do?

    \bigskip

    \begin{enumerate}[A.]
        \item Reverts any changes.
        \item Turns autocommit mode off;
        \item Commits all changes since the transaction began and turns autocommit mode on.
    \end{enumerate}

    \bigskip

    \textbf{Answer:} B

\end{enumerate}

\bigskip

\section{Databases with Frameworks}

\bigskip

\section{Quiz 1}

\bigskip

\begin{enumerate}[1.]
    \item

    What is a transaction?

    \bigskip

    \begin{enumerate}[A.]
        \item Performing a really complex INSERT statement on one line.
        \item A safe way to perform multiple statements in one single go.
    \end{enumerate}

    \bigskip

    \textbf{Answer:} B

    \item

    Please fill in the correct answer in each blank provided below.

    \bigskip

    What is the keyword used to create rows in a database table?  \_\_\_

    \bigskip

    \textbf{Answer:} INSERT

    \item

    Please fill in the correct answer in each blank provided below.

    \bigskip

    What keyword is used to update values in a database table?  \_\_\_

    \bigskip

    \textbf{Answer:} UPDATE

    \item

    If you want to update or delete specific rows in a table, what keyword will
    you have to use?

    \bigskip

    \begin{enumerate}[A.]
        \item LIKE
        \item IN
        \item WHERE
        \item IS
    \end{enumerate}

    \bigskip

    \textbf{Answer:} C

    \item

    What does an ORM do?

    \bigskip

    \begin{enumerate}[A.]
        \item Validates your SQL code to see if there are any syntax errors.
        \item Stores a complete history of your database so you can rollback to any point.
        \item Provides a convenient way for developers to perform CRUD operations in their language of choice.
    \end{enumerate}

    \bigskip

    \textbf{Answer:} C

    \item

    What's wrong with this create statement?

    \begin{lstlisting}[language=SQL]
    INSERT cars VALUES (NULL, "Fiat", "Fiat Punto");
    \end{lstlisting}

    \bigskip

    \begin{enumerate}[A.]
        \item It's missing a column name.
        \item It's missing a keyword.
        \item Nothing. It's valid syntax.
        \item It's missing a table name.

    \end{enumerate}

    \bigskip

    \textbf{Answer:} B

    \item

    Please fill in the correct answer in each blank provided below.

    \bigskip

    What is the keyword used to read from a database?  \_\_\_

    \bigskip

    \textbf{Answer:} SELECT

    \item

    What's wrong with this create statement?

    \begin{lstlisting}[language=SQL]
    UPDATE SET first_name = "Andrew" WHERE last_name = "Chalkley";
    \end{lstlisting}

    \bigskip

    \begin{enumerate}[A.]
        \item It's missing a keyword.
        \item Nothing. This is valid syntax.
        \item It's missing a table name.
        \item It's missing a column name.
    \end{enumerate}

    \bigskip

    \textbf{Answer:} C

    \item

    Please fill in the correct answer in each blank provided below.

    \bigskip

    What keyword is used to delete rows from a database table?  \_\_\_

    \bigskip

    \textbf{Answer:} DELETE

    \item

    What's wrong with this create statement?

    \begin{lstlisting}[language=SQL]
    DELETE * FROM sports_teams;
    \end{lstlisting}

    \bigskip

    \begin{enumerate}[A.]
        \item It's missing a keyword.
        \item Nothing. It's valid syntax.
        \item It's missing a column name.
        \item It doesn't need an asterisk (*).
    \end{enumerate}

    \bigskip

    \textbf{Answer:} D


\end{enumerate}




\end{document}