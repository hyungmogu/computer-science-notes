\documentclass[12pt]{article}
\usepackage[margin=2.5cm]{geometry}
\usepackage{enumerate}
\usepackage{amsfonts}
\usepackage{amsmath}
\usepackage{fancyhdr}
\usepackage{amsmath}
\usepackage{amssymb}
\usepackage{amsthm}
\usepackage{mdframed}
\usepackage{graphicx}
\usepackage{subcaption}
\usepackage{adjustbox}
\usepackage{listings}
\usepackage{xcolor}
\usepackage{booktabs}
\usepackage[utf]{kotex}
\usepackage{hyperref}

\definecolor{codegreen}{rgb}{0,0.6,0}
\definecolor{codegray}{rgb}{0.5,0.5,0.5}
\definecolor{codepurple}{rgb}{0.58,0,0.82}
\definecolor{backcolour}{rgb}{0.95,0.95,0.92}

\lstdefinestyle{mystyle}{
    backgroundcolor=\color{backcolour},
    commentstyle=\color{codegreen},
    keywordstyle=\color{magenta},
    numberstyle=\tiny\color{codegray},
    stringstyle=\color{codepurple},
    basicstyle=\ttfamily\footnotesize,
    breakatwhitespace=false,
    breaklines=true,
    captionpos=b,
    keepspaces=true,
    numbers=left,
    numbersep=5pt,
    showspaces=false,
    showstringspaces=false,
    showtabs=false,
    tabsize=1
}

\lstset{style=mystyle}

\pagestyle{fancy}
\renewcommand{\headrulewidth}{0.4pt}
\lhead{Team Treehouse}
\rhead{Java Basics Part 3 Notes}

\begin{document}
\title{Java Basics Part 3 Notes}
\author{Team Treehouse}
\maketitle

\section{Reviewing our Feedback}

\bigskip

\begin{itemize}
    \item  \textit{System.exit(...);}
    \begin{itemize}
        \item Causes the program to exit
        \item \textit{System.exit(0)} $\to$ means program exited intentionally
        \item \textit{System.exit(1)} $\to$ means program exited abnormally
    \end{itemize}

    \bigskip

    \begin{lstlisting}[language=Java]
    import java.io.Console;

    public class Introductions {
        public static void main(String[] args) {
            Console console = System.console();

            int age = 13;
            if (age < 13) {
                console.printf("Sorry. You must be 13 years to use this program");
                System.exit(0);
            }

            String name = console.readLine("Enter name:   "); // <- Let's write 'Moe' here
            String adjective = console.readLine("Enter adjective:   "); // <- and 'glad to be with his love' here :)
            console.printf("%s is very %s\n", name, adjective);
        }
    }
    \end{lstlisting}

\end{itemize}

\bigskip

\section{Exercise 1}

\bigskip

\begin{itemize}
    \item Solution included in \textit{exercise\_1.java}
\end{itemize}

\bigskip

\section{Parsing Integers}

\bigskip

\begin{itemize}
    \item \textit{Integer.parseInt(...)}
    \begin{itemize}
        \item Extracts integer from string
        \item Is also called \textbf{typecasting}

    \begin{lstlisting}[language=Java]
    import java.io.Console;

    public class Introductions {
        public static void main(String[] args) {
            Console console = System.console();

            String ageString = console.readLine("How old are you?  ");

            int age = 13;
            if (Integer.ParseInt(ageString) < 13) { // <- Here :)
                console.printf("Sorry. You must be 13 years to use this program");
                System.exit(0);
            }

            String name = console.readLine("Enter name:   "); // <- Let's write 'Moe' here
            String adjective = console.readLine("Enter adjective:   "); // <- and 'glad to be with his love' here :)
            console.printf("%s is very %s\n", name, adjective);
        }
    }
    \end{lstlisting}
    \end{itemize}
\end{itemize}

\bigskip

\section{Exercise 2}

\bigskip

\begin{itemize}
    \item Solution included in \textit{exercise\_2.java}
\end{itemize}

\bigskip

\section{Censoring Words - Using String Equality}

\bigskip

\begin{itemize}
    \item \textit{STRING\_VAR.equals('...')}
    \begin{itemize}
        \item Checks if value in \textit{STRING\_VAR1} is equal to parameter
        value
    \end{itemize}
    \begin{lstlisting}[language=Java]
    import java.io.Console;

    public class Introductions {
        public static void main(String[] args) {
            ...

            String noun = console.readLine("Enter noun:   ");

            if (noun.equals("Dork")) { // <- Here :)
                console.printf("The language is not allowed. Exiting\n");
                System.exit(0);
            }

            ...
        }
    }
    \end{lstlisting}
    \item \textit{STRING\_VAR.equalsIgnoreCase('...')}
    \begin{itemize}
        \item Checks if value in \textit{STRING\_VAR1} is equal to parameter
        \item Case is ignored
        value
    \end{itemize}

    \bigskip

    \begin{lstlisting}[language=Java]
    import java.io.Console;

    public class Introductions {
        public static void main(String[] args) {
            ...

            String noun = console.readLine("Enter noun:   ");

            if (noun.equalsIgnoreCase("Dork")) { // <- Here :)
                console.printf("The language is not allowed. Exiting\n");
                System.exit(0);
            }

            ...
        }
    }
    \end{lstlisting}
\end{itemize}

\bigskip

\section{Exercise 3}

\bigskip

\begin{itemize}
    \item Solution included in \textit{exercise\_3.java}
\end{itemize}

\bigskip

\section{Censoring Words - Using Logical ORs}

\bigskip

\begin{itemize}
    \item `||' symbol is used

    \begin{lstlisting}[language=Java]
    import java.io.Console;

    public class Introductions {
        public static void main(String[] args) {
            ...

            String noun = console.readLine("Enter noun:   ");

            if (noun.equalsIgnoreCase("Dork") || // <- Here :)
                noun.equalsIgnoreCase("Jerk")) {
                console.printf("The language is not allowed. Exiting\n");
                System.exit(0);
            }

            ...
        }
    }
    \end{lstlisting}
\end{itemize}

\bigskip

\section{Quiz 1}

\bigskip

\begin{enumerate}[1.]
    \item

    Assuming that Sara has taken the Java and Python tracks, what is the value
    of isFamiliar below:

    \bigskip

    \begin{lstlisting}[language=Java]
    boolean isFamiliar = (
        learnedJava ||
        learnedPython ||
        learnedRuby
    );
    \end{lstlisting}

    \begin{enumerate}[A.]
        \item True
        \item Yes
    \end{enumerate}

    \bigskip

    \textbf{Answer:} A

    \item

    Assuming age is set to 42, what is the value stored in the boolean answer:

    \bigskip

    \begin{lstlisting}[language=Java]
    boolean answer = (age < 40 || age > 50);
    \end{lstlisting}

    \begin{enumerate}[A.]
        \item True
        \item False
    \end{enumerate}

    \bigskip

    \textbf{Answer:} B

    \item Carlos is just getting started and he has completed the HTML and CSS courses. What answer is stored in the
    boolean isQualified below:

    \bigskip

    \begin{lstlisting}[language=Java]
    boolean isQualified = (
        learnedHTML &&
        learnedJavaScript &&
        learnedCSS
    );
    \end{lstlisting}


    \begin{enumerate}[A.]
        \item True
        \item False
    \end{enumerate}

    \bigskip

    \textbf{Answer:} B

    \item If Bob is 50, has 3 children, and works for a software company, what
    is the answer to this logical statement:

    \bigskip

    \begin{lstlisting}[language=Java]
    age > 30 || children < 3 || isEmployed
    \end{lstlisting}

    \begin{enumerate}[A.]
        \item True
        \item False
    \end{enumerate}

    \bigskip

    \textbf{Answer:} A

    \item Assuming that count below is set to 90, what is stored in the boolean
    answer:

    \bigskip

    \begin{lstlisting}[language=Java]
    boolean answer = (count > 30 && count < 120);
    \end{lstlisting}

    \begin{enumerate}[A.]
        \item True
        \item False
    \end{enumerate}

    \bigskip

    \textbf{Answer:} A

\end{enumerate}

\bigskip

\section{Looping Until the Value Passes}

\bigskip

\begin{itemize}
    \item Is done using \textit{do while} loop

    \begin{lstlisting}[language=Java]
    import java.io.Console;

    public class Introductions {
        public static void main(String[] args) {
            ...

            String noun;

            do { // <- Here :)
                noun = console.readLine("Enter a noun:   ");
                if (noun.equalsIgnoreCase("Dork") ||
                    noun.equalsIgnoreCase("Jerk")) {
                    console.printf("The language is not allowed. Try again\n");
                }
            } while(noun.equalsIgnoreCase("Dork") || noun.equalsIgnoreCase("Jerk"))


            ...
        }
    }
    \end{lstlisting}
\end{itemize}

\bigskip

\section{Exercise 4}

\bigskip

\begin{itemize}
    \item Solution included in \textit{exercise\_4.java}
\end{itemize}

\end{document}