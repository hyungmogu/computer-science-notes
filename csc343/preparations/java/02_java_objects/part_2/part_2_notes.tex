\documentclass[12pt]{article}
\usepackage[margin=2.5cm]{geometry}
\usepackage{enumerate}
\usepackage{amsfonts}
\usepackage{amsmath}
\usepackage{fancyhdr}
\usepackage{amsmath}
\usepackage{amssymb}
\usepackage{amsthm}
\usepackage{mdframed}
\usepackage{graphicx}
\usepackage{subcaption}
\usepackage{adjustbox}
\usepackage{listings}
\usepackage{xcolor}
\usepackage{booktabs}
\usepackage[utf]{kotex}
\usepackage{hyperref}

\definecolor{codegreen}{rgb}{0,0.6,0}
\definecolor{codegray}{rgb}{0.5,0.5,0.5}
\definecolor{codepurple}{rgb}{0.58,0,0.82}
\definecolor{backcolour}{rgb}{0.95,0.95,0.92}

\lstdefinestyle{mystyle}{
    backgroundcolor=\color{backcolour},
    commentstyle=\color{codegreen},
    keywordstyle=\color{magenta},
    numberstyle=\tiny\color{codegray},
    stringstyle=\color{codepurple},
    basicstyle=\ttfamily\footnotesize,
    breakatwhitespace=false,
    breaklines=true,
    captionpos=b,
    keepspaces=true,
    numbers=left,
    numbersep=5pt,
    showspaces=false,
    showstringspaces=false,
    showtabs=false,
    tabsize=1
}

\lstset{style=mystyle}

\pagestyle{fancy}
\renewcommand{\headrulewidth}{0.4pt}
\lhead{Team Treehouse}
\rhead{Java Objects Part 2 Notes}

\begin{document}
\title{Java Objects Part 2 Notes}
\author{Team Treehouse}
\maketitle

\bigskip

\section{constants}

\bigskip

\begin{itemize}
    \item Are named \textit{IN\_CAPITALIZED\_SNAKE\_CASE}
    \item Can be done using \textit{static} keyword
    \item Allows variables and methods to be exponsed without instantiation

    \begin{lstlisting}[language=Java,caption={lesson\_1/PezDispenser.java}]
    public class PezDispenser {
        public static final int MAX_PEZ = 12; // <- 1. static declared here :)
        ...
    }
    \end{lstlisting}

    \begin{lstlisting}[language=Java,caption={lesson\_1/Example.java}]
    import java.io.Console;

    public class Example {
        public static void main(String[] args) {
            ...
            System.out.printf("FUN FACT: There are %d PEZ allowed in every dispenser\n", PezDispenser.MAX_PEZ); // 2. <- And is used here :)
            ...
        }
    }
    \end{lstlisting}

    \bigskip

    \underline{\textbf{Notes:}}

    \bigskip

    \begin{itemize}
        \item Files can be compiled and displayed by typing \textit{javac Example.java \&\& java Example}
        in terminal
    \end{itemize}
\end{itemize}

\bigskip

\section{Filling the Dispenser}

\bigskip

\begin{itemize}
    \item \textit{void} keyword means nothing is returned at the end of a method

    \begin{lstlisting}[language=Java,caption={lesson\_3/PezDispenser.java}]
    public class PezDispenser {
        public void fill() { // <- This little guy here :)
            this.pezCount = MAX_PEZ;
            System.out.printf("The current count of delicious PEZ is %d\n", this.pezCount);
        }
    }
    \end{lstlisting}

    \begin{lstlisting}[language=Java,caption={lesson\_3/Example.java}]
    import java.io.Console;

    public class Example {
        public static void main(String[] args) {
            ...
            dispenser.fill(); // <- 2. Is used like this

        }
    }
    \end{lstlisting}

    \bigskip

    \underline{\textbf{Notes:}}

    \bigskip

    \begin{itemize}
        \item Files can be compiled and displayed by typing \textit{javac Example.java \&\& java Example}
        in terminal
        \item Always start with private methods, and turn to public when needed.
    \end{itemize}
\end{itemize}

\end{document}