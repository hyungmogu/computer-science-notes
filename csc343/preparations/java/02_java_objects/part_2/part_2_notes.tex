\documentclass[12pt]{article}
\usepackage[margin=2.5cm]{geometry}
\usepackage{enumerate}
\usepackage{amsfonts}
\usepackage{amsmath}
\usepackage{fancyhdr}
\usepackage{amsmath}
\usepackage{amssymb}
\usepackage{amsthm}
\usepackage{mdframed}
\usepackage{graphicx}
\usepackage{subcaption}
\usepackage{adjustbox}
\usepackage{listings}
\usepackage{xcolor}
\usepackage{booktabs}
\usepackage[utf]{kotex}
\usepackage{hyperref}

\definecolor{codegreen}{rgb}{0,0.6,0}
\definecolor{codegray}{rgb}{0.5,0.5,0.5}
\definecolor{codepurple}{rgb}{0.58,0,0.82}
\definecolor{backcolour}{rgb}{0.95,0.95,0.92}

\lstdefinestyle{mystyle}{
    backgroundcolor=\color{backcolour},
    commentstyle=\color{codegreen},
    keywordstyle=\color{magenta},
    numberstyle=\tiny\color{codegray},
    stringstyle=\color{codepurple},
    basicstyle=\ttfamily\footnotesize,
    breakatwhitespace=false,
    breaklines=true,
    captionpos=b,
    keepspaces=true,
    numbers=left,
    numbersep=5pt,
    showspaces=false,
    showstringspaces=false,
    showtabs=false,
    tabsize=1
}

\lstset{style=mystyle}

\pagestyle{fancy}
\renewcommand{\headrulewidth}{0.4pt}
\lhead{Team Treehouse}
\rhead{Java Objects Part 2 Notes}

\begin{document}
\title{Java Objects Part 2 Notes}
\author{Team Treehouse}
\maketitle

\bigskip

\section{constants}

\bigskip

\begin{itemize}
    \item Are named \textit{IN\_CAPITALIZED\_SNAKE\_CASE}
    \item Can be done using \textit{static} keyword
    \item Allows variables and methods to be exponsed without instantiation

    \begin{lstlisting}[language=Java,caption={lesson\_1/PezDispenser.java}]
    public class PezDispenser {
        public static final int MAX_PEZ = 12; // <- 1. static declared here :)
        ...
    }
    \end{lstlisting}

    \begin{lstlisting}[language=Java,caption={lesson\_1/Example.java}]
    import java.io.Console;

    public class Example {
        public static void main(String[] args) {
            ...
            System.out.printf("FUN FACT: There are %d PEZ allowed in every dispenser\n", PezDispenser.MAX_PEZ); // 2. <- And is used here :)
            ...
        }
    }
    \end{lstlisting}

    \bigskip

    \underline{\textbf{Notes:}}

    \bigskip

    \begin{itemize}
        \item Files can be compiled and displayed by typing \textit{javac Example.java \&\& java Example}
        in terminal
    \end{itemize}
\end{itemize}

\bigskip

\section{Exercise 1}

\bigskip

\begin{itemize}
    \item Solution included in \textit{exercise\_1.java}
\end{itemize}

\bigskip

\section{Filling the Dispenser}

\bigskip

\begin{itemize}
    \item \textit{void} keyword means nothing is returned at the end of a method

    \begin{lstlisting}[language=Java,caption={lesson\_3/PezDispenser.java}]
    public class PezDispenser {
        public void fill() { // <- This little guy here :)
            this.pezCount = MAX_PEZ;
            System.out.printf("The current count of delicious PEZ is %d\n", this.pezCount);
        }
    }
    \end{lstlisting}

    \begin{lstlisting}[language=Java,caption={lesson\_3/Example.java}]
    import java.io.Console;

    public class Example {
        public static void main(String[] args) {
            ...
            dispenser.fill(); // <- 2. Is used like this

        }
    }
    \end{lstlisting}

    \bigskip

    \underline{\textbf{Notes:}}

    \bigskip

    \begin{itemize}
        \item Files can be compiled and displayed by typing \textit{javac Example.java \&\& java Example}
        in terminal
        \item Always start with private methods, and turn to public when needed.
    \end{itemize}
\end{itemize}

\bigskip

\section{Exercise 2}

\bigskip

\begin{itemize}
    \item Solution included in \textit{exercise\_2.java}
\end{itemize}

\bigskip

\section{Abstraction at Play}

\bigskip

\begin{itemize}
    \item \textit{Golden Rule} Don't make users understand object internally
    \begin{itemize}
        \item Simple questions such as `is it empty?' is sufficent
    \end{itemize}

    \begin{lstlisting}[language=Java,caption={lesson\_5/PezDispenser.java}]
    public class PezDispenser {
        public boolean isEmpty() { // <- This little guy here :)
            return this.pezCount == 0;
        }

        ...
    }
    \end{lstlisting}

    \begin{lstlisting}[language=Java,caption={lesson\_5/Example.java}]
    import java.io.Console;

    public class Example {
        public static void main(String[] args) {
            ...
            if (dispenser.isEmpty()) {
                System.out.printf("Dispenser is empty"); // <- 2. with this little fellow here
            }

            ...
            if (!dispenser.isEmpty()) {
                System.out.printf("Dispenser is full\n"); // <- 3. and this guy as well
            }

        }
    }
    \end{lstlisting}

    \bigskip

    \underline{\textbf{Notes:}}

    \bigskip

    \begin{itemize}
        \item Files can be compiled and displayed by typing \textit{javac Example.java \&\& java Example}
        in terminal
    \end{itemize}

\end{itemize}

\bigskip

\section{Exercise 3}

\bigskip

\begin{itemize}
    \item Solution included in \textit{exercise\_3.java}
\end{itemize}

\bigskip

\section{Incrementing and Decrementing}

\bigskip

\begin{itemize}
    \item \textit{INT\_VARIABLE--:} Decrements the value in variable by 1
    \item \textit{INT\_VARIABLE++:} Increments the value in variable by 1

    \bigskip

    \begin{lstlisting}[language=Java,caption={lesson\_7/PezDispenser.java}]
    public class PezDispenser {
        ...
        public boolean dispense() { // <- 1. This little guy here :)
            boolean wasDispensed = false;
            if (!this.isEmpty()) {
                this.pezCount--; // <- 2. With decrement count here
                wasDispensed = true;
            }

            return wasDispensed;

        }
    }
    \end{lstlisting}

    \bigskip

    \begin{lstlisting}[language=Java,caption={lesson\_7/Example.java}]
    import java.io.Console;

    public class Example {
        public static void main(String[] args) {
            ...
            while (dispenser.dispense()) {
                System.out.println("Chomp!"); // <- 3. This will print as long as .dispensed() returns true
            }

            if (dispenser.isEmpty()) {
                System.out.println("Ate all the PEZ");
            }
        }
    }
    \end{lstlisting}

    \bigskip

    \begin{lstlisting}[language=bash]
    >>> javac Example.java && java Example
    We are making a new PEZ dispenser

    FUN FACT: There are 12 PEZ allowed in every dispenser
    Dispenser is emptyThe dispenser is Yoda
    Filling the dispenser with delicious PEZ...
    The current count of delicious PEZ is 12
    Dispenser is full
    Chomp!
    Chomp!
    Chomp!
    Chomp!
    Chomp!
    Chomp!
    Chomp!
    Chomp!
    Chomp!
    Chomp!
    Chomp!
    Chomp!
    Ate all the PEZ
    \end{lstlisting}

    \underline{\textbf{Notes:}}

    \bigskip

    \begin{itemize}
        \item Files can be compiled and displayed by typing \textit{javac Example.java \&\& java Example}
        in terminal
    \end{itemize}
\end{itemize}

\bigskip

\section{Exercise 4}

\bigskip

\begin{itemize}
    \item Solution included in \textit{exercise\_4.java}
\end{itemize}

\bigskip

\end{document}