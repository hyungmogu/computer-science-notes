\documentclass[12pt]{article}
\usepackage[margin=2.5cm]{geometry}
\usepackage{enumerate}
\usepackage{amsfonts}
\usepackage{amsmath}
\usepackage{fancyhdr}
\usepackage{amsmath}
\usepackage{amssymb}
\usepackage{amsthm}
\usepackage{mdframed}
\usepackage{graphicx}
\usepackage{subcaption}
\usepackage{adjustbox}
\usepackage{listings}
\usepackage{xcolor}
\usepackage{booktabs}
\usepackage[utf]{kotex}
\usepackage{hyperref}

\definecolor{codegreen}{rgb}{0,0.6,0}
\definecolor{codegray}{rgb}{0.5,0.5,0.5}
\definecolor{codepurple}{rgb}{0.58,0,0.82}
\definecolor{backcolour}{rgb}{0.95,0.95,0.92}

\lstdefinestyle{mystyle}{
    backgroundcolor=\color{backcolour},
    commentstyle=\color{codegreen},
    keywordstyle=\color{magenta},
    numberstyle=\tiny\color{codegray},
    stringstyle=\color{codepurple},
    basicstyle=\ttfamily\footnotesize,
    breakatwhitespace=false,
    breaklines=true,
    captionpos=b,
    keepspaces=true,
    numbers=left,
    numbersep=5pt,
    showspaces=false,
    showstringspaces=false,
    showtabs=false,
    tabsize=1
}

\lstset{style=mystyle}

\pagestyle{fancy}
\renewcommand{\headrulewidth}{0.4pt}
\lhead{Team Treehouse}
\rhead{Java Arrays Part 1 Notes}

\begin{document}
\title{Java Arrays Part 1 Notes}
\author{Team Treehouse}
\maketitle

\section{Meet Arrays}

\bigskip

\section{Declaring Arrays}

\bigskip

\begin{itemize}
    \item \textbf{Syntax:} \textit{DATA\_TYPE[] VAR\_NAME = new DATA\_TYPE[SIZE]}

    \bigskip

    \underline{\textbf{Example:}}

    \bigskip

    \begin{lstlisting}[language=Java]
    String[] friends = new String[3];
    \end{lstlisting}

    \item Defeault values
    \begin{itemize}
        \item \textit{byte}: 0
        \item \textit{short}: 0
        \item \textit{int}: 0
        \item \textit{long}: 0L
        \item \textit{float}: 0.0f
        \item \textit{double}: 0.0d
        \item \textit{char}: '\u0000'
        \item \textit{String} (or any object): null
        \item \textit{boolean}: false
    \end{itemize}
\end{itemize}

\bigskip

\section{Quiz 1}

\bigskip

\begin{enumerate}[1.]
    \item

    Which of the following statements would create an array named planets with 9
    String elements?

    \begin{enumerate}[A.]
        \item String planets = new Array(9);
        \item String[] planets = new String[9];
        \item Array[] planets = new String[9];
        \item String[] planets = new String(9);
    \end{enumerate}

    \bigskip

    \textbf{Answer:} B

    \item

    When you create a String array like this:

    \begin{lstlisting}[language=Java]
    String[] goldenGirls = new String[4];
    \end{lstlisting}

    What will the values be initialized to?

    \bigskip

    \begin{enumerate}[A.]
        \item ""
        \item null
        \item 0
    \end{enumerate}

    \bigskip

    \textbf{Answer:} B

\end{enumerate}

\bigskip

\section{Accessing Items}

\bigskip

\begin{itemize}
    \item \textbf{Syntax:} \textit{ARR\_VAR\_NAME[INDEX]}

    \bigskip

    \underline{\textbf{Example:}}

    \bigskip

    \begin{lstlisting}[language=Java]
    String[] friends = {"Pasan", "Johnathan", null};
    System.out.println(friends[0] + " is my classmate");
    \end{lstlisting}
\end{itemize}

\bigskip

\section{Quiz 2}

\bigskip

\begin{enumerate}[1.]
    \item

    Choose the code that will retrieve the last element in the following array:

    \begin{lstlisting}[language=Java]
    String[] goldenGirls = new String[4];
    goldenGirls[0] = "Blanche";
    goldenGirls[1] = "Sophia";
    goldenGirls[2] = "Rose";
    goldenGirls[3] = "Dorothy";
    \end{lstlisting}

    \begin{enumerate}[A.]
        \item goldenGirls[goldenGirls.length - 1];
        \item goldenGirls[4]
        \item goldenGirls.getLast();
    \end{enumerate}

    \bigskip

    \textbf{Answer:} A

    \item

    What code would get the first item from this array?

    \begin{lstlisting}[language=Java]
    String[] reminders = {"Baby", "Years"};
    \end{lstlisting}

    What will the values be initialized to?

    \bigskip

    \begin{enumerate}[A.]
        \item reminders[1];
        \item reminders[0];
        \item reminders{0};
    \end{enumerate}

    \bigskip

    \textbf{Answer:} B

    \item Considering the following code in the jShell REPL:

    \begin{lstlisting}[language=Java]
    jshell> beatles
    beatles ==> String[4] { "John", "Paul", "George", "Ringo" }

    jshell> System.out.println(beatles[1]);
    \end{lstlisting}

    What would be printed?

    \bigskip

    \begin{enumerate}[A.]
        \item John
        \item Paul
        \item Ringo
    \end{enumerate}

    \bigskip

    \textbf{Answer:} B

\end{enumerate}

\end{document}