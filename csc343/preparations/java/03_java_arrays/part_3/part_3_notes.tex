\documentclass[12pt]{article}
\usepackage[margin=2.5cm]{geometry}
\usepackage{enumerate}
\usepackage{amsfonts}
\usepackage{amsmath}
\usepackage{fancyhdr}
\usepackage{amsmath}
\usepackage{amssymb}
\usepackage{amsthm}
\usepackage{mdframed}
\usepackage{graphicx}
\usepackage{subcaption}
\usepackage{adjustbox}
\usepackage{listings}
\usepackage{xcolor}
\usepackage{booktabs}
\usepackage[utf]{kotex}
\usepackage{hyperref}

\definecolor{codegreen}{rgb}{0,0.6,0}
\definecolor{codegray}{rgb}{0.5,0.5,0.5}
\definecolor{codepurple}{rgb}{0.58,0,0.82}
\definecolor{backcolour}{rgb}{0.95,0.95,0.92}

\lstdefinestyle{mystyle}{
    backgroundcolor=\color{backcolour},
    commentstyle=\color{codegreen},
    keywordstyle=\color{magenta},
    numberstyle=\tiny\color{codegray},
    stringstyle=\color{codepurple},
    basicstyle=\ttfamily\footnotesize,
    breakatwhitespace=false,
    breaklines=true,
    captionpos=b,
    keepspaces=true,
    numbers=left,
    numbersep=5pt,
    showspaces=false,
    showstringspaces=false,
    showtabs=false,
    tabsize=1
}

\lstset{style=mystyle}

\pagestyle{fancy}
\renewcommand{\headrulewidth}{0.4pt}
\lhead{Team Treehouse}
\rhead{Java Arrays Part 3 Notes}

\begin{document}
\title{Java Arrays Part 3 Notes}
\author{Team Treehouse}
\maketitle

\section{Adding and Removing Items Means Copying}

\bigskip

\begin{itemize}
    \item Adding / Removing elements $\to$ new array must be declared and copy into it
    \item Adding
    \begin{itemize}
        \item \textbf{Syntax:} \textit{System.arraycopy(Object src, int srcPos, Object, dest, int destPos, int length);}
        \begin{itemize}
            \item \textbf{src:} Is the source array
            \item \textbf{srcPos:} Is the starting position of the source array
            \item \textbf{dest:} Is the destination array
            \item \textbf{destPos:} Is the starting position in the destination data
            \item \textbf{length:} Is the number of array elements to be copied
        \end{itemize}
    \end{itemize}

    \bigskip

    \underline{\textbf{Example:}}

    \bigskip

    \begin{lstlisting}[language=Java, caption={lesson\_01/Explore.java}]
    String[] classmates = {"Ben", "Johnny", "Pasan"};
    String[] classmatesAndMe = new String[4];

    System.arraycopy(classmates, 0, classmatesAndMe, 0, classmates.length);

    // Returns [Ben, Johnny, Pasan, null]
    \end{lstlisting}

    \underline{\textbf{Notes:}}

    \bigskip

    \begin{itemize}
        \item Files can be compiled and displayed by typing \textit{javac Explore.java \&\& java Explore}
        in terminal
    \end{itemize}
\end{itemize}

\bigskip

\section{Quiz 2}

\bigskip

\begin{enumerate}[1.]
    \item

    Since you can add elements to an array by making a new array and copying values into it, how do you suppose you go
    about removing items?

    \begin{enumerate}[A.]
        \item You still make a new copy, one element smaller than the current one, and simply don't copy over the value that you want to remove.
        \item I've given up all hope.
        \item You call the method deleteItemAt and pass the index you want deleted.
    \end{enumerate}

    \bigskip

    \textbf{Answer:} A

    \item

    Why can you not simply just add an item to an array?

    \bigskip

    \begin{enumerate}[A.]
        \item Polymorphism doesn't allow for growth of objects that are statically typed.
        \item The interface does provide an proper method that allows for this. It is due to method access levels.
        \item An array's length is immutable and it requires elements to be located in a contiguous order in memory.
    \end{enumerate}

    \bigskip

    \textbf{Answer:} C

\end{enumerate}

\bigskip

\section{Sorting}

\bigskip


\begin{itemize}
    \item \textbf{Syntax:} \textit{Arrays.sort(DATA\_TYPE[] arr, int from\_index, int to\_index)}
    \item \textbf{Syntax 2:} \textit{Arrays.sort(DATA\_TYPE[] arr, Comparator c)}
    \begin{itemize}
        \item Comparator is like \textit{lambda} in python
        \item \textit{Arrays.sort} is in \textit{java.util.Arrays}
        \item \textit{Comparator} is in \textit{java.util.Comparator}
    \end{itemize}

    \begin{lstlisting}[language=Java, caption={lesson\_03/Explore.java}]
    import java.util.Arrays;
    import java.util.Comparator;

    public class Explore {
        public static void main(String[] args) {
            String[] classmates = {"Ben", "Johnny", "Pasan"};

            Arrays.sort(classmates, Comparator.comparing(String::length)); // <- sorts based on length of string
            System.out.println(Arrays.toString(classmates));

            // Returns ["Ben", "Pasan", "Johnny"]
        }
    }
    \end{lstlisting}

    \underline{\textbf{Notes:}}

    \bigskip

    \begin{itemize}
        \item Files can be compiled and displayed by typing \textit{javac Explore.java \&\& java Explore}
        in terminal
    \end{itemize}
\end{itemize}

\bigskip

\section{Quiz 3}

\bigskip

\begin{enumerate}[1.]
    \item

    Assume that you have a class representing Planet. It has a method named\\
    \textit{getDistanceInAstromicalUnits} that helps figure out the distance from the Sun.

    \bigskip

    Instances of the Planet class are created and placed in an Planet[] array named
    planets.

    \bigskip

    Which code snippet would allow you to sort the planets array by distance?

    \begin{enumerate}[A.]
        \item Arrays.sort(planets);
        \item Planet.sortBy(getAstronomicalUnits());
        \item Arrays.sort(planets, Comparator.comparing(Planet::getDistanceInAstronomicalUnits));
    \end{enumerate}

    \bigskip

    \textbf{Answer:} C

\end{enumerate}

\bigskip

\section{Array Usage in Method Declarations}

\bigskip

\begin{itemize}
    \item \textbf{Syntax:} \textit{METHOD\_NAME(DATA\_TYPE ... ARG\_VAR\_NAME)}
    \begin{itemize}
        \item Functions like \textit{args} in python
        \item Combines arguements of same data type into an array
    \end{itemize}

    \begin{lstlisting}[language=Java, caption={lesson\_05/Explore.java}]

    \end{lstlisting}

    \bigskip

    \underline{\textbf{Notes:}}

    \bigskip

    \begin{itemize}
        \item Files can be compiled and displayed by typing \textit{javac Explore.java \&\& java Explore}
        in terminal
    \end{itemize}
\end{itemize}

\end{document}