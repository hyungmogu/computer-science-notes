\documentclass[12pt]{article}
\usepackage[margin=2.5cm]{geometry}
\usepackage{enumerate}
\usepackage{amsfonts}
\usepackage{amsmath}
\usepackage{fancyhdr}
\usepackage{amsmath}
\usepackage{amssymb}
\usepackage{amsthm}
\usepackage{mdframed}
\usepackage{graphicx}
\usepackage{subcaption}
\usepackage{adjustbox}
\usepackage{listings}
\usepackage{xcolor}
\usepackage{booktabs}
\usepackage[utf]{kotex}
\usepackage{hyperref}

\definecolor{codegreen}{rgb}{0,0.6,0}
\definecolor{codegray}{rgb}{0.5,0.5,0.5}
\definecolor{codepurple}{rgb}{0.58,0,0.82}
\definecolor{backcolour}{rgb}{0.95,0.95,0.92}

\lstdefinestyle{mystyle}{
    backgroundcolor=\color{backcolour},
    commentstyle=\color{codegreen},
    keywordstyle=\color{magenta},
    numberstyle=\tiny\color{codegray},
    stringstyle=\color{codepurple},
    basicstyle=\ttfamily\footnotesize,
    breakatwhitespace=false,
    breaklines=true,
    captionpos=b,
    keepspaces=true,
    numbers=left,
    numbersep=5pt,
    showspaces=false,
    showstringspaces=false,
    showtabs=false,
    tabsize=1
}

\lstset{style=mystyle}

\pagestyle{fancy}
\renewcommand{\headrulewidth}{0.4pt}
\lhead{Team Treehouse}
\rhead{Local Development Environment Part 1 Notes}

\begin{document}
\title{Local Development Environment Part 1 Notes}
\author{Team Treehouse}
\maketitle

\section{What to Expect}

\bigskip

\section{T.M.A - Too many acronyms}

\bigskip

\begin{itemize}
    \item \textbf{JDK}
    \begin{itemize}
        \item Is called \textbf{Java SE Development Kit}
        \item Is a set of tools specifically for developing Java SE Applications
        \item Is capable of creating and compiling programs
    \end{itemize}
    \item \textbf{JRE}
    \begin{itemize}
        \item Is called \textbf{Java Runtime Environment}
        \item Is a minimum set of tools that allow local Java programs to execute
        \item Contains Java Virtual Machine, Java Class Library, and the \textbf{java}
        terminal command
    \end{itemize}
\end{itemize}

\bigskip

\section{Quiz 1}

\bigskip

\begin{enumerate}[1.]
    \item

    When talking about Java SE, what does the SE stand for?

    \begin{enumerate}[A.]
        \item Strict Evaluation
        \item Something Else
        \item String Emulation
        \item Standard Edition
    \end{enumerate}

    \bigskip

    \textbf{Answer:} D

    \item

    We use the Java SE API, which is the set of libraries we use to build applications.
    What does API stand for?

    \begin{enumerate}[A.]
        \item Artificial Programmed Intelligence
        \item Application Programmer Interface
        \item Apptitude Programmed Interface
    \end{enumerate}

    \bigskip

    \textbf{Answer:} B

    \item

    Please fill in the correct answer in each blank provided below.

    \bigskip

    A minimum set of tools that allow local Java programs to execute is also known as the Java \_\_\_\_  Environment

    \bigskip

    \textbf{Answer:} Runtime

    \item

    If you are planning on developing applications locally you need only to install
    the JRE (Java Runtime Environment).


    \bigskip

    \begin{enumerate}[A.]
        \item True
        \item False
    \end{enumerate}

    \bigskip

    \textbf{Answer:} B

    \item

    A grouping of tools that allow you to create software locally on your machine.
    This is sometimes shortened to be called devkit.

    \bigskip

    \begin{enumerate}[A.]
        \item SDK
        \item JVM
        \item SE
    \end{enumerate}

    \bigskip

    \textbf{Answer:} A

    \item

    The extended subset of the tools used to develop Java SE applications uses
    the acronym JDK. What does that stand for?

    \bigskip

    \begin{enumerate}[A.]
        \item Java Development Kit
        \item Java Developed Kernel
        \item Just Delicious Kale
        \item Java Design Kinetics
    \end{enumerate}

    \bigskip

    \textbf{Answer:} A

\end{enumerate}

\bigskip

\section{The Java Virtual Machine}

\bigskip

\begin{itemize}
    \item \textit{C} $\to$ has to compile on each machine
    \item \textit{JAVA} $\to$ Can compile once and be used anywhere
    \begin{itemize}
        \item Is possible due to \textbf{JVM}, or the \textbf{java virtual machine}
    \end{itemize}
\end{itemize}


\end{document}