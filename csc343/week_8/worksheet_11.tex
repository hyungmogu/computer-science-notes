\documentclass[12pt]{article}
\usepackage[margin=2.5cm]{geometry}
\usepackage{enumerate}
\usepackage{amsfonts}
\usepackage{amsmath}
\usepackage{fancyhdr}
\usepackage{amsmath}
\usepackage{amssymb}
\usepackage{amsthm}
\usepackage{mdframed}
\usepackage{graphicx}
\usepackage{subcaption}
\usepackage{adjustbox}
\usepackage{listings}
\usepackage{xcolor}
\usepackage{booktabs}
\usepackage[utf]{kotex}
\usepackage{hyperref}
\usepackage{accents}

\definecolor{codegreen}{rgb}{0,0.6,0}
\definecolor{codegray}{rgb}{0.5,0.5,0.5}
\definecolor{codepurple}{rgb}{0.58,0,0.82}
\definecolor{backcolour}{rgb}{0.95,0.95,0.92}

\lstdefinestyle{mystyle}{
    backgroundcolor=\color{backcolour},
    commentstyle=\color{codegreen},
    keywordstyle=\color{magenta},
    numberstyle=\tiny\color{codegray},
    stringstyle=\color{codepurple},
    basicstyle=\ttfamily\footnotesize,
    breakatwhitespace=false,
    breaklines=true,
    captionpos=b,
    keepspaces=true,
    numbers=left,
    numbersep=5pt,
    showspaces=false,
    showstringspaces=false,
    showtabs=false,
    tabsize=1
}

\lstset{style=mystyle}

\pagestyle{fancy}
\renewcommand{\headrulewidth}{0.4pt}
\lhead{CSC 343}
\rhead{Worksheet 11}

\begin{document}
\title{CSC343 Worksheet 11}
\maketitle

\begin{enumerate}[1.]
    \item \textbf{Exercise 12.3.1:} Suppose our input XML document has the form
    of the product data of Figs. 12.4 and 12.5. Write XSLT stylesheets to produce
    each of the following documents.

    \begin{enumerate}[a)]
        \item An HTML file consisting of a header "Manufacturers" followed by an enumerated list of the names of all the makers of products listed in the input.
        \item An HTML file consisting of a table with headers "Model" and "Price," with a row for each PC. That row should have the proper model and price for the PC.
        \item An HTML file consisting of a table whose headers are "Model," "Price," "Speed," and "Ram" for all Laptops, followed by another table with the same headers for PC's.
        \item An XML file with root tag <PCs> and subelements having tag <PC>. This tag has attributes model, price, speed, and ram. In the output, there should be one <PC> element for each <PC> element of the input file, and the values of the attributes should be taken from the corresponding input element.
    \end{enumerate}

    \item \textbf{Exercise 12.3.2:} Suppose our input XML document has the form of the product
    data of Fig. 12.6. Write XSLT stylesheets to produce each of the following
    documents.

    \bigskip

    \begin{enumerate}[a)]
        \item An HTML file with a header for each class. Under each header is a table with column-headers "Name" and "Launched" with the appropriate entry for each ship of the class.
        \item An HTML file with root tag <Losers> and subelements <Ship>, each of whose values is the name of one of the ships that were sunk.
        \item An XML file with root tag <Ships> and subelements <Ship> for each ship. These elements each should have attributes name, class, country and num.Guns with the appropriate values taken from the input file.
        \item Repeat (c), but only list those ships that were in at least one battle.
        \item An XML file identical to the input, except that <Battle> elements should be empty, with the outcome and name of the battle as two attributes.
    \end{enumerate}
\end{enumerate}

\end{document}