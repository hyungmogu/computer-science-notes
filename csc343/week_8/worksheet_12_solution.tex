\documentclass[12pt]{article}
\usepackage[margin=2.5cm]{geometry}
\usepackage{enumerate}
\usepackage{amsfonts}
\usepackage{amsmath}
\usepackage{fancyhdr}
\usepackage{amsmath}
\usepackage{amssymb}
\usepackage{amsthm}
\usepackage{mdframed}
\usepackage{graphicx}
\usepackage{subcaption}
\usepackage{adjustbox}
\usepackage{listings}
\usepackage{xcolor}
\usepackage{booktabs}
\usepackage[utf]{kotex}
\usepackage{hyperref}
\usepackage{accents}

\definecolor{codegreen}{rgb}{0,0.6,0}
\definecolor{codegray}{rgb}{0.5,0.5,0.5}
\definecolor{codepurple}{rgb}{0.58,0,0.82}
\definecolor{backcolour}{rgb}{0.95,0.95,0.92}

\lstdefinestyle{mystyle}{
    backgroundcolor=\color{backcolour},
    commentstyle=\color{codegreen},
    keywordstyle=\color{magenta},
    numberstyle=\tiny\color{codegray},
    stringstyle=\color{codepurple},
    basicstyle=\ttfamily\footnotesize,
    breakatwhitespace=false,
    breaklines=true,
    captionpos=b,
    keepspaces=true,
    numbers=left,
    numbersep=5pt,
    showspaces=false,
    showstringspaces=false,
    showtabs=false,
    tabsize=1
}

\lstset{style=mystyle}

\pagestyle{fancy}
\renewcommand{\headrulewidth}{0.4pt}
\lhead{CSC 343}
\rhead{Worksheet 12 Solution}

\begin{document}
\title{CSC343 Worksheet 12 Solution}
\maketitle

\begin{enumerate}[1.]
    \item

    \begin{itemize}
        \item Keys
        \begin{itemize}
            \item $\{$id of molecule$\}$
            \item $\{$x position, y position, z position$\}$
        \end{itemize}
        \item Functional Dependencies
        \begin{enumerate}[1.]
            \item id of molecule $\to$ x position, y position, z position, x velocity, y velocity, z velocity
            \item x position, y position, z position $\to$ id of molecule, x velocity, y velocity, z velocity
        \end{enumerate}
    \end{itemize}

    \bigskip

    \underline{\textbf{Notes:}}

    \bigskip

    \begin{itemize}
        \item Function Dependencies
        \begin{itemize}
            \item \textit{Functional Dependency} is a relationship between two attributes
            typically between the key and other non-key attributes within a table.

            \bigskip

            \underline{\textbf{Example:}}

            \bigskip

            $\text{SIN} \to \text{Name, Address, Birthdate}$

            \bigskip

            \underline{\textbf{Example 2:}}

            \bigskip

            $\textbf{ISBN} \to \text{Title}$

        \end{itemize}

        \item Key of Relations
        \begin{itemize}
            \item One or more attributes $\{A_1,A_2,...,A_n\}$ is a key for a relation R if
            \begin{enumerate}[1.]
                \item Those attributes functionally determine all other attributes of the relation
                \item No proper subset of $\{A_1,A_2,...A_n\}$ functionally determines all other attributes of R


                \bigskip

                \underline{\textbf{Example:}}

                \bigskip

                Given relation

                \bigskip

                $R = \text{Movies1(title, year, length, genre, studioName, starName)}$

                \begin{enumerate}
                    \item $\{$title, year, starName $\}$ form a key for the relation \textbf{Movies1}
                    \item $\{$ year, starName $\}$ is not a key. Same star can be in multiple
                    movies per year
                \end{enumerate}
            \end{enumerate}

            \item Superkeys
            \begin{itemize}
                \item Means a a set of attributes that contains a key
                \item Don't need to be minimal

                \bigskip

                \underline{\textbf{Example:}}

                Given relation

                \bigskip

                $R = \text{Movies1(title, year, length, genre, studioName, starName)}$

                \begin{itemize}
                    \item $\{$ title, year, starName $\}$ is a key and superkey
                    \item $\{$ title, year, starName, title, year, length$\}$ is a superkey
                \end{itemize}
            \end{itemize}
        \end{itemize}
    \end{itemize}

    \bigskip

    \underline{\textbf{References:}}

    \bigskip

    \begin{enumerate}[1)]
        \item OpenTextBC, Chapter 11 Functional Dependencies, \href{https://opentextbc.ca/dbdesign01/chapter/chapter-11-functional-dependencies/#:~:text=A%20functional%20dependency%20(FD)%20is,determines%20the%20value%20of%20Y.}{link}
    \end{enumerate}

    \item

    \begin{enumerate}[a)]

        \item

        \bigskip

        \underline{\textbf{Notes:}}

        \bigskip

        \begin{itemize}
            \item The Splitting / Combining Rule
            \begin{itemize}
                \item Combining Rule
                \begin{itemize}
                    \item

                    $A_1,A_2,\cdots,A_n \to B_i$ for $i = 1,2,...,m$

                    to

                    $A_1,A_2,\cdots A_n \to B_1,B_2,\cdots B_m$
                \end{itemize}

                \bigskip

                \underline{\textbf{Example:}}

                \bigskip

                Given

                \bigskip

                title year $\to$ length

                title year $\to$ genre

                title year $\to$ studioName

                \bigskip

                it's combined form is

                \bigskip

                title year $\to$ length genre studioName

                \bigskip

                \item Splitting Rule
                \begin{itemize}
                    \item
                    \item

                    $A_1,A_2,\cdots A_n \to B_1,B_2,\cdots B_m$

                    to

                    $A_1,A_2,\cdots,A_n \to B_i$ for $i = 1,2,...,m$

                \end{itemize}

                \bigskip

                \underline{\textbf{Example:}}

                \bigskip

                Given

                \bigskip

                title year $\to$ length

                \bigskip

                It's splitted form is

                \bigskip

                title $\to$ length

                year $\to$ length
            \end{itemize}

            \item Trivial Functional Dependencies

            \bigskip

            \begin{itemize}
                \item A functional dependency $FD: X \to Y$ is \textbf{trivial}
                if $Y$ is a subset of $X$

                \bigskip

                \underline{\textbf{Exmaple:}}

                \bigskip

                title year $\to$ title

            \end{itemize}
        \end{itemize}
    \end{enumerate}
\end{enumerate}

\end{document}