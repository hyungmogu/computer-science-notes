\documentclass[12pt]{article}
\usepackage[margin=2.5cm]{geometry}
\usepackage{enumerate}
\usepackage{amsfonts}
\usepackage{amsmath}
\usepackage{fancyhdr}
\usepackage{amsmath}
\usepackage{amssymb}
\usepackage{amsthm}
\usepackage{mdframed}
\usepackage{graphicx}
\usepackage{subcaption}
\usepackage{adjustbox}
\usepackage{listings}
\usepackage{xcolor}
\usepackage{booktabs}
\usepackage[utf]{kotex}
\usepackage{hyperref}
\usepackage{accents}

\definecolor{codegreen}{rgb}{0,0.6,0}
\definecolor{codegray}{rgb}{0.5,0.5,0.5}
\definecolor{codepurple}{rgb}{0.58,0,0.82}
\definecolor{backcolour}{rgb}{0.95,0.95,0.92}

\lstdefinestyle{mystyle}{
    backgroundcolor=\color{backcolour},
    commentstyle=\color{codegreen},
    keywordstyle=\color{magenta},
    numberstyle=\tiny\color{codegray},
    stringstyle=\color{codepurple},
    basicstyle=\ttfamily\footnotesize,
    breakatwhitespace=false,
    breaklines=true,
    captionpos=b,
    keepspaces=true,
    numbers=left,
    numbersep=5pt,
    showspaces=false,
    showstringspaces=false,
    showtabs=false,
    tabsize=1
}

\lstset{style=mystyle}

\pagestyle{fancy}
\renewcommand{\headrulewidth}{0.4pt}
\lhead{CSC 343}
\rhead{Worksheet 12}

\begin{document}
\title{CSC343 Worksheet 12}
\maketitle

\begin{enumerate}[1.]
    \item \textbf{Exercise 3.1.2:} Consider a relation representing the present position of molecules
    in a clospd ront.ainer. The attributes are an ID for the molecule the x y
    and z coordinat.1·s of the molecule, and its velocity in the x, y, and z di~ensi;ns:
    What FD's would you expect to hold? What are the keys?

    \item \textbf{Exercise 3.2.1:} Consider a relation with schema R(A, B, C, D) and FD's
    AB $\to$ C, C $\to$ D, and D $\to$ A.

    \begin{enumerate}[a)]
        \item What are all the nontrivial FD's that follow from the given FD's? You should restrict yourself to FD's with single attributes on the right side.
        \item What are all the keys of R?
        \item What are all the superkeys for R that are not keys?
    \end{enumerate}

    \item \textbf{Exercise 3.2.2:} Repeat Exercise 3.2.1 for the following schemas and sets of
    FD's:

    \begin{enumerate}[a)]
        \item S(A, B, C, D) with FD's A$\to$ B, B $\to$ C, and B $\to$D.
        \item T(A, B, C, D) with FD's AB $\to$ C, BC $\to$ D, CD $\to$ A, and AD $\to$ B.
        \item U(A, B, C, D) with FD's A$\to$ B, B $\to$ C, C $\to$ D, and D $\to$ A.
    \end{enumerate}

    \item \textbf{Exercise 3.2.3:} Show that the following rules hold, by using the closure test
    of Section 3.2.4.

    \bigskip

    \begin{enumerate}[a)]
        \item Augmenting left sides. If $A_1 A_2 \cdots A_n \to B$ is an $FD$, and $C$ is another attribute, then $A_1 A_2 \cdots A_nC \to B$ follows.
        \item Full augmentation. If $A_1 A_2 \cdots  A_n \to B$ is an $FD$, and $C$ is another attribute, then $A_1A_2 \cdots  A_nC \to BC$ follows. Note: from this rule, the "augmentation" rule mentioned in the box of Section 3.2.7 on "A Complete Set of Inference Rules" can easily be proved.
        \item Pseudotransitivity. Suppose FD's $A_1 A_2 \cdots  A_n \to B_1B_2 \cdots  B_m$ and $C_1C_2 \cdots C_k \to D$ hold, and the $B$'s are each among the $C$'s. Then $A_1A_2 \cdots A_n E_1 E_2 \cdots E_j \to D$  holds, where the E's are all those of the C's that are not found among the B's.
        \item Addition
    \end{enumerate}

\end{enumerate}

\end{document}