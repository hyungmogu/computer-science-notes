\documentclass[12pt]{article}
\usepackage[margin=2.5cm]{geometry}
\usepackage{enumerate}
\usepackage{amsfonts}
\usepackage{amsmath}
\usepackage{fancyhdr}
\usepackage{amsmath}
\usepackage{amssymb}
\usepackage{amsthm}
\usepackage{mdframed}
\usepackage{graphicx}
\usepackage{subcaption}
\usepackage{adjustbox}
\usepackage{listings}
\usepackage{xcolor}
\usepackage{booktabs}
\usepackage[utf]{kotex}
\usepackage{hyperref}
\usepackage{accents}

\definecolor{codegreen}{rgb}{0,0.6,0}
\definecolor{codegray}{rgb}{0.5,0.5,0.5}
\definecolor{codepurple}{rgb}{0.58,0,0.82}
\definecolor{backcolour}{rgb}{0.95,0.95,0.92}

\lstdefinestyle{mystyle}{
    backgroundcolor=\color{backcolour},
    commentstyle=\color{codegreen},
    keywordstyle=\color{magenta},
    numberstyle=\tiny\color{codegray},
    stringstyle=\color{codepurple},
    basicstyle=\ttfamily\footnotesize,
    breakatwhitespace=false,
    breaklines=true,
    captionpos=b,
    keepspaces=true,
    numbers=left,
    numbersep=5pt,
    showspaces=false,
    showstringspaces=false,
    showtabs=false,
    tabsize=1
}

\lstset{style=mystyle}

\pagestyle{fancy}
\renewcommand{\headrulewidth}{0.4pt}
\lhead{CSC 343}
\rhead{Worksheet 12}

\begin{document}
\title{CSC343 Worksheet 12}
\maketitle

\begin{enumerate}[1.]
    \item \textbf{Exercise 3.1.2:} Consider a relation representing the present position of molecules
    in a clospd ront.ainer. The attributes are an ID for the molecule the x y
    and z coordinat.1·s of the molecule, and its velocity in the x, y, and z di~ensi;ns:
    What FD's would you expect to hold? What are the keys?

    \item \textbf{Exercise 3.2.1:} Consider a relation with schema R(A, B, C, D) and FD's
    AB $\to$ C, C $\to$ D, and D $\to$ A.

    \begin{enumerate}[a)]
        \item What are all the nontrivial FD's that follow from the given FD's? You should restrict yourself to FD's with single attributes on the right side.
        \item What are all the keys of R?
        \item What are all the superkeys for R that are not keys?
    \end{enumerate}

    \item \textbf{Exercise 3.2.2:} Repeat Exercise 3.2.1 for the following schemas and sets of
    FD's:

    \begin{enumerate}[a)]
        \item S(A, B, C, D) with FD's A$\to$ B, B $\to$ C, and B $\to$D.
        \item T(A, B, C, D) with FD's AB $\to$ C, BC $\to$ D, CD $\to$ A, and AD $\to$ B.
        \item U(A, B, C, D) with FD's A$\to$ B, B $\to$ C, C $\to$ D, and D $\to$ A.
    \end{enumerate}

    \item \textbf{Exercise 3.2.3:} Show that the following rules hold, by using the closure test
    of Section 3.2.4.

    \bigskip

    \begin{enumerate}[a)]
        \item Augmenting left sides. If $A_1 A_2 \cdots A_n \to B$ is an $FD$, and $C$ is another attribute, then $A_1 A_2 \cdots A_nC \to B$ follows.
        \item Full augmentation. If $A_1 A_2 \cdots  A_n \to B$ is an $FD$, and $C$ is another attribute, then $A_1A_2 \cdots  A_nC \to BC$ follows. Note: from this rule, the "augmentation" rule mentioned in the box of Section 3.2.7 on "A Complete Set of Inference Rules" can easily be proved.
        \item Pseudotransitivity. Suppose FD's $A_1 A_2 \cdots  A_n \to B_1B_2 \cdots  B_m$ and $C_1C_2 \cdots C_k \to D$ hold, and the $B$'s are each among the $C$'s. Then $A_1A_2 \cdots A_n E_1 E_2 \cdots E_j \to D$  holds, where the $E$'s are all those of the $C$'s that are not found among the $B$'s.
        \item Addition If $FD$'s $A_1A_2 \cdots A_n \to B_1B_2 \cdots B_m$ and $C_1C_2 \cdots C_k \to D_1D_2 \cdots D_j$ hold, then $FD$ $A_1A_2 \cdots A_nC_1C_2 \cdots C_K \to B_1B_2 \cdots B_m D_1D_2 \cdots D_j$ also holds. In the above, we should remove one copy of any attribute that appears among both the $A$'s and $C$'s or among both the $B$'s and $D$'s.
    \end{enumerate}

    \item \textbf{Exercise 3.2.4:} Show that each of the following are not valid rules about FD's
    by giving example relations that satisfy the given FD's (following the "if") but
    not the FD that allegedly follows (after the "then").

    \bigskip

    \begin{enumerate}[a)]
        \item If $A \to B$ then $B \to A$.
        \item If $AB \to C$ and $A \to C$, then $B \to C$.
        \item If $AB \to C$, then $A \to C$ or $B \to C$.
    \end{enumerate}

    \item \textbf{Exercise 3.2.5:} Show that if a relation has no attribute that is functionally
    determined by all the other attributes, then the relation has no nontrivial FD's
    at all.

    \item \textbf{Exercise 3.2.6:} Let $X$ and $Y$ be sets of attributes. Show that if $X \subseteq Y$, then
    $X^+ \subseteq Y^+$, where the closures are taken with respect to the same set of $FD$'s.

    \item \textbf{Exercise 3.2.7:} Prove that $(X^+)^+ = X^+$.

    \item \textbf{Exercise 3.2.9:} Find all the minimal bases for the FD's and relation of
    Example 3.11.

    \item \textbf{Exercise 3.2.10:} Suppose we have relation R(A,B,C,D,E), with some set
    of FD's, and we wish to project those FD's onto relation S(A, B, C). Give the
    FD's that hold in S if the FD's for Rare:

    \begin{enumerate}[a)]
        \item $AB \to DE$, $C \to E$, $D \to C$, and $E \to A$.
        \item $A \to D$, $BD \to E$, $AC \to E$, and $DE \to B$.
        \item $AB \to D$, $AC \to E$, $BC \to D$, $D \to A$, and $E \to B$.
        \item $A \to B$, $B \to C$, $C \to D$, $D \to E$, and $E \to A$.
    \end{enumerate}

    In each case, it is sufficient to give a minimal basis for the full set of $FD$'s of $S$.

    \item \textbf{Exercise 3.3.1:} For each of the following relation schemas and Sf'tS of FD's:

    \bigskip

    \begin{enumerate}[a)]
        \item $R(A, B, C, D)$ with $FD$'s $AB \to C$, $C \to D$, and $D \to A$.
        \item $R(A,B,C,D)$ with $FD$'s $B \to C$ and $B \to D$.
        \item $R(A, B, C, D)$ with $FD$'s $AB \to C$, $BC \to D, CD \to A and AD \to B$
        \item $R(A, B, C, D)$ with $FD$'s ,$A \to B$, $B \to C$. $C \to D$, and $D \to A$.
        \item $R(A, B, C, D, E)$ with $FD$'s $AB \to C$, $DE \to C$, and $B \to D$.
        \item $R(A, B, C, D, E)$ with $FD$'s $AB \to C$, $C \to D$, $D \to B$, and $D \to E$.
    \end{enumerate}

    do the following:

    \begin{enumerate}[i)]
        \item Indicate all the BCNF violations. Do not forget to consider FD's that are not in the given set, but follow from them. However, it is not necessary to give violations that have more than one attribute on the right side.
        \item Decompose the relations, as necessary, into collections of relations that are in BCNF.
    \end{enumerate}

    \item \textbf{Exercise 3.3.2}: We mentioned in Section 3.3.4 that we would exercise our
    option to expand the right side of an $FD$ that is a BCNF violation if possible.
    Consider a relation R whose schema is the set of attributes $\{A, B, C, D\}$ with
    $FD$'s $A \to B$ and $A \to C$. Either is a BCNF violation, because the only key
    for $R$ is $\{A, D\}$. Suppose we begin by decomposing R according to $A \to B$. Do
    we ultimately get the same result as if we first expand the BCNF violation to
    $A \to BC$? Why or why not?

    \item \textbf{Exercise 3.3.3:} Let $R$ be as in Exercise 3.3.2, but let the FD's be $A \to B$ and
    $B \to C$. Again compare decomposing using $A \to B$ first against decomposing
    by $A \to BC$ first.

    \item \textbf{Exercise 3.3.4:} Suppose we have a relation schema $R(A, B, C)$ with $FD$ $A \to
    B$. Suppose also that we decide to decompose this schema into $S(A, B)$ and
    $T(B, C)$. Give an example of an instance of relation R whose projection onto
    $S$ and $T$ and subsequent rejoining as in Section 3.4.1 does not yield the same
    relation instance. That is, $\pi_{A,B}(R)$ $\bowtie$ $\pi_{B,C}(R) \neq R$.

\end{enumerate}

\end{document}