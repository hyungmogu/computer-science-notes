\documentclass[12pt]{article}
\usepackage[margin=2.5cm]{geometry}
\usepackage{enumerate}
\usepackage{amsfonts}
\usepackage{amsmath}
\usepackage{fancyhdr}
\usepackage{amsmath}
\usepackage{amssymb}
\usepackage{amsthm}
\usepackage{mdframed}
\usepackage{graphicx}
\usepackage{subcaption}
\usepackage{adjustbox}
\usepackage{listings}
\usepackage{xcolor}
\usepackage{booktabs}
\usepackage[utf]{kotex}
\usepackage{hyperref}

\definecolor{codegreen}{rgb}{0,0.6,0}
\definecolor{codegray}{rgb}{0.5,0.5,0.5}
\definecolor{codepurple}{rgb}{0.58,0,0.82}
\definecolor{backcolour}{rgb}{0.95,0.95,0.92}

\lstdefinestyle{mystyle}{
    backgroundcolor=\color{backcolour},
    commentstyle=\color{codegreen},
    keywordstyle=\color{magenta},
    numberstyle=\tiny\color{codegray},
    stringstyle=\color{codepurple},
    basicstyle=\ttfamily\footnotesize,
    breakatwhitespace=false,
    breaklines=true,
    captionpos=b,
    keepspaces=true,
    numbers=left,
    numbersep=5pt,
    showspaces=false,
    showstringspaces=false,
    showtabs=false,
    tabsize=1
}

\lstset{style=mystyle}

\pagestyle{fancy}
\renewcommand{\headrulewidth}{0.4pt}
\lhead{CSC 343}
\rhead{Worksheet 1}

\begin{document}
\title{CSC343 Worksheet 1}
\maketitle

\noindent \textbf{Note:} This is student designed study guide to make learnings easier.
This does not reflect the course material. Please take it only as a reference.

\begin{enumerate}[1.]
    \item \textbf{Exercise 2.4.2:} Draw expression trees for each of your expressions
    of Exercise 2.4.1

    \item \textbf{Exercise 2.4.3:} This exercise builds upon Exercise 2.3.2 concerning
    World War II capital ships. Recall it involves the following relations:


    Figures 2.22 and 2.23 give some same data for these four relations. Note that unlike
    the data for Exercise 2.4.1, there are some ``dangling tuples'' in this data,
    e.g. ships mentioned in \textit{Outcomes}

    \item \textbf{Exercise 2.4.4:} Draw expression trees for each of your expressions of
    Exercise 2.4.3

    \item \textbf{Exercise 2.4.5:} What is the difference between the natural join $R \bowtie S$ and the
    theta-join $R \bowtie_C S$ where the condition $C$ is that $R.A = S.A$ for
    each attribute $A$ appearing in the schemas of both $R$ and $S$
\end{enumerate}

\bigskip

\underline{\textbf{Reference}}

\bigskip

\begin{enumerate}[1)]
    \item Stanford: CS145 - Introduction to Databases, \href{http://infolab.stanford.edu/~ullman/fcdb/aut07/index.html}{link}
\end{enumerate}

\end{document}