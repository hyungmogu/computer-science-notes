\documentclass[12pt]{article}
\usepackage[margin=2.5cm]{geometry}
\usepackage{enumerate}
\usepackage{amsfonts}
\usepackage{amsmath}
\usepackage{fancyhdr}
\usepackage{amsmath}
\usepackage{amssymb}
\usepackage{amsthm}
\usepackage{mdframed}
\usepackage{graphicx}
\usepackage{subcaption}
\usepackage{adjustbox}
\usepackage{listings}
\usepackage{xcolor}
\usepackage{booktabs}
\usepackage[utf]{kotex}
\usepackage{hyperref}
\usepackage{accents}

\definecolor{codegreen}{rgb}{0,0.6,0}
\definecolor{codegray}{rgb}{0.5,0.5,0.5}
\definecolor{codepurple}{rgb}{0.58,0,0.82}
\definecolor{backcolour}{rgb}{0.95,0.95,0.92}

\lstdefinestyle{mystyle}{
    backgroundcolor=\color{backcolour},
    commentstyle=\color{codegreen},
    keywordstyle=\color{magenta},
    numberstyle=\tiny\color{codegray},
    stringstyle=\color{codepurple},
    basicstyle=\ttfamily\footnotesize,
    breakatwhitespace=false,
    breaklines=true,
    captionpos=b,
    keepspaces=true,
    numbers=left,
    numbersep=5pt,
    showspaces=false,
    showstringspaces=false,
    showtabs=false,
    tabsize=1
}

\lstset{style=mystyle}

\pagestyle{fancy}
\renewcommand{\headrulewidth}{0.4pt}
\lhead{CSC 343}
\rhead{Worksheet 9 Solution}

\begin{document}
\title{CSC343 Worksheet 9 Solution}
\maketitle

\bigskip

\begin{enumerate}[1.]
    \item \textbf{Exercise 11.1.1:}

    \bigskip

    \begin{enumerate}[a)]
        \item

    \begin{center}
    \includegraphics[width=\linewidth]{images/worksheet_9_solution_4.png}
    \end{center}

        \bigskip

        \underline{\textbf{Notes:}}

        \begin{itemize}
            \item Semistructured data
            \begin{itemize}
                \item serves as a model suitable for \textbf{databases integration}, that is,
                for describing the data contained in two or more databases that contain similar data with
                different schemas

                \item It serves as the underlying model for notations such as XML, to be taken
                up in Section 2, that are being used to share information on the web.
            \end{itemize}

            \item Semistructured Data Representation
            \begin{itemize}
                \item is a collection of nodes

            \begin{center}
            \includegraphics[width=\linewidth]{images/worksheet_9_solution_1.png}
            \end{center}

            \begin{center}
            \includegraphics[width=\linewidth]{images/worksheet_9_solution_2.png}
            \end{center}

            \begin{center}
            \includegraphics[width=\linewidth]{images/worksheet_9_solution_3.png}
            \end{center}

            \end{itemize}

            \item
        \end{itemize}

        \item

    \begin{center}
    \includegraphics[width=\linewidth]{images/worksheet_9_solution_5.png}
    \end{center}


        \item

    \begin{center}
    \includegraphics[width=\linewidth]{images/worksheet_9_solution_6.png}
    \end{center}

    \end{enumerate}

    \item

    \begin{center}
    \includegraphics[width=\linewidth]{images/worksheet_9_solution_7.png}
    \end{center}

    \item

    \begin{center}
    \includegraphics[width=\linewidth]{images/worksheet_9_solution_8.png}
    \end{center}

    \item

    The difference is that UML must fit data into its schema, where as the semi
    structured data allows whatever schema information that is appropriate to be attached to data

    \bigskip

    \underline{\textbf{Notees:}}

    \bigskip

    \begin{itemize}
        \item Semi-structured Data
        \begin{itemize}
            \item Is schemaless
            \item Is motivated primarily by its flexibility
            \item One could enter data at will, and attach to the data whatever schema information
            you felt was appropriate for that data.
            \item Makes query processing harder
        \end{itemize}
        \item Structured Data
        \begin{itemize}
            \item Is rigid framework into which data is placed.
            \item Data must fit into schema
            \item Fixed schema allows data to be organized with data structures
            that support efficient answering of queries
            \item e.g. UML, E/R, Relational, ODL
        \end{itemize}
    \end{itemize}

    \item

    \bigskip

    \begin{itemize}
        \item XML
        \begin{itemize}
            \item is called \textit{Extensible Markup Language}
            \item is an example of semistructured data
        \end{itemize}
    \end{itemize}

\end{enumerate}

\end{document}