\documentclass[12pt]{article}
\usepackage[margin=2.5cm]{geometry}
\usepackage{enumerate}
\usepackage{amsfonts}
\usepackage{amsmath}
\usepackage{fancyhdr}
\usepackage{amsmath}
\usepackage{amssymb}
\usepackage{amsthm}
\usepackage{mdframed}
\usepackage{graphicx}
\usepackage{subcaption}
\usepackage{adjustbox}
\usepackage{listings}
\usepackage{xcolor}
\usepackage{booktabs}
\usepackage[utf]{kotex}
\usepackage{hyperref}
\usepackage{accents}

\definecolor{codegreen}{rgb}{0,0.6,0}
\definecolor{codegray}{rgb}{0.5,0.5,0.5}
\definecolor{codepurple}{rgb}{0.58,0,0.82}
\definecolor{backcolour}{rgb}{0.95,0.95,0.92}

\lstdefinestyle{mystyle}{
    backgroundcolor=\color{backcolour},
    commentstyle=\color{codegreen},
    keywordstyle=\color{magenta},
    numberstyle=\tiny\color{codegray},
    stringstyle=\color{codepurple},
    basicstyle=\ttfamily\footnotesize,
    breakatwhitespace=false,
    breaklines=true,
    captionpos=b,
    keepspaces=true,
    numbers=left,
    numbersep=5pt,
    showspaces=false,
    showstringspaces=false,
    showtabs=false,
    tabsize=1
}

\lstset{style=mystyle}

\pagestyle{fancy}
\renewcommand{\headrulewidth}{0.4pt}
\lhead{CSC 343}
\rhead{Worksheet 9}

\begin{document}
\title{CSC343 Worksheet 9}
\maketitle

\bigskip

\begin{enumerate}[1.]
    \item \textbf{Exercise 11.1.1:} Since there is no schema to design in the semistructured-data model,
    we cannot ask you design schemas to describe different situations. Rahter,
    in the following exercises we shall ask you to suggest how particular data
    might be organized to reflect certan facts.

    \bigskip

    \begin{enumerate}[a)]
        \item Add to Fig. 1 the facts that \textit{Star Wars} was directed by
        George Lucas and produced by Gary Kurtz
        \item Add to Fig. 1 information about \textit{Empire Strikes Back} and
        \textit{Return of the Jedi}, including the factds that Carrie Fisher and
        Mark Hamill appeared in these movies
        \item Add to (b) information about the studio (Fox) for these movies and
        the address of the studio (Hollywood).
    \end{enumerate}

    \item \textbf{Exercise 11.1.2:} Suggest how typical data about banks and customers,
    as in Exercise 4.1.1, could be represented in the semistructured model

    \item \textbf{Exercise 11.1.4:} Suggest how typical data about genealogy, as was
    described in Exercise 4.1.6, could be represented in the semistructured model

    \item \textbf{Exercise 11.1.5:} UML and the semistructured-data model are both "graphical"
    in nature, in the sense that they use nodes, labels, and connections among
    nodes as the medium of expression. Yet there is an essential difference between
    the two models. What is it?

    \item \textbf{Exercise 11.2.1:} Repeat Exercise 1.1 using XML
    \item \textbf{Exercise 11.2.2:} Show that any relation can be represented by an XML
    document. \textit{Hint:} Create an element for each tuple with a subelement for each
    component of that tuple.

    \item \textbf{Exercise 11.2.3:} How would you represent an empty element (one that had
    neither text nor subelements) in the database schema of Section 2.7?

    \item \textbf{Exercise 11.2.4:} In section 2.7, we gave a database schema for representing
    documents that do not have mixed content - elements that contain a mixture
    of text (\textit{\#PCDATA}) and subelements. Show how to modify the schema when
    elements can have mixed content.

    \item \textbf{Exercise 11.3.1:} Add to the document of Fig. 10 the following facts:

    \begin{enumerate}[a)]
        \item Carrie Fisher and Mark Hamil also starred \textit{The empire Strikes Back} (1980)
        and \textit{Return of the Jedi} (1983)
        \item Harrison Ford also starred in \textit{Star Wars}, in the two movies
        mentioned in (a), and the movie \textit{Firewall} (2006)
        \item Carrie Fisher also starred in \textit{Hannah and Her Sisters} (1985)
        \item Matt Damon starred in \textit{The Bourne Identity} (2002).
    \end{enumerate}

    \item \textbf{Exercise 11.3.2:} Suggest how typical data about banks and customers could be
    represented as a DTD

    \item \textbf{Exercise 11.3.3:} Suggest how typical data about players, teams,
    and fans could be represented as a DTD

    \item \textbf{Exercise 11.3.4:} Suggest how tpical data about a genealogy could be represented
    as a DTD

    \item \textbf{Exercise 11.3.5:} Using your representation from Exercise 22, devise an
    algorithm that will take any relation schema (a relation name and a list of attribute name)
    and produce a DTD describing a document that represents that relation.


    \item \textbf{Exercise 11.4.1:} Give an example of a document that conforms to the XML
    Schema definition of Fig. 12 and an example of one that has all the elements mentioned,
    but does not conform to the definition

    \item \textbf{Exercise 11.4.2:} Rewrite Fig. 12 so that there is a named complex type
    for \textit{Movies}, but no named type for \textit{Movie}.

    \item \textbf{Exercise 11.4.3:} Write the XML schema definitions of Fig. 19 and 20 as a DTD

\end{enumerate}

\end{document}