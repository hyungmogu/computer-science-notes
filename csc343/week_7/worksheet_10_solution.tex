\documentclass[12pt]{article}
\usepackage[margin=2.5cm]{geometry}
\usepackage{enumerate}
\usepackage{amsfonts}
\usepackage{amsmath}
\usepackage{fancyhdr}
\usepackage{amsmath}
\usepackage{amssymb}
\usepackage{amsthm}
\usepackage{mdframed}
\usepackage{graphicx}
\usepackage{subcaption}
\usepackage{adjustbox}
\usepackage{listings}
\usepackage{xcolor}
\usepackage{booktabs}
\usepackage[utf]{kotex}
\usepackage{hyperref}
\usepackage{accents}

\definecolor{codegreen}{rgb}{0,0.6,0}
\definecolor{codegray}{rgb}{0.5,0.5,0.5}
\definecolor{codepurple}{rgb}{0.58,0,0.82}
\definecolor{backcolour}{rgb}{0.95,0.95,0.92}

\lstdefinestyle{mystyle}{
    backgroundcolor=\color{backcolour},
    commentstyle=\color{codegreen},
    keywordstyle=\color{magenta},
    numberstyle=\tiny\color{codegray},
    stringstyle=\color{codepurple},
    basicstyle=\ttfamily\footnotesize,
    breakatwhitespace=false,
    breaklines=true,
    captionpos=b,
    keepspaces=true,
    numbers=left,
    numbersep=5pt,
    showspaces=false,
    showstringspaces=false,
    showtabs=false,
    tabsize=1
}

\lstset{style=mystyle}

\pagestyle{fancy}
\renewcommand{\headrulewidth}{0.4pt}
\lhead{CSC 343}
\rhead{Worksheet 10 Solution}

\begin{document}
\title{CSC343 Worksheet 10 Solution}
\maketitle

\bigskip

\begin{enumerate}[1.]
    \item

    \begin{enumerate}[a)]

        \item

        \bigskip

        \underline{\textbf{Notes:}}

        \begin{itemize}
            \item XPATH and Selecting Nodes
            \begin{itemize}
                \item nodename
                \begin{itemize}
                    \item Selects all nodes with the name "nodename"
                \end{itemize}
                \item /
                \begin{itemize}
                    \item Selects from the root node
                \end{itemize}
                \item //
                \begin{itemize}
                    \item Selects node in the document from the current node that
                    match the selection no matter where they are
                \end{itemize}
                \item .
                \begin{itemize}
                    \item Select the current node
                \end{itemize}
                \item ..
                \begin{itemize}
                    \item Selects the parent of the current node
                \end{itemize}
                \item @
                \begin{itemize}
                    \item Selects attributes
                \end{itemize}
            \end{itemize}

            \bigskip

            \underline{\textbf{Example:}}

    \begin{lstlisting}[language=XML]
    /StarMovieData/Star//City
    \end{lstlisting}

            \bigskip

            \begin{itemize}
                \item selects all City element in

                \bigskip

                $<$StarMovieData$>$

                \quad$<$Star$>$

                \quad\quad\textit{Here :)}

                \quad$<$/Star$>$

                $<$/StarMovieData$>$
            \end{itemize}

            \item Wildcards *
            \begin{itemize}
                \item Is used to say 'any tag'
            \end{itemize}

            \bigskip

            \underline{\textbf{Example:}}

            \bigskip

    \begin{lstlisting}[language=XML]
    /StarMovieData/*/@*
    \end{lstlisting}

            \bigskip

            \begin{itemize}
                \item `@*' means any attributes
                \item `*' means any tag
            \end{itemize}
            \item Context of Expressions
            \begin{itemize}
                \item $[...]$ means that exists or there exists
                \item $[\textit{integer}]$ selects ith child of its parent
                \item $[\textit{Tag}]$ selects elements that have one or more sublements with `Tag'
                \item$[\textit{Attribute}]$ selects elements that have attribute `Attribute'

                \bigskip

                \underline{\textbf{Example:}}

                \bigskip

    \begin{lstlisting}[language=XML]
    /StarMovieData/Star[//City = "Malibu"]/Name
    \end{lstlisting}

                \bigskip

                \begin{itemize}
                    \item Means select all Star Name that contains City with value `Malibu'
                \end{itemize}
            \end{itemize}
        \end{itemize}

    \end{enumerate}
\end{enumerate}

\end{document}