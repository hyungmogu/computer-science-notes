\documentclass[12pt]{article}
\usepackage[margin=2.5cm]{geometry}
\usepackage{enumerate}
\usepackage{amsfonts}
\usepackage{amsmath}
\usepackage{fancyhdr}
\usepackage{amsmath}
\usepackage{amssymb}
\usepackage{amsthm}
\usepackage{mdframed}
\usepackage{graphicx}
\usepackage{subcaption}
\usepackage{adjustbox}
\usepackage{listings}
\usepackage{xcolor}
\usepackage{booktabs}
\usepackage[utf]{kotex}
\usepackage{hyperref}
\usepackage{accents}

\definecolor{codegreen}{rgb}{0,0.6,0}
\definecolor{codegray}{rgb}{0.5,0.5,0.5}
\definecolor{codepurple}{rgb}{0.58,0,0.82}
\definecolor{backcolour}{rgb}{0.95,0.95,0.92}

\lstdefinestyle{mystyle}{
    backgroundcolor=\color{backcolour},
    commentstyle=\color{codegreen},
    keywordstyle=\color{magenta},
    numberstyle=\tiny\color{codegray},
    stringstyle=\color{codepurple},
    basicstyle=\ttfamily\footnotesize,
    breakatwhitespace=false,
    breaklines=true,
    captionpos=b,
    keepspaces=true,
    numbers=left,
    numbersep=5pt,
    showspaces=false,
    showstringspaces=false,
    showtabs=false,
    tabsize=1
}

\lstset{style=mystyle}

\pagestyle{fancy}
\renewcommand{\headrulewidth}{0.4pt}
\lhead{CSC 343}
\rhead{Worksheet 14 Solution}

\begin{document}
\title{CSC343 Worksheet 14 Solution}
\maketitle

\begin{enumerate}[1.]
    \item

    \begin{center}
    \includegraphics[width=\linewidth]{images/worksheet_14_solution_13.png}
    \end{center}

    \bigskip

    \begin{mdframed}
        \underline{\textbf{Correct Solution:}}

        \bigskip

        \begin{center}
        \includegraphics[width=\linewidth]{images/worksheet_14_solution_13.png}
        \end{center}

    \end{mdframed}

    \bigskip

    \underline{\textbf{Notes:}}

    \bigskip

    \begin{itemize}
        \item E/R Model
        \begin{itemize}
            \item Means \textbf{Entity Relationship Model}
            \item Entity Relationship Model(ER Modeling) is a graphical approach to database design.
            \item Is comparable to class diagram in UML
            \item Uses three principle element types:

            \begin{enumerate}[1.]
                \item Entity sets
                \begin{itemize}
                    \item Is an abstract object of some sort (i.e. entitiy)
                    \item Is not used to represent class
                    \item Is represented by rectangles
                \end{itemize}

                \begin{center}
                \includegraphics[width=0.7\linewidth]{images/worksheet_14_solution_1.png}
                \end{center}

                \item Attributes
                \begin{itemize}
                    \item Are properties of entities in a set (i.e. column name)
                    \item Each has its own primitive data types (e.g. String, integers, Reals)
                    \item Is represented by ovals
                \end{itemize}

                \begin{center}
                \includegraphics[width=0.7\linewidth]{images/worksheet_14_solution_2.png}
                \end{center}

                \item Relationships
                \begin{itemize}
                    \item Are connections among two or more entity sets (e.g. intermediary Relations like Stars In)
                    \item Is represented by diamond
                \end{itemize}

                \begin{center}
                \includegraphics[width=0.7\linewidth]{images/worksheet_14_solution_3.png}
                \end{center}
            \end{enumerate}

            \bigskip

            \underline{\textbf{Example:}}

            \bigskip

            \begin{center}
            \includegraphics[width=0.7\linewidth]{images/worksheet_14_solution_4.png}
            \end{center}
        \end{itemize}

        \item Multiway Relationships

        \begin{itemize}
            \item Connects more than two relationship sets
            \item Enables to represent relationships that otherwise is
            difficult in binary relationship
            \item Arrow $\to$ 'one'
            \item No arrow $\to$ 'many'
        \end{itemize}

        \bigskip

        \underline{\textbf{Example:}}

        \bigskip

        \begin{center}
        \includegraphics[width=\linewidth]{images/worksheet_14_solution_5.png}
        \end{center}

        \bigskip

        \underline{\textbf{Example 2:}}

        \bigskip

        \begin{center}
        \includegraphics[width=\linewidth]{images/worksheet_14_solution_6.png}
        \end{center}

        \bigskip

        \item Roles in Relationships
        \begin{itemize}
            \item Is the label of edges between the entity set and relationship
            \item Are used to clarify the sementics of relationship
        \end{itemize}

        \bigskip

        \underline{\textbf{Example:}}

        \bigskip

        \begin{center}
        \includegraphics[width=\linewidth]{images/worksheet_14_solution_7.png}
        \end{center}

        \bigskip

        \underline{\textbf{Example 2:}}

        \bigskip

        \begin{center}
        \includegraphics[width=\linewidth]{images/worksheet_14_solution_8.png}
        \end{center}

        \item Attributes on Relationships

        \begin{itemize}
            \item can be thought as a property of tuples in the relationship set
            (i.e. String, Integer, Float, Boolean)

            \bigskip

            \underline{\textbf{Example:}}

            \bigskip

            \begin{center}
            \includegraphics[width=\linewidth]{images/worksheet_14_solution_9.png}
            \end{center}

            \item Can be removed by creating an entity set with the attribute

            \bigskip

              \underline{\textbf{Example:}}

            \bigskip

            \begin{center}
            \includegraphics[width=\linewidth]{images/worksheet_14_solution_10.png}
            \end{center}
        \end{itemize}

        \item Conversting Multiway Relationships to Binary

        \bigskip

        \underline{\textbf{Example:}}

        \bigskip

        \begin{center}
        \includegraphics[width=\linewidth]{images/worksheet_14_solution_11.png}
        \end{center}


        \item Subclasses in the E/R Model

        \begin{itemize}
            \item Has its own special attributes and/or relationships
            \item All `\textit{isa}' relationship is \underline{one to one}
            \item Is represented by triangle with label `\textit{isa}' followed by
            entity set
        \end{itemize}

        \bigskip

        \underline{\textbf{Example:}}

        \bigskip

        \begin{center}
        \includegraphics[width=\linewidth]{images/worksheet_14_solution_12.png}
        \end{center}

        \bigskip



    \end{itemize}

    \item

    \begin{enumerate}[a)]
        \item

        \begin{center}
        \includegraphics[width=\linewidth]{images/worksheet_14_solution_15.png}
        \end{center}

        \bigskip

        \begin{mdframed}
            \underline{\textbf{Correct Solution:}}

            \bigskip

            \begin{center}
            \includegraphics[width=\linewidth]{images/worksheet_14_solution_17.png}
            \end{center}
        \end{mdframed}

        \item

        \begin{center}
        \includegraphics[width=\linewidth]{images/worksheet_14_solution_16.png}
        \end{center}

        \bigskip

        \begin{mdframed}
            \underline{\textbf{Correct Solution:}}

            \bigskip

            \begin{center}
            \includegraphics[width=\linewidth]{images/worksheet_14_solution_18.png}
            \end{center}
        \end{mdframed}

        \item

        \begin{center}
        \includegraphics[width=\linewidth]{images/worksheet_14_solution_19.png}
        \end{center}

        \bigskip

        \begin{mdframed}
            \underline{\textbf{Correct Solution:}}

            \bigskip

            \begin{center}
            \includegraphics[width=\linewidth]{images/worksheet_14_solution_20.png}
            \end{center}

        \end{mdframed}

        \item

        \begin{center}
        \includegraphics[width=\linewidth]{images/worksheet_14_solution_21.png}
        \end{center}

        \bigskip

        \begin{mdframed}
            \underline{\textbf{Correct Solution:}}

            \bigskip

            \begin{center}
            \includegraphics[width=\linewidth]{images/worksheet_14_solution_22.png}
            \end{center}

        \end{mdframed}

    \end{enumerate}

    \item

    \begin{center}
    \includegraphics[width=\linewidth]{images/worksheet_14_solution_23.png}
    \end{center}

    \bigskip

    \begin{mdframed}

        \underline{\textbf{Correct Solution}}

        \bigskip

        \begin{center}
        \includegraphics[width=\linewidth]{images/worksheet_14_solution_24.png}
        \end{center}

    \end{mdframed}

    \item


    \begin{enumerate}[a)]
        \item

        \begin{center}
        \includegraphics[width=\linewidth]{images/worksheet_14_solution_25.png}
        \end{center}

        \item

        \begin{center}
        \includegraphics[width=\linewidth]{images/worksheet_14_solution_26.png}
        \end{center}

        \item

        They are the same. (I need more work on providing reason).


        \bigskip

        \underline{\textbf{Notes:}}

        \bigskip

        \begin{itemize}
            \item I should ask professor about this :'(
        \end{itemize}


    \end{enumerate}

    \item

    \begin{center}
    \includegraphics[width=\linewidth]{images/worksheet_14_solution_27.png}
    \end{center}

    \item

    \begin{center}
    \includegraphics[width=\linewidth]{images/worksheet_14_solution_28.png}
    \end{center}


    \bigskip

    \begin{mdframed}
        \underline{\textbf{Correct Solution:}}

        \bigskip

        \begin{center}
        \includegraphics[width=\linewidth]{images/worksheet_14_solution_29.png}
        \end{center}
    \end{mdframed}

    \item

    \begin{center}
    \includegraphics[width=\linewidth]{images/worksheet_14_solution_30.png}
    \end{center}

    \bigskip

    \underline{\textbf{Notes:}}

    \bigskip

    \begin{itemize}
        \item I feel the need to clarify with professor if two parent subclasses can
        exist
        \item I feel the need to ask professor whether this design is valid
    \end{itemize}

    \item

    \begin{enumerate}[a)]
        \item

        \begin{center}
        \includegraphics[width=\linewidth]{images/worksheet_14_solution_31.png}
        \end{center}

        \item

        \begin{center}
        \includegraphics[width=\linewidth]{images/worksheet_14_solution_32.png}
        \end{center}

        \bigskip

        \underline{\textbf{Notes:}}

        \bigskip

        \begin{itemize}
            \item  I need to clarify with professor on one-to-many relationship.

            \bigskip

            Is it correct that the `one` side of `one-to-many' relationship represent
            foreign key in terms of SQL?

            \bigskip

            But how about the many side? What does it mean it to be many? so for example,
            ('Josh', 'Neville the father', 'Mary the mother'), ('Jay', 'Neville the father', 'Mary the mother'),
            is this one to many relationship?

            \bigskip

            In tabular terms / example what does one-to-many relationship represent
            in this context?
        \end{itemize}

    \end{enumerate}

    \item

    \begin{center}
    \includegraphics[width=\linewidth]{images/worksheet_14_solution_33.png}
    \end{center}

    \item

    \begin{center}
    \includegraphics[width=\linewidth]{images/worksheet_14_solution_34.png}
    \end{center}

    \bigskip

    \begin{mdframed}
        \underline{\textbf{Correct Solution:}}

        \bigskip

        \begin{center}
        \includegraphics[width=\linewidth]{images/worksheet_14_solution_35.png}
        \end{center}

    \end{mdframed}

    \item

    \bigskip

    Simplicity count is violated. There is more than necessary number of entitiy sets
    and attributs for address and accounts.

    \bigskip

    \begin{center}
    \includegraphics[width=\linewidth]{images/worksheet_14_solution_38.png}
    \end{center}


    \bigskip

    \underline{\textbf{Notes:}}

    \bigskip

    \begin{itemize}
        \item Design Principles

        \begin{enumerate}[1.]
            \item Faithfulness

            \begin{itemize}
                \item means design should make sense and meet its specification
                \item e.g. Adding attribute \textit{number-of-cylinders} to \textit{Stars} $\to$ NONO
            \end{itemize}

            \item Avoiding Redundancy

            \begin{itemize}
                \item \textit{Redundancy} means saying the same thing in
                two (or more) different ways
            \end{itemize}

            \bigskip

            \underline{\textbf{Example (The good example):}}

            \bigskip

            \begin{center}
            \includegraphics[width=\linewidth]{images/worksheet_14_solution_36.png}
            \end{center}

            \bigskip

            \underline{\textbf{Example (The bad example):}}

            \bigskip

            \begin{center}
            \includegraphics[width=\linewidth]{images/worksheet_14_solution_37.png}
            \end{center}

            \item Simplicty Counts

            \begin{itemize}
                \item Avoid adding more more elements than necessary
            \end{itemize}
            \item Choosing the Right Relationships

            \begin{itemize}
                \item Don't add relationships more than necessary
            \end{itemize}

            \item Picking the Right Kind of Element

            \begin{itemize}
                \item Many of the choices are between using attributes and using
                entity set / relationship combinations
            \end{itemize}
        \end{enumerate}
    \end{itemize}

    \item

    They should be combined when each studio has unique president

    \item

    \underline{\textbf{Solution:}}

    \bigskip

    \begin{center}
    \includegraphics[width=\linewidth]{images/worksheet_14_solution_39.png}
    \end{center}

    \item

    \begin{enumerate}[a)]
        \item \underline{\textbf{Solution:}}

        \bigskip

        \begin{center}
        \includegraphics[width=\linewidth]{images/worksheet_14_solution_40.png}
        \end{center}

        \item \underline{\textbf{Solution:}}

        \bigskip

        \begin{center}
        \includegraphics[width=\linewidth]{images/worksheet_14_solution_41.png}
        \end{center}

        \item \underline{\textbf{Solution:}}

        \bigskip

        \begin{center}
        \includegraphics[width=\linewidth]{images/worksheet_14_solution_42.png}
        \end{center}

        \bigskip

        \underline{\textbf{Notes:}}

        \bigskip

        \begin{itemize}
            \item Multivalued attributes is denoted by the following $^{[1]}$

            \begin{center}
            \includegraphics[width=0.7\linewidth]{images/worksheet_14_solution_43.png}
            \end{center}
        \end{itemize}

        \bigskip

        \underline{\textbf{References:}}

        \bigskip

        \begin{enumerate}[1)]
            \item OpenTextBC, The Entity Relationship Model, \href{https://opentextbc.ca/dbdesign01/chapter/chapter-8-entity-relationship-model/}{link}
        \end{enumerate}
    \end{enumerate}

    \item

    \begin{enumerate}[a)]
        \item

        \underline{\textbf{Solution:}}

        \bigskip

        \begin{enumerate}[i)]

            \item

            \begin{center}
            \includegraphics[width=0.7\linewidth]{images/worksheet_14_solution_46.png}
            \end{center}

            \item

            \begin{center}
            \includegraphics[width=0.7\linewidth]{images/worksheet_14_solution_47.png}
            \end{center}

        \end{enumerate}
        \bigskip

        \underline{\textbf{Notes:}}

        \bigskip

        \begin{itemize}
            \item Keys in the E/R Model

            \begin{itemize}
                \item \textit{key} is an attribute or a group of attributes
                whose values can be used to uniquely identify an individual entity in an
                entity set $^{[1]}$

                \item Keys are represented using `underline' under each attribute

                \begin{center}
                \includegraphics[width=0.7\linewidth]{images/worksheet_14_solution_44.png}
                \end{center}
            \end{itemize}

            \item Referential Integrity

            \begin{itemize}
                \item Means value appearing in one context must also appear in
                another
                \item Functions like Foriegn Key in SQL
                \item Is represented by a rounded arrow

                \begin{center}
                \includegraphics[width=0.7\linewidth]{images/worksheet_14_solution_45.png}
                \end{center}
            \end{itemize}
        \end{itemize}

        \bigskip

        \underline{\textbf{References:}}

        \bigskip

        \begin{enumerate}[1)]
            \item OpenTextBC, The Entity Relationship Model, \href{https://opentextbc.ca/dbdesign01/chapter/chapter-8-entity-relationship-model/}{link}
        \end{enumerate}

        \item

        \underline{\textbf{Solution:}}

        \bigskip

        \begin{enumerate}[i)]

            \item

            \begin{center}
            \includegraphics[width=0.7\linewidth]{images/worksheet_14_solution_48.png}
            \end{center}

            \item

            \begin{center}
            \includegraphics[width=0.7\linewidth]{images/worksheet_14_solution_49.png}
            \end{center}

        \end{enumerate}

        \item

        \underline{\textbf{Solution:}}

        \bigskip

        \begin{enumerate}[i)]

            \item

            \begin{center}
            \includegraphics[width=\linewidth]{images/worksheet_14_solution_50.png}
            \end{center}

            \item

            \begin{center}
            \includegraphics[width=\linewidth]{images/worksheet_14_solution_51.png}
            \end{center}


        \end{enumerate}
    \end{enumerate}

    \item

    \underline{\textbf{Solution:}}

    \bigskip

    \begin{center}
    \includegraphics[width=\linewidth]{images/worksheet_14_solution_55.png}
    \end{center}

    \bigskip


    \begin{mdframed}

        \underline{\textbf{Correct Solution:}}

        \bigskip

        \begin{center}
        \includegraphics[width=\linewidth]{images/worksheet_14_solution_57.png}
        \end{center}

        \bigskip

        \color{red}
        \begin{itemize}
            \item Yes. grade is a part of a key for enrollent.
        \end{itemize}
        \color{black}
    \end{mdframed}

    \bigskip

    \underline{\textbf{Notes:}}

    \bigskip

    \begin{itemize}
        \item Weak Entity Sets
        \begin{itemize}
            \item Is an entity set of which some or all of attributes belong to
            another entity set
            \item Is denoted by the following symbol:

            \begin{center}
            \includegraphics[width=\linewidth]{images/worksheet_14_solution_52.png}
            \end{center}

            \item Depends on a dominant entity, and it cannot exist without a strong entity. $^{[1]}$
            \item Has attributes in both weak entity sets and the entity sets

            \bigskip

            \underline{\textbf{Example:}}

            \bigskip

            \begin{center}
            \includegraphics[width=\linewidth]{images/worksheet_14_solution_56.png}
            \end{center}

            \bigskip

            \begin{itemize}
                \item Schemas for the strong entity sets

                \begin{itemize}
                    \item student (\underline{username})
                    \item assignment (\underline{shortname}, due\_date, url)
                \end{itemize}

                \item Schemas for the weak entity set

                \begin{itemize}
                    \item submission(\underline{username}, \underline{shortname}, \underline{version}, submit\_date, data)
                \end{itemize}
            \end{itemize}


        \end{itemize}

        \item Requirements for Entity Sets

        \begin{itemize}
            \item E is a weak entity if it consists of

            \begin{enumerate}[1.]
                \item Zero or more of its own attributes, and
                \item One or more many-one relationships to other (supporting) entity sets. $^{[2]}$
            \end{enumerate}

        \end{itemize}

        \begin{center}
        \includegraphics[width=\linewidth]{images/worksheet_14_solution_53.png}
        \includegraphics[width=\linewidth]{images/worksheet_14_solution_54.png}
        \end{center}

    \end{itemize}

    \bigskip

    \underline{\textbf{References:}}

    \bigskip

    \begin{enumerate}[1)]
        \item StackOverflow, Example of a strong and weak entity types, \href{https://stackoverflow.com/questions/4741967/example-of-a-strong-and-weak-entity-types}{link}
        \item Stanford, Entity-Relationship Model, \href{http://infolab.stanford.edu/~ullman/fcdb/aut07/slides/er.pdf}{link}
        \item Caltech, Converting E-R Diagrams to Relational Model, \href{http://users.cms.caltech.edu/~donnie/dbcourse/intro0607/lectures/Lecture17.pdf}{link}
    \end{enumerate}


    \item

    \begin{center}
    \includegraphics[width=\linewidth]{images/worksheet_14_solution_58.png}
    \end{center}

    \item

    \begin{enumerate}[a)]
        \item

        \underline{\textbf{Solution:}}

        \begin{center}
        \includegraphics[width=\linewidth]{images/worksheet_14_solution_59.png}
        \end{center}

        \item

        \begin{center}
        \includegraphics[width=\linewidth]{images/worksheet_14_solution_60.png}
        \end{center}



    \end{enumerate}




\end{enumerate}

\end{document}