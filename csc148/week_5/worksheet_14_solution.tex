\documentclass[12pt]{article}
\usepackage[margin=2.5cm]{geometry}
\usepackage{enumerate}
\usepackage{amsfonts}
\usepackage{amsmath}
\usepackage{fancyhdr}
\usepackage{amsmath}
\usepackage{amssymb}
\usepackage{amsthm}
\usepackage{mdframed}
\usepackage{graphicx}
\usepackage{subcaption}
\usepackage{adjustbox}
\usepackage{listings}
\usepackage{xcolor}
\usepackage{booktabs}
\usepackage[utf]{kotex}

\definecolor{codegreen}{rgb}{0,0.6,0}
\definecolor{codegray}{rgb}{0.5,0.5,0.5}
\definecolor{codepurple}{rgb}{0.58,0,0.82}
\definecolor{backcolour}{rgb}{0.95,0.95,0.92}

\lstdefinestyle{mystyle}{
    backgroundcolor=\color{backcolour},
    commentstyle=\color{codegreen},
    keywordstyle=\color{magenta},
    numberstyle=\tiny\color{codegray},
    stringstyle=\color{codepurple},
    basicstyle=\ttfamily\footnotesize,
    breakatwhitespace=false,
    breaklines=true,
    captionpos=b,
    keepspaces=true,
    numbers=left,
    numbersep=5pt,
    showspaces=false,
    showstringspaces=false,
    showtabs=false,
    tabsize=1
}

\lstset{style=mystyle}

\begin{document}
\title{CSC148 Worksheet 14 Solution}
\author{Hyungmo Gu}
\maketitle

\section*{Question 1}
\begin{enumerate}[a.]
    \item

    \adjustbox{center,valign=t}{
        \begin{tabular}{|p{10cm}|c|}
            \hline
            \textbf{Operation} & \textbf{Running time}\\
            \hline
            Insert at the front of the list & $\mathcal{O}(n)$\\
            \hline
            Insert at the end of the list & $\mathcal{O}(1)$\\
            \hline
            Look up the element at index $i$, where $0 \leq i < n$ & $\mathcal{O}(n)$\\
            \hline
        \end{tabular}
    }

    \bigskip

    \begin{mdframed}
        \underline{\textbf{Correct Solution:}}

        \bigskip

        \adjustbox{center,valign=t}{
        \begin{tabular}{|p{10cm}|c|}
            \hline
            \textbf{Operation} & \textbf{Running time}\\
            \hline
            Insert at the front of the list & $\mathcal{O}(n)$\\
            \hline
            Insert at the end of the list & $\mathcal{O}(1)$\\
            \hline
            Look up the element at index $i$, where $0 \leq i < n$ & $\color{red}\mathcal{O}(1)$\\
            \hline
        \end{tabular}
    }

    \end{mdframed}

    \item

    The inserting of an element at position $i$ requires $n - i$ elements to
    be shifted to right.

    \bigskip

    Using this fact, we can write the Big-Oh expression for inserting an item at
    index $i$ is $\mathcal{O}(n - i)$.
\end{enumerate}

\section*{Question 2}
\begin{enumerate}[a.]
    \item

    \bigskip

    \adjustbox{center,valign=t}{
        \begin{tabular}{|p{10cm}|c|}
            \hline
            \textbf{Operation} & \textbf{Running time}\\
            \hline
            Insert at the front of the linked list & $\mathcal{O}(1)$\\
            \hline
            Insert at the end of the linked list & $\mathcal{O}(n)$\\
            \hline
            Look up the element at index $i$, where $0 \leq i < n$ & $\mathcal{O}(n)$\\
            \hline
        \end{tabular}
    }

    \newpage

    \begin{mdframed}
        \underline{\textbf{Correct Solution:}}

        \bigskip

        \adjustbox{center,valign=t}{
            \begin{tabular}{|p{10cm}|c|}
                \hline
                \textbf{Operation} & \textbf{Running time}\\
                \hline
                Insert at the front of the linked list & $\mathcal{O}(1)$\\
                \hline
                Insert at the end of the linked list & $\mathcal{O}(n)$\\
                \hline
                Look up the element at index $i$, where $0 \leq i < n$ & $\color{red}\mathcal{O}(i)$\\
                \hline
            \end{tabular}
        }

    \end{mdframed}

    \item

    Without the traversal, the running time of inserting is $\mathcal{O}(1)$.

    \bigskip

    With the traversal, the running time of inserting is $\mathcal{O}(i)$.

\end{enumerate}

\section*{Question 3}
\begin{itemize}
    \item

    Unlike linked lists that store node at different memory location, array-based
    lists store elements in memory immediately one after another.

    \bigskip

    Assuming it's easier for memory to find and perform operations on elements
    located right after another, I believe it's significantly faster for
    array-based lists to insert an element at position $i$.

    \bigskip

    \begin{mdframed}
        \underline{\textbf{Correct Solution:}}

        \bigskip

        \color{red}
        Since $n - i = 1,000,000 - 500,000 = 500,000$, we can write
        $\mathcal{O}(n-i) \approx \mathcal{O}(i)$

        \bigskip

        Using this fact, we can conclude the speed of linked lists and array-based
        lists are roughly about the same.
        \color{black}
    \end{mdframed}

    \bigskip

    \underline{\textbf{Notes:}}

    \bigskip

    \begin{itemize}
        \item Noticed that professor compared the performance of linked lists
        and array-based list in terms of Big-Oh.
    \end{itemize}

\end{itemize}

\section*{Question 4}

\section*{Question 5}

\end{document}