\documentclass[12pt]{article}
\usepackage[margin=2.5cm]{geometry}
\usepackage{titling}
\usepackage{enumerate}
\usepackage{graphicx}
\usepackage{mdframed}
\usepackage{amsmath}
\usepackage{listings}
\usepackage{xcolor}
\usepackage{hyperref}
\usepackage[utf]{kotex}

\definecolor{codegreen}{rgb}{0,0.6,0}
\definecolor{codegray}{rgb}{0.5,0.5,0.5}
\definecolor{codepurple}{rgb}{0.58,0,0.82}
\definecolor{backcolour}{rgb}{0.95,0.95,0.92}

\lstdefinestyle{mystyle}{
    backgroundcolor=\color{backcolour},
    commentstyle=\color{codegreen},
    keywordstyle=\color{magenta},
    numberstyle=\tiny\color{codegray},
    stringstyle=\color{codepurple},
    basicstyle=\ttfamily\footnotesize,
    breakatwhitespace=false,
    breaklines=true,
    captionpos=b,
    keepspaces=true,
    numbers=left,
    numbersep=5pt,
    showspaces=false,
    showstringspaces=false,
    showtabs=false,
    tabsize=1
}

\lstset{style=mystyle}

\predate{}
\postdate{}

\begin{document}
\title{Lab 5: Linked Lists Solution}
\date{}
\maketitle

\section*{4) Additional exercises}
\subsection*{Generalizing \textit{\_\_getitem\_\_}}
The implementation we’ve provided for \textit{\_\_getitem\_\_} has many shortcomings
compared to Python’s built-in lists.

\bigskip

\noindent Two features that it doesn’t currently support are negative indexes and slices
(e.g., \textit{my\_list[2:5]}).

\bigskip

\noindent Your first task here is to investigate the different ways in which Python
supports these operations for built-in Python lists; you can do this by experimenting
yourself in the Python console, or by doing some reading online.

\bigskip

\noindent Then, modify the linked list implementation of \textit{\_\_getitem\_\_}
so that it handles both negative indexes and slices.

\bigskip

\noindent Note that a slice in Python is actually a class: the expression
\textit{my\_list[2:5]} is equivalent to \textit{my\_list.\_\_getitem\_\_(slice(2, 5))}.

\bigskip

\noindent Use \textit{isinstance} to determine whether the input to \textit{\_\_getitem\_\_}
is an integer or a slice.

\bigskip

\noindent The fully general method signature of \textit{\_\_getitem\_\_} should
become:

\bigskip

\begin{lstlisting}[language=python]
    def __getitem__(self, index: Union[int, slice]) -> Union[Any, LinkedList]
\end{lstlisting}

\bigskip

\noindent Note: slicing should always return a new \textit{LinkedList} object.

\bigskip

\noindent This means that for a given slice, you’ll need to create a \textit{LinkedList} and new
\textit{\_Nodes} as well, in a similar manner to how you implemented the more
powerful initializer at the end of Task 1.


\subsection*{Matplotlib Practice}
Use \textit{matplotlib} to plot the results of your timing experiments, using the same
approach as last week (See matplotlib section in lab 4).

\end{document}
