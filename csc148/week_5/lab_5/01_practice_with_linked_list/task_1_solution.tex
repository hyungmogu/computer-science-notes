\documentclass[12pt]{article}
\usepackage[margin=2.5cm]{geometry}
\usepackage{titling}
\usepackage{enumerate}
\usepackage{graphicx}
\usepackage{mdframed}
\usepackage{listings}
\usepackage{xcolor}
\usepackage{hyperref}
\usepackage[utf]{kotex}

\definecolor{codegreen}{rgb}{0,0.6,0}
\definecolor{codegray}{rgb}{0.5,0.5,0.5}
\definecolor{codepurple}{rgb}{0.58,0,0.82}
\definecolor{backcolour}{rgb}{0.95,0.95,0.92}

\lstdefinestyle{mystyle}{
    backgroundcolor=\color{backcolour},
    commentstyle=\color{codegreen},
    keywordstyle=\color{magenta},
    numberstyle=\tiny\color{codegray},
    stringstyle=\color{codepurple},
    basicstyle=\ttfamily\footnotesize,
    breakatwhitespace=false,
    breaklines=true,
    captionpos=b,
    keepspaces=true,
    numbers=left,
    numbersep=5pt,
    showspaces=false,
    showstringspaces=false,
    showtabs=false,
    tabsize=1
}

\lstset{style=mystyle}

\predate{}
\postdate{}

\begin{document}
\title{Lab 5: Linked Lists Solution}
\date{}
\maketitle

\section*{1) Practice with linked lists}

For this task: we have commented out the doctests in the methods. You will not
be able to run them until you finish step (3) of this task, at which point you
may uncomment them. We recommend you read all of the steps in this task before
you begin.

\bigskip

\begin{enumerate}[1.]
    \item In the starter code, find and read the docstring of the method \textit{\_\_len\_\_},
    and then implement it.

    \bigskip

    You already implemented this method in this week’s prep, but it’s good practice
    to implement it again. (And if you missed this week’s prep, do it now!)

    \item Then, implement the methods \textit{count}, \textit{index}, and \textit{\_\_setitem\_\_}.

    \item You might have noticed that all the doctests were commented out in the previous part.

    This is because they use a more powerful initializer than the one we’ve started with.

    \bigskip

    Your final task in this section is to implement a new initializer with the following interface:

    \bigskip

    \begin{lstlisting}[language=python]
        def __init__(self, items: list) -> None:
            """Initialize a new linked list containing the given items.

            The first node in the linked list contains the first item
            in <items>.
            """
    \end{lstlisting}

    \bigskip

    The lecture notes suggest one way to do this using \textit{append}; however,
    here we want you to try doing this without using \textit{append} (or any other helper
    method).

    \bigskip

    There are many different ways you could implement this method, but the key idea
    is that you need to loop through \textit{items}, create a new \textit{\_Node} for each
    item, link the nodes together, and initialize \textit{self.\_first}.

    \bigskip

    Spend time drawing some pictures before writing any code!

\end{enumerate}

\end{document}
