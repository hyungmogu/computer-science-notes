\documentclass[12pt]{article}
\usepackage[margin=2.5cm]{geometry}
\usepackage{enumerate}
\usepackage{amsfonts}
\usepackage{amsmath}
\usepackage{fancyhdr}
\usepackage{amsmath}
\usepackage{amssymb}
\usepackage{amsthm}
\usepackage{mdframed}
\usepackage{graphicx}
\usepackage{subcaption}
\usepackage{adjustbox}
\usepackage{listings}
\usepackage{xcolor}
\usepackage{booktabs}
\usepackage[utf]{kotex}

\definecolor{codegreen}{rgb}{0,0.6,0}
\definecolor{codegray}{rgb}{0.5,0.5,0.5}
\definecolor{codepurple}{rgb}{0.58,0,0.82}
\definecolor{backcolour}{rgb}{0.95,0.95,0.92}

\lstdefinestyle{mystyle}{
    backgroundcolor=\color{backcolour},
    commentstyle=\color{codegreen},
    keywordstyle=\color{magenta},
    numberstyle=\tiny\color{codegray},
    stringstyle=\color{codepurple},
    basicstyle=\ttfamily\footnotesize,
    breakatwhitespace=false,
    breaklines=true,
    captionpos=b,
    keepspaces=true,
    numbers=left,
    numbersep=5pt,
    showspaces=false,
    showstringspaces=false,
    showtabs=false,
    tabsize=1
}

\lstset{style=mystyle}

\begin{document}
\title{CSC148 Worksheet 13 Solution}
\author{Hyungmo Gu}
\maketitle

\section*{Question 1}
\begin{enumerate}[a.]
    \item

    The following diagram tells us the stopping condition occurs when both
    \textit{curr1} and \textit{curr2} is \textit{None}.

    \begin{center}
    \includegraphics[width=\linewidth]{images/worksheet_13_q1a_solution.png}
    \end{center}

    \bigskip

    Using this fact, the python expression involving \textit{curr1}
    and \textit{curr2} that expresses the stopping condition is

    \bigskip

    \begin{lstlisting}[language=Python]
    (curr1 is not None) and (curr2 is not None)
    \end{lstlisting}

    \item

    Python expression for the while loop condition is

    \begin{lstlisting}[language=Python]
    while (curr1 is not None) and (curr2 is not None):
        ...
    \end{lstlisting}

    \item

    The code for traversing two list is

    \begin{lstlisting}[language=Python]
    while (curr1 is not None) and (curr2 is not None):
        if curr1 is None or curr2 is None:
            return False

        if curr1.item != curr2.item:
            return False

        curr1 = curr1.next
        curr2 = curr2.next
    \end{lstlisting}

    \item

    After the loop ends, we know all items in curr1 and curr2 are identical.

    \item

    Because we know on successful loop termination, all items in curr1 and curr2
    are the same, we can use this information to conclude the two linked lists
    have the same length.

    \item

    The code that should go after the end of while loop is

    \begin{lstlisting}[language=Python]
    return True
    \end{lstlisting}

\end{enumerate}

\section*{Question 2}
\begin{enumerate}[a.]
    \item

    Initially, \textit{curr} and \textit{i} are as follows

    \bigskip

    \begin{lstlisting}[language=Python]
    curr = self._first
    i = 0
    \end{lstlisting}

\end{enumerate}



\end{document}