\documentclass[12pt]{article}
\usepackage[margin=2.5cm]{geometry}
\usepackage{titling}
\usepackage{enumerate}
\usepackage{graphicx}
\usepackage{mdframed}
\usepackage{listings}
\usepackage{xcolor}

\definecolor{codegreen}{rgb}{0,0.6,0}
\definecolor{codegray}{rgb}{0.5,0.5,0.5}
\definecolor{codepurple}{rgb}{0.58,0,0.82}
\definecolor{backcolour}{rgb}{0.95,0.95,0.92}

\lstdefinestyle{mystyle}{
    backgroundcolor=\color{backcolour},
    commentstyle=\color{codegreen},
    keywordstyle=\color{magenta},
    numberstyle=\tiny\color{codegray},
    stringstyle=\color{codepurple},
    basicstyle=\ttfamily\footnotesize,
    breakatwhitespace=false,
    breaklines=true,
    captionpos=b,
    keepspaces=true,
    numbers=left,
    numbersep=5pt,
    showspaces=false,
    showstringspaces=false,
    showtabs=false,
    tabsize=1
}

\lstset{style=mystyle}

\predate{}
\postdate{}

\begin{document}
\title{Lab 2: Introduction to Object-Oriented Programming Solution}
\date{}
\maketitle

\section*{2) Designing Classes}
\begin{enumerate}[1.]
\item \textit{Read the problem description.}
\item \textit{Decide what classes you need to design.}

\begin{lstlisting}[language=Python]
    class Race:
        pass

    class Runner:
        pass
\end{lstlisting}

\item \textit{Sample usage.}

\begin{lstlisting}[language=Python]
    class Race:
        """
        === Sample Usage ===

        Create a race registry:
        >>> r = Race()
        >>> r.categories['lt20']
        []
        >>> r.categories['lt30']
        []
        >>> r.categories['lt40']
        []
        >>> r.categories['gt40']
        []

        Registering runners:
        >>> runner_1 = Runner('Gerhard','gerhard@gmail.com')
        >>> r.register(runner_1, 'lt40')
        >>> r.categories['lt40'][0].name
        Gerhard
        >>> runner_2 = Runner('Tom','tom@gmail.com')
        >>> r.register(runner_2, 'lt30')
        >>> r.categories['lt30'][0].name
        Tom
        >>> runner_3 = Runner('Toni','toni@gmail.com')
        >>> r.register(runner_3, 'lt20')
        >>> r.categories['lt20'][0].name
        Toni
        >>> r.register(runner_1, 'lt30')
        >>> r.categories['lt30'][1].name
        Gerhard
        """
        pass


    ...


    if __name__ == '__main__':
        import doctest
        doctest.testmod()

\end{lstlisting}

\begin{mdframed}
    \underline{\textbf{Correct Solution:}}

    \bigskip

    \begin{lstlisting}[language=Python]
    class Race:
        """
        === Sample Usage ===

        Create a race registry:
        >>> r = Race()
        >>> r.runners
        []

        Registering runners:
        >>> runner_1 = Runner('Gerhard','gerhard@gmail.com', 'lt40')
        >>> r.register(runner_1)
        >>> r.runners[0].name
        'Gerhard'
        >>> runner_2 = Runner('Tom','tom@gmail.com', 'lt30')
        >>> r.register(runner_2)
        >>> r.runners[1].name
        'Tom'
        >>> runner_3 = Runner('Toni','toni@gmail.com', 'lt20')
        >>> r.register(runner_3)
        >>> r.runners[2].name
        'Toni'

        Updating runner in a speed category:
        >>> runner_4 = r.get_runner('Gerhard')
        >>> runner_4.edit_category('lt30')
        >>> runner_4.speed_category
        'lt30'

        Get all runners in a speed category:
        >>> r.get_runners('lt30')
        ['Gerhard','Tom']
        """


    class Runner:
        """
        === Sample Usage ===
        Create a runner:
        >>> runner = Runner('Gerhard', 'gerhard@gmail.com','lt30')
        >>> runner.name
        'Gerhard'
        >>> runner.email
        'gerhard@gmail.com'
        >>> runner.speed_category
        'lt30'
        """

    if __name__ == '__main__':
        import doctest
        doctest.testmod()

    \end{lstlisting}
\end{mdframed}

\item \textit{Designing the interface.}

\begin{lstlisting}[language=Python]
    class Race:
        """Race Registry

        === Attributes ===
        runners: a list of runners in race

        === Sample Usage ===

        Create a race registry:
        >>> r = Race()
        >>> r.runners
        []

        Registering runners:
        >>> runner_1 = Runner('Gerhard','gerhard@gmail.com', 'lt40')
        >>> r.register(runner_1)
        >>> r.runners[0].name
        'Gerhard'
        >>> runner_2 = Runner('Tom','tom@gmail.com', 'lt30')
        >>> r.register(runner_2)
        >>> r.runners[1].name
        'Tom'
        >>> runner_3 = Runner('Toni','toni@gmail.com', 'lt20')
        >>> r.register(runner_3)
        >>> r.runners[2].name
        'Toni'

        Updating runner in a speed category:
        >>> runner_4 = r.get_runner('Gerhard')
        >>> runner_4.edit_category('lt30')
        >>> runner_4.speed_category
        'lt30'

        Get all runners in a speed category:
        >>> runners = r.get_runners('lt30')
        >>> runners[0].name
        'Gerhard'
        >>> runners[1].name
        'Tom'
        """

        def __init__(self) -> None:
            """Initializes race registry

            >>> r = Race()
            >>> r.runners()
            []
            """
            pass

        def register(self, runner: Runner) -> None:
            """Registers runner to race

            >>> r = Race()
            >>> runner = Runner('Gerhard','gerhard@gmail.com','lt30')
            >>> r.register(runner)
            >>> r.runners[0].name
            'Gerhard'
            """
            pass

        def get_runners(self, category: str) -> None:
            """Returns list of runners in race category

            Precondition: <speed_category> in ['lt20','lt30','lt40','gt40','']

            >>> runner_1 = Runner('Gerhard','gerhard@gmail.com', 'lt40')
            >>> r.register(runner_1)
            >>> runner_2 = Runner('Tom','tom@gmail.com', 'lt20')
            >>> r.register(runner_2)
            >>> runner_3 = Runner('Toni','toni@gmail.com', 'lt20')
            >>> r.register(runner_3)
            >>> runners = r.get_runners('lt20')
            >>> runners[0].name
            'Tom'
            >>> runners[1].name
            'Toni'
            >>> r.get_runners('lt30')
            []
            """
            pass

        def get_runner(self, name: str) -> None:
            """Returns runner in race registry

            Precondition: <name> != ''

            >>> runner_1 = Runner('Gerhard','gerhard@gmail.com', 'lt40')
            >>> r.register(runner_1)
            >>> fetched_runner_1 = r.get_runners('Gerhard')
            >>> fetched_runner_1.name
            'Gerhard'
            >>> fetched_runner_2 = r.get_runners('Toni')
            >>> fetched_runner_2
            None
            """
            pass


    class Runner:
        """A runner for the race

        === Attributes ===
        name: the name of runner.
        email: the email of runner.
        speed_category: speed category runner is racing in

        === Sample Usage ===
        Create a runner:
        >>> runner = Runner('Gerhard', 'gerhard@gmail.com','lt30')
        >>> runner.name
        'Gerhard'
        >>> runner.email
        'gerhard@gmail.com'
        >>> runner.speed_category
        'lt30'
        """

        def __init__(self, name: str, email: str, speed_category: str) -> None:
            """Initialize runner

            Precondition: <name> != ''
            Precondition: <email> != ''
            Precondition: <speed_category> in ['lt20','lt30','lt40','gt40','']

            >>> runner = Runner('Gerhard', 'gerhard@gmail.com', 'lt30')
            >>> runner.name
            'Gerhard'
            >>> runner.email
            'gerhard@gmail.com'
            >>> runner.speed_category
            'lt30'
            """
            pass

        def edit_email(self, email: str) -> None:
            """Edits runner email information

            Precondition: <email> != ''

            >>> runner = Runner('Gerhard', 'gerhard@gmail.com', 'lt30')
            >>> runner.email
            'gerhard@gmail.com'
            >>> runner.edit_email('gerhard_2@gmail.com')
            >>> runner.email
            'gerhard_2@gmail.com'
            """
            pass

        def edit_category(self, speed_category: str) -> None:
            """Edits runner speed category information

            Precondition: <speed_category> in ['lt20','lt30','lt40','gt40']

            >>> runner = Runner('Gerhard', 'gerhard@gmail.com', 'lt30')
            >>> runner.speed_category
            'lt30'
            >>> runner.edit_category('lt20')
            >>> runner.speed_category
            'lt20'
            """

        def withdraw(self) -> None:
            """Withdraws runner from race

            >>> runner = Runner('Gerhard', 'gerhard@gmail.com', 'lt30')
            >>> runner.speed_category
            'lt30'
            >>> runner.withdraw()
            >>> runner.speed_category
            ''
            """
            pass

    if __name__ == '__main__':
        import doctest
        doctest.testmod()

\end{lstlisting}

\begin{mdframed}
    \underline{\textbf{Correct Solution:}}

    \bigskip

    \begin{lstlisting}[language=Python]
        class Runner:
        """A runner for the race

        === Attributes ===
        name: the name of runner.
        email: the email of runner.
        speed_category: speed category runner is racing in

        === Sample Usage ===
        Create a runner:
        >>> runner = Runner('Gerhard', 'gerhard@gmail.com','lt30')
        >>> runner.name
        'Gerhard'
        >>> runner.email
        'gerhard@gmail.com'
        >>> runner.speed_category
        'lt30'
        """
        name: str # <- Correct Solution
        email: str # <- Correct Solution
        speed_category: str # <- Correct Solution

        def __init__(self, name: str, email: str, speed_category: str) -> None:
            """Initialize runner

            Precondition: <name> != ''
            Precondition: <email> != ''
            Precondition: <speed_category> in ['lt20','lt30','lt40','gt40','']

            >>> runner = Runner('Gerhard', 'gerhard@gmail.com', 'lt30')
            >>> runner.name
            'Gerhard'
            >>> runner.email
            'gerhard@gmail.com'
            >>> runner.speed_category
            'lt30'
            """
            pass

        def edit_email(self, email: str) -> None:
            """Edits runner email information

            Precondition: <email> != ''

            >>> runner = Runner('Gerhard', 'gerhard@gmail.com', 'lt30')
            >>> runner.email
            'gerhard@gmail.com'
            >>> runner.edit_email('gerhard_2@gmail.com')
            >>> runner.email
            'gerhard_2@gmail.com'
            """
            pass

        def edit_category(self, speed_category: str) -> None:
            """Edits runner speed category information

            Precondition: <speed_category> in ['lt20','lt30','lt40','gt40']

            >>> runner = Runner('Gerhard', 'gerhard@gmail.com', 'lt30')
            >>> runner.speed_category
            'lt30'
            >>> runner.edit_category('lt20')
            >>> runner.speed_category
            'lt20'
            """

        def withdraw(self) -> None:
            """Withdraws runner from race

            >>> runner = Runner('Gerhard', 'gerhard@gmail.com', 'lt30')
            >>> runner.speed_category
            'lt30'
            >>> runner.withdraw()
            >>> runner.speed_category
            ''
            """
            pass


    class Race:
        """Race Registry

        === Attributes ===
        runners: a list of runners in race

        === Sample Usage ===

        Create a race registry:
        >>> r = Race()
        >>> r.runners
        []

        Registering runners:
        >>> runner_1 = Runner('Gerhard','gerhard@gmail.com', 'lt40')
        >>> r.register(runner_1)
        >>> r.runners[0].name
        'Gerhard'
        >>> runner_2 = Runner('Tom','tom@gmail.com', 'lt30')
        >>> r.register(runner_2)
        >>> r.runners[1].name
        'Tom'
        >>> runner_3 = Runner('Toni','toni@gmail.com', 'lt20')
        >>> r.register(runner_3)
        >>> r.runners[2].name
        'Toni'

        Updating runner in a speed category:
        >>> runner_4 = r.get_runner('Gerhard')
        >>> runner_4.edit_category('lt30')
        >>> runner_4.speed_category
        'lt30'

        Get all runners in a speed category:
        >>> runners = r.get_runners('lt30')
        >>> runners[0].name
        'Gerhard'
        >>> runners[1].name
        'Tom'
        """
        runners: List[Runner] # <- Correct Solution

        def __init__(self) -> None:
            """Initializes race registry

            >>> r = Race()
            >>> r.runners()
            []
            """
            pass

        def register(self, runner: Runner) -> None:
            """Registers runner to race

            >>> r = Race()
            >>> runner = Runner('Gerhard','gerhard@gmail.com','lt30')
            >>> r.register(runner)
            >>> r.runners[0].name
            'Gerhard'
            """
            pass

        def get_runners(self, category: str) -> None:
            """Returns list of runners in race category

            Precondition: <speed_category> in ['lt20','lt30','lt40','gt40'] # <- correct solution

            >>> r = Race() # <- Correct Solution
            >>> runner_1 = Runner('Gerhard','gerhard@gmail.com', 'lt40')
            >>> r.register(runner_1)
            >>> runner_2 = Runner('Tom','tom@gmail.com', 'lt20')
            >>> r.register(runner_2)
            >>> runner_3 = Runner('Toni','toni@gmail.com', 'lt20')
            >>> r.register(runner_3)
            >>> runners = r.get_runners('lt20')
            >>> runners[0].name
            'Tom'
            >>> runners[1].name
            'Toni'
            >>> r.get_runners('lt30')
            []
            """
            pass

        def get_runner(self, name: str) -> None:
            """Returns runner in race registry

            Precondition: <name> != ''

            >>> r = Race() # <- Correct Solution
            >>> runner_1 = Runner('Gerhard','gerhard@gmail.com', 'lt40')
            >>> r.register(runner_1)
            >>> fetched_runner_1 = r.get_runners('Gerhard')
            >>> fetched_runner_1.name
            'Gerhard'
            >>> fetched_runner_2 = r.get_runners('Toni')
            >>> fetched_runner_2
            """
            pass

    if __name__ == '__main__':
        import doctest
        doctest.testmod()

    \end{lstlisting}
\end{mdframed}

\end{enumerate}

\end{document}
