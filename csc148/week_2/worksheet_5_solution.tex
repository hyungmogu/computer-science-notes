\documentclass[12pt]{article}
\usepackage[margin=2.5cm]{geometry}
\usepackage{enumerate}
\usepackage{amsfonts}
\usepackage{amsmath}
\usepackage{fancyhdr}
\usepackage{amsmath}
\usepackage{amssymb}
\usepackage{amsthm}
\usepackage{mdframed}
\usepackage{graphicx}
\usepackage{subcaption}
\usepackage{listings}
\usepackage{xcolor}
\usepackage[utf]{kotex}

\definecolor{codegreen}{rgb}{0,0.6,0}
\definecolor{codegray}{rgb}{0.5,0.5,0.5}
\definecolor{codepurple}{rgb}{0.58,0,0.82}
\definecolor{backcolour}{rgb}{0.95,0.95,0.92}

\lstdefinestyle{mystyle}{
    backgroundcolor=\color{backcolour},
    commentstyle=\color{codegreen},
    keywordstyle=\color{magenta},
    numberstyle=\tiny\color{codegray},
    stringstyle=\color{codepurple},
    basicstyle=\ttfamily\footnotesize,
    breakatwhitespace=false,
    breaklines=true,
    captionpos=b,
    keepspaces=true,
    numbers=left,
    numbersep=5pt,
    showspaces=false,
    showstringspaces=false,
    showtabs=false,
    tabsize=1
}

\lstset{style=mystyle}

\begin{document}
\title{CSC148 Worksheet 5 Solution}
\author{Hyungmo Gu}
\maketitle

\section*{Question 1}

\begin{lstlisting}[language=Python]
    """
    from datetime import date, timedelta

    ...

    misbehaved_1 = Tweet('', date.today(), 'test')
    misbehaved.like(-1)

    yesterday = date.today() - timedelta(days=1)
    misbehaved_2 = Tweet('john', yesterday 'test')

    misbehaved_3 = Tweet('john', date.today(), '')
    """
\end{lstlisting}

\bigskip

\begin{mdframed}
    \underline{\textbf{Correct Solution:}}

    \bigskip

    \begin{lstlisting}[language=Python]
        """
        from datetime import date, timedelta

        ...

        misbehaved_1 = Tweet('', date.today(), 'test')
        misbehaved.like(-1)

        yesterday = date.today() - timedelta(days=1)
        misbehaved_2 = Tweet('john', yesterday 'test')

        misbehaved_3 = Tweet('john', date.today(), '')

        misbehaved_4 = Tweet('john doe', date.today(), 'test') # More solution!!
        """
    \end{lstlisting}

\end{mdframed}

\section*{Question 2}
\begin{itemize}
    \item A tweet must receive positive number of likes
    \item A tweet must have user id without spaces
\end{itemize}

\begin{mdframed}
    \underline{\textbf{Correct Solution:}}

    \bigskip

    \begin{itemize}
        \item A tweet must receive \color{red}non-negative\color{black}\:number of likes
        \item \color{red}A tweet user id must not be empty.\color{black}
        \item A tweet must have user id without spaces
        \item \color{red} A tweet must have content length less than or equal to 280 characters\color{black}
    \end{itemize}

\end{mdframed}

\section*{Question 3}
\begin{itemize}
    \item Fix for the property `a tweet must receive positive number of likes'

    \begin{lstlisting}[language=Python]
    class Tweet:
        ...

        def like(self, n: int) -> None:
            """Record the fact that this tweet received <n> likes.
            These likes are in addition to the ones <self> already has.
            """

            if n < 0:
                raise ValueError('n must be non-negative value')

            self.likes += n
    \end{lstlisting}

    \bigskip

    \begin{mdframed}
        \underline{\textbf{Correct Solution:}}

        \bigskip

        \begin{lstlisting}[language=Python]
            class Tweet:
            ...

            def like(self, n: int) -> None:
                """Record the fact that this tweet received <n> likes.
                These likes are in addition to the ones <self> already has.

                Precondition: n >= 0 # Correct Solution
                """

                self.likes += n
        \end{lstlisting}
    \end{mdframed}

    \item Fix for the property `a tweet user id must not be empty.'

    \begin{lstlisting}[language=Python]
    class Tweet:
        ...

        def __init__(self, who: str, when: date, what: str) -> None:
            """Initialize a new Tweet.
            """

            if who.strip() == '':
                raise ValueError("variable 'who' must not be empty")

            self.userid = who
            self.content = what
            self.created_at = when
            self.likes = 0
    \end{lstlisting}

    \bigskip

    \begin{mdframed}
        \underline{\textbf{Correct Solution:}}

        \bigskip

        \begin{lstlisting}[language=Python]
        class Tweet:
            ...

            def __init__(self, who: str, when: date, what: str) -> None:
                """Initialize a new Tweet.

                Precondition: who.strip() != '' # Correct Solution
                """

                self.userid = who
                self.content = what
                self.created_at = when
                self.likes = 0
        \end{lstlisting}
    \end{mdframed}

    \item Fix for the property `a tweet must have user id without spaces'

    \begin{lstlisting}[language=Python]
    class Tweet:
        ...

        def __init__(self, who: str, when: date, what: str) -> None:
            """Initialize a new Tweet.
            """

            if ' ' in who:
                raise ValueError("variable 'who' must not have spaces")

            self.userid = who
            self.content = what
            self.created_at = when
            self.likes = 0
    \end{lstlisting}

    \bigskip

    \begin{mdframed}
        \underline{\textbf{Correct Solution:}}

        \bigskip

        \begin{lstlisting}[language=Python]
        class Tweet:
            ...

            def __init__(self, who: str, when: date, what: str) -> None:
                """Initialize a new Tweet.

                Precondition: ' ' not in who # Correct Solution
                """

                self.userid = who
                self.content = what
                self.created_at = when
                self.likes = 0
        \end{lstlisting}
    \end{mdframed}


    \item Fix for the property 'a tweet must have content length less than or
    equal to 280 characters'

    \begin{lstlisting}[language=Python]
    class Tweet:
        ...

        def __init__(self, who: str, when: date, what: str) -> None:
            """Initialize a new Tweet.
            """

            ...

            if len(what) > 280:
                raise ValueError("variable 'what' must be 280 characters or less")

            self.userid = who
            self.content = what
            self.created_at = when
            self.likes = 0
    \end{lstlisting}

    \bigskip

    \begin{mdframed}
        \underline{\textbf{Correct Solution:}}

        \bigskip

        \begin{lstlisting}[language=Python]
        class Tweet:
            ...

            def __init__(self, who: str, when: date, what: str) -> None:
                """Initialize a new Tweet.

                Precondition: len(what) <= 280 # Correct Solution
                """

                self.userid = who
                self.content = what
                self.created_at = when
                self.likes = 0
        \end{lstlisting}
    \end{mdframed}

\end{itemize}

\bigskip

\underline{\textbf{Notes:}}

\begin{itemize}
    \item Learned that `Precondition: len(what) $<=$ 280' is called \textbf{representational invariant}.
    \item Learned that \textbf{representational invariant} is a property of instance attribute that
    every class must satisfy.
\end{itemize}

\bigskip

\section*{Question 4}

\end{document}