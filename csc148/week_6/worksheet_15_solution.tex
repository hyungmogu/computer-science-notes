\documentclass[12pt]{article}
\usepackage[margin=2.5cm]{geometry}
\usepackage{enumerate}
\usepackage{amsfonts}
\usepackage{amsmath}
\usepackage{fancyhdr}
\usepackage{amsmath}
\usepackage{amssymb}
\usepackage{amsthm}
\usepackage{mdframed}
\usepackage{graphicx}
\usepackage{subcaption}
\usepackage{adjustbox}
\usepackage{listings}
\usepackage{xcolor}
\usepackage{booktabs}
\usepackage[utf]{kotex}

\definecolor{codegreen}{rgb}{0,0.6,0}
\definecolor{codegray}{rgb}{0.5,0.5,0.5}
\definecolor{codepurple}{rgb}{0.58,0,0.82}
\definecolor{backcolour}{rgb}{0.95,0.95,0.92}

\lstdefinestyle{mystyle}{
    backgroundcolor=\color{backcolour},
    commentstyle=\color{codegreen},
    keywordstyle=\color{magenta},
    numberstyle=\tiny\color{codegray},
    stringstyle=\color{codepurple},
    basicstyle=\ttfamily\footnotesize,
    breakatwhitespace=false,
    breaklines=true,
    captionpos=b,
    keepspaces=true,
    numbers=left,
    numbersep=5pt,
    showspaces=false,
    showstringspaces=false,
    showtabs=false,
    tabsize=1
}

\lstset{style=mystyle}

\begin{document}
\title{CSC148 Worksheet 15 Solution}
\author{Hyungmo Gu}
\maketitle

\section*{Question 1}
\begin{enumerate}[a.]
    \item

    According to docstring, \textit{flatten([[0, -1], -2, [[-3, [-5], -7]]]))}
    should return

    \begin{align*}
        [0,-1,-2,-3,-5,-7]
    \end{align*}

    \item
    \adjustbox{center, valign=t}{
        \begin{tabular}{|c|c|c|}
            \hline
            \textbf{sublist} & \textbf{flatten(sublist)} & \textbf{Value of s at the end of the iteration}\\
            \hline
            N/A & N/A & [] (initial value of a)\\
            \hline
            [0,-1] & [0,-1] & \\
            \hline
            2 & 2 & \\
            \hline
            [[-3, [-5], -7]] & [-3,-5,-7] & \\
            \hline
        \end{tabular}
    }

    \bigskip

    \begin{mdframed}

        \underline{\textbf{Correct Solution:}}

        \bigskip

        \adjustbox{center, valign=t}{
            \begin{tabular}{|c|c|c|}
                \hline
                \textbf{sublist} & \textbf{flatten(sublist)} & \textbf{Value of s at the end of the iteration}\\
                \hline
                N/A & N/A & [] (initial value of a)\\
                \hline
                [0,-1] & [0,-1] & \\
                \hline
                2 & \color{red}[2]\color{black} & \\
                \hline
                [[-3, [-5], -7]] & [-3,-5,-7] & \\
                \hline
            \end{tabular}
        }
    \end{mdframed}

    \item
    \adjustbox{center, valign=t}{
        \begin{tabular}{|c|c|c|}
            \hline
            \textbf{sublist} & \textbf{flatten(sublist)} & \textbf{Value of s at the end of the iteration}\\
            \hline
            N/A & N/A & [] (initial value of a)\\
            \hline
            [0,-1] & [0,-1] & [0,-1]\\
            \hline
            2 & [2] & [0,-1,2]\\
            \hline
            [[-3, [-5], -7]] & [-3,-5,-7] & [0,-1,2,-3,-5,-7]\\
            \hline
        \end{tabular}
    }

    \item Yes, the final value of $s$ in previous problem matches the solution of
    [0,-1,2,-3,-5,-7] in problem 1.a.

    \item

    \begin{lstlisting}[language=Python]
    def flatten(obj: Union[int, List]) -> List[int]:
        """Return a (non-nested) list of the integers in <obj>.
        The integers are returned in the left-to-right order they appear
        in <obj>.

        >>> flatten(6)
        [6]
        >>> flatten([1, [-2, 3], -4])
        [1, -2, 3, -4]
        >>> flatten([[0, -1], -2, [[-3, [-5]]]])
        [0, -1, -2, -3, -5]
        """
        if isinstance(obj, int):
            return [obj]
        else:
            s = []
            for sublist in obj:
                s.extend(flatten(sublist))
            return s
    \end{lstlisting}

\end{enumerate}

\section*{Question 2}
\begin{enumerate}[a.]
    \item

    \textbf{Input that does NOT reveal an error:} $[1,[2,3,[4]],5]$

    \bigskip

    \textbf{Expected output:} $[1,2,3,4]$

    \bigskip

    \adjustbox{center, valign=t}{
        \begin{tabular}{|c|c|c|}
            \hline
            \textbf{sublist} & \textbf{flatten(sublist)} & \textbf{Value of s at the end of the iteration}\\
            \hline
            N/A & N/A & [] (initial value of a)\\
            \hline
            1 & [1] & [1] \\
            \hline
            [2,3,[4]] & [2,3,4] & [1,2,3,4]\\
            \hline
            5 & [5] & [1,2,3,4,5]\\
            \hline
        \end{tabular}
    }

    \item

    \textbf{Input that does reveal an error:} $[1,2,[2,2,3],4]$

    \bigskip

    \textbf{Expected output:} $[1,2,3,4]$

    \bigskip

    \adjustbox{center, valign=t}{
        \begin{tabular}{|c|c|c|}
            \hline
            \textbf{sublist} & \textbf{flatten(sublist)} & \textbf{Value of s at the end of the iteration}\\
            \hline
            N/A & N/A & [] (initial value of a)\\
            \hline
            1 & [1] & [1] \\
            \hline
            2 & [2] & [1,2]\\
            \hline
            [2,2,3] & [2,3] & [1,2,2,3]\\
            \hline
            4 & [4] & [1,2,2,3,4]\\
            \hline
        \end{tabular}
    }

    \bigskip

    \begin{mdframed}

    \underline{\textbf{Correct Solution:}}

    \bigskip

    \textbf{Input that does reveal an error:} $[1,2,[2,2,3],4]$

    \bigskip

    \textbf{Expected output:} $[1,2,3,4]$

    \bigskip

    \adjustbox{center, valign=t}{
        \begin{tabular}{|c|c|c|}
            \hline
            \textbf{sublist} & \textbf{flatten(sublist)} & \textbf{Value of s at the end of the iteration}\\
            \hline
            N/A & N/A & [] (initial value of a)\\
            \hline
            1 & [1] & [1] \\
            \hline
            2 & [2] & [1,2]\\
            \hline
            [2,2,3] & \color{red}[2,2,3]\color{black} & \color{red}[1,2,2,2,3]\color{black}\\
            \hline
            4 & [4] & \color{red}[1,2,2,2,3,4]\color{black}\\
            \hline
        \end{tabular}
    }

    \end{mdframed}
\end{enumerate}

\section*{Question 3}
\begin{itemize}
    \item
    The checking of the uniqueness of elements stops at sublist when it needs
    to be done a level higher.

    \begin{mdframed}

    \underline{\textbf{Correct Solution:}}

    \bigskip

    \color{red}The repeating elements are never checked.\color{black}

    \end{mdframed}
\end{itemize}

\end{document}