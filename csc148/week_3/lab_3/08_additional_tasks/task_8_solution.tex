\documentclass[12pt]{article}
\usepackage[margin=2.5cm]{geometry}
\usepackage{titling}
\usepackage{enumerate}
\usepackage{float,graphicx}
\usepackage{mdframed}
\usepackage{listings}
\usepackage{xcolor}

\definecolor{codegreen}{rgb}{0,0.6,0}
\definecolor{codegray}{rgb}{0.5,0.5,0.5}
\definecolor{codepurple}{rgb}{0.58,0,0.82}
\definecolor{backcolour}{rgb}{0.95,0.95,0.92}

\lstdefinestyle{mystyle}{
    backgroundcolor=\color{backcolour},
    commentstyle=\color{codegreen},
    keywordstyle=\color{magenta},
    numberstyle=\tiny\color{codegray},
    stringstyle=\color{codepurple},
    basicstyle=\ttfamily\footnotesize,
    breakatwhitespace=false,
    breaklines=true,
    captionpos=b,
    keepspaces=true,
    numbers=left,
    numbersep=5pt,
    showspaces=false,
    showstringspaces=false,
    showtabs=false,
    tabsize=1
}

\lstset{style=mystyle}

\predate{}
\postdate{}

\begin{document}
\title{Lab 3 Task 8 Solution}
\date{}
\maketitle

\section*{8) Additional Tasks}

\subsection*{8.1) A user player}

\begin{lstlisting}[language=Python]
    ...
    class UserPlayer(Player):

    def move(self, current: int, min_step: int,
             max_step: int, goal: int) -> int:

        amount = 0

        while True:
            amount_raw = input('Enter step amount ({}-{})'.format(min_step, max_step))

            if len(amount_raw.strip()) == 0:
                print('Please select integer value between {} and {}'.format(min_step, max_step))
                continue

            if re.search(r'[^0-9]+', amount_raw):
                print('Please select integer value between {} and {}'.format(min_step, max_step))
                continue

            amount = int(amount_raw)
            if amount < min_step or amount > max_step:
                print('Please select steps between {} and {}'.format(min_step, max_step))
                continue

            break

        return amount

    ...
    def make_player(generic_name: str) -> Player:
        ...
        return UserPlayer(name)

    ...

    if __name__ == '__main__':
        # Uncomment the lines below to check your work using
        # python_ta and doctest.
        # import python_ta
        # python_ta.check_all(config={
        #     'extra-imports': ['random'],
        #     'allowed-io': [
        #         'main',
        #         'make_player',
        #         'move',
        #         'play_one_turn'
        #     ]
        # })
        main()
\end{lstlisting}

\subsection*{8.2) A strategic player}

The solution to this problem makes following assumptions:

\begin{itemize}
    \item \textit{goal} of 21
    \item \textit{min\_step} of 1
    \item \textit{max\_step} of 3
    \item one of the player as \textit{StrategicPlayer}
    \item the other as \textit{RandomPlayer}
\end{itemize}

\bigskip

\noindent We need to create \textit{StrategicPlayer} that always wins as
player 1, and does win as player 2 when a bad move is by the other player.
Also, we need to adjust \textit{make\_player} so a player's type can be
chosen by user.

\bigskip

\begin{lstlisting}[language=Python]
    ...

    # =============== SOLUTION (Task 8.2) ================

    class StrategicPlayer(Player):

        def move(self, current: int, min_step: int,
                 max_step: int, goal: int) -> int:

            return 3
    # ====================================================

    ...

    def make_player(generic_name: str) -> Player:
        ...

        # =============== SOLUTION (Task 8.2) ================
        player_type_list = ['r', 'u', 's']

        while True:
            player_type = input(
                'Enter player type '
                '(r - Random Player, u - User Player, s - Strategic Player)')

            if player_type not in player_type_list:
                print('Please select one of the three values '
                      '({})'.format(','.join(player_type_list)))
                continue

            break

        if player_type == 'u':
            return UserPlayer(name)
        elif player_type == 's':
            return StrategicPlayer(name)
        elif player_type == 'r':
            return RandomPlayer(name)

        # ====================================================

    ...

    if __name__ == '__main__':
        # Uncomment the lines below to check your work using
        # python_ta and doctest.
        # import python_ta
        # python_ta.check_all(config={
        #     'extra-imports': ['random'],
        #     'allowed-io': [
        #         'main',
        #         'make_player',
        #         'move',
        #         'play_one_turn'
        #     ]
        # })
        main()
\end{lstlisting}

\subsection*{8.3) Tracking and reporting a player’s record}

\bigskip

We need to update \textit{Player} using \_\_str\_\_, attributes and other methods
so players' name and record are displayed at the end of each game. We also need to
update \textit{NumberGame} so winner's record are updated at the end of each game.

\bigskip

\begin{lstlisting}[language=Python]
    ...

    class Player:
        """A player in number game

        === Attributes ===
        name:
            The name of player
        # =============== SOLUTION (Task 8.3) ================
        wins:
            The number of wins
        # ====================================================

        === Representation invariants ===
        - len(name.strip()) != 0
        - 0 <= self.current <= self.goal
        - 0 < self.min_step <= self.max_step <= self.goal
        """
        name: str
        # =============== SOLUTION (Task 8.3) ================
        wins: int
        # ====================================================

        def __init__(self, name: str) -> None:
            """Initialize this Player

                Precondition:
                    - len(name.strip()) != 0
            """
            self.name = name
            # =============== SOLUTION (Task 8.3) ================
            self.wins = 0
            # ====================================================

        def __str__(self):
            return '{} - {} wins'.format(self.name, self.wins)

        # =============== SOLUTION (Task 8.3) ================
        def add_win(self):
            """Increments a win count to this player"""
            self.wins += 1
        # ====================================================

    ...

    class NumberGame:
       ...

        def play(self) -> str:
            """Play one round of this NumberGame. Return the name of the winner.

            A "round" is one full run of the game, from when the count starts
            at 0 until the goal is reached.
            """
            while self.current < self.goal:
                self.play_one_turn()
            # The player whose turn would be next (if the game weren't over) is
            # the loser. The one who went one turn before that is the winner.
            winner = self.whose_turn(self.turn - 1)
            # =============== SOLUTION (Task 8.3) ================
            winner.add_win()
            # ====================================================
            return winner.name

    ...

    if __name__ == '__main__':
        # Uncomment the lines below to check your work using
        # python_ta and doctest.
        # import python_ta
        # python_ta.check_all(config={
        #     'extra-imports': ['random'],
        #     'allowed-io': [
        #         'main',
        #         'make_player',
        #         'move',
        #         'play_one_turn'
        #     ]
        # })
        main()
\end{lstlisting}



\end{document}
