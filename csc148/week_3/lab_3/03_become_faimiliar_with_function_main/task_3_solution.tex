\documentclass[12pt]{article}
\usepackage[margin=2.5cm]{geometry}
\usepackage{titling}
\usepackage{enumerate}
\usepackage{graphicx}
\usepackage{mdframed}
\usepackage{listings}
\usepackage{xcolor}

\definecolor{codegreen}{rgb}{0,0.6,0}
\definecolor{codegray}{rgb}{0.5,0.5,0.5}
\definecolor{codepurple}{rgb}{0.58,0,0.82}
\definecolor{backcolour}{rgb}{0.95,0.95,0.92}

\lstdefinestyle{mystyle}{
    backgroundcolor=\color{backcolour},
    commentstyle=\color{codegreen},
    keywordstyle=\color{magenta},
    numberstyle=\tiny\color{codegray},
    stringstyle=\color{codepurple},
    basicstyle=\ttfamily\footnotesize,
    breakatwhitespace=false,
    breaklines=true,
    captionpos=b,
    keepspaces=true,
    numbers=left,
    numbersep=5pt,
    showspaces=false,
    showstringspaces=false,
    showtabs=false,
    tabsize=1
}

\lstset{style=mystyle}

\predate{}
\postdate{}

\begin{document}
\title{Lab 3 Task 3 Solution}
\date{}
\maketitle


\section*{3) Become familiar with function \textit{main}}
\begin{enumerate}[1.]
    \item Where is a \textit{NumberGame} constructed?

    \begin{itemize}
        \item
        By observation, we can conclude a \textit{NumberGame} constructed
        inside function \textit{main}

        \begin{lstlisting}[language=Python]
        def main() -> None:
            ...
            while True:
                g = NumberGame(goal, minimum, maximum, (p1, p2)) #<- Here!!
                winner = g.play()
                print(f'And {winner} is the winner!!!')
                print(p1)
                print(p2)
                again = input('Again? (y/n) ')
                if again != 'y':
                    return

        \end{lstlisting}
    \end{itemize}

    \item This function calls \textit{g.play} repeatedly in a loop. What about the
    game can change each time \textit{g.play} is called: the goal, the min or max
    move, the players, the moves?

    \begin{itemize}
        \item

        By observation, we can conclude that

        \begin{enumerate}[1.]
            \item the goal doesn't change
            \item the min or max move don't change
            \item the current player change as a result of whose\_turn method.

            \begin{lstlisting}[language=Python]
            def play(self) -> str:
                ...
                while self.current < self.goal:
                    self.play_one_turn() # <- In here
                ...
                winner = self.whose_turn(self.turn - 1)
                return winner.namePlayers

            def play_one_turn(self) -> None:
                ...
                next_player = self.whose_turn(self.turn) # <- Here!!
                amount = next_player.move(
                    self.current,
                    self.min_step,
                    self.max_step,
                    self.goal
                )
                self.current += amount
                self.turn += 1

                print(f'{next_player.name} moves {amount}.')
                print(f'Total is now {self.current}.')


            def whose_turn(self, turn: int) -> Player:
                ...
                if turn % 2 == 0:
                    return self.players[0]
                else:
                    return self.players[1]

            \end{lstlisting}

            \item the move changes by the \textit{move} method in \textit{play\_one\_turn}.

            \begin{lstlisting}[language=Python]
                def play(self) -> str:
                    ...
                    while self.current < self.goal:
                        self.play_one_turn()
                    ...
                    winner = self.whose_turn(self.turn - 1)
                    return winner.namePlayers

                def play_one_turn(self) -> None:
                    ...
                    next_player = self.whose_turn(self.turn)
                    amount = next_player.move( # <- Here!!
                        self.current,
                        self.min_step,
                        self.max_step,
                        self.goal
                    )
                    self.current += amount
                    self.turn += 1

                    print(f'{next_player.name} moves {amount}.')
                    print(f'Total is now {self.current}.')

                \end{lstlisting}
        \end{enumerate}
    \end{itemize}

    \item List all the places in this function where a \textit{Player} is stored,
    an instance attribute of \textit{Player} is accessed or set, or a method is
    called on a \textit{Player}.

    \bigskip

    \begin{itemize}
        \item

        We need to find all places in this function where \textit{Player} is stored,
        where an instance attribute of \textit{Player} is accessed or set, or where
        a method is called on a \textit{Player}.

        \bigskip

        First, we need to find where \textit{Player} is stored in this function.

        \bigskip

        The code tells us the type of third argument in \textit{NumberGame}
        is Tuple, and each element in the tuple is taken by the variables \textit{p1}
        and \textit{p2}.

        \bigskip

        \begin{lstlisting}[language=Python]
        def main() -> None:
            ...
            while True:
                g = NumberGame(goal, minimum, maximum, (p1, p2)) # <- Here!!
                ...
        \end{lstlisting}

        \bigskip

        Because we know each element in the tuple is of type \textit{Player}, we can
        conclude \textit{Player} is stored inside variables \textit{p1} and
        \textit{p2}

        \bigskip

        Second, we need to find where the instance instance attribute of
        \textit{Player} is accessed or set in this function.

        \bigskip

        By observation, no instance attribute of \textit{Player} is accessed
        or set.

        \bigskip

        Finally, we need to find where a method of \textit{Player} is called
        in this function.

        \bigskip

        By observation, no method of \textit{Player} is accessed or set inside this function.

    \end{itemize}

    \bigskip

    % \begin{mdframed}
    %     \underline{\textbf{Rough Work:}}

    %     \bigskip

    %     We need to find all places in this function where \textit{Player} is stored,
    %     where an instance attribute of \textit{Player} is accessed or set, or where
    %     a method is called on a \textit{Player}.

    %     \bigskip

    %     \begin{enumerate}[1.]
    %         \item Find where \textit{Player} is stored in this function.

    %         \bigskip

    %         \begin{mdframed}

    %         First, we need to find where \textit{Player} is stored in this function.

    %         \bigskip

    %         The code tells us the type of third argument in \textit{NumberGame}
    %         is Tuple, and each element in the tuple is taken by the variables \textit{p1}
    %         and \textit{p2}.

    %         \bigskip

    %         \begin{lstlisting}[language=Python]
    %         def main() -> None:
    %             ...
    %             while True:
    %                 g = NumberGame(goal, minimum, maximum, (p1, p2)) # <- Here!!
    %                 ...
    %         \end{lstlisting}

    %         \bigskip

    %         Because we know each element in the tuple is of type \textit{Player}, we can
    %         conclude \textit{Player} is stored inside variables \textit{p1} and
    %         \textit{p2}

    %         \end{mdframed}

    %         \item Find where the instance attribute of \textit{Player} is accessed
    %         or set in this function.

    %         \bigskip

    %         \begin{mdframed}

    %         Second, we need to find where the instance instance attribute of
    %         \textit{Player} is accessed or set in this function.

    %         \bigskip

    %         By observation, no instance attribute of \textit{Player} is accessed
    %         or set.

    %         \end{mdframed}

    %         \item Find where a method of \textit{Player} is called in this
    %         function.

    %         \bigskip

    %         \begin{mdframed}

    %         Finally, we need to find where a method of \textit{Player} is called
    %         in this function.

    %         \bigskip

    %         By observation, no method of \textit{Player} is
    %         accessed or set inside this function.

    %         \end{mdframed}

    %     \end{enumerate}

    %     \begin{mdframed}
    %     First, we need to find where \textit{Player} is stored in this function.

    %     \bigskip

    %     The code tells us the type of third argument in \textit{NumberGame}
    %     is Tuple, and each element in the tuple is taken by the variables \textit{p1}
    %     and \textit{p2}.

    %     \bigskip

    %     \begin{lstlisting}[language=Python]
    %     def main() -> None:
    %         ...
    %         while True:
    %             g = NumberGame(goal, minimum, maximum, (p1, p2)) # <- Here!!
    %             ...
    %     \end{lstlisting}

    %     \bigskip

    %     Because we know each element in the tuple is of type \textit{Player}, we can
    %     conclude \textit{Player} is stored inside variables \textit{p1} and
    %     \textit{p2}

    %     \bigskip

    %     Second, we need to find where the instance instance attribute of
    %     \textit{Player} is accessed or set in this function.

    %     \bigskip

    %     By observation, no instance attribute of \textit{Player} is accessed
    %     or set.

    %     \bigskip

    %     Finally, we need to find where a method of \textit{Player} is called
    %     in this function.

    %     \bigskip

    %     By observation, no method of \textit{Player} is accessed or set inside this function.

    %     \end{mdframed}

    % \end{mdframed}


\end{enumerate}

\end{document}
