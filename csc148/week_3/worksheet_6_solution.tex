\documentclass[12pt]{article}
\usepackage[margin=2.5cm]{geometry}
\usepackage{enumerate}
\usepackage{amsfonts}
\usepackage{amsmath}
\usepackage{fancyhdr}
\usepackage{amsmath}
\usepackage{amssymb}
\usepackage{amsthm}
\usepackage{mdframed}
\usepackage{graphicx}
\usepackage{subcaption}
\usepackage{listings}
\usepackage{xcolor}
\usepackage[utf]{kotex}

\definecolor{codegreen}{rgb}{0,0.6,0}
\definecolor{codegray}{rgb}{0.5,0.5,0.5}
\definecolor{codepurple}{rgb}{0.58,0,0.82}
\definecolor{backcolour}{rgb}{0.95,0.95,0.92}

\lstdefinestyle{mystyle}{
    backgroundcolor=\color{backcolour},
    commentstyle=\color{codegreen},
    keywordstyle=\color{magenta},
    numberstyle=\tiny\color{codegray},
    stringstyle=\color{codepurple},
    basicstyle=\ttfamily\footnotesize,
    breakatwhitespace=false,
    breaklines=true,
    captionpos=b,
    keepspaces=true,
    numbers=left,
    numbersep=5pt,
    showspaces=false,
    showstringspaces=false,
    showtabs=false,
    tabsize=1
}

\lstset{style=mystyle}

\begin{document}
\title{CSC148 Worksheet 6 Solution}
\author{Hyungmo Gu}
\maketitle

\section*{Question 1}
\begin{itemize}
    \item

    The two classes already defined are: Vehicle and SuperDuperManager

    \bigskip

    The additional classes required to create for this exercise are: Car,
    Helicopter and MagicCarpet

    \bigskip

    Car, Helicopter and MagicCarpet are child classes of Vehicle.

\end{itemize}

\section*{Question 2}
\begin{enumerate}[a.]
    \item The following are attributes possessed by all vehicles

    \begin{itemize}
        \item type
        \item initial\_position
        \item moves\_to
        \item move\_diagonally
        \item fuel\_usage
    \end{itemize}

    \begin{mdframed}
        \underline{\textbf{Correct Solution:}}

        \bigskip

        The following are attributes possessed by all vehicles

        \begin{itemize}
            \item \color{red}position\color{black}
            \item \color{red}fuel\color{black}
        \end{itemize}

    \end{mdframed}


    \item No. Referencing the following code in \textit{worksheet\_6\_starter\_code.py},

    \begin{lstlisting}[language=Python]
    class Vehicle:
        ...

        def __init__(self, initial_fuel: int,
                        initial_position: Tuple[int, int]) -> None:
            ...

            self.fuel = initial_fuel
            self.position = initial_position
    \end{lstlisting}

    \bigskip

    we can come up with the following examples.

    \begin{itemize}
        \item Vehicle(100, (10,20))
        \item Vehicle(50, (5, 10))
    \end{itemize}

    \bigskip

    Here, we can see the two vehicles have different value of fuel and initial position.

    \item

    \textit{fuel\_needed} not implemented because each child classes have different
    fuel consumption rate, and the method is to be defined by the child classes by
    overriding it.

    \item

    The following methods must be defined in each of its subclasses

    \bigskip

    \begin{itemize}
        \item Car
        \begin{itemize}
            \item fuel\_needed
            \item move
        \end{itemize}
        \item Helicopter
        \begin{itemize}
            \item fuel\_needed
            \item move
        \end{itemize}
        \item MagicCarpet
        \begin{itemize}
            \item \_\_init\_\_
            \item move
        \end{itemize}
    \end{itemize}

    \bigskip

    \begin{mdframed}
        \underline{\textbf{Correct Solution:}}

        \bigskip

        \begin{itemize}
            \item Car
            \begin{itemize}
                \color{red}
                \item \_\_init\_\_
                \begin{itemize}
                    \item Necessary because the parameter \textit{position} must be
                    set as optional
                    \item Necessary because \textit{self.position} must default
                    to (0,0) if the argument of position not given.
                \end{itemize}
                \color{black}
                \item fuel\_needed
                \color{red}
                \begin{itemize}
                    \item Necessary because vehicle uses fuel
                    \item Necessary because needs to define the fuel cost based
                    on it not being able to moving diagonally.
                \end{itemize}
                \color{black}
            \end{itemize}
            \item Helicopter
            \begin{itemize}
                \color{red}
                \item \_\_init\_\_
                \begin{itemize}
                    \item Necessary because the parameter \textit{position} must be
                    set as optional
                    \item Necessary because \textit{self.position} must default
                    to (3,5) if the argument of position not given.
                \end{itemize}
                \color{black}
                \item fuel\_needed
                \color{red}
                \begin{itemize}
                    \item Necessary because vehicle uses fuel
                    \item Necessary because needs to define the fuel cost based
                    on it being able to move diagonally.
                \end{itemize}
                \color{black}
            \end{itemize}
            \item MagicCarpet
            \begin{itemize}
                \item \_\_init\_\_
                \color{red}
                \begin{itemize}
                    \item Necessary to set the parameters \textit{initial\_fuel},
                    \textit{initial\_position} as optional
                    \item Necessary to randomize the value of \\\textit{self.position}.
                \end{itemize}
                \color{black}
                \item move
                \color{red}
                \begin{itemize}
                    \item Necessary to set the parameters \textit{new\_x} and \textit{new\_y}
                    as optional.
                    \item Necessary to randomize the value of new position.
                \end{itemize}
                \color{black}

            \end{itemize}
        \end{itemize}

    \end{mdframed}

\end{enumerate}

\section*{Question 3}
\begin{lstlisting}[language=Python]
    """
    Initializing SuperDuperManager:
    >>> s = SuperDuperManager()
    >>> s._vehicles
    {}

    Adding Vehicles:
    >>> s.add_vehicle('Car', '1', 100)
    >>> s._vehicles['1'].__class__.__name__
    'Car'
    >>> s.add_vehicle('Helicopter', '1', 100)
    >>> s._vehicles['1'].__class__.__name__
    'Car'

    >>> s.add_vehicle('Helicopter', '2', 100)
    >>> s._vehicles['2'].__class__.__name__
    'Helicopter'

    >>> s.add_vehicle('UnreliableMagicCarpet','3',100)
    >>> s._vehicles['3'].__class__.__name__
    'UnreliableMagicCarpet'

    Moving Vehicle:
    >>> s._vehicles['1'].position
    (0,0)
    >>> s.move_vehicle('1', 1, 1)
    >>> s._vehicles['1'].position
    (1,1)

    >>> s._vehicles['2'].position
    (3,5)
    >>> s.move_vehicle('2', 1, 1)
    >>> s._vehicles['2'].position
    (4,6)

    >>> s._vehicles['3'].position
    (4,8)
    >>> s._vehicles['3'].position
    (12,4)
    >>> s.move_vehicle('3', 1, 1)
    >>> s._vehicles['3'].position
    (100,100)

    Get Vehicle Position:
    >>> s.get_vehicle_position('1')
    (1,1)

    >>> s.get_vehicle_position('2')
    (4,6)

    >>> s.get_vehicle_position('3')
    (50,200)

    Get Vehicle Fuel:
    >>> s.get_vehicle_fuel('1')
    98

    >>> s.get_vehicle_fuel('2')
    99

    >>> s.get_vehicle_fuel('2')
    100
    """
\end{lstlisting}

\section*{Question 4}
\begin{enumerate}[a.]
    \item The instance attribute \textit{id\_} is used to keep track of vehicles.

    \bigskip

    The type of the instance attribute is string.

    \begin{mdframed}
        \underline{\textbf{Correct Solution:}}

        \bigskip

        The instance attribute \color{red}\textit{self.\_vehicles}\color{black}\:
        is used to keep track of vehicles.

        \bigskip

        The type of the instance attribute is \color{red}`dictionary'\color{black}.

    \end{mdframed}

    \bigskip

    \item The vehicles are initialized in class \textit{SuperDuperManager}'s
    \textit{add\_vehicle} method.

    \begin{mdframed}
        \underline{\textbf{Correct Solution:}}

        \bigskip

        The \color{red}instance attribute is\color{black}\:initialized in class
        \textit{SuperDuperManager}'s \color{red}\textit{\_\_init\_\_}\color{black}\:method.

    \end{mdframed}


    \item In code that keeps track of all the vehicles, the vehicles are updated
    via the methods \textit{add\_vehicle} and \textit{move\_vehicle}
\end{enumerate}

\section*{Question 5}
\begin{enumerate}[a.]
    \item If left as is, every car object would possess the instance attributes
    \textit{self.position} and \textit{self.fuel} from class Vehicle.

    \item No. Other instance attributes are not necessary. The instance attribute
    \textit{position} can be used to define its initial position, and position
    after move. As well, the attribute \textit{self.fuel} can be used to calculate
    the amount of distance

    \bigskip

    \underline{\textbf{Notes:}}

    \begin{itemize}
        \item 여보, 형모 노래 듣고 있었어요
        \item https://www.youtube.com/watch?v=5jl8mzCaCr0
        \item 여보, 형모 내 여보 많이 보고싶어요
        \item 구래두 여보, 형모 내 여보 보러 꾹 참꾸 걸어요
        \item 여보, 사랑해
        \item 여보, 고마워요 :)
    \end{itemize}

\end{enumerate}

\section*{Question 6}

\section*{Question 7}

\section*{Question 8}

\end{document}