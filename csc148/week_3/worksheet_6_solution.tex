\documentclass[12pt]{article}
\usepackage[margin=2.5cm]{geometry}
\usepackage{enumerate}
\usepackage{amsfonts}
\usepackage{amsmath}
\usepackage{fancyhdr}
\usepackage{amsmath}
\usepackage{amssymb}
\usepackage{amsthm}
\usepackage{mdframed}
\usepackage{graphicx}
\usepackage{subcaption}
\usepackage{listings}
\usepackage{xcolor}
\usepackage[utf]{kotex}

\definecolor{codegreen}{rgb}{0,0.6,0}
\definecolor{codegray}{rgb}{0.5,0.5,0.5}
\definecolor{codepurple}{rgb}{0.58,0,0.82}
\definecolor{backcolour}{rgb}{0.95,0.95,0.92}

\lstdefinestyle{mystyle}{
    backgroundcolor=\color{backcolour},
    commentstyle=\color{codegreen},
    keywordstyle=\color{magenta},
    numberstyle=\tiny\color{codegray},
    stringstyle=\color{codepurple},
    basicstyle=\ttfamily\footnotesize,
    breakatwhitespace=false,
    breaklines=true,
    captionpos=b,
    keepspaces=true,
    numbers=left,
    numbersep=5pt,
    showspaces=false,
    showstringspaces=false,
    showtabs=false,
    tabsize=1
}

\lstset{style=mystyle}

\begin{document}
\title{CSC148 Worksheet 6 Solution}
\author{Hyungmo Gu}
\maketitle

\section*{Question 1}
\begin{itemize}
    \item

    The two classes already defined are: Vehicle and SuperDuperManager

    \bigskip

    The additional classes required to create for this exercise are: Car,
    Helicopter and MagicCarpet

    \bigskip

    Car, Helicopter and MagicCarpet are child classes of Vehicle.

\end{itemize}

\section*{Question 2}
\begin{enumerate}[a.]
    \item The following are attributes possessed by all vehicles

    \begin{itemize}
        \item type
        \item initial\_position
        \item moves\_to
        \item move\_diagonally
        \item fuel\_usage
    \end{itemize}

    \begin{mdframed}
        \underline{\textbf{Correct Solution:}}

        \bigskip

        The following are attributes possessed by all vehicles

        \begin{itemize}
            \item \color{red}position\color{black}
            \item \color{red}fuel\color{black}
        \end{itemize}

    \end{mdframed}


    \item No. Referencing the following code in \textit{worksheet\_6\_starter\_code.py},

    \begin{lstlisting}[language=Python]
    class Vehicle:
        ...

        def __init__(self, initial_fuel: int,
                        initial_position: Tuple[int, int]) -> None:
            ...

            self.fuel = initial_fuel
            self.position = initial_position
    \end{lstlisting}

    \bigskip

    we can come up with the following examples.

    \begin{itemize}
        \item Vehicle(100, (10,20))
        \item Vehicle(50, (5, 10))
    \end{itemize}

    \bigskip

    Here, we can see the two vehicles have different value of fuel and initial position.

    \item

    \textit{fuel\_needed} not implemented because each child classes have different
    fuel consumption rate, and the method is to be defined by the child classes by
    overriding it.

    \item

    The following methods must be defined in each of its subclasses

    \bigskip

    \begin{itemize}
        \item Car
        \begin{itemize}
            \item fuel\_needed
            \item move
        \end{itemize}
        \item Helicopter
        \begin{itemize}
            \item fuel\_needed
            \item move
        \end{itemize}
        \item MagicCarpet
        \begin{itemize}
            \item \_\_init\_\_
            \item move
        \end{itemize}
    \end{itemize}

    \bigskip

    \begin{mdframed}
        \underline{\textbf{Correct Solution:}}

        \bigskip

        \begin{itemize}
            \item Car
            \begin{itemize}
                \color{red}
                \item \_\_init\_\_
                \begin{itemize}
                    \item Necessary because the parameter \textit{position} must be
                    set as optional
                    \item Necessary because \textit{self.position} must default
                    to (0,0) if the argument of position not given.
                \end{itemize}
                \color{black}
                \item fuel\_needed
                \color{red}
                \begin{itemize}
                    \item Necessary because vehicle uses fuel
                    \item Necessary because needs to define the fuel cost based
                    on it not being able to moving diagonally.
                \end{itemize}
                \color{black}
            \end{itemize}
            \item Helicopter
            \begin{itemize}
                \color{red}
                \item \_\_init\_\_
                \begin{itemize}
                    \item Necessary because the parameter \textit{position} must be
                    set as optional
                    \item Necessary because \textit{self.position} must default
                    to (3,5) if the argument of position not given.
                \end{itemize}
                \color{black}
                \item fuel\_needed
                \color{red}
                \begin{itemize}
                    \item Necessary because vehicle uses fuel
                    \item Necessary because needs to define the fuel cost based
                    on it being able to move diagonally.
                \end{itemize}
                \color{black}
            \end{itemize}
            \item MagicCarpet
            \begin{itemize}
                \item \_\_init\_\_
                \color{red}
                \begin{itemize}
                    \item Necessary to set the parameters \textit{initial\_fuel},
                    \textit{initial\_position} as optional
                    \item Necessary to randomize the value of \\\textit{self.position}.
                \end{itemize}
                \color{black}
                \item move
                \color{red}
                \begin{itemize}
                    \item Necessary to set the parameters \textit{new\_x} and \textit{new\_y}
                    as optional.
                    \item Necessary to randomize the value of new position.
                \end{itemize}
                \color{black}

            \end{itemize}
        \end{itemize}

    \end{mdframed}

\end{enumerate}

\section*{Question 3}
\begin{lstlisting}[language=Python]
    """
    Initializing SuperDuperManager:
    >>> s = SuperDuperManager()
    >>> s._vehicles
    {}

    Adding Vehicles:
    >>> s.add_vehicle('Car', '1', 100)
    >>> s._vehicles['1'].__class__.__name__
    'Car'
    >>> s.add_vehicle('Helicopter', '1', 100)
    >>> s._vehicles['1'].__class__.__name__
    'Car'

    >>> s.add_vehicle('Helicopter', '2', 100)
    >>> s._vehicles['2'].__class__.__name__
    'Helicopter'

    >>> s.add_vehicle('UnreliableMagicCarpet','3',100)
    >>> s._vehicles['3'].__class__.__name__
    'UnreliableMagicCarpet'

    Moving Vehicle:
    >>> s._vehicles['1'].position
    (0,0)
    >>> s.move_vehicle('1', 1, 1)
    >>> s._vehicles['1'].position
    (1,1)

    >>> s._vehicles['2'].position
    (3,5)
    >>> s.move_vehicle('2', 1, 1)
    >>> s._vehicles['2'].position
    (4,6)

    >>> s._vehicles['3'].position
    (4,8)
    >>> s._vehicles['3'].position
    (12,4)
    >>> s.move_vehicle('3', 1, 1)
    >>> s._vehicles['3'].position
    (100,100)

    Get Vehicle Position:
    >>> s.get_vehicle_position('1')
    (1,1)

    >>> s.get_vehicle_position('2')
    (4,6)

    >>> s.get_vehicle_position('3')
    (50,200)

    Get Vehicle Fuel:
    >>> s.get_vehicle_fuel('1')
    98

    >>> s.get_vehicle_fuel('2')
    99

    >>> s.get_vehicle_fuel('2')
    100
    """
\end{lstlisting}

\section*{Question 4}
\begin{enumerate}[a.]
    \item The instance attribute \textit{id\_} is used to keep track of vehicles.

    \bigskip

    The type of the instance attribute is string.

    \begin{mdframed}
        \underline{\textbf{Correct Solution:}}

        \bigskip

        The instance attribute \color{red}\textit{self.\_vehicles}\color{black}\:
        is used to keep track of vehicles.

        \bigskip

        The type of the instance attribute is \color{red}`dictionary'\color{black}.

    \end{mdframed}

    \bigskip

    \item The vehicles are initialized in class \textit{SuperDuperManager}'s
    \textit{add\_vehicle} method.

    \begin{mdframed}
        \underline{\textbf{Correct Solution:}}

        \bigskip

        The \color{red}instance attribute is\color{black}\:initialized in class
        \textit{SuperDuperManager}'s \color{red}\textit{\_\_init\_\_}\color{black}\:method.

    \end{mdframed}


    \item In code that keeps track of all the vehicles, the vehicles are updated
    via the methods \textit{add\_vehicle} and \textit{move\_vehicle}
\end{enumerate}

\section*{Question 5}
\begin{enumerate}[a.]
    \item If left as is, every car object would possess the instance attributes
    \textit{self.position} and \textit{self.fuel} from class Vehicle.

    \item No. Other instance attributes are not necessary.

    \bigskip

    Looking at the table provided at the beginning of worksheet, car class needs
    information about initial position, final position, fuel usage, and remaining fuel amount.

    \bigskip

    For initial and final position, the instance attribute \textit{position} is used.

    \bigskip

    For fuel usage, this is embedded inside \textit{fuel\_needed} method. So,
    the additional attribute is not necessary.

    \bigskip

    For fuel remaining, the instance attribute \textit{self.fuel} is used.

    \bigskip

    \underline{\textbf{Notes:}}

    \begin{itemize}
        \item 여보, 형모 노래 듣고 있었어요
        \item https://www.youtube.com/watch?v=5jl8mzCaCr0
        \item 여보, 형모 내 여보 많이 보고싶어요
        \item 구래두 여보, 형모 내 여보 보러 꾹 참꾸 걸어요
        \item 여보, 사랑해
        \item 여보, 고마워요 :)
    \end{itemize}

\end{enumerate}

\section*{Question 6}
\begin{enumerate}[a.]
    \item Car inherits the following methods from class Vehicle: \textit{\_\_init\_\_},
    \textit{fuel\_needed} and \textit{move}.
    \item Of these inherited methods, \textit{fuel\_needed} needs to be
    implemented by overriding it.
    \item

    The two methods needs overriding: \textit{\_\_init\_\_} and \textit{fuel\_needed}.

    \bigskip

    For \textit{\_\_init\_\_}, because we need to set the parameter \textit{initial\_position}
    as optional while keeping the rest of the code the same, we want to call the
    parent method as a helper

    \begin{lstlisting}[language=Python]
    class Car:

        def __init__(self, initial_fuel: int,
                     initial_position: Tuple[int, int] = (0,0)) -> None:

            super().__init__()

    \end{lstlisting}

    \bigskip

    For \textit{fuel\_needed}, because we need to replace contents within, the
    parent class method shouldn't be called as a helper.

    \begin{lstlisting}[language=Python]
    class Car:
        def fuel_needed(self, new_x: int, new_y: int) -> int:
            # New lines of code here

    \end{lstlisting}

    \bigskip

    \underline{\textbf{Notes:}}

    \begin{itemize}
        \item 여보!!!!!!!!
        \item 사랑해요 여보
        \item 형모 당신만을 사랑해
    \end{itemize}


\end{enumerate}


\section*{Question 7}

Code is also included in \textit{worksheet\_6\_solution.py}.

\begin{lstlisting}[language=Python]
    from worksheet_6_starter_code import Vehicle, SuperDuperManager

    """
    Question 7
    """

    class Car(Vehicle):

        def __init__(self, initial_fuel: int,
                        initial_position: Tuple[int, int] = (0,0)) -> None:
            super().__init__()

        def fuel_needed(self, new_x: int, new_y: int) -> int:

            old_x = self.position[0]
            old_y = self.position[1]

            delta_x = abs(old_x - new_x)
            delta_y = abs(old_y - new_y)

            return delta_x + delta_y

\end{lstlisting}

\section*{Question 8}

\subsection*{1) Helicopter}

\begin{itemize}
    \item \underline{\textbf{Question 5:}}
    \begin{enumerate}[a.]
        \item If left as is, every helicopter objects would possess instance attributes
        \textit{self.position} and \textit{self.fuel} from class Vehicle.

        \bigskip

        This is the same as Car class.

        \item No. With the same reasoning as Car class, \textit{self.position}
        and \textit{self.fuel} are sufficient to do all required operations.

    \end{enumerate}
    \item \underline{\textbf{Question 6:}}
    \begin{enumerate}[a.]
        \item Just like Car class, Helicopter Class inherits the following methods
        from class Vehicle: \textit{\_\_init\_\_}, \textit{fuel\_needed} and \textit{move}.

        \item Just like Car class, \textit{fuel\_needed} needs to be
        implemented by overriding it.

        \item With the same reasoning as Car class, the two methods needs
        overriding: \textit{\_\_init\_\_} and \textit{fuel\_needed}.

    \end{enumerate}

\end{itemize}

\begin{lstlisting}[language=Python]
    from worksheet_6_starter_code import Vehicle, SuperDuperManager
    import math

    """
    Question 8.1
    """

    class Helicopter(Vehicle):

        def __init__(self, initial_fuel: int,
                        initial_position: Tuple[int, int] = (0,0)) -> None:
            super().__init__()

        def fuel_needed(self, new_x: int, new_y: int) -> int:

            old_x = self.position[0]
            old_y = self.position[1]

            delta_x = abs(old_x - new_x)
            delta_y = abs(old_y - new_y)

            return math.ceil(math.sqrt(delta_x**2 + delta_y**2))

\end{lstlisting}


\bigskip

\subsection*{2) UnreliableMagicCarpet}

\begin{itemize}
    \item \underline{\textbf{Question 5:}}
    \begin{enumerate}[a.]
        \item If left as is, every UnreliableMagicCarpet object would possess
        the instance\\ attributes \textit{self.position} and \textit{self.fuel} from class Vehicle.

        \item No. With the same reasoning as Car class, other instance attributes are
        not necessary.
    \end{enumerate}
    \item \underline{\textbf{Question 6:}}
    \begin{enumerate}[a.]
        \item UnreliableMagicCarpet inherits the following methods from class Vehicle: \textit{\_\_init\_\_},
        \textit{fuel\_needed} and \textit{move}.
        \item No inherited methods must be implemented. All methods are implemented by parent class.

        \begin{mdframed}
            \underline{\textbf{Correct Solution}}

            \bigskip

            \color{red}\textit{fuel\_needed} needs implementation. This is to suppress `NotImplemented' error\color{black}.
        \end{mdframed}

        \item

        The two methods needs overriding: \textit{\_\_init\_\_}, and \textit{move}.

        \bigskip

        For \textit{\_\_init\_\_}, the following needs modification

        \begin{itemize}
            \item \textit{initial\_position}: This argument needs to be set as optional
            \item \textit{self.position}: This instance attribute needs to be set a random integer value
        \end{itemize}

        \bigskip

        Because we want to change the above while keeping information about \textit{self.fuel},
        we want to use parent class method as a helper.

        \bigskip

        \begin{lstlisting}[language=Python]
        import random
        from datetime import datetime

        class UnreliableMagicCarpet:

            def __init__(self, initial_fuel: int,
                         initial_position: Tuple[int, int] = (0,0)) -> None:
                super().__init__()

                random.seed(datetime.now())
                self.position = (random.randint(-100,100), random.randint(-100,100))

        \end{lstlisting}

        Now, for \textit{move}, we need to replace entire contents within.

        \bigskip

        Then, it follows from this fact that parent method doesn't need to be
        called as a helper.

        \bigskip

        \begin{lstlisting}[language=Python]
        class UnreliableMagicCarpet:
            def move(self, new_x: int, new_y: int) -> None:
                # New lines of code here

        \end{lstlisting}

        \begin{mdframed}
            \underline{\textbf{Correct Solution}}

            \bigskip

            The two methods needs overriding: \textit{\_\_init\_\_}, and \textit{move}.

            \bigskip

            For \textit{\_\_init\_\_}, the following needs modification

            \begin{itemize}
                \item \textit{initial\_position}: This argument needs to be set as optional
                \item \textit{self.position}: This instance attribute needs to be set a random integer value
            \end{itemize}

            \bigskip

            Because we want to change the above while keeping information about \textit{self.fuel},
            we want to use parent class method as a helper.

            \bigskip

            \begin{lstlisting}[language=Python]
            import random
            from datetime import datetime

            class UnreliableMagicCarpet:

                def __init__(self, initial_fuel: int,
                             initial_position: Tuple[int, int] = (0,0)) -> None:
                    super().__init__()

                    random.seed(datetime.now())
                    self.position = (random.randint(-100,100), random.randint(-100,100))

            \end{lstlisting}

            \color{red}
            Now, for \textit{fuel\_needed}, we need to replace entire contents within.

            \bigskip

            It follows from this fact that parent method doesn't need to be
            called as a helper.
            \color{black}

            \bigskip

            \begin{lstlisting}[language=Python]
            class UnreliableMagicCarpet:
                def fuel_needed(self, new_x: int, new_y: int) -> None:
                    # New lines of code here

            \end{lstlisting}

            \bigskip

            Now, for \textit{move}, we need to replace entire contents within.

            \bigskip

            Then, it follows from this fact that parent method doesn't need to be
            called as a helper.

            \bigskip

            \begin{lstlisting}[language=Python]
            class UnreliableMagicCarpet:
                def move(self, new_x: int, new_y: int) -> None:
                    # New lines of code here

            \end{lstlisting}
        \end{mdframed}

    \end{enumerate}
\end{itemize}


\begin{lstlisting}[language=Python]
    from worksheet_6_starter_code import Vehicle, SuperDuperManager
    import math

    """
    Question 8.2
    """

    class UnreliableMagicCarpet:

        def __init__(self, initial_fuel: int,
                        initial_position: Tuple[int, int] = (0,0)) -> None:
            super().__init__()

            random.seed(datetime.now())
            self.position = (random.randint(-100,100), random.randint(-100,100))

        def move(self, new_x: int, new_y: int) -> None:
            self.position = (random.randint(-100,100), random.randint(-100,100))

\end{lstlisting}

\bigskip

\begin{mdframed}
    \underline{\textbf{Correct Solution}}

    \bigskip

    \begin{lstlisting}[language=Python]

    from worksheet_6_starter_code import Vehicle, SuperDuperManager
    import math

    """
    Question 8.2
    """

    class UnreliableMagicCarpet:

        def __init__(self, initial_fuel: int,
                        initial_position: Tuple[int, int] = (0,0)) -> None:
            super().__init__()

            random.seed(datetime.now())
            self.position = (random.randint(-10,10), random.randint(-10,10)) #<- Correct solution

        def fuel_needed(self, new_x: int, new_y: int) -> int:
            return 0 # <- Correct Solution

        def move(self, new_x: int, new_y: int) -> None:
            new_x = self.position[0] + random.randint(-2, 2) # <- Correct Solution
            new_y = self.position[1] + random.randint(-2, 2) # <- Correct Solution
            self.position = (new_x, new_y) # <- Correct Solution

    \end{lstlisting}

\end{mdframed}


\end{document}