\documentclass[12pt]{article}
\usepackage[margin=2.5cm]{geometry}
\usepackage{enumerate}
\usepackage{amsfonts}
\usepackage{amsmath}
\usepackage{fancyhdr}
\usepackage{amsmath}
\usepackage{amssymb}
\usepackage{amsthm}
\usepackage{mdframed}
\usepackage{graphicx}
\usepackage{listings}
\usepackage{xcolor}

\definecolor{codegreen}{rgb}{0,0.6,0}
\definecolor{codegray}{rgb}{0.5,0.5,0.5}
\definecolor{codepurple}{rgb}{0.58,0,0.82}
\definecolor{backcolour}{rgb}{0.95,0.95,0.92}

\lstdefinestyle{mystyle}{
    backgroundcolor=\color{backcolour},
    commentstyle=\color{codegreen},
    keywordstyle=\color{magenta},
    numberstyle=\tiny\color{codegray},
    stringstyle=\color{codepurple},
    basicstyle=\ttfamily\footnotesize,
    breakatwhitespace=false,
    breaklines=true,
    captionpos=b,
    keepspaces=true,
    numbers=left,
    numbersep=5pt,
    showspaces=false,
    showstringspaces=false,
    showtabs=false,
    tabsize=1
}

\lstset{style=mystyle}

\begin{document}
\title{Rectangle Exercise Solution}
\author{Hyungmo Gu}
\maketitle

\section*{Part 1}
\begin{enumerate}
    \item

    \textbf{Class Name:} Rectangle

    \bigskip

    \textbf{One Line Summary:} A rectangle is defined by its top-left coordinates
    as well as its width and height.

    \item

    \begin{lstlisting}[language=Python]
    Rectangle(10,20,300,400)
    \end{lstlisting}

    \item

    \textbf{Headers:}

    \bigskip

    \begin{itemize}
        \item translate\_left(self, num):
        \item translate\_right(self, num):
        \item translate\_up(self, num):
        \item translate\_down(self, num):
        \item is\_equal(self, rect):
        \item is\_falling\_within\_another\_rectangle(self, rect):
        \item is\_overlapping(self, rect):
    \end{itemize}

    \begin{lstlisting}[language=Python]
    class Rectangle:
        """A rectangle is defined by its top-left coordinates as well as its width and height."""

        def translate_left(self, num):
            """Translate Rectangle to left by <num>
            @type self: Rectangle
            @type num: int
            @rtype: None
            >>> rect = Rectangle(10,20,300,400)
            >>> rect.translate_left(10)
            """
        def translate_right(self, num):
            """Translate Rectangle to right by <num>
            @type self: Rectangle
            @type num: int
            @rtype: None
            >>> rect = Rectangle(10,20,300,400)
            >>> rect.translate_right(10)
            """

        def translate_up(self, num):
            """Translate Rectangle to up by <num>
            @type self: Rectangle
            @type num: int
            @rtype: None
            >>> rect = Rectangle(10,20,300,400)
            >>> rect.translate_up(10)
            """

        def translate_down(self, num):
            """Translate Rectangle to down by <num>
            @type self: Rectangle
            @type num: int
            @rtype: None
            >>> rect = Rectangle(10,20,300,400)
            >>> rect.translate_down(10)
            """

        def is_equal(self, rect):
            """Return whether <rect> and <self> have the same coordinate and size
            @type self: Rectangle
            @type rect: Rectangle
            @rtype: bool
            >>> rect_1 = Rectangle(10,20,300,400)
            >>> rect_2 = Rectangle(10,20,300,400)
            >>> rect_3 = Rectangle(15,25,300,400)
            >>> rect_1.is_equal(rect_2)
            True
            >>> rect_1.is_equal(rect_3)
            False
            """

        def is_falling_within_another_rectangle(self, rect):
            """Return whether <self> is inside <rect>
            @type self: Rectangle
            @type rect: Rectangle
            @rtype: bool
            >>> rect_1 = Rectangle(10,20,300,400)
            >>> rect_2 = Rectangle(15,15,100,50)
            >>> rect_2.is_falling_within_another_rectangle(rect_1)
            True
            """

        def is_overlapping(self, rect):
            """Returns whether <self> has overlapping region with <rect>
            @type self: Rectangle
            @type rect: Rectangle
            @rtype: bool
            >>> rect_1 = Rectangle(10,20,300,400)
            >>> rect_2 = Rectangle(0,0,300,400)
            >>> rect_1.is_overlapping(rect_2)
            True
            """
    \end{lstlisting}

    \bigskip

    \textbf{Notes:}

    \begin{itemize}
    \item What should be written for \textbf{@rtype} if nothing is returned?
    \item What should be written for \textbf{@type} if a parameter is of type class?
    \item What should be written for \textbf{@type} if a parameter is \textbf{self}?
    \item When writing an example, should class instantiation also be included like below?

    \begin{lstlisting}[language=Python]
        def is_overlapping(self, rect):
            """
            ...
            >>> rect_1 = Rectangle(10,20,300,400)
            >>> rect_2 = Rectangle(0,0,300,400)
            ...
            """
    \end{lstlisting}
    \end{itemize}
\end{enumerate}

\section*{Part 2}

\end{document}