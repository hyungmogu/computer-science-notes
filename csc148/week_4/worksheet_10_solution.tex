\documentclass[12pt]{article}
\usepackage[margin=2.5cm]{geometry}
\usepackage{enumerate}
\usepackage{amsfonts}
\usepackage{amsmath}
\usepackage{fancyhdr}
\usepackage{amsmath}
\usepackage{amssymb}
\usepackage{amsthm}
\usepackage{mdframed}
\usepackage{graphicx}
\usepackage{subcaption}
\usepackage{listings}
\usepackage{xcolor}
\usepackage{booktabs}
\usepackage[utf]{kotex}

\definecolor{codegreen}{rgb}{0,0.6,0}
\definecolor{codegray}{rgb}{0.5,0.5,0.5}
\definecolor{codepurple}{rgb}{0.58,0,0.82}
\definecolor{backcolour}{rgb}{0.95,0.95,0.92}

\lstdefinestyle{mystyle}{
    backgroundcolor=\color{backcolour},
    commentstyle=\color{codegreen},
    keywordstyle=\color{magenta},
    numberstyle=\tiny\color{codegray},
    stringstyle=\color{codepurple},
    basicstyle=\ttfamily\footnotesize,
    breakatwhitespace=false,
    breaklines=true,
    captionpos=b,
    keepspaces=true,
    numbers=left,
    numbersep=5pt,
    showspaces=false,
    showstringspaces=false,
    showtabs=false,
    tabsize=1
}

\lstset{style=mystyle}

\begin{document}
\title{CSC148 Worksheet 10 Solution}
\author{Hyungmo Gu}
\maketitle

\section*{Question 1}
\begin{enumerate}[a.]
    \item

    The following code must be changed.

    \begin{itemize}
        \item \textit{self.\_items.append(item)} in \textit{push()} method must
        be changed to \textit{self.\_items.insert(0,item)}
        \item \textit{self.\_items.pop()} in \textit{pop()} method must be changed
        to \textit{self.\_items.pop(0)}.
    \end{itemize}

    \item

    \begin{lstlisting}[language=Python,caption={worksheet\_10\_q1b\_solution.py},captionpos=b]
    class Stack:
        """A last-in-first-out (LIFO) stack of items.
        Stores data in first-in, last-out order. When removing an item from the
        stack, the most recently-added item is the one that is removed.
        """
        # === Private Attributes ===
        # _items:
        # The items stored in the stack. The end of the list represents
        # the top of the stack.
        _items: List

        def __init__(self) -> None:
            """Initialize a new empty stack.
            """
            self._items = []

        def is_empty(self) -> bool:
            """Return whether this stack contains no items.
            >>> s = Stack()
            >>> s.is_empty()
            True
            >>> s.push(
            hello
            )
            >>> s.is_empty()
            False
            """

            return self._items == []

        def push(self, item: Any) -> None:
            """Add a new element to the top of this stack.
            """
            # =========== Solution (Question 1.b) ===========
            self._items.insert(0,item)
            # ===============================================

        def pop(self) -> Any:
            """Remove and return the element at the top of this stack.
            >>> s = Stack()
            >>> s.push(
            hello
            )
            >>> s.push(
            goodbye
            )
            >>> s.pop()

            goodbye

            """
            # =========== Solution (Question 1.b) ===========
            self._items.pop(0)
            # ===============================================
    \end{lstlisting}

\end{enumerate}

\section*{Question 2}
\begin{itemize}
    \item

    The following changes in docstring must be made.

    \begin{enumerate}[1.]
        \item The line `The items stored in the stack. The end of the list represents the top of the stack.'
        under the description of \textit{\_items} in private attribute should be changed to
        `The items stored in the stack. The end of the list represents the \textit{bottom} of the stack.'

        \item The line `Add a new element to the top of this stack.' in \textit{push()}
        method must be changed to `Add a new element to the \textit{bottom} of this stack.'
        \item The line `Remove and return the element at the top of this stack.'
        in \textit{pop()} method must be changed to `Remove and return the
        element at the \textit{bottom} of this stack.'
    \end{enumerate}
\end{itemize}

\bigskip

\begin{mdframed}
    \underline{\textbf{Correct Solution:}}

    \bigskip

    \begin{itemize}
        \item

        The following changes in docstring must be made.

        \begin{enumerate}[1.]
            \item The line `The items stored in the stack. The \color{red}front\color{black}\:of the list represents the top of the stack.'
            under the description of \textit{\_items} in private attribute should be changed to
            `The items stored in the stack. The end of the list represents the \color{red}top\color{black}\:of the stack.'
            \color{red}
                \begin{itemize}
                    \item This is because stack is LIFO. Last element in is the first to come out.
                    \item Last element added and removed are now at the beginning of the list.
                \end{itemize}
            \color{black}
        \end{enumerate}
    \end{itemize}

\end{mdframed}

\bigskip

\underline{\textbf{Notes:}}

\bigskip

\begin{itemize}
    \item 형모야. 쪼금만이라도 무너지지 않고 여보에게 더 빨리 갈 수 만있다면...
    \item 형모야. 내 여보 있어
    \item 형모야. 괜찮아
    \item 형모야. 차분히...
\end{itemize}


\section*{Question 3}

\section*{Question 4}

\end{document}