\documentclass[12pt]{article}
\usepackage[margin=2.5cm]{geometry}
\usepackage{titling}
\usepackage{enumerate}
\usepackage{graphicx}
\usepackage{mdframed}
\usepackage{listings}
\usepackage{xcolor}
\usepackage{hyperref}
\usepackage[utf]{kotex}

\definecolor{codegreen}{rgb}{0,0.6,0}
\definecolor{codegray}{rgb}{0.5,0.5,0.5}
\definecolor{codepurple}{rgb}{0.58,0,0.82}
\definecolor{backcolour}{rgb}{0.95,0.95,0.92}

\lstdefinestyle{mystyle}{
    backgroundcolor=\color{backcolour},
    commentstyle=\color{codegreen},
    keywordstyle=\color{magenta},
    numberstyle=\tiny\color{codegray},
    stringstyle=\color{codepurple},
    basicstyle=\ttfamily\footnotesize,
    breakatwhitespace=false,
    breaklines=true,
    captionpos=b,
    keepspaces=true,
    numbers=left,
    numbersep=5pt,
    showspaces=false,
    showstringspaces=false,
    showtabs=false,
    tabsize=1
}

\lstset{style=mystyle}

\predate{}
\postdate{}

\begin{document}
\title{Lab 4: Abstract Data Type Solution}
\date{}
\maketitle

\section*{1) Stack review}
\begin{enumerate}[1.]
    \item Write a function that takes a stack of integers and removes all of the items which are greater than 5.
    The other items in the stack, and their relative order, should remain unchanged.

    \begin{lstlisting}[language=Python,caption={task\_1\_solution.py},captionpos=b]
    ...
    def remove_big(s: Stack) -> None:
        """Remove the items in <stack> that are greater than 5.

        Do not change the relative order of the other items.

        >>> s = Stack()
        >>> s.push(1)
        >>> s.push(29)
        >>> s.push(8)
        >>> s.push(4)
        >>> remove_big(s)
        >>> s.pop()
        4
        >>> s.pop()
        1
        >>> s.is_empty()
        True
        """
        side_stack = Stack()

        # 1. Move all elements in stack 1 to stack 2
        while not s.is_empty():
            # 1.1 While moving elements, if element value is greater than 5, then pass
            popped_element = s.pop()

            if type(popped_element) == int and popped_element > 5:
                continue

            side_stack.push(popped_element)

        # 2. Move back all elements from stack 2 back to stack 1.
        while not side_stack.is_empty():
            s.push(side_stack.pop())
    ...
    \end{lstlisting}

    \item Write a function that takes a stack and returns a new stack that contains each item in the old stack twice in a row.
    We’ll leave it up to you to decide what order to put the copies into in the new stack.

    \bigskip

    Note that because the docstring doesn’t say that the old stack will be mutated,
    the old stack should remain unchanged when the function returns.
\end{enumerate}

\bigskip



\end{document}
