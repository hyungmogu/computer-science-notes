\documentclass[12pt]{article}
\usepackage[margin=2.5cm]{geometry}
\usepackage{titling}
\usepackage{enumerate}
\usepackage{graphicx}
\usepackage{mdframed}
\usepackage{listings}
\usepackage{xcolor}
\usepackage{hyperref}
\usepackage[utf]{kotex}

\definecolor{codegreen}{rgb}{0,0.6,0}
\definecolor{codegray}{rgb}{0.5,0.5,0.5}
\definecolor{codepurple}{rgb}{0.58,0,0.82}
\definecolor{backcolour}{rgb}{0.95,0.95,0.92}

\lstdefinestyle{mystyle}{
    backgroundcolor=\color{backcolour},
    commentstyle=\color{codegreen},
    keywordstyle=\color{magenta},
    numberstyle=\tiny\color{codegray},
    stringstyle=\color{codepurple},
    basicstyle=\ttfamily\footnotesize,
    breakatwhitespace=false,
    breaklines=true,
    captionpos=b,
    keepspaces=true,
    numbers=left,
    numbersep=5pt,
    showspaces=false,
    showstringspaces=false,
    showtabs=false,
    tabsize=1
}

\lstset{style=mystyle}

\predate{}
\postdate{}

\begin{document}
\title{Lab 4: Abstract Data Type Solution}
\date{}
\maketitle

\section*{3) Running timing experiments}
\begin{enumerate}[1.]
    \item Your first task is to open \textit{timequeue.py} and follow the instructions
    contained within it to complete the timing experiment.

    \item After you’ve run your experiment, you should notice that your two queue
    operations \textit{enqueue} and \textit{dequeue} behave quite differently.

    \bigskip

    While one seems to take the same amount of time no matter how many items are
    in the queue, the other takes longer and longer as the number of items are in
    the queue.

    \bigskip

    Compare your notes with other groups.
    Which end of a Python list seems to be the “slow” end? Do you have a guess
    as to why this might be the case? (If you don't: don't worry! You'll learn
    about this in later weeks.)

\end{enumerate}



\end{document}
