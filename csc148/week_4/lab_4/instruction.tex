\documentclass[12pt]{article}
\usepackage[margin=2.5cm]{geometry}
\usepackage{titling}
\usepackage{enumerate}
\usepackage{graphicx}
\usepackage{mdframed}
\usepackage{listings}
\usepackage{xcolor}
\usepackage{hyperref}
\usepackage[utf]{kotex}

\definecolor{codegreen}{rgb}{0,0.6,0}
\definecolor{codegray}{rgb}{0.5,0.5,0.5}
\definecolor{codepurple}{rgb}{0.58,0,0.82}
\definecolor{backcolour}{rgb}{0.95,0.95,0.92}

\lstdefinestyle{mystyle}{
    backgroundcolor=\color{backcolour},
    commentstyle=\color{codegreen},
    keywordstyle=\color{magenta},
    numberstyle=\tiny\color{codegray},
    stringstyle=\color{codepurple},
    basicstyle=\ttfamily\footnotesize,
    breakatwhitespace=false,
    breaklines=true,
    captionpos=b,
    keepspaces=true,
    numbers=left,
    numbersep=5pt,
    showspaces=false,
    showstringspaces=false,
    showtabs=false,
    tabsize=1
}

\lstset{style=mystyle}

\predate{}
\postdate{}

\begin{document}
\title{Lab 4: Abstract Data Type}
\date{}
\maketitle

\section*{1) Stack review}
Open \textit{mystack.py} and first review the given stack implementation and the
\textit{size} function we discussed in lecture.

\bigskip

\noindent Complete the following tasks.

\bigskip

\noindent Note that you should write these as top-level functions, not stack methods.

\bigskip

\noindent While you may use a temporary stack (as we did in lecture for \textit{size}),do not use
any other Python compound data structures, like lists.

\bigskip

\begin{enumerate}[1.]
    \item Write a function that takes a stack of integers and removes all of the items which are greater than 5.
    The other items in the stack, and their relative order, should remain unchanged.

    \item Write a function that takes a stack and returns a new stack that contains each item in the old stack twice in a row.
    We’ll leave it up to you to decide what order to put the copies into in the new stack.

    \bigskip

    Note that because the docstring doesn’t say that the old stack will be mutated,
    the old stack should remain unchanged when the function returns.
\end{enumerate}

\begin{lstlisting}[language=Python,caption={mystack.py},captionpos=b]
    """CSC148 Lab 4: Abstract Data Types

    === CSC148 Winter 2020 ===
    Department of Computer Science,
    University of Toronto

    === Module Description ===
    In this module, you will write two different functions that operate on a Stack.
    Pay attention to whether or not the stack should be modified.
    """
    from typing import Any, List


    ################################################################
    # Task 1: Practice with stacks
    ################################################################
    class Stack:
        """A last-in-first-out (LIFO) stack of items.

        Stores data in a last-in, first-out order. When removing an item from the
        stack, the most recently-added item is the one that is removed.
        """
        # === Private Attributes ===
        # _items:
        #     The items stored in this stack. The end of the list represents
        #     the top of the stack.
        _items: List

        def __init__(self) -> None:
            """Initialize a new empty stack."""
            self._items = []

        def is_empty(self) -> bool:
            """Return whether this stack contains no items.

            >>> s = Stack()
            >>> s.is_empty()
            True
            >>> s.push('hello')
            >>> s.is_empty()
            False
            """
            return self._items == []

        def push(self, item: Any) -> None:
            """Add a new element to the top of this stack."""
            self._items.append(item)

        def pop(self) -> Any:
            """Remove and return the element at the top of this stack.

            Raise an EmptyStackError if this stack is empty.

            >>> s = Stack()
            >>> s.push('hello')
            >>> s.push('goodbye')
            >>> s.pop()
            'goodbye'
            """
            if self.is_empty():
                raise EmptyStackError
            else:
                return self._items.pop()


    class EmptyStackError(Exception):
        """Exception raised when an error occurs."""
        pass


    def size(s: Stack) -> int:
        """Return the number of items in s.

        >>> s = Stack()
        >>> size(s)
        0
        >>> s.push('hi')
        >>> s.push('more')
        >>> s.push('stuff')
        >>> size(s)
        3
        """
        side_stack = Stack()
        count = 0
        # Pop everything off <s> and onto <side_stack>, counting as we go.
        while not s.is_empty():
            side_stack.push(s.pop())
            count += 1
        # Now pop everything off <side_stack> and back onto <s>.
        while not side_stack.is_empty():
            s.push(side_stack.pop())
        # <s> is restored to its state at the start of the function call.
        # We consider that it was not mutated.
        return count


    # TODO: implement this function!
    def remove_big(s: Stack) -> None:
        """Remove the items in <stack> that are greater than 5.

        Do not change the relative order of the other items.

        >>> s = Stack()
        >>> s.push(1)
        >>> s.push(29)
        >>> s.push(8)
        >>> s.push(4)
        >>> remove_big(s)
        >>> s.pop()
        4
        >>> s.pop()
        1
        >>> s.is_empty()
        True
        """
        pass


    # TODO: implement this function!
    def double_stack(s: Stack) -> Stack:
        """Return a new stack that contains two copies of every item in <stack>.

        We'll leave it up to you to decide what order to put the copies into in
        the new stack.

        >>> s = Stack()
        >>> s.push(1)
        >>> s.push(29)
        >>> new_stack = double_stack(s)
        >>> s.pop()  # s should be unchanged.
        29
        >>> s.pop()
        1
        >>> s.is_empty()
        True
        >>> new_items = []
        >>> new_items.append(new_stack.pop())
        >>> new_items.append(new_stack.pop())
        >>> new_items.append(new_stack.pop())
        >>> new_items.append(new_stack.pop())
        >>> sorted(new_items)
        [1, 1, 29, 29]
        """
        pass


    if __name__ == '__main__':
        import doctest
        doctest.testmod()

\end{lstlisting}


\section*{2) Queues}

\section*{3) Running timing experiments}

\section*{4) Additional exercises}

\end{document}
