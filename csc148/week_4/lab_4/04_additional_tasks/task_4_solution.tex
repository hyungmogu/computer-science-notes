\documentclass[12pt]{article}
\usepackage[margin=2.5cm]{geometry}
\usepackage{titling}
\usepackage{enumerate}
\usepackage{graphicx}
\usepackage{mdframed}
\usepackage{listings}
\usepackage{xcolor}
\usepackage{hyperref}
\usepackage[utf]{kotex}

\definecolor{codegreen}{rgb}{0,0.6,0}
\definecolor{codegray}{rgb}{0.5,0.5,0.5}
\definecolor{codepurple}{rgb}{0.58,0,0.82}
\definecolor{backcolour}{rgb}{0.95,0.95,0.92}

\lstdefinestyle{mystyle}{
    backgroundcolor=\color{backcolour},
    commentstyle=\color{codegreen},
    keywordstyle=\color{magenta},
    numberstyle=\tiny\color{codegray},
    stringstyle=\color{codepurple},
    basicstyle=\ttfamily\footnotesize,
    breakatwhitespace=false,
    breaklines=true,
    captionpos=b,
    keepspaces=true,
    numbers=left,
    numbersep=5pt,
    showspaces=false,
    showstringspaces=false,
    showtabs=false,
    tabsize=1
}

\lstset{style=mystyle}

\predate{}
\postdate{}

\begin{document}
\title{Lab 4: Abstract Data Type Solution}
\date{}
\maketitle

\section*{4) Additional Tasks}
\subsection*{Graphing your results}
\begin{enumerate}[1.]
    \item Implement \textit{time\_queue\_lists}, a modified version of your timing
    experiment function that returns a tuple containing three lists:

    \begin{itemize}
        \item A list of queue sizes it tried
        \item A list of the corresponding times to run enqueue for each queue size
        \item A list of the corresponding times to run dequeue for each queue size
    \end{itemize}

    \bigskip

    Note that each of your lists should have the same length.

    \bigskip

    \begin{mdframed}
        \begin{lstlisting}[language=Python,caption={task\_4\_q1\_solution.py},captionpos=b]
        ...
        def time_queue_lists() -> Tuple[List[int], List[float], List[float]]:
            """Run timing experiments for Queue.enqueue and Queue.dequeue.

            Return lists storing the results of the experiments.  See the lab
            handout for further details.
            """
            queue_sizes = [10000, 20000, 40000, 80000, 160000]
            enqueue_time_list = []
            dequeue_time_list = []

            trials = 200

            for queue_size in queue_sizes:
                queues = _setup_queues(queue_size, trials)

                time = 0
                for queue in queues:
                    time += timeit('queue.enqueue(1)', number=1, globals=locals())

                print(f'enqueue: Queue size {queue_size:>7}, time {time}')
                enqueue_time_list.append(time)

            for queue_size in queue_sizes:
                queues = _setup_queues(queue_size, trials)

                time = 0
                for queue in queues:
                    time += timeit('queue.dequeue()', number=1, globals=locals())

                print(f'dequeue: Queue size {queue_size:>7}, time {time}')
                dequeue_time_list.append(time)

            return (queue_sizes, enqueue_time_list, dequeue_time_list)

        ...
        \end{lstlisting}
    \end{mdframed}

    \item To actually plot the data, we’ll use the Python library \textit{matplotlib},
    which is an extremely powerful and popular library for plotting all sorts of
    data.

    \bigskip

    If you’re on a Teaching Lab machine, you already have this library installed.

    \bigskip

    If you’re on your own machine, you should have already installed this library
    by following the CSC148 Software Guide. (Look for the section on installing
    Python libraries.)

    \bigskip

    Add the statement \textit{import matplotlib.pyplot as plt} to the top of \textit{timequeue.py},
    and make sure you can still run your file without error.

    \bigskip

    \begin{mdframed}

    \begin{lstlisting}[language=Python,caption={task\_4\_q2\_solution.py},captionpos=b]
    ...
    import matplotlib.pyplot as plt
    ...
    \end{lstlisting}

    \bigskip

    \underline{\textbf{Note:}}

    \begin{itemize}
        \item
        If \textit{matplotlib} is missing, install by typing \textit{pip3 install matplotlib}
        in terminal or windows command line.
    \end{itemize}

    \end{mdframed}

    \item To get a basic 2-D plot of your timing data, work your way through the
    first part of \href{https://matplotlib.org/users/pyplot_tutorial.html}{this guide} (Links to an external site.). (Ignore all of the
    references to “numpy”, which is another Python library we aren’t using in this
    course. Also ignore the other sections after the first one; the whole tutorial
    is pretty long!)

    \bigskip

    You can use an x-axis range of 0-200000 and a y-axis range of 0-0.02 (feel
    free to adjust the y-axis depending on how long the experiments take to run
    on your computer).

    \item If you still have time, explore! There’s lots of customization you can
    do with \textit{matplotlib} to make your graphs really pretty.
\end{enumerate}

\subsection*{Undo and redo}

\end{document}
