\documentclass[12pt]{article}
\usepackage[margin=2.5cm]{geometry}
\usepackage{titling}
\usepackage{enumerate}
\usepackage{graphicx}
\usepackage{mdframed}
\usepackage{listings}
\usepackage{xcolor}
\usepackage{hyperref}
\usepackage[utf]{kotex}

\definecolor{codegreen}{rgb}{0,0.6,0}
\definecolor{codegray}{rgb}{0.5,0.5,0.5}
\definecolor{codepurple}{rgb}{0.58,0,0.82}
\definecolor{backcolour}{rgb}{0.95,0.95,0.92}

\lstdefinestyle{mystyle}{
    backgroundcolor=\color{backcolour},
    commentstyle=\color{codegreen},
    keywordstyle=\color{magenta},
    numberstyle=\tiny\color{codegray},
    stringstyle=\color{codepurple},
    basicstyle=\ttfamily\footnotesize,
    breakatwhitespace=false,
    breaklines=true,
    captionpos=b,
    keepspaces=true,
    numbers=left,
    numbersep=5pt,
    showspaces=false,
    showstringspaces=false,
    showtabs=false,
    tabsize=1
}

\lstset{style=mystyle}

\predate{}
\postdate{}

\begin{document}
\title{Lab 4: Abstract Data Type Solution}
\date{}
\maketitle

\section*{2) Queues}
\begin{enumerate}[1.]
    \item Implement \textit{Queue} class found in \textit{myqueue.py}

    \begin{lstlisting}[language=Python,caption={task\_2\_q1\_solution.py},captionpos=b]
    ...
    class Queue:
        """A first-in-first-out (FIFO) queue of items.

        Stores data in a first-in, first-out order. When removing an item from the
        queue, the least recently-added item (i.e. the oldest item in the Queue)
        is the one that is removed.
        # === Private Attributes ===
        # _items:
        #     The items stored in this queue. The front of the list represents
        #     the front of the queue.
        """
        _items: List
        def __init__(self) -> None:
            """Initialize a new empty queue."""
            self._items = []

        def is_empty(self) -> bool:
            """Return whether this queue contains no items.

            >>> q = Queue()
            >>> q.is_empty()
            True
            >>> q.enqueue('hello')
            >>> q.is_empty()
            False
            """
            return self._items == []

        def enqueue(self, item: Any) -> None:
            """Add <item> to the back of this queue.
            """
            self._items.append(item)

        def dequeue(self) -> Optional[Any]:
            """Remove and return the item at the front of this queue.

            Return None if this Queue is empty.
            (We illustrate a different mechanism for handling an erroneous case.)

            >>> q = Queue()
            >>> q.enqueue('hello')
            >>> q.enqueue('goodbye')
            >>> q.dequeue()
            'hello'
            """
            if self.is_empty():
                raise EmptyStackError
            else:
                return self._items.pop(0)
    ...
    \end{lstlisting}

    \item Complete functions \textit{product} and \textit{product\_star} in \textit{myqueue.py}

    \bigskip

    \begin{lstlisting}[language=Python,caption={task\_2\_q2\_solution.py},captionpos=b]
    ...
    def product_star(integer_queue: Queue) -> int:
        """Return the product of integers in the queue.

        Precondition: integer_queue contains only integers.

        >>> primes = [2, 3, 5, 7, 11, 13, 17, 19, 23, 29]
        >>> prime_line = Queue()
        >>> for prime in primes:
        ...     prime_line.enqueue(prime)
        ...
        >>> product_star(prime_line)
        6469693230
        >>> prime_line.is_empty()
        False
        """
        side_queue = Queue()
        output = 1

        if integer_queue.is_empty():
            return 0

        # 1. Move elements from integer_queue to side_queue
        while not integer_queue.is_empty():
            dequeued_element = integer_queue.dequeue()

            # 1.1 While moving elements, multiply each of them to output.
            output *= dequeued_element
            side_queue.enqueue(dequeued_element)

        # 2. Move back elements from side_queue to integer_queue
        while not side_queue.is_empty():
            integer_queue.enqueue(side_queue.dequeue())

        return output
    ...
    \end{lstlisting}

    \bigskip

    \underline{\textbf{Notes:}}

    \bigskip

    \begin{itemize}
        \item 오늘 형모 마음도 날씨도 봄 처럼 따뜻해요.
        \item 여보, 형모 사랑하는 내 여보 하고 손잡고 같이 걸을 수 있게 해줘서 고마워요 :)
        \item 형모 마음이 설레요
        \item 허허허허허허허
        \item 형모야. 오늘 무너지지 않고, 설레이는 마음 갖고, 차분히
        \item 화이팅 :)
    \end{itemize}

\end{enumerate}

\end{document}
