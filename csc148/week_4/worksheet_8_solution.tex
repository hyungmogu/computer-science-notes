\documentclass[12pt]{article}
\usepackage[margin=2.5cm]{geometry}
\usepackage{enumerate}
\usepackage{amsfonts}
\usepackage{amsmath}
\usepackage{fancyhdr}
\usepackage{amsmath}
\usepackage{amssymb}
\usepackage{amsthm}
\usepackage{mdframed}
\usepackage{graphicx}
\usepackage{subcaption}
\usepackage{listings}
\usepackage{xcolor}
\usepackage{booktabs}
\usepackage[utf]{kotex}

\definecolor{codegreen}{rgb}{0,0.6,0}
\definecolor{codegray}{rgb}{0.5,0.5,0.5}
\definecolor{codepurple}{rgb}{0.58,0,0.82}
\definecolor{backcolour}{rgb}{0.95,0.95,0.92}

\lstdefinestyle{mystyle}{
    backgroundcolor=\color{backcolour},
    commentstyle=\color{codegreen},
    keywordstyle=\color{magenta},
    numberstyle=\tiny\color{codegray},
    stringstyle=\color{codepurple},
    basicstyle=\ttfamily\footnotesize,
    breakatwhitespace=false,
    breaklines=true,
    captionpos=b,
    keepspaces=true,
    numbers=left,
    numbersep=5pt,
    showspaces=false,
    showstringspaces=false,
    showtabs=false,
    tabsize=1
}

\lstset{style=mystyle}

\begin{document}
\title{CSC148 Worksheet 8 Solution}
\author{Hyungmo Gu}
\maketitle

\section*{Question 1}
\begin{itemize}
    \item
    No. It's not a good solution.

    \bigskip

    The code is trying to count the number of elements in list.

    \bigskip

    The \textit{for} loop takes $\Theta(n)$ time, and this is not an efficient solution.

    \bigskip

    We can do better than that by reducing the runtime to $\Theta(1)$ by using \textit{len(...)} function.

    \bigskip

    \begin{mdframed}
        \underline{\textbf{Correct Solution:}}

        \bigskip
        No. It's not a good solution.

        \bigskip

        \color{red}
        Stacks are not iterable
        \color{black}
    \end{mdframed}

\end{itemize}

\section*{Question 2}
\begin{itemize}
    \item

    Yes. This is a good solution.

    \bigskip

    The quick points are

    \begin{itemize}
        \item The method is trying to determine the number of elements in \textit{Stack}.
        \item \textit{pop()} method removes an element from stack. This works as an indexing variable for the while loop.
        \item \textit{is\_empty()} method checks for the condition of stack not having any elements.
        This allows while loop to terminate after using stack's \textit{pop()} method sufficient number of times.
        \item \textit{count} variable allows the number of elements to be counted, as it is being
        removed from \textit{Stack} by \textit{pop} method.
    \end{itemize}

    \bigskip

    \begin{mdframed}
        \underline{\textbf{Correct Solution:}}

        \bigskip
        \color{red}
        No. This is not a good solution.
        \color{black}

        \bigskip

        The quick points are

        \color{red}
        \begin{itemize}
            \item The code uses \textit{pop()} method.
            \item \textit{pop()} method causes \textit{Stack} to mutate in
            number of elements, and the next time the \textit{size} function is called,
            it will return 0.
            \item \textit{size()} function should not affect the number of elements in stack.
        \end{itemize}
        \color{black}
    \end{mdframed}
\end{itemize}

\section*{Question 3}
\begin{itemize}
    \item

    This is a good solution if the instance attribute \textit{\_items}
    is using list to store items.

    \bigskip

    Going further, this is a good solution for any iterable objects with
    \textit{\_\_len\_\_} method (it should be correctly defined as well!).


    \begin{mdframed}
        \underline{\textbf{Correct Solution:}}

        \bigskip
        \color{red}
        No. This is not a good solution.

        \bigskip

        \textit{s.\_item} is a private attribute, and private attribute should
        not be used outside of \textit{Stack}.

        \color{black}
    \end{mdframed}
\end{itemize}

\section*{Question 4}
\begin{itemize}
    \item

    No. This is not a good solution.

    \bigskip

    The quick points are

    \begin{itemize}
        \item Parameter \textit{s} is of type \textit{Stack}
        \item \textit{Stack} is a class
        \item Class passes function by reference. That is, changes made to class
        inside function also affects outside.
        \item The variable \textit{s\_copy} is pointing to \textit{s}
        \begin{itemize}
            \item Unlike with string and integers, this doesn't copy class (this is a huge bad)
        \end{itemize}
        \item \textit{pop()} method is used, and this causes \textit{s} outside of
        function to have size 0 by the end of operation
    \end{itemize}
\end{itemize}

\section*{Question 5}

\begin{lstlisting}[language=Python,caption={worksheet\_8\_solution.py},captionpos=b]
    from typing import Any

    class Stack:
        """A last-in-first-out (LIFO) stack of items.
        Stores data in last-in, first-out order. When removing an item from the
        stack, the most recently-added item is the one that is removed.
        """
        def __init__(self) -> None:
            """Initialize a new empty stack."""
            self._item = []

        def is_empty(self) -> bool:
            """Return whether this stack contains no items.
            >>> s = Stack()
            >>> s.is_empty()
            True
            >>> s.push('hello')
            >>> s.is_empty()
            False
            """

            if len(self._item) != 0:
                return False

            return True

        def push(self, item: Any) -> None:
            """Add a new element to the top of this stack.
            """
            self._item.append(item)

        def pop(self) -> Any:
            """Remove and return the element at the top of this stack.
            >>> s = Stack()
            >>> s.push('hello')
            >>> s.push('goodbye')
            >>> s.pop()
            'goodbye'
            """
            if len(self._item) == 0:
                return None

            item = self._item.pop()
            return item

    if __name__ == '__main__':
        import doctest
        doctest.testmod()
\end{lstlisting}

\end{document}