\documentclass[12pt]{article}
\usepackage{enumerate}
\usepackage{amsfonts}
\usepackage{amsmath}
\usepackage{fancyhdr}
\usepackage{amssymb}
\usepackage{mdframed}

\begin{document}
\title{Worksheet 10 Review}
\maketitle

\section*{Question 1}
\begin{enumerate}[a.]
    \item
    \begin{align}
        (165)_8 &= 5 \cdot 8^0 + 6 \cdot 8^1 + 1 \cdot 8^2\\
        &= 5 + 48 + 64\\
        &= 53 + 64\\
        &= 117
    \end{align}

    \item
    \setcounter{equation}{0}
    \begin{align}
        (B4)_16 &= 4 \cdot 16^0 + 11 \cdot 16^1\\
        &= 4 + (11 \cdot 16)\\
        &= 4 + 176\\
        &= 180
    \end{align}
\end{enumerate}

\section*{Question 2}
\begin{enumerate}[a.]
    \item
    \setcounter{equation}{0}
    \begin{align*}
        357 \div 2 &= 178, \text{remainder \textbf{1}},\\
        178 \div 2 &= 89, \text{remainder \textbf{0}},\\
        89 \div 2 &= 44, \text{remainder \textbf{1}},\\
        44 \div 2 &= 22, \text{remainder \textbf{0}},\\
        22 \div 2 &= 11, \text{remainder \textbf{0}},\\
        11 \div 2 &= 5, \text{remainder \textbf{1}},\\
        5 \div 2 &= 2, \text{remainder \textbf{1}},\\
        2 \div 2 &= 1, \text{remainder \textbf{0}},\\
        1 \div 2 &= 0, \text{remainder \textbf{1}}
    \end{align*}

    Combining it together, the binary representation of 357 is $(101100101)_2$

    \item

    \begin{align*}
        1 \cdot 2^0 + 0 \cdot 2^1 + 1 \cdot 2^2 &= \frac{1 + 0 + 4}{8^0} = 5\\
        0 \cdot 2^3 + 0 \cdot 2^4 + 1 \cdot 2^5 &= \frac{0 + 0 + 32}{8^1} = 4\\
        1 \cdot 2^6 + 0 \cdot 2^7 + 1 \cdot 2^8 &= \frac{64 + 0 + 256}{8^2} = 5
    \end{align*}

    Combining it together, the octal representation of $(101100101)_2$ is
    $(545)_8$.

    \item

    \begin{align*}
        357 \div 16 &= 22, \text{remainder \textbf{5}},\\
        22 \div 16 &= 1, \text{remainder \textbf{5}},\\
        1 \div 16 &= 0, \text{remainder \textbf{1}},\\
    \end{align*}

    Combining it together, the hexadecimal representation of 357 is $(155)_{16}$.

    \begin{mdframed}
        \underline{\textbf{Correct Solution:}}

        \bigskip

        \begin{align*}
            357 \div 16 &= 22, \text{remainder \textbf{5}},\\
            22 \div 16 &= 1, \text{remainder \color{red}\textbf{6}}\color{black},\\
            1 \div 16 &= 0, \text{remainder \textbf{1}},\\
        \end{align*}

        Combining it together, the hexadecimal representation of 357 is \color{red}\textbf{$(165)_{16}$}\color{black}.

    \end{mdframed}

\end{enumerate}

\section*{Question 3}
\begin{enumerate}[a.]
    \item
    \setcounter{equation}{0}
    \begin{align}
        0.375 \times 2 = 0.75 + &\textbf{0}\\
        0.75 \times 2 = 0.5 + &\textbf{1}\\
        0.5 \times 2 = 0 + &\textbf{1}
    \end{align}

    Combining the above, the binary representation of 0.375 is $(0.011)_2$.

    \textbf{Notes:}
    \begin{itemize}
        \item \textbf{Converting decimal to binary}

        \begin{align}
            0.8125 \times 2 = 0.625 + &\textbf{1}\\
            0.625 \times 2 = 0.25 + &\textbf{1}\\
            0.25 \times 2 = 0.5 + &\textbf{0}\\
            0.5 \times 2 = 0 + &\textbf{1}
        \end{align}

        Binaries read \textit{top to bottom}
    \end{itemize}

    \item

    \setcounter{equation}{0}
    \begin{align}
        0.1 \times 2 = 0.2 + &\textbf{0}\\
        0.2 \times 2 = 0.4 + &\textbf{0}\\
        0.4 \times 2 = 0.8 + &\textbf{0}\\
        0.8 \times 2 = 0.6 + &\textbf{1}\\
        0.6 \times 2 = 0.2 + &\textbf{1}\\
        0.2 \times 2 = 0.4 + &\textbf{0}\\
        0.4 \times 2 = 0.8 + &\textbf{0}\\
        0.8 \times 2 = 0.6 + &\textbf{1}
    \end{align}

    Combining the above, the binary representation of 0.1 is $(0.0\overline{0011})_2$.

\end{enumerate}

\section*{Question 4}
\begin{enumerate}[a.]
    \item
    \setcounter{equation}{0}
    \begin{align}
        \sum\limits_{i=0}^{\infty} \left(\frac{1}{2}\right) &= \frac{\frac{1}{2}}{1 - \frac{1}{2}}\\
        &= \frac{\frac{1}{2}}{\frac{1}{2}}\\
        &= 1
    \end{align}

    \item

    Since $1^{st}$ 1 repeats every 4 decimal places, and $2^{nd}$ 1 repeats every
    5 decimal places, we have

    \setcounter{equation}{0}
    \begin{align}
        (0.0\overline{0011})_2 &= \sum\limits_{i=1}^{\infty} \left(\frac{1}{2}\right)^{4i} + \sum\limits_{i=1}^{\infty} \left(\frac{1}{2}\right)^{4i + 1}\\
        &= \sum\limits_{i=1}^{\infty} \frac{1}{16} + \frac{1}{2} \cdot \sum\limits_{i=1}^{\infty} \frac{1}{16}\\
        &= \frac{1}{15} + \frac{1}{30}\\
        &= \frac{3}{30}\\
        &= \frac{1}{10}
    \end{align}
\end{enumerate}

\end{document}