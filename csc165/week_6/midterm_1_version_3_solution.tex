\documentclass[12pt]{article}
\usepackage{enumerate}
\usepackage{amsfonts}
\usepackage{amsmath}
\usepackage{fancyhdr}
\usepackage{amssymb}

\begin{document}
\title{Midterm 1 Version 3 Solution}
\maketitle

\section*{Question 1}
\begin{enumerate}[a.]
    \item

    Since $S_1 = \{ab,ba,aab,bba,baa,\dots\}$ and $S_2=\{aaa,aab,aba,baa,abb,bab,
    bba\}$,

    \bigskip

    $S_2 \setminus S_1 = \{aaa,aab,aba,bab\}$

    \bigskip

    \textbf{Correct Solution:}

    Since $S_1 = \{ab,ba,aab,abb,bba,baa,\dots\}$ and $S_2=\{aaa,aab,aba,baa,abb,bab,
    bba,bbb\}$,

    $S_2 \setminus S_1 = \{aaa,aba,bab,bbb\}$

    \item

    See table below

    \begin{tabular}{c|c|c|c|c|c}
        $p$ & $q$ & $r$ & $\neg r$ & $p \Rightarrow q$ & $(p \Rightarrow q) \Leftrightarrow \neg r$\\
        \hline
        T & T & T & F & T & F\\
        \hline
        T & T & F & T & T & T\\
        \hline
        T & F & T & F & F & T\\
        \hline
        F & T & T & F & T & F\\
        \hline
        T & F & F & T & F & F\\
        \hline
        F & T & F & T & T & T\\
        \hline
        F & F & T & F & T & F\\
        \hline
        F & F & F & T & T & T
    \end{tabular}

    \item

    Let $x = \underline{\hspace{1.5cm}}$, and $y \in \mathbb{N}$.

    \bigskip

    We will prove that $P(x)$ is true and $Q(x,y)$ or $Q(x, y+1)$ is false.

    \bigskip

    \textbf{Correct Solution:}

    Negation: $\exists x \in \mathbb{N}, \forall y \in \mathbb{N}, P(x) \land
    (\neg Q(x,y) \land \neg Q(x,y+1))$

    \bigskip

    Let $x = \underline{\hspace{1.5cm}}$ and $y \in \mathbb{N}$.

    We will prove that $P(x)$ is true, and both $Q(x,y)$ and $Q(x,y+1)$ are false.

\end{enumerate}

\section*{Question 2}
\begin{enumerate}[a.]
    \item

    $\forall x \in T, Canadian(x) \land Star(x)$

    \bigskip

    \textbf{Correct Solution:}

    $\forall x \in T, Canadian(x) \Rightarrow Star(x)$

    \item

    $\forall x \in T, Canadian(x) \Rightarrow \forall y \in T,\:\neg Canadian(y) \land Defeated(x,y)$

    \bigskip

    \textbf{Correct Solution:}

    $\forall x \in T, Canadian(x) \Rightarrow (\forall y \in T,\:\neg Canadian(y) \Rightarrow Defeated(x,y))$

    \item

    $\exists x \in T, Canadian(x) \land Star(x) \Rightarrow \forall y \in T,\:
    \exists z \in T,\: y \neq z \land Canadian(y) \land Defeated(y,z)$

    \bigskip

    \textbf{Correct Solution:}

    $\exists x \in T, Canadian(x) \land Star(x) \Rightarrow (\forall y \in T,\:
    Canadian(y) \Rightarrow \exists z \in T, y \neq z \land Defeated(y,z))$

    \item

    $\exists x \in T,\:Canadian(x) \land Star(x) \land (\forall y \in T,\: x \neq y
    \land Canadian(y) \land \neg Star(y))$

    \textbf{Correct Solution:}

    $\exists x \in T,\:Canadian(x) \land Star(x) \land (\forall y \in T,\: x \neq y
    \land Canadian(y) \Rightarrow \neg Star(y))$

\end{enumerate}

\section*{Question 3}
\begin{enumerate}[a.]
    \item

    $\forall n \in \mathbb{N},\:\exists k \in \mathbb{N},\:n > 1 \land n = 2k + 1 \Rightarrow
    \exists p,q \in \mathbb{Z}^{+},\:n = p^2 - q^2$

    \item

    Let $n \in \mathbb{N}$. Assume $n > 1$, and that there exists $k \in \mathbb{Z}$
    such that $n = 2k + 1$.

    \bigskip

    Also, let $p = k + 1$ and $q = k$.

    \bigskip

    Then,
    \setcounter{equation}{0}
    \begin{align}
        p^2 - q^2 &= (k+1)^2 - k^2\\
        &= k^2 + 2k + 1 - k^2\\
        &= 2k + 1\\
        &= n
    \end{align}

    Then, it follows from above that the statement  $\forall n \in \mathbb{N},\:
    \exists k \in \mathbb{N},\:n > 1 \land n = 2k + 1 \Rightarrow \exists p,q \in
    \mathbb{Z}^{+},\:n = p^2 - q^2$ is true.

\end{enumerate}


\section*{Question 4}
\begin{itemize}
    \item

    Let $d,n \in \mathbb{N}$. Assume $d \mid n$ and $d \neq n$.

    \bigskip

    Then, $\exists k \in \mathbb{N}$,
    \setcounter{equation}{0}
    \begin{align}
        n &= kd
    \end{align}

    \bigskip

    Since $k \in \mathbb{N}$, there are two cases of divisors. One is when $k = 1$,
    and the other is $k \geq 2$.

    \bigskip

    Since $n \neq d$, $k \geq 2$.

    \bigskip

    Then,

    \begin{align}
        n &= kd\\
        &\geq 2d
    \end{align}

    \bigskip

    Then,

    \begin{align}
        \frac{n}{2} &\geq d
    \end{align}

    Then, it follows from above that the statement $\forall d,n \in \mathbb{N}$,
    $d \mid n \land d \neq n \Rightarrow d \leq \frac{n}{2}$ is true.

\end{itemize}

\end{document}