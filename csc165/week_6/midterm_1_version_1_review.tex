\documentclass[12pt]{article}
\usepackage{enumerate}
\usepackage{amsfonts}
\usepackage{amsmath}
\usepackage{fancyhdr}
\usepackage{amssymb}
\usepackage{mdframed}

\begin{document}
\title{Midterm 1 Version 1 Review}
\maketitle

\section*{Question 1}
\begin{enumerate}[a.]
    \item

    Because we know

    $S_1 = \{ aa,bb,cc,aab,aac,aaa,bba,bbb,bbc,cca,ccb,ccc,aaaa,\dots \}$
    and $S_2$ is a set of all strings over U with length 3, we can conclude

    \begin{align*}
        S_1 \cap S_2 = \{aaa,aab,aac,bba,bbb,bbc,cca,ccb,ccc\}
    \end{align*}

    \item

    See table below

    \begin{tabular}{|c|c|c|c|c|c|}
        \hline
        $p$ & $q$ & $r$ & $\neg r$ & $p \lor q$ & $p \lor q \Rightarrow \neg r$\\
        \hline
        T & T & T & F & T & F\\
        \hline
        T & T & F & T & T & T\\
        \hline
        T & F & T & F & T & F\\
        \hline
        F & T & T & F & T & F\\
        \hline
        T & F & F & T & T & T\\
        \hline
        F & T & F & T & T & T\\
        \hline
        F & F & T & F & F & T\\
        \hline
        F & F & F & T & F & T\\
        \hline
    \end{tabular}

    \item

    Let $x \in \mathbb{N}$, and $y = \underline{\hspace{1.5cm}}$.

    \bigskip

    We will prove that predicate $P(x,y)$ is true, or predicate $Q(x,y)$ is
    true.

    \begin{mdframed}
        \underline{\textbf{Correct Solution:}}

        \color{red}
        \textbf{Let $x = \underline{\hspace{1.5cm}}$, and $y \in \mathbb{N}$.}

        \bigskip

        We will prove that \color{red}\textbf{both predicates $P(x,y)$ and $Q(x,y)$
        are false.}\color{black}

        \color{black}

    \end{mdframed}


    \textbf{Notes:}
    \begin{itemize}
        \item How can I proceed a proof when there is $\lor$ on R.H.S of the statement?
        What's the general structure of proof given this symbol?
    \end{itemize}
\end{enumerate}

\section*{Question 2}

\section*{Question 3}

\section*{Question 4}

\end{document}