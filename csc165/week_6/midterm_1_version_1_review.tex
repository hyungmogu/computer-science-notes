\documentclass[12pt]{article}
\usepackage{enumerate}
\usepackage{amsfonts}
\usepackage{amsmath}
\usepackage{fancyhdr}
\usepackage{amssymb}
\usepackage{mdframed}

\begin{document}
\title{Midterm 1 Version 1 Review}
\maketitle

\section*{Question 1}
\begin{enumerate}[a.]
    \item

    Because we know

    $S_1 = \{ aa,bb,cc,aab,aac,aaa,bba,bbb,bbc,cca,ccb,ccc,aaaa,\dots \}$
    and $S_2$ is a set of all strings over U with length 3, we can conclude

    \begin{align*}
        S_1 \cap S_2 = \{aaa,aab,aac,bba,bbb,bbc,cca,ccb,ccc\}
    \end{align*}

    \item

    See table below

    \begin{tabular}{|c|c|c|c|c|c|}
        \hline
        $p$ & $q$ & $r$ & $\neg r$ & $p \lor q$ & $p \lor q \Rightarrow \neg r$\\
        \hline
        T & T & T & F & T & F\\
        \hline
        T & T & F & T & T & T\\
        \hline
        T & F & T & F & T & F\\
        \hline
        F & T & T & F & T & F\\
        \hline
        T & F & F & T & T & T\\
        \hline
        F & T & F & T & T & T\\
        \hline
        F & F & T & F & F & T\\
        \hline
        F & F & F & T & F & T\\
        \hline
    \end{tabular}

    \item

    Let $x \in \mathbb{N}$, and $y = \underline{\hspace{1.5cm}}$.

    \bigskip

    We will prove that predicate $P(x,y)$ is true, or predicate $Q(x,y)$ is
    true.

    \begin{mdframed}
        \underline{\textbf{Correct Solution:}}

        \color{red}
        \textbf{Let $x = \underline{\hspace{1.5cm}}$, and $y \in \mathbb{N}$.}

        \bigskip

        We will prove that \color{red}\textbf{both predicates $P(x,y)$ and $Q(x,y)$
        are false.}\color{black}

        \color{black}

    \end{mdframed}


    \textbf{Notes:}
    \begin{itemize}
        \item How can I proceed a proof when there is $\lor$ on R.H.S of the statement?
        What's the general structure of proof given this symbol?
    \end{itemize}
\end{enumerate}

\section*{Question 2}
\begin{enumerate}[a.]
    \item $\exists x \in P,\:Student(x) \land Attends(x)$
    \item

    $\forall x \in P,\:\exists y \in P, Student(y) \land Attends(y) \Rightarrow Loves(x,y)$

    \begin{mdframed}
        \underline{\textbf{Correct Solution:}}

        \bigskip

        $\forall x \in P,\:\exists y \in P, Student(y) \land Attends(y) \color{red}\land Loves(x,y)\color{black}$
    \end{mdframed}

    \textbf{Notes:}
    \begin{itemize}
        \item When should $\Rightarrow$ be used, and when should $\land$ be used?
    \end{itemize}

    \item

    $\forall x \in P,\: Student(x) \land Attends(x) \Rightarrow Loves(x,x)$

    \item

    $\forall x_1,x_2 \in P,\: x_1 \neq x_2 \Rightarrow Loves(x_1,x_2) \land
    Loves(x_2,x_1) \Rightarrow \neg Attends(x_1) \lor \neg Attends(x_2)$

    \begin{mdframed}
        \underline{\textbf{Correct Solution:}}

        \bigskip

        $\forall x_1,x_2 \in P,\: x_1 \neq x_2 \color{red}\land\color{black} Loves(x_1,x_2) \land
        Loves(x_2,x_1) \Rightarrow \neg Attends(x_1) \lor \neg Attends(x_2)$
    \end{mdframed}
\end{enumerate}

\section*{Question 3}
\begin{enumerate}[a.]
    \item

    $\forall a,b,c \in \mathbb{Z}, \exists l,m,n \in \mathbb{Z}, b = la \land c = mb
    \Rightarrow c = na$

    \item

    Let $a,b,c \in \mathbb{Z}$. Assume there is some $l,m,n \in \mathbb{Z}$,
    $b = la$ and $c = mb$.

    \bigskip

    We want to show there is some $n \in \mathbb{Z}$, $c = na$.

    \bigskip

    Because we know $c = mb$ and $b = la$, we can conclude that

    \begin{align}
        c &= mb\\
        &= (ml)a
    \end{align}

    \bigskip

    Since $ml \in \mathbb{Z}$, we can choose $n = ml$.

    \bigskip

    Then,

    \begin{align}
        c &= na
    \end{align}

\end{enumerate}

\section*{Question 4}
\begin{itemize}
    \item

    Let $x,y \in \mathbb{R}$.

    \bigskip

    We want to show $\lfloor x + y \rfloor \geq \lfloor x \rfloor + \lfloor y \rfloor$.

    \bigskip

    Because we know $x \in \mathbb{R}$, by using fact 1 on $x$, we can conclude there
    is $\epsilon \in \mathbb{R}$ and $0 \leq \epsilon < 1$, $x = \lfloor x \rfloor + \epsilon$.

    \bigskip

    Then,

    \setcounter{equation}{0}
    \begin{align}
        \lfloor x + y \rfloor &\geq \bigl\lfloor \lfloor x \rfloor + \epsilon + y \bigr\rfloor
    \end{align}

    \bigskip

    Then, because we know $\lfloor x \rfloor \in \mathbb{Z}$ and $y \in \mathbb{R}$,
    by using fact 2, we can conclude

    \begin{align}
        \lfloor x + y \rfloor &\geq \lfloor x \rfloor + \lfloor \epsilon + y \rfloor
    \end{align}

    \bigskip

    Then, since $\lfloor \epsilon + y \rfloor \geq \lfloor y \rfloor$,

    \begin{align}
        \lfloor x + y \rfloor &\geq \lfloor x \rfloor + \lfloor y \rfloor
    \end{align}

\end{itemize}

\end{document}