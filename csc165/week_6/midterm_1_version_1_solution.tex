\documentclass[12pt]{article}
\usepackage{enumerate}
\usepackage{amsfonts}
\usepackage{amsmath}
\usepackage{fancyhdr}
\usepackage{amssymb}

\begin{document}
\title{Midterm 1 Version 1 Solution}
\maketitle

\section*{Question 1}
\begin{enumerate}[a.]
    \item

    $S_1 = \{aa,bb,cc,aaa,aab,aac,bba,bbb,bbc,cca,ccb,ccc,\dots\}$

    \bigskip

    Since $S_2$ is a set of elements with length 3,

    $S_1 \cap S_2 = \{aaa,aab,aac,bba,bbb,bbc,cca,ccb,ccc\}$

    \item

    See below

    \begin{tabular}{c|c|c|c|c|c}
        $p$ & $q$ & $r$ & $\neg r$ & $(p \lor q)$ & $(p \lor q) \Rightarrow \neg r$\\
        \hline
        T & T & T & F & T & F\\
        \hline
        T & F & F & T & T & T\\
        \hline
        F & T & F & T & T & T\\
        \hline
        F & F & T & F & F & T\\
        \
        T & T & F & T & T & T\\
        \hline
        T & F & T & F & T & T\\
        \hline
        F & T & T & F & T & T\\
        \hline
        F & F & F & T & F & F
    \end{tabular}

    \item

    \textbf{Negation:} $\exists x \in \mathbb{N}, \forall y \in \mathbb{N}, \neg
    P(x,y) \land \neg Q(x,y)$.

    \bigskip

    Let $x = \underline{\hspace{1.5cm}}$, and $y \in \mathbb{N}$.

    \bigskip

    We will prove that predicate $P$ and $Q$ are not true.

\end{enumerate}

\section*{Question 2}
\begin{enumerate}[a.]
    \item

    $\exists x \in P,\: Student(x) \land Attends(x)$

    \item

    $\forall x \in P,\:\exists y \in P,\:Student(y) \land Attends(y) \land Loves(x,y)$

    \item

    $\forall x \in P,\: Student(x) \land Attends(x) \Rightarrow Loves(x,x)$

    \item

    $\forall x_1,x_2 \in P,\:x_1 \neq x_2 \land Loves(x_1,x_2) \Rightarrow
    (Attends(x_1) \land \neg Attends(x_2)) \lor (\neg Attends(x_2) \land Attends(x_1))$

    \bigskip

    \textbf{Correct Solution:}

    $\forall x_1,x_2 \in P,\:x_1 \neq x_2 \land Loves(x_1,x_2) \land Loves(x_1,x_2) \Rightarrow
    \neg Attends(x_1) \lor \neg Attends(x_2)$


\end{enumerate}

\section*{Question 3}
\begin{enumerate}[a.]
    \item

    $\forall a,b,c \in \mathbb{Z},\:\exists k,l \in \mathbb{Z},\:b = ka \land c = lb
    \Rightarrow \exists m \in \mathbb{Z},\:c = ma$

    \item

    Let $a,b,c \in \mathbb{Z}$, and $k = \frac{b}{a},\:l = \frac{c}{b} \in \mathbb{Z}$.
    Assume, $b = ka$ and $c = lb$.

    \bigskip

    Then,

    \begin{align}
        c &= lb\\
        &= \left( \frac{c}{b} \right) a\\
        &= \left( \frac{c}{b} \right) \left( \frac{b}{a} \right) a\\
        &= \left[ \left( \frac{c}{b} \right) \left( \frac{b}{a} \right) \right] a
    \end{align}

    \bigskip

    Since $\left( \frac{c}{b} \right),\:\left( \frac{b}{a} \right) \in \mathbb{Z}$,
    $\left( \frac{c}{b} \right) \left( \frac{b}{a} \right) \in \mathbb{Z}$.

    \bigskip

    Then, it follows from the definition of divisibility that $a$ divides $c$.

\end{enumerate}

\section*{Question 4}
\begin{itemize}
    \item

    Let $x,y \in \mathbb{R}$.

    \bigskip

    Then, there exists $\epsilon_1,\:\epsilon_2 \in \mathbb{R},\:0 \leq
    \epsilon_1,\:\epsilon_2 < 1 \land x = \lfloor x \rfloor + \epsilon_1 \land y =
    \lfloor y \rfloor + \epsilon_2$ by fact 1.

    \bigskip

    Then,
    \setcounter{equation}{0}
    \begin{align}
        \lfloor x + y \rfloor &= \Big\lfloor \lfloor x \rfloor + \epsilon_1 = \lfloor y \rfloor + \epsilon_2 \Big\rfloor\\
        &= \Big\lfloor (\lfloor x \rfloor + \lfloor y \rfloor) + (\epsilon_1 + \epsilon_2) \Big\rfloor
    \end{align}

    \bigskip

    Then,

    \begin{align}
        \Big\lfloor (\lfloor x \rfloor + \lfloor y \rfloor) + (\epsilon_1 + \epsilon_2) \Big\rfloor &= (\lfloor x \rfloor + \lfloor y \rfloor) + \Big\lfloor \epsilon_1 + \epsilon_2 \Big\rfloor
    \end{align}

    by fact 2.

    \bigskip

    Then,

    \begin{align}
        \lfloor x + y \rfloor &\geq \lfloor x \rfloor + \lfloor y \rfloor
    \end{align}

    \bigskip

    Then, it follows from above that the statement $\forall x,y\in \mathbb{R}, \lfloor x + y
    \rfloor \geq \lfloor x \rfloor + \lfloor y \rfloor$ is true.


\end{itemize}

\end{document}