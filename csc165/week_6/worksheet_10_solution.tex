\documentclass[12pt]{article}
\usepackage{enumerate}
\usepackage{amsfonts}
\usepackage{amsmath}
\usepackage{fancyhdr}
\usepackage{amssymb}

\begin{document}
\title{Worksheet 10 Solution}
\maketitle

\section*{Question 1}
\begin{enumerate}[a.]
    \item

    \begin{align}
        (165)_8 &= 5 \times 8^0 + 6 \times 8^1 + 1 \times 8^2\\
        &= 5 + 48 + 64\\
        &= 117
    \end{align}

    \item

    \textbf{Reference Table}

    \bigskip

    \begin{tabular}{ c|c|c|c|c|c|c|c|c|c|c|c|c|c|c|c|c }
        Number & 1 & 2 & 3 & 4 & 5 & 6 & 7 & 8 & 9 & A & B & C & D & E & F & G\\
        \hline
        Value & 1 & 2 & 3 & 4 & 5 & 6 & 7 & 8 & 9 & 10 & 11 & 12 & 13 & 14 & 15 & 16
    \end{tabular}

    \setcounter{equation}{0}
    \begin{align}
        (B4)_{16} &= 4 \times 16^0 + 11 \times 16^1\\
        &= 4 + 176\\
        &= 180
    \end{align}

\end{enumerate}

\section*{Question 2}
\begin{enumerate}[a.]
    \item

    \begin{align*}
        357 \div 2 = 178 &, \text{remainder}\:\textbf{1}\\
        178 \div 2 = 89 &, \text{remainder}\:\textbf{0}\\
        89 \div 2 = 44 &, \text{remainder}\:\textbf{1}\\
        44 \div 2 = 22 &, \text{remainder}\:\textbf{0}\\
        22 \div 2 = 11 &, \text{remainder}\:\textbf{0}\\
        11 \div 2 = 5 &, \text{remainder}\:\textbf{1}\\
        5 \div 2 = 2 &, \text{remainder}\:\textbf{1}\\
        2 \div 2 = 1 &, \text{remainder}\:\textbf{0}\\
        1 \div 2 = 0 &, \text{remainder}\:\textbf{1}
    \end{align*}

    Hence, the binary representation of 357 is (101100101).

    \item

    \begin{align*}
        357 \div 8 = 44 &, \text{remainder}\:\textbf{5}\\
        44 \div 8 = 5 &, \text{remainder}\:\textbf{4}\\
        5 \div 8 = 0 &, \text{remainder}\:\textbf{5}
    \end{align*}

    Hence, the octal representation of 357 is $(545)_8$

    \item

    \begin{align*}
        357 \div 16 = 22 &, \text{remainder}\:\textbf{5}\\
        22 \div 16 = 1 &, \text{remainder}\:\textbf{6}\\
        1 \div 16 = 0 &, \text{remainder}\:\textbf{1}
    \end{align*}

    Hence, the hexadecimal representation of 357 is $(165)_{16}$

\end{enumerate}

\section*{Question 3}
\begin{enumerate}[a.]
    \item

    \begin{align*}
        0.375 \times 2 = 0.750 &, + \textbf{0}\\
        0.750 \times 2 = 0.5 &, + \textbf{1}\\
        0.5 \times 2 = 0 &, + \textbf{1}
    \end{align*}

    Hence, the binary representation of 0.375 is $(0.011)_2$.

    \item

    \begin{align*}
        \frac{1}{10} \times 2 &= \frac{2}{10} + \textbf{0}\\
        \frac{2}{10} \times 2 &= \frac{4}{10} + \textbf{0}\\
        \frac{4}{10} \times 2 &= \frac{8}{10} + \textbf{0}\\
        \frac{8}{10} \times 2 &= \frac{6}{10} + \textbf{1}\\
        \frac{6}{10} \times 2 &= \frac{2}{10} + \textbf{1}\\
        \frac{2}{10} \times 2 &= \frac{4}{10} + \textbf{0}\\
        \frac{4}{10} \times 2 &= \frac{8}{10} + \textbf{0}\\
        \frac{8}{10} \times 2 &= \frac{6}{10} + \textbf{1}
    \end{align*}

    Hence, the binary representation of $\frac{1}{10}$ is $(0.0\overline{0011})_2$.

\end{enumerate}


\end{document}