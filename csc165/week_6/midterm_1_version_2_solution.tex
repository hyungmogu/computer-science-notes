\documentclass[12pt]{article}
\usepackage{enumerate}
\usepackage{amsfonts}
\usepackage{amsmath}
\usepackage{fancyhdr}
\usepackage{amssymb}

\begin{document}
\title{Midterm 1 Version 2 Solution}
\maketitle

\section*{Question 1}
\begin{enumerate}[a.]
    \item

    Since

    $S_1 = \{1,2,3,5,7,11,13,17,19,23,29\}$, and $S_2 = \{1,2,3,5,6,10,15,30\}$,

    $S_1 \cap S_2 = \{1,2,3,5\}$

    \item

    See the table below

    \begin{tabular}{c|c|c|c|c|c}
        $p$ & $q$ & $r$ & $\neg p$ & $\neg p \Leftrightarrow q$ & $(\neg p \Leftrightarrow q) \Rightarrow r$\\
        \hline
        T & T & T & F & F & T\\
        \hline
        T & T & F & F & F & T\\
        \hline
        T & F & T & F & T & T\\
        \hline
        F & T & T & T & T & F\\
        \hline
        T & F & F & F & T & F\\
        \hline
        F & F & T & T & F & T\\
        \hline
        F & F & F & T & F & T
    \end{tabular}

    \item

    Let $x \in \mathbb{N}$. Assume $P(x)$.

    \bigskip

    We will prove that there is a natural number $y$ such that the predicate
    $Q(x,y)$ is true.

\end{enumerate}

\section*{Question 2}
\begin{enumerate}[a.]
    \item

    $\forall x \in P,\:Cat(x) \land Loves(x,x)$

    \item

    $\forall x \in P,\: \exists y \in P,\: Cat(x) \land Cute(y) \land Loves(x,y)$

    \item

    $\exists x \in P,\: Cat(x) \land Cute(x) \Rightarrow \forall y \in P, Cat(y)
    \land Cute(y)$

\end{enumerate}

\section*{Question 3}

\section*{Question 4}

\end{document}