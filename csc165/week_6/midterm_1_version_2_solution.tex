\documentclass[12pt]{article}
\usepackage{enumerate}
\usepackage{amsfonts}
\usepackage{amsmath}
\usepackage{fancyhdr}
\usepackage{amssymb}

\begin{document}
\title{Midterm 1 Version 2 Solution}
\maketitle

\section*{Question 1}
\begin{enumerate}[a.]
    \item

    Since

    $S_1 = \{1,2,3,5,7,11,13,17,19,23,29\}$, and $S_2 = \{1,2,3,5,6,10,15,30\}$,

    $S_1 \cap S_2 = \{1,2,3,5\}$

    \item

    See the table below

    \begin{tabular}{c|c|c|c|c|c}
        $p$ & $q$ & $r$ & $\neg p$ & $\neg p \Leftrightarrow q$ & $(\neg p \Leftrightarrow q) \Rightarrow r$\\
        \hline
        T & T & T & F & F & T\\
        \hline
        T & T & F & F & F & T\\
        \hline
        T & F & T & F & T & T\\
        \hline
        F & T & T & T & T & F\\
        \hline
        T & F & F & F & T & F\\
        \hline
        F & F & T & T & F & T\\
        \hline
        F & F & F & T & F & T
    \end{tabular}

    \bigskip

    \textbf{Correct Solution:}

    \begin{tabular}{c|c|c|c|c|c}
        $p$ & $q$ & $r$ & $\neg p$ & $\neg p \Leftrightarrow q$ & $(\neg p \Leftrightarrow q) \Rightarrow r$\\
        \hline
        T & T & T & F & F & T\\
        \hline
        T & T & F & F & F & T\\
        \hline
        T & F & T & F & T & T\\
        \hline
        *F & *T & *T & *T & *T & *T\\
        \hline
        T & F & F & F & T & F\\
        \hline
        *F & *T & *F & *T & *T & *F\\
        \hline
        F & F & T & T & F & T\\
        \hline
        F & F & F & T & F & T
    \end{tabular}

    $* = \text{Incorrect/missing solution}$

    \item

    Let $x \in \mathbb{N}$. Assume $P(x)$.

    \bigskip

    We will prove that there is a natural number $y$ such that the predicate
    $Q(x,y)$ is true.

    \bigskip

    \textbf{Correct Solution:}

    Let $x \in \mathbb{N}$, and $y = \underline{\hspace{1.5cm}}$. Assume $P(x)$.

    \bigskip

    We will prove that the predicate $Q(x,y)$ is true.

\end{enumerate}

\section*{Question 2}
\begin{enumerate}[a.]
    \item

    $\forall x \in P,\:Cat(x) \land Loves(x,x)$

    \bigskip

    \textbf{Correct Solution:}

    $\forall x \in P,\:Cat(x) \Rightarrow Loves(x,x)$

    \item

    $\forall x \in P,\: \exists y \in P,\: Cat(x) \land Cute(y) \land Loves(x,y)$

    \bigskip

    \textbf{Correct Solution:}

    $\forall x \in P,\: Cat(x) \land Cute(x) \Rightarrow (\forall y \in P, Cat(y)
    \Rightarrow Cute(y))$

    \item

    $\exists x \in P,\: Cat(x) \land Cute(x) \Rightarrow \forall y \in P, Cat(y)
    \land Cute(y)$

    \item

    $\forall p_1,p_2 \in P,\:p_1 \neq p_2 \land Loves(p_1,p_2) \land Loves(p_2,p_1)
    \Rightarrow (Cat(p_1) \land \neg Cat(p_2)) \lor (\neg Cat(p_1) \land Cat(p_2))$

\end{enumerate}

\section*{Question 3}
\begin{enumerate}[a.]
    \item

    $\exists n \in \mathbb{N},\: n > 1 \Rightarrow \forall x \in \mathbb{R},\:
    \lfloor nx \rfloor = n \lfloor x \rfloor$

    \bigskip

    \textbf{Correct Solution:}

    $\exists n \in \mathbb{N},\:n>1 \land (\forall x \in \mathbb{R}^{+},\:
    \lfloor nx \rfloor = n \lfloor x \rfloor)$

    \item

    \textbf{Negation:} $\forall n \in \mathbb{N},\: n > 1 \land (\exists x \in
    \mathbb{R},\: \lfloor nx \rfloor \neq n \lfloor x \rfloor)$

    \bigskip

    Let $n = 2$, $x = 0.5$.

    \bigskip

    Then,

    \begin{align}
        \lfloor nx \rfloor &= \lfloor 2(0.5) \rfloor\\
        &= 1
    \end{align}

    And,

    \begin{align}
        n \lfloor x \rfloor &= 2 \lfloor 0.5 \rfloor\\
        &= 2(0)\\
        &= 0
    \end{align}

    Since $\lfloor nx \rfloor \neq n \lfloor x \rfloor$,the predicate logic is
    false.

    \bigskip

    \textbf{Correction, First Case ($n \leq 1$):}

    \bigskip

    Assume $n \leq 1$. Then, the first case of the negation is true.

    \bigskip

    \textbf{Correction, Second Case ($\exists x \in \mathbb{R}, n \lfloor x \rfloor
    \neq \lfloor nx \rfloor$):}

    \bigskip

    Let $n \in \mathbb{N}$, $x = \frac{1}{n}$. Assume $n > 1$.

    \bigskip

    Then,
    \setcounter{equation}{0}
    \begin{align}
        n \lfloor x \rfloor &= n \lfloor \frac{1}{n} \rfloor\\
        &= 0
    \end{align}

    And,

    \begin{align}
        \lfloor nx \rfloor &= \lfloor \frac{1}{n} \rfloor\\
        &= \lfloor 1 \rfloor\\
        &= 1
    \end{align}

    Since $n\lfloor x \rfloor \neq \lfloor nx \rfloor$, the second case of negation
    is also false.

    \bigskip

    Then, it follows from the negation that the statement is false.

\end{enumerate}

\section*{Question 4}
\begin{itemize}
    \item

    Let $a,b \in \mathbb{N}$. Assume $b \mid a$ and $b \mid (a + 2)$.

    \bigskip

    Then, $\exists k,l \in \mathbb{Z}$,
    \setcounter{equation}{0}
    \begin{align}
        a &= kb\\
        (a + 2) &= lb
    \end{align}

    by the definition of divisibility.

    \bigskip

    Then,

    \begin{align}
        2 &= (l - k)b
    \end{align}

    \bigskip

    Since $(l - k) \in \mathbb{Z}$ and $b \in \mathbb{N}$, the only possible
    combinations that make up 2 are 1 and 2.

    \bigskip

    Then it follows from above that $b = 1$ or $b = 2$.
\end{itemize}

\end{document}