\documentclass[12pt]{article}
\usepackage{enumerate}
\usepackage{amsfonts}
\usepackage{amsmath}
\usepackage{fancyhdr}
\usepackage{amssymb}

\begin{document}
\title{Worksheet 12 Solution}
\maketitle

\section*{Question 1}
\begin{enumerate}[a.]
    \item

    $c,n_0 \in \mathbb{R}^{+},\:\forall n \in \mathbb{N},\: n \geq n_0 \Rightarrow
    g(n) \leq c$, where $g: \mathbb{N} \to \mathbb{R}^{\geq 0}$

    \item

    Let $c = \frac{277}{2}$, $n_0 = 1$, $n \in \mathbb{N}$, $f,g: \mathbb{N} \to
    \mathbb{R}^{\geq 0}$, $g(n) = 100 + \frac{77}{n+1}$, $f(n) = 1$. Assume $n \geq n_0$

    \bigskip

    Then,

    \begin{align}
        g(n) = 100 + \frac{77}{n+1} &\leq 100 + \frac{77}{n+1}\\
        &\leq 100 + \frac{77}{2}\\
        &\leq \frac{277}{2}\\
        &\leq c \cdot 1\\
        &\leq c f(x)
    \end{align}

    \bigskip

    The, it follows from the definition of Big-Oh that the statement $100 +
    \frac{77}{n+1} \in \mathcal{O}(1)$ is true.
\end{enumerate}

\section*{Question 2}
\begin{itemize}
    \item

    \textbf{Expanded Statement:} $f,g:\mathbb{N} \to \mathbb{R}^{\geq 0}$,
    $(\exists c,n_0 \in \mathbb{R}^{+},\:\forall n \in \mathbb{N},\:n \geq n_0
    \Rightarrow g(n) \leq cf(n)) \Rightarrow (\exists d,m_0 \in \mathbb{R}^{+},\:
    \forall m \in \mathbb{N},\: m \geq m_0 \Rightarrow dg(n) \leq f(n))$.

    \bigskip

    Let $f,g: \mathbb{N} \to \mathbb{R}^{\geq 0}$, $n_0 = 1$, $c = \frac{1}{d}$,
    $n \in \mathbb{N}, m_0 = 1$. Assume $n \geq n_0$, $g(n) \leq cf(n)$ and
    $m \geq m_0$.

    \bigskip

    Then,
    \setcounter{equation}{0}
    \begin{align}
        g(n) &\leq cf(n)\\
        g(n) &\leq \frac{1}{d}f(n)\\
        dg(n) &\leq f(n)
    \end{align}

    \bigskip

    Then,

    \begin{align}
        dg(m) &\leq f(m)
    \end{align}

    by changing variable from n to m.

    \bigskip

    Then, it follows from the definition of Omega that the statement
    $,f,g: \mathbb{N} \to \mathbb{R}^{\geq 0}$, $g \in \mathcal{O}(f) \Rightarrow
    \Omega (g)$ is true.

\end{itemize}

\section*{Question 3}
\begin{itemize}
    \item

    Let $g: \mathbb{N} \to \mathbb{R}^{\geq 0}$, $a \in \mathbb{R}^{\geq 0}$,
    $m \in \mathbb{N}$, $c_2 \gg a$, $c_1 = \frac{1}{c_2}$. Assume $g \in \Omega (1)$,
    $m \geq m_0$.

    \bigskip

    Then
    \setcounter{equation}{0}
    \begin{align}
        a + g &\leq a + c_2 g\\
        &< c_2 g
    \end{align}

    and,

    \begin{align}
        a + g &\geq g\\
        &> c_1 g
    \end{align}

    \bigskip

    Then, by the definition of theta, the statement $\forall g: \mathbb{N} \to
    \mathbb{R}^{\geq 0}$, and $a \in \mathbb{R}^{\geq 0}$, $g \in \Omega(1)
    \Rightarrow a + g \in \Theta(g)$ is true.

\end{itemize}

\section*{Question 4}
\begin{enumerate}
    \item

    $g \notin \mathcal{O}(f): \forall c,n_0 \in \mathbb{R}^{+},\:\exists n \in \mathbb{N},\:(
    n \geq n_0) \land (g(n) > cf(n))$

    \item

    Let $c,n_0 \in \mathbb{R}^{+}$, and $n = n_0 + c^{\frac{1}{a-b}}$. Assume $a > b$.

    \bigskip

    \textbf{Note:} Need to ask how $n = n_0 + c^{\frac{1}{a-b}} \in \mathbb{N}$.

    \bigskip

    Then, $n \geq n_0$.

    \bigskip

    And
    \setcounter{equation}{0}
    \begin{align}
        cn^b &< (n_0 + c^{\frac{1}{a-b}})^{a-b}n^b\\
        &< (n_0 + c^{\frac{1}{a-b}})^{a-b}(n_0 + c^{\frac{1}{a-b}})^b\\
        &< (n_0 + c^{\frac{1}{a-b}})^{a-b+b}\\
        &< (n_0 + c^{\frac{1}{a-b}})^a\\
        &< n^a
    \end{align}

    Then, it follows that the statement $\forall a,b \in
    \mathbb{R}^{+}, a > b \Rightarrow n^a \notin \mathcal{O}(n^b)$ is true.

\end{enumerate}

\end{document}