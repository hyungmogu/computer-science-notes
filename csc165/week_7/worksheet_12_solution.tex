\documentclass[12pt]{article}
\usepackage{enumerate}
\usepackage{amsfonts}
\usepackage{amsmath}
\usepackage{fancyhdr}
\usepackage{amssymb}

\begin{document}
\title{Worksheet 12 Solution}
\maketitle

\section*{Question 1}
\begin{enumerate}[a.]
    \item

    $c,n_0 \in \mathbb{R}^{+},\:\forall n \in \mathbb{N},\: n \geq n_0 \Rightarrow
    g(n) \leq c$, where $g: \mathbb{N} \to \mathbb{R}^{\geq 0}$
\end{enumerate}

\section*{Question 2}
\begin{itemize}
    \item

    Let $c = \frac{277}{2}$, $n_0 = 1$, $n \in \mathbb{N}$, $f,g: \mathbb{N} \to
    \mathbb{R}^{\geq 0}$, $g(n) = 100 + \frac{77}{n+1}$, $f(n) = 1$. Assume $n \geq n_0$

    \bigskip

    Then,

    \begin{align}
        g(n) = 100 + \frac{77}{n+1} &\leq 100 + \frac{77}{n+1}\\
        &\leq 100 + \frac{77}{2}\\
        &\leq \frac{277}{2}\\
        &\leq c \cdot 1\\
        &\leq c f(x)
    \end{align}

    \bigskip

    The, it follows from the definition of Big-Oh that the statement $100 +
    \frac{77}{n+1} \in \mathcal{O}(1)$ is true.

\end{itemize}

\section*{Question 3}

\section*{Question 4}

\end{document}