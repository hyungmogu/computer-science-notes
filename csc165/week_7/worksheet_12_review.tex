\documentclass[12pt]{article}
\usepackage{enumerate}
\usepackage{amsfonts}
\usepackage{fancyhdr}
\usepackage{amsmath}
\usepackage{amssymb}
\usepackage{amsthm}

\begin{document}
\title{Worksheet 12 Review}
\maketitle

\section*{Question 1}
\begin{enumerate}[a.]
    \item

    $g \in \mathcal{O}(1):\:\exists c,n_0 \in \mathbb{R}^{+}, \forall n \in
    \mathbb{N}, n \geq n_0 \Rightarrow g(n) \leq c$, where $g: \mathbb{N} \to \mathbb{R}^{\geq 0}$

    \bigskip

    \textbf{Notes:}

    \begin{itemize}
        \item

        $g \in \mathcal{O}(f):\:\exists c,n_o \in \mathbb{R}^{+},\:\forall n \in
        \mathbb{N},\:n \geq n_0 \Rightarrow g(n) \leq cf(n)$, where $f,g:\mathbb{N} \to \mathbb{R}^{\geq 0}$

    \end{itemize}

    \item

    \textbf{Predicate Logic} $\exists c,n_0 \in \mathbb{R}^{+},\:\forall n \in
    \mathbb{N},\:n \geq n_0 \Rightarrow g(n) \leq c$, where $g: \mathbb{N} \to \mathbb{R}^{\geq 0}$

    \bigskip

    \begin{proof}
        Let $n_0 = 1$, $c = 200$ and $g(n) = 100 + \frac{77}{n+1}$. Assume $n \geq n_0$.

        \bigskip

        We will prove the statement by showing

        \begin{align}
            100 + \frac{77}{n+1} &\leq c
        \end{align}

        \bigskip

        It follows from the fact $n_0 \geq 1$ that we can write

        \begin{align}
            100 + \frac{77}{n+1} &\leq 100 + \frac{77}{1+1}\\
            &\leq 100 + \frac{77}{2}\\
            &\leq 100 + 77\\
            &\leq 100 + 100\\
            &\leq 200
        \end{align}

        \bigskip

        Then,

        \begin{align}
            100 + \frac{77}{n+1} &\leq c
        \end{align}

        by the fact that $c = 200$.
    \end{proof}

\end{enumerate}

\section*{Question 2}
\begin{itemize}
    \item

    \textbf{Predicate Logic:} $\forall f,g:\mathbb{N} \to \mathbb{R}^{\geq 0},\:
    (\exists c_0,n_0 \in \mathbb{R}^{+}, \forall n \in \mathbb{N}, \:n \geq
    n_0 \Rightarrow g(n) \leq c_0f(n)) \Rightarrow (\exists d_0,m_0 \in
    \mathbb{R}^{+},\:\forall n \in \mathbb{N},\:n \geq m_0 \Rightarrow f(n) \geq d g(n))$

    \bigskip

    \begin{proof}

    Let $f,g:\mathbb{N} \to \mathbb{R}^{\geq 0}$. Let $c = 2$, $n_0 = 1$ and
    $n \in \mathbb{N}$. Assume $n \geq m_0$. Let $d = \frac{1}{c}$ and $m_0 = n_0$.
    Assume $n \geq m_0$.

    \bigskip

    We will prove that $d_0g(n) \leq f(n)$ given $g(n) \leq c_0f(n))$.

    \bigskip

    It follows from the assumption $g(n) \leq f(n)$ that we can write

    \setcounter{equation}{0}
    \begin{align}
        g(n) &\leq cf(n)\\
        \frac{1}{2} g(n) &\leq f(n)\\
        \frac{1}{2} g(n) &\leq f(n)
    \end{align}

    \bigskip

    Then since $d = \frac{1}{2}$,

    \begin{align}
        d \cdot g(n) &\leq f(n)
    \end{align}
    \end{proof}
\end{itemize}

\section*{Question 3}
\begin{itemize}
    \item
    \textbf{Predicate Logic:} $\forall g:\mathbb{N} \to \mathbb{R}^{\geq 0}$,
    $\forall a \in \mathbb{R}^{\geq 0}$, $(\exists c,n_0 \in \mathbb{R}^{+},\:
    \forall n \in \mathbb{N},\:n \geq n_0 \Rightarrow g(n) \geq c) \Rightarrow
    (\exists c_1,c_2,n_1 \in \mathbb{R}^{+}, \forall n \in \mathbb{N}, n \geq n_1
    \Rightarrow c_1g(n) \leq  a + g(n) \leq c_2g(n))$

    \bigskip

    \begin{proof}
        Let $g:\mathbb{N} \to \mathbb{R}^{\geq 0}$, and $a \in \mathbb{R}^{\geq 0}$.
        Assume $g \in \Omega(1)$, that is there exists $c,n_0 \in \mathbb{R}^{+}$,
        for every $n \in \mathbb{N}$ such that if $n \geq n_0$, $g(n) \geq c$. Let
        $c_1 = \frac{1}{2}$, $c_2 = \left( \frac{a}{c} + 1 \right)$ and
        $n_1 = n_0$. Assume $n \geq n_1$.

        \bigskip

        We will prove $c_1g(n) \leq a + g(n) \leq c_2g(n)$ by diving into two parts,
        first by proving $c_1g(n) \leq a + g(n)$ is true, and then
        second by proving $a + g(n) \leq c_2g(n)$. Then, we will combine the two
        at the end to finish.

        \bigskip

        \textbf{Part 1 ($c_1g(n) \leq a + g(n)$):}

        \bigskip

        It follows from the fact $a \in \mathbb{R}^{+}$ that we can write

        \setcounter{equation}{0}
        \begin{align}
            a + g(n) &\geq g(n)\\
            &\geq \frac{1}{2} \cdot g(n)
        \end{align}

        \bigskip

        Then, because we know $c_1 = \frac{1}{2}$, we can conclude

        \begin{align}
            a + g(n) &\geq c_1 \cdot g(n)
        \end{align}

        \textbf{Part 2 ($a + g(n) \leq c_2g(n)$):}

        \bigskip

        Using the value $c_2 = \left( \frac{a}{c} + 1 \right)$, we can write

        \begin{align}
            c_2g(n) &= \left( \frac{a}{c} + 1 \right) \cdot g(n)\\
            &= \frac{a}{c} \cdot g(n) + g(n)
        \end{align}

        \bigskip

        Then,

        \begin{align}
            c_2g(n) &\geq \frac{a}{c} \cdot c + g(n)
        \end{align}

        by the assumption that $g(n) \geq c$.

        \bigskip

        Then,

        \begin{align}
            c_2g(n) &\geq a + g(n)
        \end{align}

        \bigskip

        Since both $a + g(n) \leq c_2g(n)$ and $c_1g(n) \leq a + g(n)$ are true,
        we can conclude that the inequality $c_1g(n) \leq  a + g(n) \leq c_2g(n)$ is true.

    \end{proof}

    \bigskip

    \textbf{Notes:}

    \begin{itemize}
        \item Noticed professor uses english phrase when expanding assumption.

        \item
        $g \in \Theta(f):\: g \in \mathcal{O}(f) \land g \in \Omega(f)$

        or

        $g \in \Theta(f):\:\exists c_1,c_2,n_1 \in \mathbb{R}^{+}, \forall n \in \mathbb{N}, n \geq n_1
        \Rightarrow c_1g(n) \leq f(n) \leq c_2g(n)$, where $f,g:\:\mathbb{N} \to \mathbb{R}^{\geq 0}$

        \item
        $g \in \Omega(f):\:\exists c,n_o \in \mathbb{R}^{+},\:\forall n \in
        \mathbb{N},\:n \geq n_0 \Rightarrow g(n) \geq cf(n)$, where $f,g:\mathbb{N} \to \mathbb{R}^{\geq 0}$

        \item

        $g \in \mathcal{O}(f):\:\exists c,n_o \in \mathbb{R}^{+},\:\forall n \in
        \mathbb{N},\:n \geq n_0 \Rightarrow g(n) \leq cf(n)$, where $f,g:\mathbb{N} \to \mathbb{R}^{\geq 0}$

    \end{itemize}
\end{itemize}

\section*{Question 4}
\begin{enumerate}[a.]
    \item
    $g \notin \mathcal{O}(f):\:\exists g,f:\mathbb{N} \to \mathbb{R}^{\geq 0},
    \forall c,n_0 \in \mathbb{R}^{+}, \exists n \in \mathbb{N}, (n \geq n_0) \land
    (g(n) > cf(n))$

    \item
    \textbf{Predicate Logic:} $\forall a,b \in \mathbb{R}^{+}, a > b \Rightarrow
    n^a \notin \mathcal{O}(n^b)$

    \bigskip

    \textbf{Expanded Predicate Logic:} $\forall a,b \in \mathbb{R}^{+}, a > b \Rightarrow
    \forall c,n_0 \in \mathbb{R}^{+},\:\exists n \in \mathbb{N},\: (n \geq n_0) \land (n^a > cn^b)$

    \bigskip

    \begin{proof}

    Let $a, \in \mathbb{R}^{+}$. Assume $a > b$. Let $c, n_0 \in \mathbb{R}^{+}$
    and $n = (n_0 + c)$.

    \bigskip

    We will prove the statement $(n \geq n_0) \land (n^a > cn^b)$ in two parts
    one for $(n \geq n_0)$ and the other for $(n^a > cn^b)$. Then, we will
    combine the two parts to finish.

    \bigskip

    \textbf{Part 1} ($n \geq n_0$):

    \bigskip

    Because we know $n_0,c \in \mathbb{R}^{+}$, we can conclude $n_0,c > 0$.

    \bigskip

    Using the fact $n_0,c > 0$, we can calculate

    \setcounter{equation}{0}
    \begin{align}
        n_0 + c &\geq n_0
    \end{align}

    \bigskip

    Then, because we know $n_0 + c = n$, we can conclude

    \begin{align}
        n &\geq n_0
    \end{align}

    \bigskip

    \textbf{Part 2} ($n^a > cn^b$):

    \bigskip

    Since $n = (n_0 + c)$, we can calculate that

    \begin{align}
        cn^b &= c(n_0 + c)^b
    \end{align}

    \bigskip

    Then,

    \begin{align}
        cn^b < (c + n_0)(n_0 + c)^b
    \end{align}

    by the fact $c,n_0 \in \mathbb{R}^{+}$ and $c + n_0 > c$.

    \bigskip

    Then,

    \begin{align}
        cn^b &< (c + n_0)^{a-b}(n_0 + c)^b
    \end{align}

    by the fact $a > b$.

    \bigskip

    Then,

    \begin{align}
        cn^b &< (c + n_0)^{a-b+b}\\
        cn^b &< (c + n_0)^{a}
    \end{align}

    \bigskip

    Then, because we know $c + n_0 = n$, we can conclude

    \begin{align}
        cn^b &< n^a
    \end{align}

    \bigskip

    Since part 1 and part 2 are both true, we can conclude $(n \geq n_0) \land (n^a > cn^b)$ is true.

    \end{proof}

    \bigskip

    \textbf{Notes:}

    \begin{itemize}
        \item
        $\forall x,y \in \mathbb{R}^{+}, x > y \Leftrightarrow \log x > \log y$
    \end{itemize}

\end{enumerate}

\end{document}