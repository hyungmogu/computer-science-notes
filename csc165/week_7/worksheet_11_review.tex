\documentclass[12pt]{article}
\usepackage{enumerate}
\usepackage{amsfonts}
\usepackage{fancyhdr}
\usepackage{amsmath}
\usepackage{amssymb}
\usepackage{amsthm}
\usepackage{mdframed}

\begin{document}
\title{Worksheet 11 Review}
\maketitle

\section*{Question 1}
\begin{enumerate}[a.]
    \item

    $\forall a,b \in \mathbb{R}^{+},\: a \leq b \Rightarrow (\exists c,n_0 \in
    \mathbb{R}^{+},\:\forall n \in \mathbb{N},\:n \geq n_0 \Rightarrow g(n) \leq cf(n))$

    \begin{mdframed}
        \underline{\textbf{Correct Solution:}}

        \bigskip

        $\forall a,b \in \mathbb{R}^{+},\: a \leq b \Rightarrow (\exists c,n_0 \in
        \mathbb{R}^{+},\:\forall n \in \mathbb{N},\:n \geq n_0 \Rightarrow
        \color{red} n^a \color{black} \leq c\color{red} n^b \color{black})$

    \end{mdframed}

    \item

    \begin{proof}
        Let $a,b \in \mathbb{R}^{+}$, $n \in \mathbb{N}$, $c = 1$, and $n_0 = 1$. Assume
        $a \leq b$ and $n > n_0$.

        \bigskip

        We will prove the statement by showing $n^a \leq cn^b$.

        \bigskip

        Because we know $n \geq 1$, we can conclude that

        \begin{align}
            n^a &\leq n^b
        \end{align}

        \bigskip

        Then, it follows from the fact $c = 1$ that

        \begin{align}
            n^a &\leq cn^b
        \end{align}

    \end{proof}

    \bigskip

    \textbf{Notes:}

    \begin{itemize}
        \item Professor used $\forall a,b \in \mathbb{R}^{+},
        a \leq b \Rightarrow n^a \leq n^b$ as a fact given $n \geq 1$.
        \item I don't feel comfortable using the above fact with $a,b \in \mathbb{R}^{+}$.
        \item What facts can be used intuitively?
        \item Given $a \in \mathbb{R}^{+}$, is $1 \leq n \Rightarrow [1]^a \leq n^a$
        also true? Can this be used as a fact?

    \end{itemize}



\end{enumerate}

\section*{Question 2}

\section*{Question 3}

\end{document}