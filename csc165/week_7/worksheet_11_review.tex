\documentclass[12pt]{article}
\usepackage{enumerate}
\usepackage{amsfonts}
\usepackage{fancyhdr}
\usepackage{amsmath}
\usepackage{amssymb}
\usepackage{amsthm}
\usepackage{mdframed}

\begin{document}
\title{Worksheet 11 Review}
\maketitle

\section*{Question 1}
\begin{enumerate}[a.]
    \item

    $\forall a,b \in \mathbb{R}^{+},\: a \leq b \Rightarrow (\exists c,n_0 \in
    \mathbb{R}^{+},\:\forall n \in \mathbb{N},\:n \geq n_0 \Rightarrow g(n) \leq cf(n))$

    \begin{mdframed}
        \underline{\textbf{Correct Solution:}}

        \bigskip

        $\forall a,b \in \mathbb{R}^{+},\: a \leq b \Rightarrow (\exists c,n_0 \in
        \mathbb{R}^{+},\:\forall n \in \mathbb{N},\:n \geq n_0 \Rightarrow
        \color{red} n^a \color{black} \leq c\color{red} n^b \color{black})$

    \end{mdframed}

    \item

    \begin{proof}
        Let $a,b \in \mathbb{R}^{+}$, $n \in \mathbb{N}$, $c = 1$, and $n_0 = 1$. Assume
        $a \leq b$ and $n > n_0$.

        \bigskip

        We will prove the statement by showing $n^a \leq cn^b$.

        \bigskip

        Because we know $n \geq 1$, we can conclude that

        \begin{align}
            n^a &\leq n^b
        \end{align}

        \bigskip

        Then, it follows from the fact $c = 1$ that

        \begin{align}
            n^a &\leq cn^b
        \end{align}

    \end{proof}

    \bigskip

    \begin{mdframed}
        \underline{\textbf{Attempt 2:}}

        \bigskip

        Let $a,b \in \mathbb{R}^{+}$, $n \in \mathbb{N}$, $c = 1$, and $n_0 = 1$. Assume
        $a \leq b$ and $n > n_0$.

        \bigskip

        We will prove the statement by showing $n^a \leq cn^b$.

        \bigskip
        \color{red}
        Because we know $n \geq 1$, we can conclude

        \setcounter{equation}{0}
        \begin{align}
            n^a &\geq 1^a\\
            n^a &\geq 1
        \end{align}

        \bigskip

        Then, because we know $\frac{b}{a} \geq 1$, we can conclude

        \begin{align}
            n^a &\leq \left[ n^a \right]^{\frac{b}{a}}\\
            n^a &\leq n^b
        \end{align}

        \color{black}

        \bigskip

        Then, it follows from the fact $c = 1$ that

        \begin{align}
            n^a &\leq cn^b
        \end{align}

    \end{mdframed}

    \bigskip

    \textbf{Notes:}

    \begin{itemize}
        \item Professor used $\forall a,b \in \mathbb{R}^{+},
        a \leq b \Rightarrow n^a \leq n^b$ as a fact given $n \geq 1$.
        \item I don't feel comfortable using the above fact with $a,b \in \mathbb{R}^{+}$.
        \item What facts can be used intuitively?
        \item Given $a \in \mathbb{R}^{+}$, is $1 \leq n \Rightarrow [1]^a \leq n^a$
        also true? Can this be used in proof as a fact?

    \end{itemize}

\end{enumerate}

\section*{Question 2}
\begin{itemize}
    \item

    \textbf{Predicate Logic:} $\forall a,b \in \mathbb{R}^{+},\:a > 1 \land \:b > 1
    \Rightarrow (\exists c, n_0 \in \mathbb{R}^{+},\:n \geq n_0 \Rightarrow \log_a n \leq \log_b n)$

    \begin{proof}

        Let $a,b \in \mathbb{R}^{+}$, $c = 2 \log_a b$, and $n_0 = 1$. Assume $a > 1$,
        $b > 1$, and $n \geq n_0$.

        \bigskip

        We will prove that given $n_0$ and $c$, $\log_a n \leq c \cdot \log_b n$.

        \bigskip

        It follows from the change of base rule $\log_b n = \frac{\log_a n}{\log_a b}$
        that

        \setcounter{equation}{0}
        \begin{align}
            \log_a n \cdot 1 &= \log_a n \cdot \frac{\log_a b}{\log_a b}\\
            &= \log_b n \cdot \log_a b\\
            &\leq 2 \log_a b \cdot \log_b n
        \end{align}

        \bigskip

        Then, since $c = 2 \cdot \log_a b$,

        \begin{align}
            \log_a n &\leq c \cdot \log_b n
        \end{align}

    \end{proof}

    \bigskip

    \begin{mdframed}
        \underline{\textbf{Attempt 2:}}

        \bigskip

        Let $a,b \in \mathbb{R}^{+}$. \color{red} Assume $a > 1$, $b > 1$.
        Let $c = 2 \log_a b$, and $n_0 = 1$. Assume $n \geq n_0$\color{black}.

        \bigskip

        We will prove that given $n_0$ and $c$, $\log_a n \leq c \cdot \log_b n$.

        \bigskip

        \color{red}
        Change of base rule fact tells us the following
        \setcounter{equation}{0}
        \begin{align}
            \forall a,b \in \mathbb{R}^{+}, \forall n \in \mathbb{N},
            a \neq 1 \land b \neq 1 \Rightarrow \log_b n = \frac{\log_a n}{\log_a b}
        \end{align}

        Using this fact, we can write
        \color{black}

        \setcounter{equation}{0}
        \begin{align}
            \log_a n \cdot 1 &= \log_a n \cdot \frac{\log_a b}{\log_a b}\\
            &= \log_b n \cdot \log_a b\\
            &\leq 2 \log_a b \cdot \log_b n
        \end{align}

        \bigskip

        Then, since $c = 2 \cdot \log_a b$,

        \begin{align}
            \log_a n &\leq c \cdot \log_b n
        \end{align}

    \end{mdframed}

    \bigskip

    \textbf{Notes:}

    \begin{itemize}
        \item Change of base rule

        \begin{align}
            \forall a,b,n \in \mathbb{R}^{+}, a \neq 1 \land b \neq 1 \Rightarrow
            \log_b n = \frac{\log_a n}{\log_a b}
        \end{align}

        \item Noticed professor uses 'Let' and 'Assume' twice to introduce headers
        for the statement and $\log_a n \in \mathcal{O}(\log_b n)$ separately.

        \bigskip

        \begin{mdframed}
        Let $a,b \in \mathbb{R}^{+}$. Assume that $a > 1$ and $b > 1$.
        Let $n_0 = 1$, and let $c = \frac{1}{\log_b a}$. Let $n \in \mathbb{N}$,
        and assume that $n \geq n_0$. We want to show that $\log_a n \leq c \cdot \log_b n$.
        \end{mdframed}

        \item Noticed if $\log_a n = c \cdot \log_b n$ is true, then the following
        is also true

        \begin{enumerate}[1.]
            \item $\log_a n \leq c \cdot \log_b n$
            \item $\log_a n \geq c \cdot \log_b n$
        \end{enumerate}

        \item Noticed professor uses the phrase

        \bigskip

        \begin{mdframed}

        \underline{\hspace{1cm}} fact tells us the following

        \begin{align*}
            \bigl\{ \dots \bigr\}
        \end{align*}

        Using this rule, we can write

        \begin{align*}
            \bigl\{ \dots \bigr\}
        \end{align*}

        \end{mdframed}

        to introduce an external fact to a proof.

        \item

        $g \in \mathcal{O}(f):\:\exists c,n_o \in \mathbb{R}^{+},\:\forall n \in
        \mathbb{N},\:n \geq n_0 \Rightarrow g(n) \leq cf(n)$, where $f,g:\mathbb{N} \to \mathbb{R}^{\geq 0}$

    \end{itemize}

\end{itemize}

\section*{Question 3}
\begin{itemize}
    \item

    \textbf{Predicate Logic:} $\forall f,g: \mathbb{N} \to \mathbb{R}^{\geq 0}$,
    $c_0 = 1$, $n_0 = 1$. Assume $n \geq n_0$, and $g(n) \leq c_0f(n)$. Let $d_0 = c_0 + 1$,
    and $m_0 = n_0$. Assume $m \geq m_0$.

    \bigskip

    \begin{proof}
        Let $f,g: \mathbb{N} \to \mathbb{R}^{\geq 0},\:c_0 = 1, n_0 = 1$. Assume $n \geq n_0$,
        and $g(n) \leq c_0f(n)$. Let $d_0 = c_0 + 1$ and $m_0 = n_0$. Assume $m \geq m_0$.

        \bigskip

        We will prove the statement by starting from the assumption $g(n) \leq c_0f(n)$,
        and show that $(f + g)(n) \leq d_0f(n)$.

        \bigskip

        It follows from the assumption $g(n) \leq c_0f(n)$ that we can write

        \setcounter{equation}{0}
        \begin{align}
            g(n) &\leq c_0f(n)\\
            g(n) + f(n) &\leq c_0f(n) + f(n)\\
            g(n) + f(n) &\leq f(n)(c_0 + 1)
        \end{align}

        \bigskip

        Then, since $d_0 = c_0 + 1$,

        \begin{align}
            f(n) + g(n) &\leq d_0f(n)
        \end{align}

        \bigskip

        The \textbf{sum of $f$ and $g$} fact tells us the following

        \begin{align}
            \forall n \in \mathbb{N}, (f+g)(n) &= f(n) + g(n)
        \end{align}

        \bigskip

        Using this fact, we can write

        \begin{align}
            (f + g)(n) &\leq d_0f(n)
        \end{align}
    \end{proof}

\end{itemize}

\end{document}