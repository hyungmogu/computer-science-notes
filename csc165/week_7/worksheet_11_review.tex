\documentclass[12pt]{article}
\usepackage{enumerate}
\usepackage{amsfonts}
\usepackage{fancyhdr}
\usepackage{amsmath}
\usepackage{amssymb}
\usepackage{amsthm}
\usepackage{mdframed}

\begin{document}
\title{Worksheet 11 Review}
\maketitle

\section*{Question 1}
\begin{enumerate}[a.]
    \item

    $\forall a,b \in \mathbb{R}^{+},\: a \leq b \Rightarrow (\exists c,n_0 \in
    \mathbb{R}^{+},\:\forall n \in \mathbb{N},\:n \geq n_0 \Rightarrow g(n) \leq cf(n))$

    \begin{mdframed}
        \underline{\textbf{Correct Solution:}}

        \bigskip

        $\forall a,b \in \mathbb{R}^{+},\: a \leq b \Rightarrow (\exists c,n_0 \in
        \mathbb{R}^{+},\:\forall n \in \mathbb{N},\:n \geq n_0 \Rightarrow
        \color{red} n^a \color{black} \leq c\color{red} n^b \color{black})$

    \end{mdframed}

    \item

    \begin{proof}
        Let $a,b \in \mathbb{R}^{+}$, $n \in \mathbb{N}$, $c = 1$, and $n_0 = 1$. Assume
        $a \leq b$ and $n > n_0$.

        \bigskip

        We will prove the statement by showing $n^a \leq cn^b$.

        \bigskip

        Because we know $n \geq 1$, we can conclude that

        \begin{align}
            n^a &\leq n^b
        \end{align}

        \bigskip

        Then, it follows from the fact $c = 1$ that

        \begin{align}
            n^a &\leq cn^b
        \end{align}

    \end{proof}

    \bigskip

    \begin{mdframed}
        \underline{\textbf{Attempt 2:}}

        \bigskip

        Let $a,b \in \mathbb{R}^{+}$, $n \in \mathbb{N}$, $c = 1$, and $n_0 = 1$. Assume
        $a \leq b$ and $n > n_0$.

        \bigskip

        We will prove the statement by showing $n^a \leq cn^b$.

        \bigskip
        \color{red}
        Because we know $n \geq 1$, we can conclude

        \setcounter{equation}{0}
        \begin{align}
            n^a &\geq 1^a\\
            n^a &\geq 1
        \end{align}

        \bigskip

        Then, because we know $\frac{b}{a} \geq 1$, we can conclude

        \begin{align}
            n^a &\leq \left[ n^a \right]^{\frac{b}{a}}\\
            n^a &\leq n^b
        \end{align}

        \color{black}

        \bigskip

        Then, it follows from the fact $c = 1$ that

        \begin{align}
            n^a &\leq cn^b
        \end{align}

    \end{mdframed}

    \bigskip

    \textbf{Notes:}

    \begin{itemize}
        \item Professor used $\forall a,b \in \mathbb{R}^{+},
        a \leq b \Rightarrow n^a \leq n^b$ as a fact given $n \geq 1$.
        \item I don't feel comfortable using the above fact with $a,b \in \mathbb{R}^{+}$.
        \item What facts can be used intuitively?
        \item Given $a \in \mathbb{R}^{+}$, is $1 \leq n \Rightarrow [1]^a \leq n^a$
        also true? Can this be used in proof as a fact?

    \end{itemize}

    \item

    \textbf{Predicate Logic:} $\forall a,b \in \mathbb{R}^{+},\:a > 1 \land \:b > 1
    \Rightarrow (\exists c, n_0 \in \mathbb{R}^{+},\:n \geq n_0 \Rightarrow \log_a n \leq \log_b n)$

    \begin{proof}

        Let $a,b \in \mathbb{R}^{+}$, $c = 2 \log_a b$, and $n_0 = 1$. Assume $a > 1$,
        $b > 1$, and $n \geq n_0$.

        \bigskip

        We will prove that given $n_0$ and $c$, $\log_a n \leq c \cdot \log_b n$.

        \bigskip

        It follows from the change of base rule $\log_b n = \frac{\log_a n}{\log_a b}$
        that

        \setcounter{equation}{0}
        \begin{align}
            \log_a n \cdot 1 &= \log_a n \cdot \frac{\log_a b}{\log_a b}\\
            &= \log_b n \cdot \log_a b\\
            &\leq 2 \log_a b \cdot \log_b n
        \end{align}

        \bigskip

        Then, since $c = 2 \cdot \log_a b$,

        \begin{align}
            \log_a n &\leq c \cdot \log_b n
        \end{align}

    \end{proof}

    \textbf{Notes:}

    \begin{itemize}
        \item Change of base rule

        \begin{align}
            \log_b x &= \frac{\log_a x}{\log_a b}
        \end{align}

    \end{itemize}

\end{enumerate}

\section*{Question 2}

\section*{Question 3}

\end{document}