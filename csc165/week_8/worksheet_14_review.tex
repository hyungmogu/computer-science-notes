\documentclass[12pt]{article}
\usepackage{enumerate}
\usepackage{amsfonts}
\usepackage{amsmath}
\usepackage{fancyhdr}
\usepackage{amssymb}
\usepackage{listings}

\begin{document}
\title{Worksheet 14 Review}
\maketitle

\section*{Question 1}
\begin{enumerate}[a.]
    \item

    Since the inner loop starts at $j = 0$ and finishes at $j = n-1$ with $j$
    increasing by 1 per iteration, we can conclude that the inner loop has

    \begin{align}
        \lceil n - 1 - 0 + 1 \rceil &= n
    \end{align}

    iterations.

    \bigskip

    Since the inner loop takes 1 step per iteration, we can conclude that the
    inner loop has the total cost of

    \begin{align}
        n \cdot 1 &= n
    \end{align}

    steps.

    \bigskip

    For the outer loop, because it starts at $i = 0$ and ends at $i = n-1$ with $i$
    increasing by 5 per iteration, we can conclude that the outer loop has

    \begin{align}
        \left\lceil \frac{n-1-0+1}{5} \right\rceil = \left\lceil \frac{n}{5} \right\rceil
    \end{align}

    iterations.

    \bigskip

    Since each iteration in the outer loop takes $n$ steps, we can conclude the
    outer loop has the total cost of

    \begin{align}
        n \cdot n &= n^2
    \end{align}

    steps.

    \bigskip

    Since we are ignoring the cost of the loop variables, the total cost of the
    algorithm is $n^2 + n$ steps.

    \bigskip

    Then, because we know the algorithm takes total of $n^2 + n$ steps,
    we can conclude the algorithm has the runtime of $\Theta(n^2)$.

\end{enumerate}

\section*{Question 2}

\end{document}