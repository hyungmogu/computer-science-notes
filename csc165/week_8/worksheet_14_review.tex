\documentclass[12pt]{article}
\usepackage{enumerate}
\usepackage{amsfonts}
\usepackage{amsmath}
\usepackage{fancyhdr}
\usepackage{amssymb}
\usepackage{listings}
\usepackage{xcolor}

\definecolor{codegreen}{rgb}{0,0.6,0}
\definecolor{codegray}{rgb}{0.5,0.5,0.5}
\definecolor{codepurple}{rgb}{0.58,0,0.82}
\definecolor{backcolour}{rgb}{0.95,0.95,0.92}

\lstdefinestyle{mystyle}{
    backgroundcolor=\color{backcolour},
    commentstyle=\color{codegreen},
    keywordstyle=\color{magenta},
    numberstyle=\tiny\color{codegray},
    stringstyle=\color{codepurple},
    basicstyle=\ttfamily\footnotesize,
    breakatwhitespace=false,
    breaklines=true,
    captionpos=b,
    keepspaces=true,
    numbers=left,
    numbersep=5pt,
    showspaces=false,
    showstringspaces=false,
    showtabs=false,
    tabsize=2
}

\lstset{style=mystyle}

\begin{document}
\title{Worksheet 14 Review}
\maketitle

\section*{Question 1}
\begin{enumerate}[a.]
    \item

    Since the inner loop starts at $j = 0$ and finishes at $j = n-1$ with $j$
    increasing by 1 per iteration, we can conclude that the inner loop has

    \begin{align}
        \lceil n - 1 - 0 + 1 \rceil &= n
    \end{align}

    iterations.

    \bigskip

    Since the inner loop takes 1 step per iteration, we can conclude that the
    inner loop has the total cost of

    \begin{align}
        n \cdot 1 &= n
    \end{align}

    steps.

    \bigskip

    For the outer loop, because it starts at $i = 0$ and ends at $i = n-1$ with $i$
    increasing by 5 per iteration, we can conclude that the outer loop has

    \begin{align}
        \left\lceil \frac{n-1-0+1}{5} \right\rceil = \left\lceil \frac{n}{5} \right\rceil
    \end{align}

    iterations.

    \bigskip

    Since each iteration in the outer loop takes $n$ steps, we can conclude the
    outer loop has the total cost of

    \begin{align}
        n \cdot n &= n^2
    \end{align}

    steps.

    \bigskip

    Since we are ignoring the cost of the loop variables, the total cost of the
    algorithm is $n^2 + n$ steps.

    \bigskip

    Then, because we know the algorithm takes total of $n^2 + n$ steps,
    we can conclude the algorithm has the runtime of $\Theta(n^2)$.

    \item

    We will determine the exact cost and theta of this algorithm by first
    calculating the exact cost of inner loop 1

    \begin{lstlisting}[language=Python]
    j = 1
    while j < n:
        j = j * 3
    \end{lstlisting}

    and then, calculating the exact cost of inner loop 2

    \begin{lstlisting}[language=Python]
    k = 0
    while k < n:
        k = k + 2
    \end{lstlisting}

    and then, calculating the exact cost of the outer loop using the information
    from the exact cost of inner loop 1 and inner loop 2

    \begin{lstlisting}[language=Python]
    i = 4
    while i < n:
        j = 1
        while j < n:
            j = j * 3
        k = 0
        while k < n:
            k = k + 2
        i = i + 1
    \end{lstlisting}

    and then, we will finish off by calculating the theta of the outer loop.

    \bigskip

    \textbf{Part 1 (Calculating the exact cost of loop 1):}

    \bigskip

    Because we kow $j = j \cdot 3$, we can calculate

    \begin{align*}
        i_1 &= 3\\
        i_2 &= 9\\
        i_3 &= 27\\
        &\vdots\\
        i_j &= 3^j
    \end{align*}

    \bigskip

    Then, using the fact that loop termination occurs when $i_j \geq n$, we can
    conclude

    \setcounter{equation}{0}
    \begin{align}
        3^j &\geq n\\
        j &\geq \log_3 n
    \end{align}

    \bigskip

    Since we are looking for the smallest value of $j$ resulting in loop termination,
    we can conclude the value of $j$ is $\lceil \log_3 n \rceil$.

    \bigskip

    Since the inner loop 1 takes constant step per iteration, we can conclude that
    the loop has exact cost of

    \begin{align}
        \lceil \log_3 n \rceil \cdot 1 &= \lceil \log_3 n \rceil
    \end{align}

    steps.

    \bigskip

    \textbf{Part 2 (Calculating the exact cost of loop 2):}

    \bigskip

    \textbf{Part 3 (Calculating the exact cost of outer loop):}

    \bigskip

    \textbf{Part 4 (Calculating Theta):}

\end{enumerate}

\section*{Question 2}

\end{document}