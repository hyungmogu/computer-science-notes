\documentclass[12pt]{article}
\usepackage{enumerate}
\usepackage{amsfonts}
\usepackage{fancyhdr}
\usepackage{amsmath}
\usepackage{amssymb}
\usepackage{amsthm}
\usepackage{mdframed}

\begin{document}
\title{Worksheet 13 Review}
\maketitle

\section*{Question 1}
\begin{enumerate}[a.]
    \item

    Since the loop starts from $i = 0$ and ends at $i = n - 1$. The loop has

    \begin{align}
        n - 1 - 0 + 1 &= n
    \end{align}

    iterations.

    \bigskip

    Since each iteration runs 5 steps, the loop has total cost of

    \begin{align}
        5 \cdot n &= 5n
    \end{align}

    steps.

    \bigskip

    Because we know $i = 0$ at line 2 has cost of 1, we can conclude that the
    algorithm has total cost of $5n + 1$ steps.

    \bigskip

    \begin{mdframed}
        \underline{\textbf{Correct Solution:}}

        \bigskip

        \color{red}
        Because we know the loop starts from $i = 0$ and ends at
        $i = n - 1$ with $i$ increasing by 5 per iteration, we can conclude the
        loop has

        \begin{align}
            \left\lceil \frac{n}{5} \right\rceil
        \end{align}

        \color{black}

        iterations.

        \bigskip

        \color{red}
        Since each iteration takes constant time, the loop has
        runtime of $\Theta(n)$
        \color{black}

    \end{mdframed}

    \bigskip

    \textbf{Notes:}

    \begin{itemize}
        \item How does professor begin a proof after 'We will prove that...' or
        at the beginning of each case/parts?

        \item Noticed professor doesn't provide a detailed explanation for the
        number of iterations.

        \item Realized the goal of this problem is to determine the exact cost and runtime
        of each loop.

        \bigskip

        \begin{mdframed}
            There are $\left\lceil \frac{n}{5} \right\rceil$ iterations. $\cdots$
        \end{mdframed}

    \end{itemize}


    \item

    Because we know the loop starts at $i = 4$ and ends at $i = n - 1$ with $i$
    increasing by 1 per iteration, we can conclude that the loop has

    \setcounter{equation}{0}
    \begin{align}
        \lceil n - 1 4 + 1 \rceil &= n - 4
    \end{align}

    iterations.

    \bigskip

    Since each iteration takes a constant time, we can conclude that the loop has
    runtime of $\mathcal{O}(n)$.

    \bigskip

    \begin{mdframed}
        \underline{\textbf{Correct Solution:}}

        \bigskip

        Because we know the loop starts at $i = 4$ and ends at $i = n - 1$ with $i$
        increasing by 1 per iteration, we can conclude that the loop has
        \color{red}\textbf{at most}\color{black}

        \setcounter{equation}{0}
        \begin{align}
            \lceil n - 1 4 + 1 \rceil &= n - 4
        \end{align}

        iterations.

        \bigskip

        Since each iteration takes a constant time, we can conclude that the loop has
        runtime of \color{red}$\Theta(n)$\color{black}.

    \end{mdframed}

    \bigskip

    \textbf{Notes:}

    \begin{itemize}
        \item Noticed professor doesn't count $i = 4$ or $i = 0$ in question 1.a
        to the total cost of algorithm. But in later parts of the question, the two
        are considered. I wonder if the line with constant runtime
        should always be accounted for, or if it can be ignored in certain circumstances.
        If latter, when can the constants be ignored?

    \end{itemize}

    \item

    Since the loop starts at $i = 0$ and ends at $i = n-1$ with $i$ increasing
    by $\frac{n}{10}$ per iteration, we can conclude that the loop has at most

    \setcounter{equation}{0}
    \begin{align}
        \left\lceil \frac{(n-1-0+1)}{\frac{n}{10}} \right\rceil &= \left\lceil \frac{n \cdot 10}{n} \right\rceil\\
        &= 10
    \end{align}

    iterations.

    \bigskip

    Since each iteration takes constant time, we can conclude that the loop has
    runtime of $\Theta(1)$.

    \item

    For the case where $n^2 \leq 20$, it's omitted because we are assuming the
    value of $n$ is asymptotically large.

    \bigskip

    For the case where $n^2 > 20$, because we know the loop runs from $i = 20$ to
    $i = n^2 - 1$ with $i$ increasing by 3 per iteration, we can conclude that the
    loop has at most

    \setcounter{equation}{0}
    \begin{align}
        \left\lceil \frac{n^2 - 1 - 20 + 1}{3} \right\rceil &= \left\lceil \frac{n^2 - 20}{3} \right\rceil
    \end{align}

    iterations.

    \bigskip

    Since the loop takes constant time per iteration, the loop has total runtime
    of $\Theta(n^2)$.

    \item

    Since we are considering asymptotic runtime or the case where $n$ is very large,
    the case where $n^2 \leq 20$ will be omitted.

    \bigskip

    For the case where $n^2 > 20$, because we know the first loop runs from $i = 20$
    and ends at $i = n^2 - 1$ with $i$ increasing by 3 per iteration, we can conclude
    that the first loop runs

    \setcounter{equation}{0}
    \begin{align}
        \left\lceil \frac{n^2 -1 - 20 + 1}{3} \right\rceil &= \left\lceil \frac{n^2 - 20}{3} \right\rceil
    \end{align}

    iterations

    \bigskip

    For the second loop, because we know the loop runs from $j = 0$ to $j = n-1$
    with $j$ increasing by 0.01 per iteration, we can conclude that the second loop
    runs at most

    \begin{align}
        \left\lceil \frac{n - 1 - 0 + 1}{\frac{1}{100}} \right\rceil &= \left\lceil 100 \cdot n \right\rceil\\
        &= 100 \cdot n
    \end{align}

    iterations

    \bigskip

    Since each iteration takes a constant time in both of the loops, the total
    runtime of the loops in this algorithm is

    \begin{align}
        \Theta \left( \left\lceil \frac{n^2 - 20}{3} \right\rceil + 100 \cdot n \right) &= \Theta(n^2)
    \end{align}

    \begin{mdframed}
        \underline{\textbf{Correct Solution:}}

        \bigskip

        For the case where $n^2 \leq 20$, it's omitted because we are assuming the
        value of $n$ is asymptotically large.

        \bigskip

        For the case where $n^2 > 20$, because we know the loop runs from $i = 20$ to
        $i = n^2 - 1$ with $i$ increasing by 3 per iteration, we can conclude that the
        loop has at most

        \setcounter{equation}{0}
        \begin{align}
            \left\lceil \frac{n^2 - 1 - 20 + 1}{3} \right\rceil &= \left\lceil \frac{n^2 - 20}{3} \right\rceil
        \end{align}

        iterations.

        \bigskip

        Since the loop takes constant time per iteration, the loop has total runtime
        of $\Theta(n^2)$.

        \item

        Since we are considering asymptotic runtime or the case where $n$ is very large,
        the case where $n^2 \leq 20$ will be omitted.

        \bigskip

        For the case where $n^2 > 20$, because we know the first loop runs from $i = 20$
        and ends at $i = n^2 - 1$ with $i$ increasing by 3 per iteration, we can conclude
        that the first loop runs

        \setcounter{equation}{0}
        \begin{align}
            \left\lceil \frac{n^2 -1 - 20 + 1}{3} \right\rceil &= \left\lceil \frac{n^2 - 20}{3} \right\rceil
        \end{align}

        iterations

        \bigskip

        \color{red}
        Since the first loop takes a constant time per iteration, we can conclude
        the first loop has runtime of $\Theta(n^2)$.
        \color{black}

        \bigskip

        For the second loop, because we know the loop runs from $j = 0$ to $j = n-1$
        with $j$ increasing by 0.01 per iteration, we can conclude that the second loop
        runs at most

        \begin{align}
            \left\lceil \frac{n - 1 - 0 + 1}{\frac{1}{100}} \right\rceil &= \left\lceil 100 \cdot n \right\rceil\\
            &= 100 \cdot n
        \end{align}

        iterations

        \bigskip

        \color{red}
        Since the first loop takes a constant time per iteration, we can conclude
        the second loop has runtime of $\Theta(n)$.
        \color{black}

        \bigskip

        \color{red}
        Since $n \in \Theta(n^2)$, the total runtime of algorithm is $\Theta(n^2)$.
        \color{black}

    \end{mdframed}

    \bigskip

    \textbf{Notes:}

    \begin{itemize}
        \item Noticed professor computes the theta of each loop, and then compare
        to choose the biggest theta. Professor calls it \textbf{one version of
        the “sum” Big-Oh/Omega/Theta theorem.}

        \begin{itemize}
            \item Ah. this is also how $\in$ in $n \in \Theta(n)$ is used.
        \end{itemize}

    \end{itemize}

\end{enumerate}

\section*{Question 2}
\begin{enumerate}[a.]
    \item

    It follows from the fact $i_k = i \cdot 2$ that we can conclude

    \begin{align*}
    i_3 &= 8\\
    i_4 &= 16\\
    i_k &= 2^k
    \end{align*}

    \item

    Using the fact that loop termination occurs when $i_k \geq n$, we can
    calculate

    \setcounter{equation}{0}
    \begin{align}
        2^k &\geq n\\
        \log 2^k &\geq \log n\\
        k &\geq \log n
    \end{align}

    \bigskip

    Since the loop terminates when $k$ is greater than or equal to $\log n$, we
    can conclude that the loop has $\lceil \log n \rceil$ iterations.

    \item

    Since the loop runs $\lceil \log n \rceil$ iterations, and constant time is
    taken per iteration, we can conclude that the algorithm has runtime of
    $\Theta(\log n)$.

    \item

    We did not initialize $i = 0$ to prevent the loop from running indefinitely.

\end{enumerate}

\section*{Question 3}
\begin{itemize}
    \item

    \begin{proof}
        Let $k \in \mathbb{Z}^{+}$.

        \bigskip

        Using the fact $i = i \cdot i$, we can calculate

        \setcounter{equation}{0}
        \begin{align}
            i_1 &= 4 = 2^2\\
            i_2 &= 16 = 2^4\\
            i_3 &= 256 = 2^8\\
            i_k &= 2^{2^k}
        \end{align}

        \bigskip

        Since the loop terminates when $i_k \geq n$, we can conclude

        \begin{align}
            2^{2^k} &\geq n\\
            \log 2^{2^k} &\geq \log n\\
            2^k &\geq \log n\\
            \log 2^k &\geq \log \log n\\
            k &\geq \log \log n
        \end{align}

        \bigskip

        Since we are looking for the smallest $k$ when the loop terminates, the smallest
        $k$ possible is $\lceil \log \log n \rceil$.

        \bigskip

        Since the loop takes constant time per iteration, we can conclude the
        runtime of the algorithm is $\Theta(\log \log n)$
    \end{proof}

\end{itemize}

\end{document}