\documentclass[12pt]{article}
\usepackage{enumerate}
\usepackage{amsfonts}
\usepackage{fancyhdr}
\usepackage{amsmath}
\usepackage{amssymb}
\usepackage{amsthm}
\usepackage{mdframed}

\begin{document}
\title{Worksheet 13 Review}
\maketitle

\section*{Question 1}
\begin{enumerate}[a.]
    \item

    Since the loop starts from $i = 0$ and ends at $i = n - 1$. The loop has

    \begin{align}
        n - 1 - 0 + 1 &= n
    \end{align}

    iterations.

    \bigskip

    Since each iteration runs 5 steps, the loop has total cost of

    \begin{align}
        5 \cdot n &= 5n
    \end{align}

    steps.

    \bigskip

    Because we know $i = 0$ at line 2 has cost of 1, we can conclude that the
    algorithm has total cost of $5n + 1$ steps.

    \bigskip

    \begin{mdframed}
        \underline{\textbf{Correct Solution:}}

        \bigskip

        \color{red}
        Because we know the loop starts from $i = 0$ and ends at
        $i = n - 1$ with $i$ increasing by 5 per iteration, we can conclude the
        loop has

        \begin{align}
            \left\lceil \frac{n}{5} \right\rceil
        \end{align}

        \color{black}

        iterations.

        \bigskip

        \color{red}
        Since each iteration takes constant time, the loop has
        runtime of $\Theta(n)$
        \color{black}

    \end{mdframed}

    \bigskip

    \textbf{Notes:}

    \begin{itemize}
        \item How does professor begin a proof after 'We will prove that...' or
        at the beginning of each case/parts?

        \item Noticed professor doesn't provide a detailed explanation for the
        number of iterations.

        \item Realized the goal of this problem is to determine the exact cost and runtime
        of each loop.

        \bigskip

        \begin{mdframed}
            There are $\left\lceil \frac{n}{5} \right\rceil$ iterations. $\cdots$
        \end{mdframed}

    \end{itemize}

    \item

    Because we know the loop starts at $i = 4$ and ends at $i = n - 1$ with $i$
    increasing by 1 per iteration, we can conclude that the loop has

    \setcounter{equation}{0}
    \begin{align}
        \lceil n - 1 4 + 1 \rceil &= n - 4
    \end{align}

    iterations.

    \bigskip

    Since each iteration takes a constant time, we can conclude that the loop has
    runtime of $\mathcal{O}(n)$.

\end{enumerate}

\section*{Question 2}

\section*{Question 3}

\section*{Question 4}

\end{document}