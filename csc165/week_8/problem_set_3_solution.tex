\documentclass[12pt]{article}
\usepackage{enumerate}
\usepackage{amsfonts}
\usepackage{amsmath}
\usepackage{fancyhdr}
\usepackage{amssymb}

\begin{document}
\title{Problem Set 3 Solution}
\maketitle

\section*{Question 1}
\begin{enumerate}
    \item

    Let $x \in \mathbb{R}$.

    \bigskip

    \textbf{Base Case (n = 0):}

    \bigskip

    Let $n = 0$.

    \bigskip

    Then,

    \begin{align}
        a_0 &= 0
    \end{align}

    \bigskip

    Then it follows from above that the base case holds.

    \bigskip

    \textbf{Inductive Case ($n > 0$):}

    \bigskip

    Let $k \in \mathbb{N}$, and assume $a_n = x\prod\limits_{i=0}^{n-1} a_i$.

    \bigskip

    Then,
    \setcounter{equation}{0}
    \begin{align}
        x\prod\limits_{i=0}^{n-1} a_i \cdot a_n &= x\prod\limits_{i=0}^{n} a_i\\
        &= a_{n+1}
    \end{align}

    \bigskip

    Then it follows from above that the recursive sequence of numbers is true for
    all natural numbers.

    \item

    From the following table

    \begin{tabular}{c|c|c|c}
        String Length & Number of Even (Digit Sum) & Number of Odd (Digit Sum) & Total\\
        \hline
        1 & 2 & 1 & 3\\
        \hline
        2 & 5 & 4 & 9\\
        \hline
        3 & 14 & 13 & 27
    \end{tabular}

    \bigskip

    we see that $E_n = \frac{3^n + 1}{2}$ and $O_n = \frac{3^n - 1}{2}$.

    \bigskip

    As well, we see that the number of new elements in $E_{n+1}$ is $3^n$.

    \bigskip

    Now, we will prove that $E_n$ and $O_n$ are true for all natural numbers
    using the induction hypothesis.

    \bigskip

    \textbf{Base Case (n = 1):}

    \bigskip

    Let $n = 1$.

    \bigskip

    Then, $E_n = \frac{4}{2} = 2$ and $O_n = \frac{2}{2} = 1$.

    \bigskip

    Since the result matches to data in table, the base case holds.

    \bigskip

    \textbf{Inductive Case:}

    Let $n \in \mathbb{N}$. Assume $E_n = \frac{3^n + 1}{2}$ and $O_n = \frac{3^n - 1}{2}$.

    \bigskip

    Then,
    \setcounter{equation}{0}
    \begin{align}
        E_{n+1} &= \frac{3^n + 1}{2} + 3^n\\
        &= \frac{3^n + 1}{2} + \frac{2 \cdot 3^n}{2}\\
        &= \frac{3 \cdot 3^n + 1}{2}\\
        &= \frac{3^{n+1} + 1}{2}
    \end{align}

    \bigskip

    Then, it follows from above that the inductive step for $E_n$ holds.

    \bigskip

    Similarly, for $O_n$,

    \begin{align}
        O_{n+1} &= \frac{3^n - 1}{2} + 3^n\\
        &= \frac{3^n - 1}{2} + \frac{2 \cdot 3^n}{2}\\
        &= \frac{3 \cdot 3^n - 1}{2}\\
        &= \frac{3^{n+1} - 1}{2}
    \end{align}

    \bigskip

    Then, it follows from above that the inductive step for $O_n$ holds.

    \bigskip

    Then, it follows from the definition of induction hypothesis that the value of
    $E_n$ and $O_n$ are true for all $n$.

\end{enumerate}

\section*{Question 2}
\begin{enumerate}[a.]
    \item

    Since first 1 repeats every $4i - 1$ times and the second 1 repeats every $4i$ times,

    \setcounter{equation}{0}
    \begin{align}
        (0.\overline{0011})_2 &= \sum\limits_{i=1}^{\frac{n}{4}} \left( \frac{1}{2} \right)^{4i} + \sum\limits_{i=1}^{\frac{n}{4}} \left( \frac{1}{2} \right)^{4i-1}\\
        &= \sum\limits_{i=1}^{\frac{n}{4}} \left( \frac{1}{16} \right)^{i} + 2 \cdot \sum\limits_{i=1}^{\frac{n}{4}} \left( \frac{1}{16} \right)^{i}\\
        &= \frac{1}{16} \cdot \sum\limits_{i=0}^{\frac{n}{4} - 1} \left( \frac{1}{16} \right)^{i} + \sum\limits_{i=0}^{\frac{n}{4}-1} \left( \frac{1}{16} \right)^{i}\\
        &= \frac{3}{16} \cdot \sum\limits_{i=0}^{\frac{n}{4} - 1} \left( \frac{1}{16} \right)^{i}
    \end{align}

    \bigskip

    Then,

    \begin{align}
        \frac{3}{16} \cdot \sum\limits_{i=0}^{\frac{n}{4} - 1} \left( \frac{1}{16} \right)^{i} &= \frac{3}{16} \cdot \left(\frac{1 - \frac{1}{16}^{\frac{n}{4}}}{1 - (\frac{1}{16})} \right)
    \end{align}

    by using the formula $\forall n \in \mathbb{Z}^{+} and r \in \mathbb{R},\: r \neq 1 \Rightarrow \sum\limits_{i=0}^{n-1} r^i = \frac{1-r^n}{1 - r}$.

    \bigskip

    Then,

    \begin{align}
        \frac{3}{16} \cdot \left(\frac{1 - \frac{1}{16}^{\frac{n}{4}}}{1 - (\frac{1}{16})} \right) &= (\frac{1 - \frac{1}{2}^n}{\frac{15}{16}})\\
        &= \frac{1}{5} \cdot \left( 1 - \frac{1}{2}^n \right)\\
        &= \frac{1}{5} \cdot \left( \frac{2^n - 1}{2^n} \right)
    \end{align}

    \bigskip

    Then,

    \begin{align}
        0.2 - \frac{1}{5} \cdot \left( \frac{2^n - 1}{2^n} \right) &= \frac{1}{5} - \frac{1}{5} \cdot \left( \frac{2^n - 1}{2^n} \right)\\
        &= \frac{2^n}{5 \cdot 2^n} - \frac{1}{5} \cdot \left( \frac{2^n - 1}{2^n} \right)\\
        &= \frac{1}{5 \cdot 2^n}
    \end{align}

    Then, it follows from above that $\forall n \in \mathbb{Z}^{+},\: 4 \mid n
    \Rightarrow \frac{1}{5 \cdot 2^n}$

    \item

    Let $n \in \mathbb{Z}^{+}$, and $x \in \{x \mid x \in \mathbb{R}^{+}, 0
    \leq x < 1\}$.

    \bigskip

    We will prove that the statement $\forall n \in \mathbb{Z}^{+}$, $\forall x
    \in S$, $\exists x_1 \in S$, $FB(n, x_1)\land 0 \leq x - x_1 < 1$ is true
    using induction hypothesis.

    \bigskip

    Let $n = 1$.

    \textbf{Case 1 ($0 \leq x < 0.5$, from ${x \mid x \in \mathbb{R}, 0 \leq x < 1}$):}

    \bigskip

    Let $x_1 = 0$.

    \bigskip

    Then,
    \setcounter{equation}{0}
    \begin{align}
        0 &= (0.0)_2\\
        &= \sum\limits_{i=1}^1 \frac{b_i}{2}
    \end{align}

    by the fact that $b_i = 0$.

    \bigskip

    Then, it follows from above that $FB(1,x_1)$ is true.

    \bigskip

    Now we will prove that $0 \leq x - x_1 < \frac{1}{2}$ is true.

    \bigskip

    Let $x_1 = 0$. Assume $0 \leq x < 0.5$.

    \bigskip

    Then,

    \begin{align}
        0 &\leq x < 0.5\\
        0 - x_1 &\leq x - x_1 < \frac{1}{2} - x_1\\
        0 &\leq x - x_1 < \frac{1}{2}
    \end{align}

    \bigskip

    Then, it follows from above that $FB(n,x_1) \land 0 \leq x - x_1 < \frac{1}{2}$
    hold for the base case with $0 \leq x < 0.5$.


\end{enumerate}

\section*{Question 3}

\section*{Question 4}

\end{document}