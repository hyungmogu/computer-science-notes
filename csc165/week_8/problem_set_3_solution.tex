\documentclass[12pt]{article}
\usepackage{enumerate}
\usepackage{amsfonts}
\usepackage{amsmath}
\usepackage{fancyhdr}
\usepackage{amssymb}

\begin{document}
\title{Problem Set 3 Solution}
\maketitle

\section*{Question 1}
\begin{enumerate}
    \item

    Let $x \in \mathbb{R}$.

    \bigskip

    \textbf{Base Case (n = 0):}

    \bigskip

    Let $n = 0$.

    \bigskip

    Then,

    \begin{align}
        a_0 &= 0
    \end{align}

    \bigskip

    Then it follows from above that the base case holds.

    \bigskip

    \textbf{Inductive Case ($n > 0$):}

    \bigskip

    Let $k \in \mathbb{N}$, and assume $a_n = x\prod\limits_{i=0}^{n-1} a_i$.

    \bigskip

    Then,
    \setcounter{equation}{0}
    \begin{align}
        x\prod\limits_{i=0}^{n-1} a_i \cdot a_n &= x\prod\limits_{i=0}^{n} a_i\\
        &= a_{n+1}
    \end{align}

    \bigskip

    Then it follows from above that the recursive sequence of numbers is true for
    all natural numbers.

    \item

    From the following table

    \begin{tabular}{c|c|c|c}
        String Length & Number of Even (Digit Sum) & Number of Odd (Digit Sum) & Total\\
        \hline
        1 & 2 & 1 & 3\\
        \hline
        2 & 5 & 4 & 9\\
        \hline
        3 & 14 & 13 & 27
    \end{tabular}

    \bigskip

    we see that $E_n = \frac{3^n + 1}{2}$ and $O_n = \frac{3^n - 1}{2}$.

    \bigskip

    As well, we see that the number of new elements in $E_{n+1}$ is $3^n$.

    \bigskip

    Now, we will prove that $E_n$ and $O_n$ are true for all natural numbers
    using the induction hypothesis.

    \bigskip

    \textbf{Base Case (n = 1):}

    \bigskip

    Let $n = 1$.

    \bigskip

    Then, $E_n = \frac{4}{2} = 2$ and $O_n = \frac{2}{2} = 1$.

    \bigskip

    Since the result matches to data in table, the base case holds.

    \bigskip

    \textbf{Inductive Case:}

    Let $n \in \mathbb{N}$. Assume $E_n = \frac{3^n + 1}{2}$ and $O_n = \frac{3^n - 1}{2}$.

    \bigskip

    Then,
    \setcounter{equation}{0}
    \begin{align}
        E_{n+1} &= \frac{3^n + 1}{2} + 3^n\\
        &= \frac{3^n + 1}{2} + \frac{2 \cdot 3^n}{2}\\
        &= \frac{3 \cdot 3^n + 1}{2}\\
        &= \frac{3^{n+1} + 1}{2}
    \end{align}

    \bigskip

    Then, it follows from above that the inductive step for $E_n$ holds.

    \bigskip

    Similarly, for $O_n$,

    \begin{align}
        O_{n+1} &= \frac{3^n - 1}{2} + 3^n\\
        &= \frac{3^n - 1}{2} + \frac{2 \cdot 3^n}{2}\\
        &= \frac{3 \cdot 3^n - 1}{2}\\
        &= \frac{3^{n+1} - 1}{2}
    \end{align}

    \bigskip

    Then, it follows from above that the inductive step for $O_n$ holds.

    \bigskip

    Then, it follows from the definition of induction hypothesis that the value of
    $E_n$ and $O_n$ are true for all $n$.

\end{enumerate}

\section*{Question 2}
\begin{enumerate}[a.]
    \item

    Since first 1 repeats every $4i - 1$ times and the second 1 repeats every $4i$ times,

    \setcounter{equation}{0}
    \begin{align}
        (0.\overline{0011})_2 &= \sum\limits_{i=1}^{\frac{n}{4}} \left( \frac{1}{2} \right)^{4i} + \sum\limits_{i=1}^{\frac{n}{4}} \left( \frac{1}{2} \right)^{4i-1}\\
        &= \sum\limits_{i=1}^{\frac{n}{4}} \left( \frac{1}{16} \right)^{i} + 2 \cdot \sum\limits_{i=1}^{\frac{n}{4}} \left( \frac{1}{16} \right)^{i}\\
        &= \frac{1}{16} \cdot \sum\limits_{i=0}^{\frac{n}{4} - 1} \left( \frac{1}{16} \right)^{i} + \sum\limits_{i=0}^{\frac{n}{4}-1} \left( \frac{1}{16} \right)^{i}\\
        &= \frac{3}{16} \cdot \sum\limits_{i=0}^{\frac{n}{4} - 1} \left( \frac{1}{16} \right)^{i}
    \end{align}

    \bigskip

    Then,

    \begin{align}
        \frac{3}{16} \cdot \sum\limits_{i=0}^{\frac{n}{4} - 1} \left( \frac{1}{16} \right)^{i} &= \frac{3}{16} \cdot \left(\frac{1 - \frac{1}{16}^{\frac{n}{4}}}{1 - (\frac{1}{16})} \right)
    \end{align}

    by using the formula $\forall n \in \mathbb{Z}^{+} and r \in \mathbb{R},\: r \neq 1 \Rightarrow \sum\limits_{i=0}^{n-1} r^i = \frac{1-r^n}{1 - r}$.

    \bigskip

    Then,

    \begin{align}
        \frac{3}{16} \cdot \left(\frac{1 - \frac{1}{16}^{\frac{n}{4}}}{1 - (\frac{1}{16})} \right) &= (\frac{1 - \frac{1}{2}^n}{\frac{15}{16}})\\
        &= \frac{1}{5} \cdot \left( 1 - \frac{1}{2}^n \right)\\
        &= \frac{1}{5} \cdot \left( \frac{2^n - 1}{2^n} \right)
    \end{align}

    \bigskip

    Then,

    \begin{align}
        0.2 - \frac{1}{5} \cdot \left( \frac{2^n - 1}{2^n} \right) &= \frac{1}{5} - \frac{1}{5} \cdot \left( \frac{2^n - 1}{2^n} \right)\\
        &= \frac{2^n}{5 \cdot 2^n} - \frac{1}{5} \cdot \left( \frac{2^n - 1}{2^n} \right)\\
        &= \frac{1}{5 \cdot 2^n}
    \end{align}

    Then, it follows from above that $\forall n \in \mathbb{Z}^{+},\: 4 \mid n
    \Rightarrow \frac{1}{5 \cdot 2^n}$

    \item

    Let $n \in \mathbb{Z}^{+}$, and $x \in \{x \mid x \in \mathbb{R}^{+}, 0
    \leq x < 1\}$.

    \bigskip

    We will prove that the statement $\forall n \in \mathbb{Z}^{+}$, $\forall x
    \in S$, $\exists x_1 \in S$, $FB(n, x_1)\land 0 \leq x - x_1 < 1$ is true
    using induction hypothesis.

    \bigskip

    Let $n = 1$.

    \textbf{Case 1} ($0 \leq x < 0.5$, from $S = {x \mid x \in \mathbb{R}, 0 \leq x < 1}$):

    \bigskip

    Let $x_1 = 0$.

    \bigskip

    Then,
    \setcounter{equation}{0}
    \begin{align}
        0 &= (0.0)_2\\
        &= \sum\limits_{i=1}^1 \frac{b_i}{2}
    \end{align}

    by the fact that $b_i = 0$.

    \bigskip

    Then, it follows from above that $FB(1,x_1)$ is true.

    \bigskip

    Now we will prove that $0 \leq x - x_1 < \frac{1}{2}$ is true.

    \bigskip

    Let $x_1 = 0$. Assume $0 \leq x < 0.5$.

    \bigskip

    Then,

    \begin{align}
        0 &\leq x < 0.5\\
        0 - x_1 &\leq x - x_1 < \frac{1}{2} - x_1\\
        0 &\leq x - x_1 < \frac{1}{2}
    \end{align}

    \bigskip

    Then, it follows from above that $FB(n,x_1) \land 0 \leq x - x_1 < \frac{1}{2}$
    hold for the base case with $0 \leq x < 0.5$.

    \bigskip

    \textbf{Case 2} ($0.5 \leq x < 1$ from $S = \{x \mid x \in \mathbb{R}^{\geq 0}, 0 \leq x < 1\}$):

    \bigskip

    First, we will prove that $FB(1,x_1)$ is true.

    \bigskip

    Let $x_1 = 0.5$.

    \bigskip

    Then,

    \begin{align}
        0.5 &= \frac{1}{2}\\
        &= \sum\limits_{i=1}^1 \frac{b_i}{2}
    \end{align}

    where $b_i = 1$.

    \bigskip

    Then, it follows from the definition of finite fractional binary representation
    that $x$ has fractional binary representation with 1 bits, and $FB(1,x_1)$ is
    true.

    \bigskip

    Now, we will prove that $0 \leq x - x_1 < 0.5$.

    \bigskip

    Let $x_1 = 0.5$. Assume $0.5 \leq x < 1$.

    \bigskip

    Then,

    \begin{align}
        0.5 - x_1 &\leq x - x_1 < 1 - x_1\\
        0 &\leq x - x_1 < 0.5
    \end{align}

    \bigskip

    Then, it follows from above that $ 0 \leq x - x_1 < 0.5$ is true.

    \bigskip

    Then, since $0 \leq x - x_1 < 0.5$ is true and $FB(1,x_1)$ is true, $FB(1,x_1) \land
    0 \leq x - x_1 < 0.5$ is true for the case $0.5 \leq x < 1$.

    \bigskip

    Then, by combining results from case 1 and case 2, we can conclude that the
    statement holds for the base case.

    \bigskip

    Now, let $n \in \mathbb{Z}^{+}$, and $x \in S$. Assume $\exists x_1 \in S$,
    $FB(n, x_1) \land 0 \leq x - x_1 \leq \frac{1}{2^n}$.


    Then, we will prove that the statement $\forall n \in \mathbb{Z}^{+}$, $\forall x
    \in S$, $FB(n,x_1) \land 0 \leq x - x_1 \leq \frac{1}{2^n}$ is true for inductive
    case by separating $0 \leq x - x_1 \leq \frac{1}{2^n}$ into following cases.

    \bigskip

    \textbf{Case 1} ($0 \leq x - x_1 < \frac{1}{2^{n+1}}$):

    \bigskip

    First, we will prove that $FB(n+1,x_2)$ is true.

    \bigskip

    Let $x_2 = x_1$.

    \bigskip

    Then,

    \begin{align}
        x_2 &= \sum\limits_{i=1}^n \frac{b_i}{2}\\
        &= \sum\limits_{i=1}^n \frac{b_i}{2} + \frac{b_{i+1}}{2}\\
        &= \sum\limits_{i=1}^{n+1} \frac{b_i}{2}
    \end{align}

    by setting $b_{i+1} = 0$.

    \bigskip

    Then, it follows from above that $FB(n+1, x_2)$ is true.

    \bigskip

    Now, we will prove that $0 \leq x - x_2 < \frac{1}{2^{n+1}}$ is true.

    \bigskip

    Let $x_2 = x_1$. Assume $0 \leq x - x_1 < \frac{1}{2^{n+1}}$.

    \bigskip

    Then, it follows from assumption that $0 \leq x - x_2 < \frac{1}{2^{n+1}}$ is
    true.

    \bigskip

    Then, since $FB(n+1, x_2)$ is true and $0 \leq x - x_2 < \frac{1}{2^{n+1}}$ is true,
    $FB(n+1, x_2) \land 0 \leq x - x_2 < \frac{1}{2^{n+1}}$ is true for the case
    $0 \leq x - x_1 < \frac{1}{2^{n+1}}$.

    \bigskip

    \textbf{Case 2} ($\frac{1}{2^{n+1}} \leq x - x_1 \leq \frac{1}{2^{n}}$):

    \bigskip

    First, we will prove that $FB(n+1,x_2)$ is true.

    \bigskip

    Let $x_2 = x_1 - \frac{1}{2^{n+1}}$.

    \bigskip

    Then,

    \begin{align}
        x_2 &= \sum\limits_{i=1}^n \frac{b_i}{2^i} - \frac{1}{2^{n+1}}\\
        &= \sum\limits_{i=1}^n \frac{b'_i}{2}
    \end{align}

    by the fact that $b'_i = b_i$, $b'_n = 0$ and $b'_{n+1} = 1$.

    \bigskip

    Then, it follows from the definition of finite fractional binary representation that
    $FB{n+1,x_2}$ is true.

    \bigskip

    Now, we will prove that $0 \leq x - x_2 \leq \frac{1}{2^{n+1}}$ is true.

    \bigskip

    Let $n \in \mathbb{Z}^{+}$, $x \in \mathbb{S}$, $x_2 = x_1 - \frac{1}{2^{n+1}}$.
    Assume $\frac{1}{2^{n+1}} \leq x - x_1 \leq \frac{1}{2^{n}}$.

    \bigskip

    Then,

    \begin{align}
        \frac{1}{2^{n+1}} &\leq x - x_1 \leq \frac{1}{2^{n}}\\
        \frac{1}{2^{n+1}} - \frac{1}{2^{n+1}} &\leq x - x_1 - \frac{1}{2^{n+1}} \leq \frac{1}{2^{n}} - \frac{1}{2^{n+1}}\\
        0 &\leq x - x_2 \leq \frac{1}{2^{n}} - \frac{1}{2^{n+1}}\\
        0 &\leq x - x_2 \leq \frac{2}{2^{n+1}} - \frac{1}{2^{n+1}}\\
        0 &\leq x - x_2 \leq \frac{1}{2^{n+1}}
    \end{align}

    \bigskip

    Then it follows from above that $0 \leq x - x_2 \leq \frac{1}{2^{n+1}}$
    is true.

    \bigskip

    Then, since $FB(n+1,x_2)$ is true, and $0 \leq x - x_2 \leq \frac{1}{2^{n+1}}$ is true,
    $FB(n+1,x_2) \land 0 \leq x - x_2 \leq \frac{1}{2^{n+1}}$ is true for the case
    $\frac{1}{2^{n+1}} \leq x - x_1 \leq \frac{1}{2^{n}}$.

    \bigskip

    Then, because the statement is true in both case 1 and case 2, the statement
    at inductive step holds.

    \bigskip

    Then, it follows from induction hypothesis that the statement is true for all
    natural numbers.

\end{enumerate}

\section*{Question 3}
\begin{enumerate}[a.]
    \item

    \textbf{Definition of Big-Oh:} $g \in \mathcal{O}(f): \exists c,n_0 \in \mathbb{R}^{+},\:\forall n \in \mathbb{N},\:
    n \geq n_0 \Rightarrow g(n) \leq cf(n), where f,g: \mathbb{N} \to \mathbb{R}^{\geq 0}$

    \bigskip

    Since $a \leq b \land c \leq d \Rightarrow a + b \leq c + d$, we will prove
    that $n^4 + 165n^3 \leq c(n^4 - n^2)$ by validating the following
    inequalities, then combining it together

    \setcounter{equation}{0}
    \begin{align}
        n^4 &\leq \frac{c}{3}n^4\\
        165n^3 &\leq \frac{c}{3}n^4\\
        cn^2 &\leq \frac{c}{3}n^4
    \end{align}

    \bigskip

    \textbf{Validating} $(n^4 \leq \frac{c}{3}n^4)$:

    \bigskip

    Let $n \in \mathbb{N}$, $n_0 = 2$, and $c = 249$. Assume $n \geq n_0$.

    \bigskip

    Then,

    \begin{align}
        n^4 &\leq \frac{249}{3}n^4\\
        n^4 &\leq 83n^4
    \end{align}

    \bigskip

    Then, it follows from above that $n^4 \leq \frac{c}{3}n^4$ holds.

    \bigskip

    \textbf{Validating} $(165n^3 \leq \frac{c}{3}n^4)$:

    Let $n \in \mathbb{N}$, $n_0 = 2$, and $c = 249$. Assume $n \geq n_0$.

    \bigskip

    Then,

    \begin{align}
        165n^3 &\leq \frac{c}{3}n^4\\
        165n^3 &\leq \frac{249}{3}n^4\\
        165n^3 &\leq 83n^4
    \end{align}

    \bigskip

    Then,

    \begin{align}
        165n^3 &\leq 83n^4\\
        165(2)^3 &\leq 83(2)^4
    \end{align}

    because of the fact $n_0 = 2$ and the assumption $n \geq n_0$

    \bigskip

    Then,

    \begin{align}
        165(2)^3 &\leq 83(2)^4\\
        165 \cdot 8 &\leq 83 \cdot 16\\
        1320 &\leq 1328
    \end{align}

    \bigskip

    Then, it follows from above that $165n^3 \leq \frac{c}{3}n^4$ holds.

    \bigskip

    \textbf{Validating} $(cn^2 \leq \frac{c}{3}n^4)$:

    \bigskip

    Let $n \in \mathbb{N}$, $n_0 = 2$, and $c = 249$. Assume $n \geq n_0$.

    \bigskip

    Then,

    \begin{align}
        cn^2 &\leq \frac{c}{3}n^4\\
        249n^2 &\leq \frac{249}{3}n^4\\
        249n^2 &\leq 83n^4\\
    \end{align}

    \bigskip

    Then,

    \begin{align}
        249n^2 &\leq 83n^4\\
        249(2)^2 &\leq 83(2)^4\\
    \end{align}

    because of the fact $n_0 = 2$ and the assumption $n \geq n_0$

    \bigskip

    Then,

    \begin{align}
        249n^2 &\leq 83n^4\\
        249 \cdot 4 &\leq 83 \cdot 16\\
        996 &\leq 1328
    \end{align}

    \bigskip

    Then, it follows from above that $cn^2 \leq \frac{c}{3}n^4$ holds.

    \bigskip

    Then, because we know $n^4 \leq \frac{c}{3}n^4$, $165n^3 \leq \frac{c}{3}n^4$,
    and $cn^2 \leq \frac{c}{3}n^4$ are true, we can conclude that
    $n^4 + 165n^3 + cn^2 \leq cn^4$ is true.

    \bigskip

    Then,

    \begin{align}
        n^4 + 165n^3 + cn^2 &\leq cn^4\\
        n^4 + 165n^3 &\leq c(n^4 - n^2)
    \end{align}

    \bigskip

    Then, it follows from the definition of Big-Oh that the statement $n^4 + 165n^3
    \in \mathcal{O}(n^4 - n^2)$ is true.

    \item

    % For this question, I should validate each statements separately
    % then combine together using cases

    \textbf{Negation:} $\forall f: \mathbb{N} \to \mathbb{R}^{+},\:\exists g:\mathbb{N}
    \to \mathbb{R}^{\geq 0},\:(\forall c,n_0 \in \mathbb{R}^{+},\:\exists n \in \mathbb{N},
    \:n \geq n_0 \land (g(n) > cf(n))) \land (\forall d,m_0 \in \mathbb{R}^{+},\:\exists
    m \in \mathbb{N},\:m \geq m_0 \land (g(n) < cf(n)))$

    \bigskip

    Let

    \begin{align*}
        g(n) = \left \{
        \begin{array}{ll}
            0 & \text{if n even}\\
            nf(n) & \text{otherwise}\\
        \end{array} \right.
    \end{align*}

    \bigskip

    We will prove the statement by separating into cases, and combine them
    together in the end.

    \bigskip

    \textbf{Case 1} ($g \notin \mathcal{O}(f):\forall c,n_0 \in \mathbb{R}^{+},\:
    \exists n \in \mathbb{N},\:n \geq n_0 \land g(n) > cf(n)$):

    \bigskip

    Let $c,n_0 \in \mathbb{R}^{+}$, $n = 2 \cdot \lceil max(c,n_0) \rceil$.

    \bigskip

    Because there are two parts in $max(c,n_0)$, we will prove by separating into
    cases $n_0 > c$, and $c > n_0$ and combine the result together in the end.

    \bigskip

    Consider the case $n_0 > c$.

    \bigskip

    Since $n_0 > c$ and $n = 2\lceil n_0 \rceil$, we can conclude that
    $n \geq n_0$.

    \bigskip

    Then,

    \setcounter{equation}{0}
    \begin{align}
        g(n) &= nf(n)\\
        &= 2\lceil n_0 \rceil f(n)\\
        &> 2cf(n)\\
        &> cf(n)
    \end{align}

    \bigskip

    Then, since $g(n) > cf(n)$ is true and $n \geq n_0$ is true, $n \geq n_0 \land
    g(n) > cf(n)$ is true.

    \bigskip

    Now, consider the case $c > n_0$.

    \bigskip

    Since $c > n_0$ and $n = 2\lceil c \rceil$, we can conclude that
    $n \geq n_0$.

    \bigskip

    Then,

    \begin{align}
        g(n) &= nf(n)\\
        &= 2\lceil c \rceil f(n)\\
        &> 2cf(n)\\
        &> cf(n)
    \end{align}

    \bigskip

    Then, since $g(n) > cf(n)$ is true and $n \geq n_0$ is true given $n_0 > c$,
    and $c > n_0$, $n \geq n_0 \land g(n) > cf(n)$ is true.

    \bigskip

    Then, it follows from above that the statement $g \notin \mathcal{O}(f)$ is
    true.

    \bigskip

    \textbf{Case 2} ($g \notin \Omega (f):\:\forall c,n_0 \in \mathbb{R}^{+},\:
    \forall n \in \mathbb{N},\:n \geq n_0 \land g(n) < cf(n)$):

    \bigskip

    Let $c,n_0 \in \mathbb{R}^{+}$, $n = \lceil n_0 \rceil + 1$.

    \bigskip

    Then, we can conclude that $n \geq n_0$.

    \bigskip

    Then,
    \setcounter{equation}{0}
    \begin{align}
        0 = g(n) < cf(n)
    \end{align}

    by the fact that no values in codomain of $f$ is 0.

    \bigskip

    Then, $n \geq n_0 \land g(n) < cf(n)$ is true by the fact that
    ($n \geq n_0$) is true and ($g(n) < cf(n)$) is true.

    \bigskip

    Then, it follows from above that the statement $g \notin \Omega(f)$ is true.

    \bigskip

    Since $g \notin \Omega(f)$  is true and $g \notin \mathcal{O}(f)$ is true given
    $g$ and $n$, it follows from the negation of statement that the statement $\exists
    f: \mathbb{N} \to \mathbb{R}^{+},\:(\forall g: \mathbb{N} \to \mathbb{R}^{\geq 0},\:
    g \in \mathcal{O}(f) \lor g \in \Omega(f))$ is false.

\end{enumerate}

\section*{Question 4}
\begin{enumerate}[a.]
    \item

    Let $n_0 = c^{-\frac{1}{b-a}}$. Assume $n \geq n_0$.

    \bigskip

    Then,
    \setcounter{equation}{0}
    \begin{align}
        n &\geq n_0\\
        n &\geq c^{-\frac{1}{b-a}}\\
        \left[n\right]^{-(b-a)}&\leq \left[c^{-\frac{1}{b-a}}\right]^{-(b-a)}\\
        n^{a-b} &\leq c
    \end{align}

    \bigskip

    Then,

    \begin{align}
        cf(n) &= cn^b\\
        &\geq n^{a-b}n^b\\
        &\geq n^{a-b+b}\\
        &\geq n^a
    \end{align}

    Then, it follows from the definition of little-oh that the statement
    $\forall a,b \in \mathbb{R}^{+}, a < b \Rightarrow n^a \in o(n^b)$ is true.

    \item

    \textbf{Predicate logic in expanded form:} $\forall f,g:\mathbb{N} \to \mathbb{R}^{+},\:
    (\forall c \in \mathbb{R}^{+},\:\exists n_0 \in \mathbb{R}^{+},\:\forall n \in
    \mathbb{N},\: n \geq n_0 \Rightarrow g(n) \leq cf(n)) \Rightarrow (\forall d,m_0
    \in \mathbb{R}^{+}, \exists m \in \mathbb{N},\: m \geq m_0 \land f(m) > dg(m))$

    \bigskip

    Rough idea
    \setcounter{equation}{0}
    \begin{align}
        \frac{1}{2}g(n) < g(n) \leq cf(n)
    \end{align}


\end{enumerate}

\end{document}