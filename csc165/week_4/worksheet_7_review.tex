\documentclass[12pt]{article}
\usepackage{enumerate}
\usepackage{amsfonts}
\usepackage{amsmath}
\usepackage{fancyhdr}
\usepackage{amssymb}
\usepackage{mdframed}

\begin{document}
\title{Worksheet 7 Solution}
\maketitle

\section*{Question 1}
\begin{enumerate}
    \item

    Assume that $n \leq 1$.

    \bigskip

    Then, it follows from the assumption that the statement holds for the case
    $n \leq 1$.

    \begin{mdframed}
        \underline{\textbf{Correct Solution:}}

        Assume that $n \leq 1$.

        \bigskip

        Then, the assumption satisfies the first part of the OR we want to prove.

    \end{mdframed}

    \bigskip

    \textbf{Notes:}
    \begin{itemize}
        \item the professor specifically states the assumption satisfies the first
        part of the OR we want to prove.

    \end{itemize}

    \item

    Assume $\exists k,d \in \mathbb{N}$, $n = kd \land d \neq 1 \land d \neq n$.

    \bigskip

    Let $a = d$ and $b = k$.

    \bigskip

    We will divide proof into parts and combine them together.

    \bigskip

    \textbf{Part 1} ($n \nmid a$):

    Since $\frac{1}{k} \cdot n = d$, $k$ must be 1 for $n$ to divide $d$.

    \bigskip

    Then, because we know $d \neq n$, we can conclude that $n \nmid a$.

    \bigskip

    \textbf{Part 2} ($n \nmid b$):

    Since $\frac{1}{d} \cdot n = k$, $d$ must be 1 for $n$ to divide $k$.

    \bigskip

    Then, because we know $d \neq 1$, we can conclude $n \nmid b$.

    \textbf{Part 3} ($n \mid ab$):

    Since $ab = n$ and $\forall n \in \mathbb{N},\:n \mid n$, we can conclude that $n \mid ab$.

    \bigskip

    Then, it follows from the result of part 1, part 2 and part 3 that the second
    part of the OR is true.

    \begin{mdframed}

        \underline{\textbf{Correct Solution:}}

        \bigskip

        Assume $\exists d \in \mathbb{N}$, $k \in \mathbb{Z}$, $n = dk \land
        d \neq 1 \land d \neq n$, and n > 1.

        \bigskip

        Let $a = d$ and $b = k$.

        \bigskip

        We will prove this statement by dividing into cases and combining them
        together.

        \bigskip

        \textbf{Case 1} ($n \mid ab$):

        \bigskip

        Because we know $n = ab$ and $n \mid n$ by fact 1 , we can conclude
        $n \mid ab$.

        \bigskip

        \textbf{Case 2} ($n \nmid a$):

        \bigskip

        Because we know $d \geq 1$ from $d \in \mathbb{N}$ and $n > 1$ in
        assumption, we can conclude $k \geq 1$.

        \bigskip

        Then,

        \begin{align}
            n &= dk\\
            n &> d
        \end{align}

        where $'>'$ sign is due to the assumption $d \neq n$.

        \bigskip

        Then,

        \begin{align}
            d < 1 \lor n \nmid d
        \end{align}

        by contrapositive of fact 2.

        \bigskip

        Since the first part of OR is not true, we can conclude $n \nmid a$.

        \bigskip

        \textbf{Case 3} ($n \nmid b$):

        Because we know $n = dk$, $d \geq 1$ from $d \in \mathbb{N}$ and $n > 1$ in
        assumption, we can conclude $k \geq 1$.

        \bigskip

        Then because we know $d \neq n \land d \neq 1$ and $n = dk$, we can conclude
        $k \neq n \land k \neq 1$.

        Then,

        \begin{align}
            n &= dk\\
            n &> k
        \end{align}

        where $'>'$ sign is due to the fact $k \neq n \land k \neq 1$.

        \bigskip

        Then,

        \begin{align}
            b < 1 \lor n \nmid y
        \end{align}

        by contrapositive of fact 2.

        \bigskip

        Since the first part of OR is not true, and we can conclude $n \nmid b$.

    \end{mdframed}

    \textbf{Notes:}
    \begin{itemize}
        \item \textbf{Definition of Divisibility:} Let $a,d \in \mathbb{Z}$. There exists $k \in \mathbb{Z}$, $n = dk$
        \item \textbf{Contrapositive of Fact 2:} $\forall x,y \in \mathbb{N}, 1 > x \lor x > y \Rightarrow y < 1 \lor x \nmid y$

    \end{itemize}


\end{enumerate}

\section*{Question 2}

\section*{Question 3}

\end{document}