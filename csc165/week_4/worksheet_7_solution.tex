\documentclass[12pt]{article}
\usepackage{enumerate}
\usepackage{amsfonts}
\usepackage{amsmath}
\usepackage{fancyhdr}
\usepackage{amssymb}

\begin{document}
\title{Worksheet 7 Solution}
\maketitle

\section*{Question 1}
\begin{enumerate}[a.]
    \item

    \textbf{Case 1 ($n \geq 1$):}

    \bigskip

    No more proof required. This is exactly what we want to show.

    \bigskip

    \textbf{Case 2 ($\exists d \in \mathbb{N}, d \mid n \land d \neq 1 \land d \neq n$):}

    \bigskip

    Let $a = d$ and $b = k$.

    \bigskip

    Because we know $a \mid n$, and is written in form $n = ab, k \in \mathbb{Z}$,
    we can conclude that $k \mid n$.

    \bigskip

    Because we know $\forall n \in \mathbb{Z}^{+}$, and $d \in \mathbb{Z}, d \mid
    n \Rightarrow d \leq n$, $a \leq n$ and $b \leq n$.

    \bigskip

    Then the only combination where $n \mid a$ and $n \mid b$ are true is when
    $a = n$ and $b = 1$, and vice versa, by the fact that any lower value of $a$
    or $b$ results in non-interger value.

    \bigskip

    Then it follows from the assumption $a \neq 1 \land a \neq n$ and $b \neq 1 \land b \neq n$
    that $n \nmid a$ and $n \nmid b$



\end{enumerate}

\section*{Question 2}

\section*{Question 3}

\end{document}