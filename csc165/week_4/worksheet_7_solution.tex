\documentclass[12pt]{article}
\usepackage{enumerate}
\usepackage{amsfonts}
\usepackage{amsmath}
\usepackage{fancyhdr}
\usepackage{amssymb}

\begin{document}
\title{Worksheet 7 Solution}
\maketitle

\section*{Question 1}
\begin{enumerate}[a.]
    \item

    \textbf{Case 1 ($n \geq 1$):}

    \bigskip

    No more proof required. This is exactly what we want to show.

    \bigskip

    \textbf{Case 2 ($\exists d \in \mathbb{N}, d \mid n \land d \neq 1 \land d \neq n$):}

    \bigskip

    Let $a = d$ and $b = k$.

    \bigskip

    Because we know $\forall n \in \mathbb{Z}^{+}$, and $l \in \mathbb{Z}, l \mid
    n \Rightarrow l \leq n$, $a \leq n$.

    \bigskip

    Then $n \mid a$ is true only when $a = n$ and $b = 1$, by the fact that any
    lower value of $a$ results in non-integer value.

    \bigskip

    Then it follows from the assumption $a \neq 1 \land a \neq n$ that $n \nmid a$.

    \bigskip

    The same logic holds for $n \nmid b$.

    \bigskip

    Lastly, since $n = ab$, and $\forall x \in \mathbb{Z},\:x \mid x$, $n \mid ab$.


\end{enumerate}

\section*{Question 2}

\begin{enumerate}[a.]
    \item

    Let $n,m \in \mathbb{N}$. Assume $Prime(n)$, and $n \nmid m$.

    \bigskip

    Then,

    \begin{align}
        gcd(n,m) &= 1
    \end{align}

    by fact 2 (i.e. $\forall n,p \in \mathbb{Z}, Prime(p) \land p \nmid n \Rightarrow
    gcd(p,n) = 1$).

    \bigskip

    Then $\exists r,s \in \mathbb{Z}$,

    \begin{align}
        1 = gcd(n,m) = rn + sm
    \end{align}

    by fact 6 (i.e. $\forall n,m \in \mathbb{N}, \exists r,s \in \mathbb{Z},
    rn+sm = gcd(n,m)$).

    \bigskip

    Then, it follows from above that the statement $\forall n,m \in \mathbb{N},\:
    Prime(n) \land n \nmid m \Rightarrow (\exists r,s \in \mathbb{Z}, rn+sm = 1)$ is
    true.

    \item

    Let $n,m \in \mathbb{N}$. Assume $Prime(n)$ and $(\exists r,s \in \mathbb{Z}, rn+sm = 1)$.

    \bigskip

    Then,

    \begin{align}
        gcd(n,m) = 1
    \end{align}

    by fact 6 (i.e. $\forall n,m \in \mathbb{N},\:\exists r,s \in \mathbb{Z},\: rn+sm = gcd(n,m)$).

    \bigskip

    Then, 1 is the maximum number that divides both $n$ and $m$, by the definition
    of GCD.

    \bigskip

    It follows from the above that $n \mid m$ only when $n = 1$.

    \bigskip

    Since $n$ is prime and $n > 1$, the above is not possible, and $n \nmid m$.


\end{enumerate}


\section*{Question 3}

\begin{enumerate}[a.]
    \item

    Let $x \in \mathbb{Z}$.

    \bigskip

    Then,
    \setcounter{equation}{0}
    \begin{align}
        x &= x\\
        x &= (1)x
    \end{align}

    \bigskip

    Then, it follows from the definition of divisibility that x divides x.

    \item

    Let $x,y \in \mathbb{N}$. Assume $y \geq 1$ and $x \mid y$.

    \bigskip

    Then $\exists k \in \mathbb{Z}$,

    \setcounter{equation}{0}
    \begin{align}
        y &= kx
    \end{align}

    \bigskip

    Then, because we know $y \geq 1$, and $x \geq 1$, we can conclude that $k \geq 1$.

    \bigskip

    Then it follows from the above that

    \begin{align}
        1 \leq x \leq kx = y
    \end{align}

    \item

    Let $n,p \in \mathbb{Z}$. Assume $Prime(p)$ and $p \nmid n$.

    \bigskip

    Because we know from the definition of prime number, the common divisors
    available for $p$ are 1 and $p$.

    \bigskip

    Also, because we know $\forall n \in \mathbb{Z}, n \mid n$, we can conclude
    that $1 \mid n$.

    \bigskip

    Since $p \nmid n$, but $1 \mid p$ and $1 \mid n$, $gcd(p,n) = 1$

    \item

    Let $n,m \in \mathbb{N}$.

    \bigskip

    \textbf{Case 1 ($n \neq 0, m = 0$):}

    \bigskip

    Assume $n \neq 0$ and $m = 0$.

    \bigskip

    Since $n \mid n$ (by fact 1) and $n \mid m$, $n$ is a common divisor, and

    \bigskip
    \setcounter{equation}{0}
    \begin{align}
        gcd(n,m) &= n
    \end{align}

    by the definition of greatest common divisor.

    \bigskip

    Since, $n \in \mathbb{N}$ and $n \mid gcd(n,m)$ (by fact 1),

    \begin{align}
        1 &\leq gcd(n,m) \leq n
    \end{align}

    by fact 2.

    \bigskip

    \textbf{Case 2 ($n = 0, m \neq 0$):}

    \bigskip

    The inequality $gcd(n,m) \geq 1$ holds using the same logic as case 1.

    \bigskip

    \textbf{Case 3 ($n \neq 0, m \neq 0$):}

    \bigskip

    Let $n,m \in \mathbb{N}$. Assume $n \neq 0$ and $m \neq 0$.

    \bigskip

    Since 1 is the smallest divisor that exists in both $n$ and $m$,
    \setcounter{equation}{0}
    \begin{align}
        gcd(n,m) &\geq 1
    \end{align}


\end{enumerate}

\end{document}