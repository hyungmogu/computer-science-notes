\documentclass[12pt]{article}
\usepackage{enumerate}
\usepackage{amsfonts}
\usepackage{amsmath}
\usepackage{fancyhdr}
\usepackage{amssymb}

\begin{document}
\title{Worksheet 6 Solution}
\maketitle

\section*{Question 1}
\begin{enumerate}[a.]
    \item

    $P(123) \land (\forall n \in \mathbb{N}, P(n) \Rightarrow n \leq 123)$

    \item

    $isCD(x,y,d)$: $\exists x,y,d \in \mathbb{Z},\:d \mid x \land d \mid y$

    $isGCD(x,y,d)$: $\exists x,y,d \in \mathbb{Z},\:(x = 0 \land y = 0 \land d = 0)
    \lor ((x \neq 0 \lor y \neq 0) \land isCD(x,y,d) \land \forall e \in \mathbb{Z}, \:e > d \Rightarrow \neg isCD(x,y,e))$

    \item

    Statement: $\forall x \in \mathbb{Z}^{+}, IsGCD(x,0,x)$

    \bigskip

    For the value $x$, because we know $x \mid x$, and $\forall n \in \mathbb{Z}^{+}$ and $\forall
    d \in \mathbb{Z}, d \mid n \Rightarrow d \leq n$, $x$ is the biggest divisor of $x$

    For the value $0$, because we know anything that divides 0 is 0, and
    $\exists k \in \mathbb{Z},\: 0 = k \times 0$, $k$ can be chosen to be $x$.

    Then, it follows from the definition of GCD that the statement $IsGCD(x,0,x)$ is true.

    \item

    $\forall a,b \in \mathbb{Z}, (a \neq 0 \lor b \neq 0) \Rightarrow \exists p,q \in \mathbb{Z},\:gcd(a,b) = ap + qb \land \forall m \in \mathbb{Z}, m < gcd(a,b) \land m \neq ap + qb$


\end{enumerate}

\section*{Question 2}

\section*{Question 3}

\end{document}