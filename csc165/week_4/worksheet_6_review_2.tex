\documentclass[12pt]{article}
\usepackage[margin=2.5cm]{geometry}
\usepackage{enumerate}
\usepackage{amsfonts}
\usepackage{fancyhdr}
\usepackage{amsmath}
\usepackage{amssymb}
\usepackage{amsthm}
\usepackage{mdframed}


\begin{document}
\title{Worksheet 6 Review 2}
\author{Hyungmo Gu}
\maketitle

\section*{Question 1}
\begin{enumerate}[a.]
    \item

    $\forall x \in \mathbb{N},\:P(123) \land P(x) \Rightarrow x \leq 123$

    \bigskip

    \begin{mdframed}
        \underline{\textbf{Correct Solution:}}

        \bigskip

        $P(123) \land (\color{red}\forall x \in \mathbb{N},\color{black} P(x) \Rightarrow x \leq 123)$
    \end{mdframed}

    \item

    $IsCD(x,y,d):\:d \mid x \land d \mid y$, where $x,y,d \in \mathbb{Z}$

    \bigskip

    $IsGCD(x,y,d):\:\forall n \in \mathbb{N},\:IsCD(x,y,n) \Rightarrow \exists d
    \in \mathbb{N},\:IsCD(x,y,d) \land n \leq d$

    \bigskip

    \begin{mdframed}
        \underline{\textbf{Correct Solution:}}

        \bigskip

        $IsCD(x,y,d):\:d \mid x \land d \mid y$, where $x,y,d \in \mathbb{Z}$

        \bigskip

        $IsGCD(x,y,d):\color{red}(x = 0 \land y = 0 \Rightarrow d = 0)
        \land (\color{red}x \neq 0 \land y \neq 0 \Rightarrow IsCD(x,y,d) \land (\forall d_1 \in \mathbb{Z},\:IsCD(x,y,d_1) \Rightarrow d_1 \leq d))$, \color{red}where $x,y,d \in \mathbb{Z}$\color{black}
    \end{mdframed}

    \bigskip

    \textbf{Notes:}

    \begin{itemize}
        \item Realized the definition of $IsGCD$ extends from previous question
        \item Noticed professor defines if...else conditions in a predicate logic the following way

        \begin{mdframed}
        \begin{align*}
        (\text{case 1} \Rightarrow \text{statement 1}) \land (\text{case 2} \Rightarrow \text{statement 2})
        \end{align*}
        \end{mdframed}

        \item Hm... I feel puzzled about $\land$ operator used in between cases (
        i.e. $(x = 0 \land y = 0 \Rightarrow d = 0) \land (x \neq 0 ...)$).
        At glimpse, I felt $\lor$ is more appropriate since if this case is not true,
        then we want other case should be true.
    \end{itemize}

    \item

    \textbf{Statement:} $IsCD(x,0,x) \land (\forall d_1 \in \mathbb{Z}, IsCD(x,0,d_1) \Rightarrow d_1 \leq x)$

    \begin{mdframed}
        \underline{\textbf{Pseudoproof:}}

        \bigskip

        Let $x \in \mathbb{Z}^{+}$

        \bigskip

        We need to prove $x$ is a common divisor to both 0 and $x$, and we need
        to prove all common divisors $d_1$ of 0 and $x$ is less than or equal to $x$.

        \bigskip

        \begin{enumerate}[1.]
            \item Show $IsCD(x,0,x)$

            \bigskip

            We need to show there is $k_1 \in \mathbb{Z}$ such that
            $x = k_1 \cdot x$ and we need to show $k_2 \in \mathbb{Z}$ such that
            $0 = k_2 \cdot x$.

            \bigskip

            Let $k_1 = 1$ and $k_2 = 0$.

            \bigskip

            \begin{itemize}
                \item Show $x = k_1 \cdot x$ and $0 = k_2 \cdot 0$

                \begin{mdframed}
                Then, we can calculate that

                \begin{align}
                    x &= 1 \cdot x = k_1 \cdot x\\
                    0 &= 0 \cdot x = k_2 \cdot x
                \end{align}
                \end{mdframed}
            \end{itemize}

            \item Show $\forall d_1 \in \mathbb{Z}, IsCD(x,0,d_1) \Rightarrow d_1 \leq x$

            \bigskip

            Let $d_1 \in \mathbb{Z}$ and assume $d_1 \mid x$ and $d_1 \mid 0$.

            \bigskip

            We need to show $d_1 \leq x$.

            \bigskip

            \begin{enumerate}[1.]
                \item Use fact `$\forall n \in \mathbb{Z}^+,\:\forall d \in \mathbb{Z},\:d \mid n \Rightarrow d \leq n$' to show $d_1 \leq x$.

                \begin{mdframed}
                The hint tells us

                \begin{align}
                    \forall n \in \mathbb{Z}^+,\:\forall d \in \mathbb{Z},\:d \mid n \Rightarrow d \leq n
                \end{align}

                \bigskip

                Because we know from assumption that $d_1 \mid x$, by using the hint,
                we can conclude

                \begin{align}
                    d_1 \leq x
                \end{align}
                \end{mdframed}
            \end{enumerate}

        \end{enumerate}

        \bigskip



        \bigskip



    \end{mdframed}

\end{enumerate}

\section*{Question 2}

\section*{Question 3}

\end{document}