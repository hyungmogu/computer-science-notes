\documentclass[12pt]{article}
\usepackage[margin=2.5cm]{geometry}
\usepackage{enumerate}
\usepackage{amsfonts}
\usepackage{fancyhdr}
\usepackage{amsmath}
\usepackage{amssymb}
\usepackage{amsthm}
\usepackage{mdframed}


\begin{document}
\title{Worksheet 6 Review 2}
\author{Hyungmo Gu}
\maketitle

\section*{Question 1}
\begin{enumerate}[a.]
    \item

    $\forall x \in \mathbb{N},\:P(123) \land P(x) \Rightarrow x \leq 123$

    \bigskip

    \begin{mdframed}
        \underline{\textbf{Correct Solution:}}

        \bigskip

        $P(123) \land (\color{red}\forall x \in \mathbb{N},\color{black} P(x) \Rightarrow x \leq 123)$
    \end{mdframed}

    \item

    $IsCD(x,y,d):\:d \mid x \land d \mid y$, where $x,y,d \in \mathbb{Z}$

    \bigskip

    $IsGCD(x,y,d):\:\forall n \in \mathbb{N},\:IsCD(x,y,n) \Rightarrow \exists d
    \in \mathbb{N},\:IsCD(x,y,d) \land n \leq d$

    \bigskip

    \begin{mdframed}
        \underline{\textbf{Correct Solution:}}

        \bigskip

        $IsCD(x,y,d):\:d \mid x \land d \mid y$, where $x,y,d \in \mathbb{Z}$

        \bigskip

        $IsGCD(x,y,d):\color{red}(x = 0 \land y = 0 \Rightarrow d = 0)
        \land (\color{red}x \neq 0 \land y \neq 0 \Rightarrow IsCD(x,y,d) \land (\forall d_1 \in \mathbb{Z},\:IsCD(x,y,d_1) \Rightarrow d_1 \leq d))$, \color{red}where $x,y,d \in \mathbb{Z}$\color{black}
    \end{mdframed}

    \bigskip

    \textbf{Notes:}

    \begin{itemize}
        \item Realized the definition of $IsGCD$ extends from previous question
        \item Noticed professor defines if...else conditions in a predicate logic the following way

        \begin{mdframed}
        \begin{align*}
        (\text{case 1} \Rightarrow \text{statement 1}) \land (\text{case 2} \Rightarrow \text{statement 2})
        \end{align*}
        \end{mdframed}

        \item Hm... I feel puzzled about $\land$ operator used in between cases (
        i.e. $(x = 0 \land y = 0 \Rightarrow d = 0) \land (x \neq 0 ...)$).
        At glimpse, I felt $\lor$ is more appropriate since if this case is not true,
        then we want other case should be true.
    \end{itemize}

    \item

    \textbf{Statement:} $IsCD(x,0,x) \land (\forall d_1 \in \mathbb{Z}, IsCD(x,0,d_1) \Rightarrow d_1 \leq x)$

    \begin{proof}
        Let $x \in \mathbb{Z}^{+}$

        \bigskip

        We need to prove $x$ is a common divisor to both 0 and $x$, and we need
        to prove all common divisors $d_1$ of 0 and $x$ is less than or equal to $x$.

        \bigskip

        First, we need to show there is $k_1 \in \mathbb{Z}$ such that
        $x = k_1 \cdot x$ and we need to show $k_2 \in \mathbb{Z}$ such that
        $0 = k_2 \cdot x$.

        \bigskip

        Let $k_1 = 1$ and $k_2 = 0$.

        \bigskip

        Then, we can calculate that

        \begin{align}
            x &= 1 \cdot x = k_1 \cdot x\\
            0 &= 0 \cdot x = k_2 \cdot x
        \end{align}

        \bigskip

        Now, we need to show all integers $d_1$ that is a common divisor to both 0 and $x$
        is less than equal to $x$.

        \bigskip

        Let $d_1 \in \mathbb{Z}$ and assume $d_1 \mid x$ and $d_1 \mid 0$.

        \bigskip

        We need to show $d_1 \leq x$.

        \bigskip

        The hint tells us

        \begin{align}
            \forall n \in \mathbb{Z}^+,\:\forall d \in \mathbb{Z},\:d \mid n \Rightarrow d \leq n
        \end{align}

        \bigskip

        Because we know from assumption that $d_1 \mid x$, by using the hint,
        we can conclude

        \begin{align}
            d_1 \leq x
        \end{align}
    \end{proof}

    \bigskip

    \begin{mdframed}
        \underline{\textbf{Pseudoproof:}}

        \bigskip

        Let $x \in \mathbb{Z}^{+}$

        \bigskip

        We need to prove $x$ is a common divisor to both 0 and $x$, and we need
        to prove all common divisors $d_1$ of 0 and $x$ is less than or equal to $x$.

        \bigskip

        \begin{enumerate}[1.]
            \item Show $IsCD(x,0,x)$

            \bigskip

            We need to show there is $k_1 \in \mathbb{Z}$ such that
            $x = k_1 \cdot x$ and we need to show $k_2 \in \mathbb{Z}$ such that
            $0 = k_2 \cdot x$.

            \bigskip

            Let $k_1 = 1$ and $k_2 = 0$.

            \bigskip

            \begin{itemize}
                \item Show $x = k_1 \cdot x$ and $0 = k_2 \cdot 0$

                \begin{mdframed}
                Then, we can calculate that

                \begin{align}
                    x &= 1 \cdot x = k_1 \cdot x\\
                    0 &= 0 \cdot x = k_2 \cdot x
                \end{align}
                \end{mdframed}
            \end{itemize}

            \item Show $\forall d_1 \in \mathbb{Z}, IsCD(x,0,d_1) \Rightarrow d_1 \leq x$

            \bigskip

            Let $d_1 \in \mathbb{Z}$ and assume $d_1 \mid x$ and $d_1 \mid 0$.

            \bigskip

            We need to show $d_1 \leq x$.

            \bigskip

            \begin{enumerate}[1.]
                \item Use fact `$\forall n \in \mathbb{Z}^+,\:\forall d \in \mathbb{Z},\:d \mid n \Rightarrow d \leq n$' to show $d_1 \leq x$.

                \begin{mdframed}
                The hint tells us

                \begin{align}
                    \forall n \in \mathbb{Z}^+,\:\forall d \in \mathbb{Z},\:d \mid n \Rightarrow d \leq n
                \end{align}

                \bigskip

                Because we know from assumption that $d_1 \mid x$, by using the hint,
                we can conclude

                \begin{align}
                    d_1 \leq x
                \end{align}
                \end{mdframed}
            \end{enumerate}

        \end{enumerate}

    \end{mdframed}

    \item $\forall a,b \in \mathbb{Z},\:(a \neq 0) \lor (b \neq 0) \Rightarrow \exists
    p,q \in \mathbb{Z},\:pa+qb = gcd(a,b)$

\end{enumerate}

\section*{Question 2}
\begin{enumerate}[a.]
    \item

    \begin{proof}
        \bigskip

        Assume $Even(n)$. That is $\exists k \in \mathbb{Z},\:n = 2k$.

        \bigskip

        We need to show there is an integer $k_1$ such that $n^2 - 3n = 2k_1$.

        \bigskip

        Let $k_1 = (2k^2 - 3k)$.

        \bigskip

        The assumption tells us $n = 2k$.

        \bigskip

        Then, by using this fact, we can write
        \setcounter{equation}{0}
        \begin{align}
            n^2 - 3n &= (2k)^2 - 3(2k)\\
            &= 4k^2 - 6k\\
            &= 2(2k^2 - 3k)\\
            &= 2k_1
        \end{align}

    \end{proof}

    \bigskip

    \begin{mdframed}
        \underline{\textbf{Pseudoproof:}}

        \bigskip

        Assume $Even(n)$. That is $\exists k \in \mathbb{Z},\:n = 2k$.

        \bigskip

        We need to show there is an integer $k_1$ such that $n^2 - 3n = 2k_1$.

        \bigskip

        Let $k_1 = (2k^2 - 3k)$.

        \bigskip

        \begin{itemize}
            \item Show $n^2 - 3n = 2k_1$ by using assumption.

            \bigskip

            \begin{mdframed}
            The assumption tells us $n = 2k$.

            \bigskip

            Then, by using this fact, we can write

            \begin{align}
                n^2 - 3n &= (2k)^2 - 3(2k)\\
                &= 4k^2 - 6k\\
                &= 2(2k^2 - 3k)\\
                &= 2k_1
            \end{align}
            \end{mdframed}

        \end{itemize}

    \end{mdframed}

    \item

    \begin{proof}

        In this case, assume $Odd(n)$. That is $\exists k \in \mathbb{Z},\:n = 2k - 1$.

        \bigskip

        We need to show there is an integer $k_1$ such that $n^2 - 3n = 2k_1$.

        \bigskip

        Let $k_1 = (2k^2 - 5k + 2)$.

        \bigskip

        The assumption tells us $n = 2k-1$.

        \bigskip

        Then, by using this fact, we can write

        \setcounter{equation}{0}
        \begin{align}
            n^2 - 3n &= (2k-1)^2 - 3(2k-1)\\
            &= 4k^2-4k+1-6k+3\\
            &= 4k^2 -10k + 4\\
            &= 2(2k^2 - 5k + 2)\\
            &= 2k_1
        \end{align}

    \end{proof}

    \bigskip

    \begin{mdframed}
        \underline{\textbf{Pseudoproof:}}

        \bigskip

        Assume $Odd(n)$. That is $\exists k \in \mathbb{Z},\:n = 2k - 1$.

        \bigskip

        We need to show there is an integer $k_1$ such that $n^2 - 3n = 2k_1$.

        \bigskip

        Let $k_1 = (2k^2 - 5k + 2)$.

        \bigskip

        \begin{itemize}
            \item Show $n^2 - 3n = 2k_1$ by using assumption.

            \bigskip

            \begin{mdframed}
            The assumption tells us $n = 2k-1$.

            \bigskip

            Then, by using this fact, we can write

            \begin{align}
                n^2 - 3n &= (2k-1)^2 - 3(2k-1)\\
                &= 4k^2-4k+1-6k+3\\
                &= 4k^2 -10k + 4\\
                &= 2(2k^2 - 5k + 2)\\
                &= 2k_1
            \end{align}
            \end{mdframed}

        \end{itemize}

    \end{mdframed}

    \bigskip

    \textbf{Notes:}

    \begin{itemize}
        \item Noticed professor uses predicate logic when expanding definition in
        assumption.

        \begin{mdframed}
        Assume that $n$ is odd, i.e. $\exists k \in \mathbb{Z},\:n = 2k -1$.
        \end{mdframed}

    \end{itemize}

\end{enumerate}

\section*{Question 3}
\begin{enumerate}[a.]
    \item $\forall a,b \in \mathbb{N}$, $Prime(b) \Rightarrow 1 \geq gcd(a,b) \lor gcd(a,b) \geq b$
    \item

    \bigskip

    \begin{mdframed}
        \underline{\textbf{Pseudoproof:}}

        \bigskip

        Let $a,b \in \mathbb{N}$. Assume $Prime(b)$. That is, $p > 1 \land (\forall
        d \in \mathbb{N},\:d \mid p \Rightarrow d = 1 \lor d = p)$.

        \bigskip

        We will prove $1 \geq gcd(a,b)$ or $gcd(a,b) \geq b$ using proof by cases.

        \bigskip

        \textbf{Case 1 ($b \mid a$):}

        \bigskip

        In this case, assume $b$ divides $a$. That is, $\exists k \in \mathbb{Z},\:a = kb$.

        \bigskip

        We need to prove $gcd(a,b) \geq b$.

        \bigskip

        \begin{enumerate}[1.]
            \item Show $IsGCD(a,b,b)$, i.e. $ IsCD(a,b,b) \land (\forall d_1 \in \mathbb{Z},\:IsCD(a,b,d_1) \Rightarrow d_1 \leq b))$

            \bigskip

            First, we need to show $b$ is the greatest common divisor to both $a$ and $b$. That is,
            $IsCD(a,b,b) \land (\forall d_1 \in \mathbb{Z},\:IsCD(a,b,d_1) \Rightarrow d_1 \leq b))$

            \bigskip

            \begin{itemize}
                \item Show $IsCD(a,b,b)$

                \bigskip

                \begin{mdframed}
                Starting with showing $IsCD(a,b,b)$, the assumption tells us
                $b \mid a$, and we know $b \mid b$.

                \bigskip

                Then, it follows from these facts that $b$ is a common divisor to
                both $a$ and $b$.

                \end{mdframed}

                \item Show $\forall d_1 \in \mathbb{Z},\:IsCD(a,b,d_1) \Rightarrow d_1 \leq b$

                \bigskip

                \begin{mdframed}

                Next for showing $(\forall d_1 \in \mathbb{Z},\:IsCD(a,b,d_1)
                \Rightarrow d_1 \leq b))$, the definition of prime number
                tells us $b$ has two non-negative divisors 1 and $b$.

                \bigskip

                Because we know $1 \mid a$ and $b \mid a$, we can conclude 1 and
                $b$ are the only non-negative common divisor to both $a$ and $b$.

                \bigskip

                Since $1 < b$, $b = b$ and all other common divisors are less than 0,
                we can conclude all common divisors to both $a$ and $b$ are less than or equal to $b$.

                \end{mdframed}
            \end{itemize}

            \item Show $b \leq gcd(a,b)$ by using the fact $b \mid b$ and $\forall n \in
            \mathbb{Z}^{+},\:\forall d \in \mathbb{Z},\:d \mid n \Rightarrow d \leq n$.

            \begin{mdframed}

            Now, we need to show $b \leq gcd(a,b)$.

            \bigskip

            The hint from question 1.c tells us

            \begin{align}
                \forall n \in \mathbb{Z}^{+},\:\forall d \in \mathbb{Z},\:d \mid n \Rightarrow d \leq n
            \end{align}

            Since we know $b$ divides $b$, by using this fact, we can write

            \begin{align}
                b &\leq b
            \end{align}

            Because we know $b = gcd(a,b)$, we can conclude

            \begin{align}
                b &\leq gcd(a,b)
            \end{align}

            \end{mdframed}
        \end{enumerate}

        \bigskip

        \textbf{Case 2 ($b \nmid a$):}

        \bigskip

        In this case, assume $b$ doesn't divide $a$.

        \bigskip

        We need to prove $1 \geq gcd(a,b)$.

        \bigskip

        \begin{enumerate}[1.]
            \item Show $IsGCD(a,b,1)$

            First, we need to show 1 is the greatest common divisor to both $a$
            and $b$.

            \begin{itemize}
                \item Find all possible common divisors to both $a$ and $b$.

                \bigskip

                \begin{mdframed}

                The assumption tells us $b$ is a prime number, and so, from definition,
                we know $b$ has two non-negative divisors 1 and $b$.

                \end{mdframed}

                \item Show 1 is the only common divisor to $a$ and $b$.

                \begin{mdframed}

                Because we know $b \nmid a$ from assumption and $1 \mid a$, we
                can conclude 1 is the only non-negative common divisor to both
                $a$ and $b$.

                \end{mdframed}

                \item Conclude $gcd(a,b) = 1$.
                \begin{mdframed}

                Because we know all common divisors to both $a$ and $b$
                are less than or equal to 1, we can conclude $gcd(a,b) = 1$.

                \end{mdframed}
            \end{itemize}

            \item Show $gcd(a,b) \leq 1$ by using the fact $\forall n \in
            \mathbb{Z}^{+},\:\forall d \in \mathbb{Z},\:d \mid n \Rightarrow d \leq n$.

            \begin{mdframed}

            Now, we need to show $1 \geq gcd(a,b)$.

            \bigskip

            The hint from question 1.c tells us

            \begin{align}
                \forall n \in \mathbb{Z}^{+},\:\forall d \in \mathbb{Z},\:d \mid n \Rightarrow d \leq n
            \end{align}

            Since we know 1 divides 1, by using this fact, we can write

            \begin{align}
                1 &\geq 1
            \end{align}

            Because we know $1 = gcd(a,b)$, we can conclude

            \begin{align}
                1 &\geq gcd(a,b)
            \end{align}

            \end{mdframed}

        \end{enumerate}

    \end{mdframed}

    \bigskip

    \textbf{Notes:}

    \begin{itemize}
        \item $Prime(p):\:p > 1 \land (\forall d \in \mathbb{N},\:d \mid p \Rightarrow d = 1 \lor d = p)$
    \end{itemize}
\end{enumerate}

\end{document}