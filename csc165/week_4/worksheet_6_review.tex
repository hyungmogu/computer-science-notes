\documentclass[12pt]{article}
\usepackage{enumerate}
\usepackage{amsfonts}
\usepackage{amsmath}
\usepackage{fancyhdr}
\usepackage{amssymb}

\begin{document}
\title{Worksheet 6 Review}
\maketitle

\section*{Question 1}
\begin{enumerate}[a.]
    \item

    $\forall n \in \mathbb{N},\:P(123) \land \neg (n > 123 \Rightarrow P(n))$

    \bigskip

    \textbf{Correct Solution:}

    $\:P(123) \land (\forall n \in \mathbb{N}, P(n) \Rightarrow n \leq 123)$

    \item

    $IsCD(x,y):\:\exists x,y,d \in \mathbb{Z},\: d \mid x \land d \mid y$

    \bigskip

    $IsGCD(x,y):\:\exists x,y,d \in \mathbb{Z}, d \mid x \land d \mid y \land
    (\forall n \in \mathbb{N}, n > d \Rightarrow n \nmid x \lor n \nmid y)$

    \bigskip

    \textbf{Correct Solution:}

    $IsGCD(x,y,d):\exists x,y,d \in \mathbb{Z}, (x = 0 \land y = 0 \Rightarrow d = 0) \land (x \neq 0 \lor y \neq 0 \Rightarrow
    IsCD(x,y,d) \Rightarrow \forall d' \in \mathbb{Z}, IsCD(x,y,d') \Rightarrow d' \leq d)$

    \item

    Let $a = x$, $b = 0$, $d = x$ and $d' \in \mathbb{Z}$. Assume $IsCD(x,y,d')$.

    \bigskip

    Because we know $x \mid x$ and $x \mid 0$, we can conclude that $d$ is a common
    divisor to $a$ and $b$.

    \bigskip

    Since $d' \mid a$ and $d' \mid b$, and since $\forall n \in \mathbb{Z}^{+}$,
    $\forall d \in \mathbb{Z},\:d \mid n \Rightarrow d \leq n$, we can conclude
    that

    \begin{align}
        d' &\leq a
    \end{align}

    \bigskip

    Then,

    \begin{align}
        d' \leq d
    \end{align}

    by the fact that $d = a$.

    \bigskip

    Then it follows from above that the statement $\forall x \in \mathbb{Z}^{+}$,
    $IsGCD(x,0,x)$ is true.

    \item

    \textbf{Attempt 1:}

    $a,b \in \mathbb{Z},\:a \neq 0 \lor b \neq 0 \Rightarrow (\exists d \in \mathbb{Z},
    \:d=GCD(a,b) \Rightarrow \forall d' \in \mathbb{Z}^{+}, \exists p,q \in \mathbb{Z})$

    \textbf{Attempt 2:}

    $a,b \in \mathbb{Z},\:\exists d \in \mathbb{Z},\:(a \neq 0 \lor b \neq 0)
    \land d=GCD(a,b) \Rightarrow \exists p,q \in \mathbb{Z},\:d = ap + bq \land
    d > 0 \land (\forall d' \in \mathbb{Z}^{+}, d' = ap + bq \Rightarrow d' \geq d))$

\end{enumerate}

\section*{Question 2}
\begin{enumerate}[a.]
    \item

    Let $n \in \mathbb{Z}$. Assume that $\exists k \in \mathbb{Z}, n = 2k$.

    \bigskip

    Then,
    \setcounter{equation}{0}
    \begin{align}
        n^2 - 3n &= 4n^2 - 6n\\
        &= 2(n^2 - 3n)\\
        &= 2m
    \end{align}

    where $m = n^2 - 3n \in \mathbb{Z}$.

    \bigskip

    Then, by definition of even number, $n^2 - 3n$ is even.

    \item

    Let $n \in \mathbb{Z}$. Assume $\exists k \in \mathbb{Z},\:n = 2k - 1$.

    \bigskip

    Then,
    \setcounter{equation}{0}
    \begin{align}
        n^2 - 3n &= (2k - 1)^2 - 3 \cdot (2k - 1)\\
        &= 4k^2 - 4k + 1 -6k +3\\
        &= 4k^2 - 10k + 4\\
        &= 2(2k^2 - 5k + 2)\\
        &= 2m
    \end{align}

    where $m = 2k^2 - 5k + 2 \in \mathbb{Z}$.

    \bigskip

    Then, it follows from the definition of even number that $n^2-3n$ is even.
\end{enumerate}

\section*{Question 3}
\begin{enumerate}[a.]
    \item

    $\forall a,b \in \mathbb{N},\:Prime(b) \Rightarrow 1 \geq gcd(a,b) \lor
    gcd(a,b) \geq b$

    \item

    Let $a,b \in \mathbb{N}$. Assume $Prime(b)$.

    \bigskip

    We will prove the statement by considering two cases, when $b \mid a$, and
    when $b \nmid a$.

    \bigskip

    \textbf{Case 1} ($b \mid a$):

    \bigskip

    Assume $b \mid a$.

    \bigskip

    Since $b$ is a prime number, there are to possible divisors $b$ and 1.

    \bigskip

    Since $b \mid a$ and $b \mid b$, $b = gcd(a,b)$.

    \bigskip

    Since $b \mid gcd(a,b)$, by the fact $\forall n \in \mathbb{Z}^{+},
    d \in \mathbb{Z},\: d \mid n \Rightarrow d \leq n$, we can conclude that

    \setcounter{equation}{0}
    \begin{align}
        b &\leq gcd(a,b)
    \end{align}

    \textbf{Case 2} ($b \nmid a$):

    \bigskip

    Assume $b \nmid a$.

    \bigskip

    Since $b$ is a prime number, $b$ has two divisors $b$ and $1$.

    \bigskip

    Since $b \nmid a$ and $1 \mid a$, $1 = gcd(a,b)$.

    \bigskip

    Since $gcd(a,b) \mid 1$, by the fact $\forall n \in \mathbb{Z}^{+},
    d \in \mathbb{Z},\: d \mid n \Rightarrow d \leq n$, we can conclude that

    \setcounter{equation}{0}
    \begin{align}
        1 &\geq gcd(a,b)
    \end{align}

\end{enumerate}

\end{document}