\documentclass[12pt]{article}
\usepackage{enumerate}
\usepackage{amsfonts}
\usepackage{amsmath}
\usepackage{fancyhdr}
\usepackage{amssymb}
\usepackage{mdframed}

\begin{document}
\title{Worksheet 7 Review 2}
\maketitle

\section*{Question 1}
\begin{enumerate}[a.]
    \item

    In this case assume that $n \leq 1$.

    \bigskip

    We want to show $n \leq 1$.

    \bigskip

    Since the assumption tells us $n \leq 1$, we can conclude this is true.

    \item

    \begin{mdframed}

    \underline{\textbf{Pseudoproof:}}

    \bigskip

    Let $a = d$ and $b = k$. Assume there exists $d \in \mathbb{N}$ where
    $(\exists k \in \mathbb{Z}, n = dk) \land d \neq 1 \land d \neq n$. Assume $n > 1$

    \bigskip

    We need to prove that $n \nmid a$, $n \nmid b$ and $n \mid ab$.

    \bigskip

    \begin{enumerate}[1.]
        \item Show $n \nmid a$.

        \bigskip

        First, we need to show $n \nmid a$.

        \bigskip

        \begin{enumerate}[1.]

            \item Show $n \geq d$.

            \bigskip

            \begin{mdframed}
            The fact 2 tells us

            \begin{align}
                \forall x,y \in \mathbb{N}, y \geq 1 \land x \mid y \Rightarrow 1 \leq x \leq y
            \end{align}

            and we know from headers that $d \mid n$, $n > 1$, and $n,d \in \mathbb{N}$.

            \bigskip

            Then, by using these facts, we can write

            \begin{align}
                1 \leq d \leq n
            \end{align}

            \end{mdframed}

            \item Show that for $n$ to divide $d$, $n = d$.

            \begin{mdframed}
            Now, the definition of divisibility tells us for $n$ to
            divide $d$, there must be some $k_1 \in \mathbb{Z}$ such that
            $d$ is equal to $k_1 \cdot n$.

            \bigskip

            Then, since we know $n \geq d$, by using these facts, we can conclude
            the definition of divisibility is satisfied when $k_1 = 1$, or
            when $n = d$.

            \end{mdframed}

            \item Conclude $n \nmid a$.

            \begin{mdframed}
            Then, since we know from header that $n \neq d$, we can conclude
            $n \nmid d$.
            \end{mdframed}
        \end{enumerate}

        \begin{mdframed}
        First, we need to show $n \nmid a$.

        \bigskip

        The fact 2 tells us

        \begin{align}
            \forall x,y \in \mathbb{N}, y \geq 1 \land x \mid y \Rightarrow 1 \leq x \leq y
        \end{align}

        and we know from headers that $d \mid n$, $n > 1$, and $n,d \in \mathbb{N}$.

        \bigskip

        Then, by using these facts, we can write

        \begin{align}
            1 \leq d \leq n
        \end{align}

        \bigskip

        Now, the definition of divisibility tells us for $n$ to
        divide $d$, there must be some $k_1 \in \mathbb{Z}$ such that
        $d$ is equal to $k_1 \cdot n$.

        \bigskip

        Then, since we know $n \geq d$, by using these facts, we can conclude
        the definition of divisibility is satisfied only when $k_1 = 1$, or
        when $n = d$.

        \bigskip

        Then, since we know from the header that $n \neq d$, we can conclude
        $n \nmid d$.

        \bigskip

        Then, since we know $d = a$ from the header, we can conclude $n \nmid a$.

        \end{mdframed}

        \item Show $n \nmid b$
        \begin{itemize}

            \item Show $k \mid n$

            \item Show $k \geq 1$.

            \begin{mdframed}
            The header tells us $n > 1$ $d \geq 0$, and we know from
            assumption that $n = dk$.

            \bigskip

            Since the facts tell us $k \leq 0$ results in $n \leq 0$ and this
            cannot happen, we can conclude $k \geq 1$.

            \end{mdframed}

            \item Show $n \geq k$ using the fact $k \mid n$ and $k \geq 1$.

            \bigskip

            \begin{mdframed}
            The fact 2 tells us

            \begin{align}
                \forall x,y \in \mathbb{N}, y \geq 1 \land x \mid y \Rightarrow 1 \leq x \leq y
            \end{align}

            and we know from headers that $d \mid n$, $n > 1$, and $n,d \in \mathbb{N}$.

            \bigskip

            Then, by using these facts, we can write

            \begin{align}
                1 \leq d \leq n
            \end{align}

            \end{mdframed}

            \item Show that for $n$ to divide $k$, $n = k$.

            \begin{mdframed}
            Now, the definition of divisibility tells us for $n$ to
            divide $k$, there must be some $k_1 \in \mathbb{Z}$ such that
            $d$ is equal to $k_1 \cdot n$.

            \bigskip

            Then, since we know $n \geq d$, by using these facts, we can conclude
            the definition of divisibility is satisfied when $k_1 = 1$, or
            when $n = d$.

            \end{mdframed}

            \item Conclude $n \nmid b$.

            \begin{mdframed}
            Then, since we know from header that $n \neq d$, we can conclude
            $n \nmid d$.
            \end{mdframed}


        \end{itemize}
        \item Show $n \mid ab$

    \end{enumerate}

    \end{mdframed}


\end{enumerate}

\section*{Question 2}

\section*{Question 3}

\end{document}