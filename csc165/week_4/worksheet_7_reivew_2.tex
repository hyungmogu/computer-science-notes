\documentclass[12pt]{article}
\usepackage[margin=2.5cm]{geometry}
\usepackage{enumerate}
\usepackage{amsfonts}
\usepackage{fancyhdr}
\usepackage{amsmath}
\usepackage{amssymb}
\usepackage{amsthm}
\usepackage{mdframed}
\usepackage[utf]{kotex}

\begin{document}
\title{Worksheet 7 Review 2}
\author{Hyungmo Gu}
\maketitle

\section*{Question 1}
\begin{enumerate}[a.]
    \item

    In this case assume that $n \leq 1$.

    \bigskip

    We want to show $n \leq 1$.

    \bigskip

    Since the assumption tells us $n \leq 1$, we can conclude this is true.

    \item

    \begin{proof}

    Let $a = d$ and $b = k$. Assume there exists $d \in \mathbb{N}$ where
    $(\exists k \in \mathbb{Z}, n = dk) \land d \neq 1 \land d \neq n$. Assume $n > 1$

    \bigskip

    We need to prove that $n \nmid a$, $n \nmid b$ and $n \mid ab$.

    \bigskip

    We will do so in parts.

    \bigskip

    \underline{\textbf{Part 1 (Proving $n \nmid a$):}}

    \bigskip

    We need to prove $n \nmid a$.

    \bigskip

    First, we need to show $n \geq d$.

    \bigskip

    The fact 2 tells us

    \begin{align}
        \forall x,y \in \mathbb{N}, y \geq 1 \land x \mid y \Rightarrow 1 \leq x \leq y
    \end{align}

    and we know from headers that $d \mid n$, $n > 1$, and $n,d \in \mathbb{N}$.

    \bigskip

    Then, by using these facts, we can write

    \begin{align}
        1 \leq d \leq n
    \end{align}

    \bigskip

    Second, we need to show $n = d$.

    \bigskip

    The definition of divisibility tells us for $n$ to
    divide $d$, there must be some $k_1 \in \mathbb{Z}$ such that
    $d$ is equal to $k_1 \cdot n$.

    \bigskip

    Then, since we know $n \geq d$, by using these facts, we can conclude
    the definition of divisibility is satisfied only when $k_1 = 1$, or
    when $n = d$.

    \bigskip

    Finally, since we know from the header that $n \neq d$, we can conclude
    $n \nmid d$.

    \bigskip

    Then, since we know $d = a$ from the header, we can conclude $n \nmid a$.

    \bigskip

    \underline{\textbf{Part 2 (Proving $n \nmid b$):}}

    \bigskip

    We need to prove $n \nmid b$.

    \bigskip

    First, we need to show $k \mid n$.

    \bigskip

    The assumption tells us $n = kd$.

    \bigskip

    Then, it follows from the definition of divisibility that $k \mid d$.

    \bigskip

    Second, we need to show $k \geq 1$.

    \bigskip

    The header tells us $n > 1$ $d \geq 0$, and we know from
    assumption that $n = dk$.

    \bigskip

    Since the facts tell us $k \leq 0$ results in $n \leq 0$ and this
    cannot happen, we can conclude $k \geq 1$.

    \bigskip

    Third, we need to show $n \geq k$ using the fact $k \geq 1$ and $k \mid n$.

    \bigskip

    The fact 2 tells us

    \begin{align}
        \forall x,y \in \mathbb{N}, y \geq 1 \land x \mid y \Rightarrow 1 \leq x \leq y
    \end{align}

    \bigskip

    Since we know $k \mid n$, $n > 1$ and $k,n \in \mathbb{N}$, we can conclude
    $k \leq n$.

    \bigskip

    Fourth, we need to show $n = k$.

    \bigskip

    The definition of divisibility tells us for $n$ to
    divide $k$, there must be some $k_1 \in \mathbb{Z}$ such that
    $k$ is equal to $k_1 \cdot n$.

    \bigskip

    Then, using the fact $n \geq k$, we can conclude the definition of
    divisibility is satisfied only when $k_1 = 1$, or $n = k$.

    \bigskip

    Finally, since we know from the header that $n \neq k$, we can conclude
    $n \nmid k$.

    \bigskip

    Then, it follows from the fact $k = b$, we can conclude $n \nmid b$.

    \bigskip

    \underline{\textbf{Part 3 (Proving $n \mid ab$):}}

    \bigskip

    We need to prove $n \mid ab$.

    \bigskip

    The fact 1 tells us

    \begin{align}
        \forall x \in \mathbb{Z}, x \mid x
    \end{align}

    \bigskip

    Since we know $n \in \mathbb{N}$, we can conclude $n \mid n$.

    \bigskip

    Then, since we know $n = dk$, $d = a$ and $k = b$, we can conclude
    $n \mid ab$.

    \end{proof}

    \bigskip

    \begin{mdframed}

    \underline{\textbf{Pseudoproof:}}

    \bigskip

    Let $a = d$ and $b = k$. Assume there exists $d \in \mathbb{N}$ where
    $(\exists k \in \mathbb{Z}, n = dk) \land d \neq 1 \land d \neq n$. Assume $n > 1$

    \bigskip

    We need to prove that $n \nmid a$, $n \nmid b$ and $n \mid ab$.

    \bigskip

    We will do so in parts.

    \bigskip

    \begin{enumerate}[1.]
        \item Show $n \nmid a$.

        \bigskip

        First, we need to show $n \nmid a$.

        \bigskip

        \begin{enumerate}[1.]

            \item Show $n \geq d$.

            \bigskip

            \begin{mdframed}
            The fact 2 tells us

            \begin{align}
                \forall x,y \in \mathbb{N}, y \geq 1 \land x \mid y \Rightarrow 1 \leq x \leq y
            \end{align}

            and we know from headers that $d \mid n$, $n > 1$, and $n,d \in \mathbb{N}$.

            \bigskip

            Then, by using these facts, we can write

            \begin{align}
                1 \leq d \leq n
            \end{align}

            \end{mdframed}

            \item Show that for $n$ to divide $d$, $n = d$.

            \begin{mdframed}
            Now, the definition of divisibility tells us for $n$ to
            divide $d$, there must be some $k_1 \in \mathbb{Z}$ such that
            $d$ is equal to $k_1 \cdot n$.

            \bigskip

            Then, since we know $n \geq d$, by using these facts, we can conclude
            the definition of divisibility is satisfied when $k_1 = 1$, or
            when $n = d$.

            \end{mdframed}

            \item Conclude $n \nmid a$.

            \begin{mdframed}
            Then, since we know from header that $n \neq d$, we can conclude
            $n \nmid d$.
            \end{mdframed}
        \end{enumerate}

        \begin{mdframed}

        \underline{\textbf{Part 1 (Proving $n \nmid a$):}}

        \bigskip

        We need to prove $n \nmid a$.

        \bigskip

        First, we need to show $n \geq d$.

        \bigskip

        The fact 2 tells us

        \begin{align}
            \forall x,y \in \mathbb{N}, y \geq 1 \land x \mid y \Rightarrow 1 \leq x \leq y
        \end{align}

        and we know from headers that $d \mid n$, $n > 1$, and $n,d \in \mathbb{N}$.

        \bigskip

        Then, by using these facts, we can write

        \begin{align}
            1 \leq d \leq n
        \end{align}

        \bigskip

        Second, we need to show $n = d$.

        \bigskip

        The definition of divisibility tells us for $n$ to
        divide $d$, there must be some $k_1 \in \mathbb{Z}$ such that
        $d$ is equal to $k_1 \cdot n$.

        \bigskip

        Then, since we know $n \geq d$, by using these facts, we can conclude
        the definition of divisibility is satisfied only when $k_1 = 1$, or
        when $n = d$.

        \bigskip

        Finally, since we know from the header that $n \neq d$, we can conclude
        $n \nmid d$.

        \bigskip

        Then, since we know $d = a$ from the header, we can conclude $n \nmid a$.

        \end{mdframed}

        \item Show $n \nmid b$
        \begin{itemize}

            \item Show $k \mid n$

            \bigskip

            First, we need to show $k \mid n$.

            \bigskip

            \begin{itemize}
                \item State $n = kd$.
                \begin{mdframed}
                The assumption tells us $n = kd$.
                \end{mdframed}

                \item Show $k \mid n$ by using the definition of divisibility
                \begin{mdframed}
                Then, it follows from the definition of divisibility that $k \mid d$.
                \end{mdframed}
            \end{itemize}

            \bigskip

            \begin{mdframed}

            First, we need to show $k \mid n$.

            \bigskip

            The assumption tells us $n = kd$.

            \bigskip

            Then, it follows from the definition of divisibility that $k \mid d$.
            \end{mdframed}

            \item Show $k \geq 1$.

            \bigskip

            Second, we need to show $k \geq 1$.

            \bigskip

            \begin{mdframed}
            Second, we need to show $k \geq 1$.

            \bigskip

            The header tells us $n > 1$ $d \geq 0$, and we know from
            assumption that $n = dk$.

            \bigskip

            Since the facts tell us $k \leq 0$ results in $n \leq 0$ and this
            cannot happen, we can conclude $k \geq 1$.

            \end{mdframed}

            \item Show $n \geq k$ using the fact $k \mid n$ and $k \geq 1$.

            \bigskip

            Third, we need to show $n \geq k$.

            \bigskip

            \begin{mdframed}
            \bigskip

            Third, we need to show $n \geq k$.

            \bigskip

            The fact 2 tells us

            \begin{align}
                \forall x,y \in \mathbb{N}, y \geq 1 \land x \mid y \Rightarrow 1 \leq x \leq y
            \end{align}

            \bigskip

            Since we know $k \mid n$, $n > 1$ and $k,n \in \mathbb{N}$, we can conclude
            $k \leq n$.

            \end{mdframed}

            \item Show that for $n$ to divide $k$, $n = k$.

            \bigskip

            Fourth, we need to show $n = k$.

            \bigskip

            \begin{mdframed}
            Fourth, we need to show $n = k$.

            \bigskip

            The definition of divisibility tells us for $n$ to
            divide $k$, there must be some $k_1 \in \mathbb{Z}$ such that
            $k$ is equal to $k_1 \cdot n$.

            \bigskip

            Then, using the fact $n \geq k$, we can conclude the definition of
            divisibility is satisfied only when $k_1 = 1$, or $n = k$.

            \end{mdframed}

            \item Conclude $n \nmid a$.

            \begin{mdframed}
            Finally, since we know from the header that $n \neq k$, we can conclude
            $n \nmid k$.

            \bigskip

            It follows from the fact $k = b$, we can conclude $n \nmid b$.
            \end{mdframed}

        \end{itemize}

        \begin{mdframed}

        \underline{\textbf{Part 2 (Proving $n \nmid b$):}}

        \bigskip

        We need to show $n \nmid b$.

        \bigskip

        First, we need to show $k \mid n$.

        \bigskip

        The assumption tells us $n = kd$.

        \bigskip

        Then, it follows from the definition of divisibility that $k \mid d$.

        \bigskip

        Second, we need to show $k \geq 1$.

        \bigskip

        The header tells us $n > 1$ $d \geq 0$, and we know from
        assumption that $n = dk$.

        \bigskip

        Since the facts tell us $k \leq 0$ results in $n \leq 0$ and this
        cannot happen, we can conclude $k \geq 1$.

        \bigskip

        Third, we need to show $n \geq k$ using the fact $k \geq 1$ and $k \mid n$.

        \bigskip

        The fact 2 tells us

        \begin{align}
            \forall x,y \in \mathbb{N}, y \geq 1 \land x \mid y \Rightarrow 1 \leq x \leq y
        \end{align}

        \bigskip

        Since we know $k \mid n$, $n > 1$ and $k,n \in \mathbb{N}$, we can conclude
        $k \leq n$.

        \bigskip

        Fourth, we need to show $n = k$.

        \bigskip

        The definition of divisibility tells us for $n$ to
        divide $k$, there must be some $k_1 \in \mathbb{Z}$ such that
        $k$ is equal to $k_1 \cdot n$.

        \bigskip

        Then, using the fact $n \geq k$, we can conclude the definition of
        divisibility is satisfied only when $k_1 = 1$, or $n = k$.

        \bigskip

        Finally, since we know from the header that $n \neq k$, we can conclude
        $n \nmid k$.

        \bigskip

        Then, it follows from the fact $k = b$, we can conclude $n \nmid b$.

        \end{mdframed}


        \item Show $n \mid ab$

        We need to show $n \mid ab$.

        \begin{itemize}
            \item State fact 1

            \begin{mdframed}
            The fact 1 tells us

            \begin{align}
                \forall x \in \mathbb{Z}, x \mid x
            \end{align}
            \end{mdframed}

            \item Show $n \mid n$

            \begin{mdframed}
            Since we know $n \in \mathbb{N}$, we can conclude $n \mid n$.
            \end{mdframed}

            \item Show $n \mid ab$ using the fact $n = dk$ where $a = d$ and $b = k$.

            \begin{mdframed}
            Then, since we know $n = dk$, $d = a$ and $k = b$, we can conclude
            $n \mid ab$.
            \end{mdframed}
        \end{itemize}

        \begin{mdframed}

        \underline{\textbf{Part 3 (Proving $n \mid ab$):}}

        \bigskip

        We need to show $n \mid ab$.

        \bigskip

        The fact 1 tells us

        \begin{align}
            \forall x \in \mathbb{Z}, x \mid x
        \end{align}

        \bigskip

        Since we know $n \in \mathbb{N}$, we can conclude $n \mid n$.

        \bigskip

        Then, since we know $n = dk$, $d = a$ and $k = b$, we can conclude
        $n \mid ab$.
        \end{mdframed}

    \end{enumerate}

    \end{mdframed}

    \bigskip

    \textbf{Notes:}

    \begin{itemize}
        \item Made some serious errors (i.e. show n = a or n = b) :(.
        \item How can a proof be organized so it's structurally clear so moe
        3 months from now can say I understand this proof? I used first, second
        and third to show steps involved but I still feel something is missing...
        \item Can I write a predicate logic for proving $n \nmid b$ or $n \nmid a$?
        (i.e. ... $\Rightarrow n \nmid b$)?
    \end{itemize}


\end{enumerate}

\section*{Question 2}
\begin{enumerate}[a.]
    \item

    \bigskip

    \begin{proof}
        Let $m,n \in \mathbb{N}$. Assume $Prime(n)$ and $n \nmid m$.

        \bigskip

        We need to prove there are some integer numbers $r$ and $s$ such that $rn + sm = 1$.

        \bigskip

        First, we need to show $gcd(n,m) = 1$.

        \bigskip

        The fact 3 tells us
        \setcounter{equation}{0}
        \begin{align}
            \forall n,m \in \mathbb{Z},\:Prime(n) \land n \nmid m \Rightarrow gcd(n,m) = 1
        \end{align}

        \bigskip

        Because we know from assumption that $n$ is prime and $n \nmid m$, we can write
        $gcd(n,m) = 1$.

        \bigskip

        Finally, the fact 6 tells us

        \begin{align}
            \forall n,m \in \mathbb{N},\:\exists r,s \in \mathbb{Z},\:rn + sm = gcd(n,m)
        \end{align}

        \bigskip

        Since $gcd(n,m) = 1$, we can conclude

        \begin{align}
            gcd(n,m) = rn + sm  = 1
        \end{align}

    \end{proof}

    \bigskip

    \begin{mdframed}

    \underline{\textbf{Pseudoproof:}}

    \bigskip

    Let $m,n \in \mathbb{N}$. Assume $Prime(n)$ and $n \nmid m$.

    \bigskip

    We need to prove $\exists r,s \in \mathbb{Z}, rn + sm = 1$.

    \bigskip

    \begin{enumerate}[1.]
        \item Show $gcd(n,m) = 1$, using fact 3

        \bigskip

        First, we need to show $gcd(n,m) = 1$.

        \bigskip

        \begin{mdframed}
        First, we need to show $gcd(n,m) = 1$.

        \bigskip

        The fact 3 tells us

        \begin{align}
            \forall n,m \in \mathbb{Z},\:Prime(n) \land n \nmid m \Rightarrow gcd(n,m) = 1
        \end{align}

        \bigskip

        Because we know from assumption that $n$ is prime and $n \nmid m$, we can write
        $gcd(n,m) = 1$.
        \end{mdframed}

        \item Show $rn + sm = gcd(n,m) = 1$ using fact 6

        \bigskip

        \begin{mdframed}
        Finally, the fact 6 tells us

        \begin{align}
            \forall n,m \in \mathbb{N},\:\exists r,s \in \mathbb{Z},\:rn + sm = gcd(n,m)
        \end{align}

        \bigskip

        Since $gcd(n,m) = 1$, we can conclude

        \begin{align}
            gcd(n,m) = rn + sm  = 1
        \end{align}
        \end{mdframed}

    \end{enumerate}

    \end{mdframed}

    \bigskip

    \textbf{Notes:}

    \begin{itemize}
        \item Noticed that professor doesn't put $\exists$ symbols in 'we need to prove that...'.

        \begin{mdframed}
            Let $n, m \in \mathbb{N}$. Assume that $n$ is prime and that $n - m$.
            We want to prove there exist $r, s \in \mathbb{Z}$, $rn + sm = 1$.

        \end{mdframed}
        \item 형모야. 오늘도 사랑하는 내 여보 향해 화이팅 :)
        \item 오늘 캘거리에 구름이 많은데 날씨가 굉장히 밝구나.
        \item 오오오오오!!!!
    \end{itemize}

    \item

    \underline{\textbf{Contrapositive of Statement:}} $\forall n,m \in \mathbb{N},\:
    n \mid m \Rightarrow \neg Prime(n) \lor (\forall r,s \in \mathbb{Z},\:rn + sm \neq 1)$

    \bigskip

    \begin{mdframed}
        \underline{\textbf{Pseudoproof:}}

        \bigskip

        Let $n,m \in \mathbb{N}$. Assume $n \mid m$ and assume $n$ is prime, i.e
        $n > 1 \land (\forall d \in \mathbb{N}, d \mid n \Rightarrow d = 1 \lor d = n, where n \in \mathbb{N})$

        \bigskip

        We need to show for every $r,s \in \mathbb{Z}$, $rn + sm \neq 1$.

        \bigskip

        \begin{enumerate}[1.]
            \item Show $gcd(n,m) \geq 1$ using fact 4.

            First, we need to show $gcd(n,m) \geq 1$.

            \begin{mdframed}
            First, we need to show $gcd(n,m) \geq 1$.

            \bigskip

            The fact 4 tells us

            \begin{align}
                \forall n,m \in \mathbb{N},\: n \neq 0 \lor m \neq 0 \Rightarrow gcd(m,n) \geq 1
            \end{align}

            \bigskip

            Because we know from assumption that $n > 1$, we can write
            $gcd(n,m) \geq 1$.
            \end{mdframed}

            \item Show $gcd(n,m) = n$

            \bigskip

            Second, we need to show $gcd(n,m) = n$.

            \begin{mdframed}
                Second, we need to show $gcd(n,m) = n$.

                \bigskip

                The definition of greatest common divisor tells us

                \begin{align}
                    \forall n,m \in \mathbb{Z},\:IsCD(n,m,n) \land (\forall d_1 \in \mathbb{Z},\: IsCD(n,m,d_1) \Rightarrow d_1 \leq n)
                \end{align}

                \bigskip

                Because we know $n$ is a common divisor to both $n$ and $m$, and $n$
                is the highest value that divides $n$ and $m$, we can conclude
                $gcd(n,m) = n$.

            \end{mdframed}

            \item Show $gcd(n,m) \neq 1$ using definition of prime.

            \bigskip

            Third, we need to show $gcd(n,m) \neq 1$.

            \bigskip

            \begin{mdframed}

            Third, we need to show $gcd(n,m) \neq 1$.

            \bigskip

            Because we know from assumption that $n > 1$, we can conclude
            $gcd(n,m) \neq 1$.
            \end{mdframed}

            \item Show $\forall r,s \in \mathbb{Z}, rn + sm \geq n$.

            \begin{mdframed}

            \end{mdframed}

            \item Conclude $rn + sm \neq 1$ using the fact $n > 1$.

        \end{enumerate}
    \end{mdframed}

\end{enumerate}

\section*{Question 3}

\end{document}