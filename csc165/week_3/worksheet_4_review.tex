\documentclass[12pt]{article}
\usepackage{enumerate}
\usepackage{amsfonts}
\usepackage{amsmath}
\usepackage{fancyhdr}
\usepackage{amssymb}

\begin{document}
\title{Worksheet 4 Review}
\maketitle

\section*{Question 1}
\begin{enumerate}[a.]
    \item

    $\exists n \in \mathbb{N},\:n > 3 \land n^2 - 1.5n \geq 5$

    \item

    The variable is existentially quantified

    \item

    When introduced, the variable's value should be a \textbf{concrete natural number}.

    \item

    Let $n = 5$.

    \bigskip

    Then $n > 3$, and

    \begin{align}
        n^2 - 1.5n &= 25 - 7.5\\
        &= 17.5 \geq 5
    \end{align}

    \bigskip

    Then, it follows from above that the statement $\exists n \in \mathbb{N}$,
    $n > 3 \land n^2 -1.5n \geq 5$ is true.

    \item

    $\forall n \in \mathbb{N}, n > 3 \Rightarrow n^2-1.5n > 4$

    \bigskip

    $\Rightarrow$ should be used, because it allows the scoping of the set $\mathbb{N}$.

    \item

    Universally Quantified

    \item

    The variable's value should be an arbitrary natural number.

    \item

    The assumption made is $n > 3$. It is determined by seeing the lhs of
    $\Rightarrow$.

    \item

    Let $n \in \mathbb{N}$. Assume $n > 3$.

    \bigskip

    Then,
    \setcounter{equation}{0}
    \begin{align}
        n &> 3\\
        (n - 0.75)^2 &> (3 - 0.75)^2\\
        n^2 - 1.5n + 0.5625 &> 5.0625\\
        n^2 - 1.5n &> 4.5\\
        n^2 - 1.5n &> 4
    \end{align}

    \bigskip

    Then, it follows from above that the statement $\forall n \in \mathbb{N},\:
    n > 3 \Rightarrow n^2 - 1.5n > 4$.

\end{enumerate}

\section*{Question 2}
\begin{enumerate}[a.]
    \item

    $\forall n \in \mathbb{N},\:n > 5 \Rightarrow 2 \mid n \land 3 \mid n$

    \item

    $\exists n \in \mathbb{N}, (n > 5) \land (2 \nmid n \lor 3 \nmid n)$

    \item

    Let $n = 7$.

    \bigskip

    Then, $2 \nmid n \lor 3 \nmid n$.

    \bigskip

    Then, it follows from the negation that the statement $\forall n \in \mathbb{N},\:
    n >5 \Rightarrow 2 \mid n \land 3 \mid n$ is false.
\end{enumerate}

\section*{Question 3}
\begin{enumerate}[a.]
    \item

    Let $x \in \mathbb{R}$, and $y = -x + 165$.

    \bigskip

    Then, the statement $\forall x \in \mathbb{R},\: \exists y \in \mathbb{R},\:
    x + y < 165$ is true.

    \item

    Let $y = 166$, and $x \in \mathbb{N}$.

    \bigskip

    Then the statement $\exists y \in \mathbb{N},\:\forall x \in \mathbb{N},\:
    x+y >165$ is true.


\end{enumerate}


\end{document}