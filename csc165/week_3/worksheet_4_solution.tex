\documentclass[12pt]{article}
\usepackage{enumerate}
\usepackage{amsfonts}
\usepackage{amsmath}
\usepackage{fancyhdr}
\usepackage{amssymb}

\begin{document}
\title{Worksheet 4 Solution}
\maketitle

\section*{Question 1}
\begin{enumerate}[a.]
    \item $\exists n \in \mathbb{N}, (n > 3) \land (n^2-1.5n \geq 5)$
    \item The variable is existentially quantified
    \item Concrete natural number
    \item

    Let n = 5.

    Then,

    \begin{equation}
        (5)^2 - 1.5(5)
    \end{equation}

    Then,

    \begin{equation}
        25 - 7.5
    \end{equation}

    Then,

    \begin{equation}
        17.5
    \end{equation}

    which is greater than 5. So, the statement is True

    \item

    $\forall n \in \mathbb{N}, n > 3 \Rightarrow n^2 - 1.5n > 4$

    Here $\Rightarrow$ should be used because $n > 3$ is a given, and we are using it to show that the statement $n^2 - 1.5n > 4$ is True

    \item

    The variable is universally quantified

    \item

    In this proof the variable must be \textbf{arbitrary} natural number

    \item

    The assumption made is that the any natural number greater than 3 satisfies the statement $n^2 - 1.5n > 4$.

    This assumption is made since the predicate logic is the proof of an implication

    \item

    Let $n \in \mathbb{N}$ be an arbitruary number of $\mathbb{N}$, and assume $n > 3$. Then,

    \setcounter{equation}{0}
    \begin{equation}
        n^2 > 3n
    \end{equation}

    \begin{equation}
        n^2 -1.5n > 3n - 1.5n
    \end{equation}

    \begin{equation}
        n^2 -1.5n > 1.5n
    \end{equation}

    Because we know that $n > 3$, we can conclude

    \begin{equation}
        n^2 -1.5n > 1.5(3)
    \end{equation}

    \begin{equation}
        n^2 -1.5n > 4.5
    \end{equation}

    It follows that the statement $\forall n \in \mathbb{N}, n > 3 \Rightarrow n^2 - 1.5n > 4$ is true.

\end{enumerate}

\section*{Question 2}
\begin{enumerate}[a.]
    \item $\forall n \in \mathbb{N}, n > 5 \Rightarrow 2 \mid n \lor 3 \mid n$
    \item $\exists n \in \mathbb{N}, (n > 5) \land (2 \nmid n \lor 3 \nmid n)$
    \item

    Let $n = 7$.

    Since 7 is a prime number, 7 is not divisible by both 2 and 3.

    It follows from the above that the original statement $\forall n \in \mathbb{N}, n > 5 \Rightarrow 2 \mid n \lor 3 \mid n$ is False
\end{enumerate}



\end{document}