\documentclass[12pt]{article}
\usepackage{enumerate}
\usepackage{amsfonts}
\usepackage{amsmath}
\usepackage{fancyhdr}
\usepackage{amssymb}

\begin{document}
\title{Worksheet 5 Review}
\maketitle

\section*{Question 1}
\begin{itemize}
    \item

    \textbf{Predicate Logic:} $\forall x,y \in \mathbb{Z}, Odd(x) \land Odd(y)
    \Rightarrow Odd(xy)$

    \bigskip

    Let $x,y \in \mathbb{Z}$. Assume $Odd(x)$ and $Odd(y)$.

    \bigskip

    Then, $\exists k,m \in \mathbb{Z}$,

    \begin{align}
        x &= 2k - 1\\
        y &= 2m - 1
    \end{align}

    \bigskip

    Then,

    \begin{align}
        xy &= (2k-1)(2m-1)\\
        xy &= (4km - 2k - 2m + 2) - 1\\
        xy &= 2(2km - k - m + 1) - 1\\
        xy &= 2o - 1
    \end{align}

    by setting $o = 2km - k - m + 1$.

    \bigskip

    Since, $o \in \mathbb{Z}$, it follows from the definition of odd that the statement
    $\forall x,y \in \mathbb{Z}, Odd(x) \land Odd(y) \Rightarrow Odd(xy)$ is true.
\end{itemize}

\section*{Question 2}
\begin{enumerate}[a.]
    \item

    $\forall n,m \in \mathbb{Z},\:Even(n) \land Odd(m) \Rightarrow m^2 - n^2 = m + n$

    \item

    The flaw is that the value $k$ in $n = 2k$ and $m = 2k + 1$ cannot be the same.
\end{enumerate}
\section*{Question 3}
\begin{enumerate}[a.]
    \item

    $Dom(f,g): \forall n \in \mathbb{Z},\: g(n) \leq f(n)$, where $f,g: \mathbb{N}
    \to \mathbb{R}^{\geq 0}$

    \item

    Let $f(n) = 3n$, $g(n) = n$, and $n \in \mathbb{N}$.

    \bigskip

    Then,
    \setcounter{equation}{0}
    \begin{align}
        g(n) = n &\leq n + n + n\\
        &\leq 3n\\
        &\leq f(n)
    \end{align}

    Then, it follows from the definition of '\textbf{is dominated by}' that g is
    dominated by f.

    \item

    \textbf{Negation:} $\neg Dom(f,g): \exists n \in \mathbb{Z},\:g(n) > f(n)$, where
    $f,g: \mathbb{N} \to \mathbb{R}^{\geq 0}$

    \bigskip

    Let $n = 1$, $f(n) = 3n$, and $g(n) = n$.

    \bigskip

    Then,
    \setcounter{equation}{0}
    \begin{align}
        n + 165 &= (1) + 165\\
        &= 166\\
        &> 1\\
        &> (1)^2\\
        &> n^2
    \end{align}

    \bigskip

    Then it follows from the negation of $Dom(f,g)$ that g is not dominated by f.

    \item

    \textbf{Predicate Logic:} $\forall f,g:\mathbb{N} \to \mathbb{R}^{\geq 0}$,
    $f(n) = n^2 \land g(n) = n + 165 \Rightarrow (\exists m \in \mathbb{N},
    g(m) > f(m))$

    \bigskip

    Let $f,g:\mathbb{N} \to \mathbb{R}^{\geq 0}$ and $n = 1$. Assume $f(n) = n^2$,
    and $g(n) = n + 165$.

    \bigskip

    Then,
    \setcounter{equation}{0}
    \begin{align}
        g(1) = (1) + 165 &= 166\\
        &> 1\\
        &>(1)^2\\
        &> f(1)
    \end{align}

    \bigskip

    Then, it follows from above statement that g is not dominated by f.

\end{enumerate}

\section*{Question 4}
\begin{itemize}
    \item

    Let $x \in \mathbb{R}^{\geq 0}$, and $\epsilon = x - \lfloor x \rfloor$.
    Assume $x \geq 4$.

    \bigskip

    Then,

    \setcounter{equation}{0}
    \begin{align}
        (\lfloor x \rfloor)^2 &= (x - \epsilon)^2\\
        &= x^2 - 2x\epsilon + \epsilon^2
    \end{align}

    \bigskip

    Since

    \begin{align}
        x &\geq 4\\
        x^2 &\geq 4x\\
        \frac{1}{2}x^2 &\geq 2x\\
    \end{align}

    ,and

    \begin{align}
        \frac{1}{2}x^2 &\geq 2x\epsilon
    \end{align}

    by using the fact $\forall x \in \mathbb{R},\: 0 \leq x - \lfloor x \rfloor < 1$,

    \begin{align}
        (\lfloor x \rfloor)^2 &= x^2 - 2x\epsilon + \epsilon^2\\
        &\geq \frac{1}{2}x^2 + \epsilon^2\\
        &\geq \frac{1}{2}x^2
    \end{align}

    \bigskip

    Then it follows from above that the statement $\forall x \in \mathbb{R}^{\geq 0}$,
    $x \geq 4 \Rightarrow (\lfloor x \rfloor)^2 \geq \frac{1}{2}x^2$ is true.

\end{itemize}

\end{document}