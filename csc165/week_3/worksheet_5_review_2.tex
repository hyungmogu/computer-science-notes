\documentclass[12pt]{article}
\usepackage{enumerate}
\usepackage{amsfonts}
\usepackage{amsmath}
\usepackage{fancyhdr}
\usepackage{amsmath}
\usepackage{amssymb}
\usepackage{amsthm}
\usepackage{mdframed}

\begin{document}
\title{Worksheet 5 Review 2}
\maketitle

\section*{Question 1}
\begin{itemize}

    \item

    \textbf{Statement:} $\forall m,n \in \mathbb{Z},\:(\exists k_1 \in \mathbb{Z},\:m=2k_1+1)
    \land (\exists k_2 \in \mathbb{Z},\:n = 2k_2 + 1) \Rightarrow (\exists k_3 \in \mathbb{Z},\:mn = 2k_3 + 1)$

    \begin{proof}

    Let $m,n \in \mathbb{Z}$. Assume there is an integer $k_1$ such that $m = 2k_1 + 1$.
    Assume there is an integer $k_2$ such that $n = 2k_2 + 1$. Let $k_3 = (2k_1k_2) + k_1 + k_2$.

    \bigskip

    We need to prove $mn = 2k_3 + 1$.

    \bigskip

    The assumption tells us $m = 2k_1 + 1$ and $n = 2k_2 + 1$.

    \bigskip

    By using these facts and then multiplying them together, we can conclude

    \begin{align}
        mn &= (2k_1 + 1)(2k_2 + 1)\\
        &= 4k_1k_2 + 2k_1 + 2k_2 + 1\\
        &= 2[(2k_1k_2) + k_1 + k_2] + 1\\
        &= 2k_3 + 1
    \end{align}

    \end{proof}

    \bigskip

    \textbf{Notes:}

    \begin{itemize}
        \item Noticed professor pre-calculates the value of $k_3$ as roughwork
        before writing proof
        \item Noticed professor uses `That is...' when expanding definition in writing

        \bigskip

        \begin{mdframed}
            ... and assume they are both odd. That is, we assume there exists $k_1,k_2 \in \mathbb{Z}$
            such that $m = 2k_1 -1$ and $n = 2k_2 - 1$.
        \end{mdframed}

        \item Noticed professor uses 'i.e. ...' when expanding definition in writing.

        \bigskip

        \begin{mdframed}
            We need to prove that $mn$ is odd, i.e. there exists $k_3$ such that
            $mn = 2k_3 + 1$.
        \end{mdframed}

        \item Noticed professor defines the header for R.H.S of $\Rightarrow$
        operator after `We need to prove that ...'

        \bigskip

        \begin{mdframed}
            We need to prove that $mn$ is odd, i.e. there exists $k_3$ such that
            $mn = 2k_3 + 1$.

            \bigskip

            \textbf{Let $k_3 = 2k_1k_2 - k_1 - k_2 + 1$}
        \end{mdframed}


    \end{itemize}

    % \bigskip

    % \begin{mdframed}

    %     \underline{\textbf{Pseudoproof:}}

    %     \bigskip

    %     Let $m,n \in \mathbb{Z}$. Assume there is an integer $k_1$ such that $m = 2k_1 + 1$.
    %     Assume there is an integer $k_2$ such that $n = 2k_2 + 1$. Let $k_3 = (2k_1k_2) + k_1 + k_2$.

    %     \bigskip

    %     We need to prove $mn = 2k_3 + 1$.

    %     \begin{enumerate}[1.]
    %         \item Show $mn = 2k_3 + 1$
    %         \begin{itemize}
    %             \item Show by multiplying $m = 2k_1 + 1$ and $n = 2k_2 + 1$ together.

    %             \begin{mdframed}
    %                 \begin{align}
    %                 mn &= (2k_1 + 1)(2k_2 + 1)\\
    %                 &= 4k_1k_2 + 2k_1 + 2k_2 + 1\\
    %                 &= 2[(2k_1k_2 + k_1 + k_2)] + 1
    %                 &= 2k_3 + 1
    %                 \end{align}
    %             \end{mdframed}
    %         \end{itemize}
    %     \end{enumerate}

    % \end{mdframed}

\end{itemize}

\section*{Question 2}
\begin{enumerate}[a.]
    \item

    \textbf{Predicate Logic:} $\forall m,n \in \mathbb{Z},\:Even(m) \land Odd(n) \Rightarrow m^2 - n^2 = m + n$

    \bigskip

    \textbf{Predicate Logic Expanded:} $\forall m,n \in \mathbb{Z},\:(\exists k_1
    \in \mathbb{Z},\:m = 2k_1) \land (\exists k_2 \in \mathbb{Z},\: n = 2k_2 + 1)
    \Rightarrow m^2 - n^2 = m + n$

    \item The value of $k$ used for $m$ and $n$ must not be under the same variable.

\end{enumerate}

\section*{Question 3}
\begin{enumerate}[a.]
    \item $Dom(f,g):\:\forall n \in \mathbb{N},\:g(n) \leq f(n)$, where $f,g:\mathbb{N} \to \mathbb{R}^{\geq 0}$

    \bigskip

    \textbf{Notes:}

    \begin{itemize}
        \item \textbf{Definition of is Dominated By:} Let $f,g:\mathbb{N} \to
        \mathbb{R}^{\geq 0}$. We say that $g$ is \textbf{is dominated by} $f$
        (or $f$ \textbf{dominates} $g$) when for every natural number $n$, $g(n) \leq f(n)$.
    \end{itemize}

    \item

    \begin{proof}
    Let $f(n) = 3n$ and $g(n) = n$.

    \bigskip

    We need to prove that $g$ is dominated by $f$, i.e. for every natural number $n$,
    $g(n) \leq f(n)$.

    \bigskip

    The header tells us $g(n) = n$ and $f(n) = 3n$.

    \bigskip

    Starting from $g(n)$, we can conclude

    \setcounter{equation}{0}
    \begin{align}
        g(n) = n &\leq 3n\\
        &= f(n)
    \end{align}

    \end{proof}

    \bigskip

    \begin{mdframed}
        \underline{\textbf{Correct Solution:}}

        \bigskip

        Let \color{red}$n \in \mathbb{N}$\color{black}, $f(n) = 3n$ and $g(n) = n$.

        \bigskip

        We need to prove that $g$ is dominated by $f$, i.e. for every natural number $n$,
        $g(n) \leq f(n)$.

        \bigskip

        The header tells us $g(n) = n$ and $f(n) = 3n$.

        \bigskip

        \color{red}Since $n \geq 0$ from the fact $n \in \mathbb{N}$\color{black},
        starting from $g(n)$, we can conclude

        \setcounter{equation}{0}
        \begin{align}
            g(n) = n &\leq 3n\\
            &= f(n)
        \end{align}


    \end{mdframed}

\end{enumerate}

\section*{Question 4}

\end{document}