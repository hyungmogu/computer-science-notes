\documentclass[12pt]{article}
\usepackage{enumerate}
\usepackage{amsfonts}
\usepackage{amsmath}
\usepackage{fancyhdr}
\usepackage{amsmath}
\usepackage{amssymb}
\usepackage{amsthm}
\usepackage{mdframed}

\begin{document}
\title{Worksheet 5 Review 2}
\maketitle

\section*{Question 1}
\begin{itemize}

    \item

    \textbf{Statement:} $\forall m,n \in \mathbb{Z},\:(\exists k_1 \in \mathbb{Z},\:m=2k_1+1)
    \land (\exists k_2 \in \mathbb{Z},\:n = 2k_2 + 1) \Rightarrow (\exists k_3 \in \mathbb{Z},\:mn = 2k_3 + 1)$

    \begin{proof}

    Let $m,n \in \mathbb{Z}$. Assume there is an integer $k_1$ such that $m = 2k_1 + 1$.
    Assume there is an integer $k_2$ such that $n = 2k_2 + 1$. Let $k_3 = (2k_1k_2) + k_1 + k_2$.

    \bigskip

    We need to prove $mn = 2k_3 + 1$.

    \bigskip

    The assumption tells us $m = 2k_1 + 1$ and $n = 2k_2 + 1$.

    \bigskip

    By using these facts and then multiplying them together, we can conclude

    \begin{align}
        mn &= (2k_1 + 1)(2k_2 + 1)\\
        &= 4k_1k_2 + 2k_1 + 2k_2 + 1\\
        &= 2[(2k_1k_2) + k_1 + k_2] + 1\\
        &= 2k_3 + 1
    \end{align}

    \end{proof}

    \bigskip

    \textbf{Notes:}

    \begin{itemize}
        \item Noticed professor pre-calculates the value of $k_3$ as roughwork
        before writing proof
    \end{itemize}

    % \bigskip

    % \begin{mdframed}

    %     \underline{\textbf{Pseudoproof:}}

    %     \bigskip

    %     Let $m,n \in \mathbb{Z}$. Assume there is an integer $k_1$ such that $m = 2k_1 + 1$.
    %     Assume there is an integer $k_2$ such that $n = 2k_2 + 1$. Let $k_3 = (2k_1k_2) + k_1 + k_2$.

    %     \bigskip

    %     We need to prove $mn = 2k_3 + 1$.

    %     \begin{enumerate}[1.]
    %         \item Show $mn = 2k_3 + 1$
    %         \begin{itemize}
    %             \item Show by multiplying $m = 2k_1 + 1$ and $n = 2k_2 + 1$ together.

    %             \begin{mdframed}
    %                 \begin{align}
    %                 mn &= (2k_1 + 1)(2k_2 + 1)\\
    %                 &= 4k_1k_2 + 2k_1 + 2k_2 + 1\\
    %                 &= 2[(2k_1k_2 + k_1 + k_2)] + 1
    %                 &= 2k_3 + 1
    %                 \end{align}
    %             \end{mdframed}
    %         \end{itemize}
    %     \end{enumerate}

    % \end{mdframed}

\end{itemize}

\section*{Question 2}

\section*{Question 3}

\section*{Question 4}

\end{document}