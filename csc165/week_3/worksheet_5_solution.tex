\documentclass[12pt]{article}
\usepackage{enumerate}
\usepackage{amsfonts}
\usepackage{amsmath}
\usepackage{fancyhdr}
\usepackage{amssymb}

\begin{document}
\title{Worksheet 5 Solution}
\maketitle

\section*{Question 1}
\begin{itemize}
    \item

    $\forall n,p \in \mathbb{N}, Odd(n) \land Odd(p) \Rightarrow Odd(n \times p)$

    \bigskip

    Let $n,p \in \mathbb{Z}$, and assume $n,p$ are odd numbers.

    \bigskip

    Then, $\exists k,m \in \mathbb{Z},\:n = 2k - 1,\: p = 2m -1$ by the definition of
    odd numbers

    \bigskip

    Then,

    \begin{align}
        n \times p &= (2k-1)(2m-1)\\
        &= 2k2m - 2k - 2m + 1 \\
        &=(2k2m - 2k - 2m + 2) - 1 \\
        &=2(2km - k - m + 1) - 1\\
        &=2l - 1
    \end{align}

    where $l = 2km - k - m + 1$.

    \bigskip

    Since $l \in \mathbb{Z}$, it follows from the definition of odd number that the product
    of two odd numbers is odd.

\end{itemize}

\section*{Question 2}

\begin{enumerate}[a.]
    \item

    $\forall m,n \in \mathbb{Z},\:Even(m) \land Odd(n) \Rightarrow m^2-n^2 = m + n$

    \item

    The flaw is in the same value $k$. This implies that the statement is true
    only when $n$ is 1 less than $m$. This doesn't mean it's true for all even and
    odd numbers.


\end{enumerate}

\section*{Question 3}
\begin{enumerate}[a.]
    \item

    $Dom(f,g):\forall n \in \mathbb{N}, g(n) \leq f(n)$, where $f,g:\:\mathbb{N} \to \mathbb{R}^{\geq0}$

    \item

    Let $n \in \mathbb{R}^{\geq0}$, $f(n) = 3n$ and $g(n) = n$.

    \bigskip

    Then,

    \begin{align}
        g(n) = n &\leq n + n + n\\
        &\leq 3n\\
        &\leq f(n)
    \end{align}

    \bigskip

    Then, it follows from the definition that $f$ dominates $g$.

    \item

    Predicate Logic: $\exists n \in \mathbb{N},\:g(n) > f(n)$.

    \bigskip

    Let $n = 1$.

    \bigskip

    Then,


    \begin{align}
        g(1) = (1) + 165 = 166 &> 1\\
        &> f(1)
    \end{align}

    \bigskip

    Then, it follows from negation of the definition that $f$ does not dominate $g$.

    \item

    Predicate Logic: $\:\exists x \in \mathbb{R}^{\geq 0},\:\exists n \in \mathbb{N},\:g(n) > f(n)$

    \bigskip

    Let $x = 1$ and $n = 1$.

    \bigskip

    Then,

    \begin{align}
        g(1) = (1) + 1 = 2 &> 1\\
        &> f(1)
    \end{align}

    \bigskip

    Then, it follows from negation of the definition that $f$ does not dominate $g$.


\end{enumerate}

\section*{Question 4}

    \begin{itemize}
        \item

        Let $x \in \mathbb{R}^{\geq 0}$, $\epsilon = x - \lfloor x \rfloor$, and
        assume $x \geq 4$.

        \bigskip

        Then,

        \begin{align}
            (\lfloor x \rfloor)^2 &\geq 4\\
            (x + \epsilon)^2 &\geq 4
        \end{align}

        \bigskip

        by the fact that $\epsilon$ can be rewritten as $\lfloor x \rfloor = x + \epsilon$.

        \bigskip

        Then,

        \begin{align}
            (x + \epsilon)^2 &\geq 4\\
            x^2 + 2x\epsilon + \epsilon^2 &\geq 4\\
        \end{align}


    \end{itemize}

\end{document}