\documentclass[12pt]{article}
\usepackage{enumerate}
\usepackage{amsfonts}
\usepackage{amsmath}
\usepackage{fancyhdr}
\usepackage{amssymb}

\begin{document}
\title{Worksheet 5 Solution}
\maketitle

\section*{Question 1}
\begin{itemize}
    \item

    $\forall n,p \in \mathbb{N}, Odd(n) \land Odd(p) \Rightarrow Odd(n \times p)$

    \bigskip

    Let $n,p \in \mathbb{Z}$, and assume $n,p$ are odd numbers.

    \bigskip

    Then, $\exists k,m \in \mathbb{Z},\:n = 2k - 1,\: p = 2m -1$ by the definition of
    odd numbers

    \bigskip

    Then,

    \begin{align}
        n \times p &= (2k-1)(2m-1)\\
        &= 2k2m - 2k - 2m + 1 \\
        &=(2k2m - 2k - 2m + 2) - 1 \\
        &=2(2km - k - m + 1) - 1\\
        &=2l - 1
    \end{align}

    where $l = 2km - k - m + 1$.

    \bigskip

    Since $l \in \mathbb{Z}$, it follows from the definition of odd number that the product
    of two odd numbers is odd.

\end{itemize}

\section*{Question 2}

\begin{enumerate}[a.]
    \item

    $\forall m,n \in \mathbb{Z},\:Even(m) \land Odd(n) \Rightarrow m^2-n^2 = m + n$

    \item

    The flaw is in the same value $k$. This implies that the statement is true
    only when $n$ is 1 less than $m$. This doesn't mean it's true for all even and
    odd numbers.


\end{enumerate}


\section*{Question 3}

\section*{Question 4}

\end{document}