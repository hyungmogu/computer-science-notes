\documentclass[12pt]{article}
\usepackage{enumerate}
\usepackage{amsfonts}
\usepackage{amsmath}
\usepackage{fancyhdr}

\begin{document}
\title{Problem Set 0 Solution}
\author{Hyungmo Gu}
\date{\today}
\maketitle

\section*{Question 1}
    \begin{itemize}
        \item Solution complete. Please see above
    \end{itemize}

\section*{Question 2}
    \begin{itemize}
        \item CSC 165
        \item Mathematical Expression and Reasoning for Computer Science
        \item David Liu
    \end{itemize}

\section*{Question 3}
    \begin{itemize}
        \item
            $S_1 = \{\,x \mid x \in \mathbb{Z},\:x < 30\,\}$

            $S_2 = \{\,0, 1, 9, 10, 11, 19, 20, 21, 29, 30, 31 \dotsc\}$

            So,

            $S_1 \cap S_2 = \{\,0, 1, 9, 10, 19, 20, 21, 29\,\}$


    \end{itemize}


\section*{Question 4}
    \begin{tabular}{c | c | c | c | c | c | c}
        p & q & r & $\neg q$ & $p \lor \neg q $ & $p \iff r$ & $(p \lor \neg q) \Rightarrow (p \iff r)$ \\
        \hline
        T & T & T & F & T & T & T \\
        T & T & F & F & T & F & F \\
        T & F & T & T & T & T & T \\
        F & T & T & F & F & F & T \\
        T & F & F & T & T & F & F \\
        F & T & F & F & F & T & T \\
        F & F & T & T & T & F & F \\
        F & F & F & T & T & T & T \\
        \hline
    \end{tabular}


\section*{Question 5}
    \begin{enumerate}[i)]
        \item
        \begin{align*}
            5n + \frac{2n \cdot (n - 1)}{2} &= 165165 \\
            5n + \frac{2n^2 - 2n}{2} &= 165165 \\
            n^2 + 5n - n &= 165165 \\
            n^2 + 4n &= 165165 \\
            n^2 + 4n + 4 &= 165169 \\
            (n + 2)^2 &= 165169 \\
            n &= \sqrt{165169} - 2 \\
            n &= 404.409891612 \\
            n &\approx 405
        \end{align*}
    \end{enumerate}

\end{document}