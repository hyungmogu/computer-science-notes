\documentclass[12pt]{article}
\usepackage[margin=2.5cm]{geometry}
\usepackage{enumerate}
\usepackage{amsfonts}
\usepackage{amsmath}
\usepackage{fancyhdr}
\usepackage{mdframed}

\begin{document}
\title{Worksheet 3 Review 2}
\author{Hyungmo Gu}
\maketitle

\section*{Question 1}
\begin{enumerate}[a.]
    \item $Correct(my\_prog) \land Python(my\_prog)$
    \item

    $\forall x \in P,\: \neg Correct(x) \Rightarrow Python(x)$

    \begin{mdframed}
        \underline{\textbf{Correct Solution:}}

        $\color{red}\exists\color{black} x \in P,\: \neg Correct(x) \color{red}\land\color{black} Python(x)$
    \end{mdframed}

    \item $\forall x \in P, Python(x) \Rightarrow \neg Correct(x)$
    \item $\forall x \in P, \neg Correct(x) \Rightarrow Python(x)$
    \item There is a program that is written in $Python$ and is $Correct$
    \item All programs are not written in $Python$ and is $Correct$
    \item There is a program that is $Correct$ and not written in $Python$
    \item All programs that are correct is not written in $Python$, and all
    programs that are $Correct$ is not written in $Python$.
\end{enumerate}

\section*{Question 2}
\begin{enumerate}[a.]
    \item Either all programs that are written in $Python$ is $Correct$, or
    all programs that are written in $Python$ are not $Correct$
    \item $(\exists x \in P,\:Python(x) \land Correct(x)) \Rightarrow
    (\forall x \in P, Python(x) \land Correct(x))$
    \item The difference is that in statement 1, each divisibility claims can be
    validated with different natural numbers where as in statement 2,
    the two claims must be validated with a single natural number.

    \bigskip

    The statement 1 is true, where as statement 2 is false (consider counter example
    of $x = 7$)
\end{enumerate}

\section*{Question 3}

\section*{Question 4}

\section*{Question 5}

\end{document}