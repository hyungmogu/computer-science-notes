\documentclass[12pt]{article}
\usepackage{enumerate}
\usepackage{amsfonts}
\usepackage{amsmath}
\usepackage{fancyhdr}
\usepackage{mdframed}

\begin{document}
\title{Worksheet 2 Review}
\maketitle

\section*{Question 1}
\begin{enumerate}[a.]
    \item

    One example is $x = \text{Aizah}$ and $y = \text{Aizah}$.

    \bigskip

    There are more than one possible answer. The following examples also show
    truthiness of the statement.

    \begin{itemize}
        \item $x = \text{Carlos}$ and $y = \text{Carlos}$
        \item $x = \text{Ellen}$ and $y = \text{Ellen}$
    \end{itemize}

    \item

    One example is $x = \text{Betty}$ and $y = \text{Aizah}$.

    \bigskip

    There are more than one possible answer. The following examples also show
    also show truthiness of the statement.

    \bigskip

    \underline{\textbf{Part 1 ($\neg Rich(x)$ - True, $\neg SameDept(x,y)$ - False):}}

    \begin{itemize}
        \item $x = \text{Betty}$, $y = \text{Betty}$
        \item $x = \text{Betty}$, $y = \text{Doug}$
        \item $x = \text{Doug}$, $y = \text{Aizah}$
        \item $x = \text{Doug}$, $y = \text{Betty}$
        \item $x = \text{Doug}$, $y = \text{Doug}$
        \item $x = \text{Flo}$, $y = \text{Ellen}$
        \item $x = \text{Flo}$, $y = \text{Flo}$
    \end{itemize}

    \bigskip

    \underline{\textbf{Part 2 ($\neg Rich(x)$ - False, $\neg SameDept(x,y)$ - True):}}

    \begin{itemize}
        \item $x = \text{Aizah}$, $y = \text{Carlos}$
        \item $x = \text{Aizah}$, $y = \text{Ellen}$
        \item $x = \text{Aizah}$, $y = \text{Flo}$
        \item $x = \text{Carlos}$, $y = \text{Aizah}$
        \item $x = \text{Carlos}$, $y = \text{Betty}$
        \item $x = \text{Carlos}$, $y = \text{Doug}$
        \item $x = \text{Carlos}$, $y = \text{Ellen}$
        \item $x = \text{Carlos}$, $y = \text{Flo}$
        \item $x = \text{Ellen}$, $y = \text{Aizah}$
        \item $x = \text{Ellen}$, $y = \text{Betty}$
        \item $x = \text{Ellen}$, $y = \text{Carlos}$
        \item $x = \text{Ellen}$, $y = \text{Doug}$
    \end{itemize}

    \bigskip

    \underline{\textbf{Part 3 ($\neg Rich(x)$ - True, $\neg SameDept(x,y)$ - True):}}

    \begin{itemize}
        \item $x = \text{Betty}$, $y = \text{Carlos}$
        \item $x = \text{Betty}$, $y = \text{Ellen}$
        \item $x = \text{Betty}$, $y = \text{Flo}$
        \item $x = \text{Doug}$, $y = \text{Carlos}$
        \item $x = \text{Doug}$, $y = \text{Ellen}$
        \item $x = \text{Doug}$, $y = \text{Flo}$
        \item $x = \text{Flo}$, $y = \text{Aizh}$
        \item $x = \text{Flo}$, $y = \text{Betty}$
        \item $x = \text{Flo}$, $y = \text{Carlos}$
    \end{itemize}

    \item

    This statement is true. This is because in each department there is an
    individual who is rich. For example, in sales, there is Aizah. In HR, there
    is Carlos. In design, there is Ellen. So, for every employee $y$, we can choose
    person who is rich in the same department.

    \item

    Consider an example where $y$ is not in the same department as $x$, say $x = \text{Aizah}$
    and $y = \text{Carlos}$. This sets the statement $Rich(x) \land SameDept(x,y)$ to false.

    \bigskip

    \textbf{Notes:}

    \begin{itemize}
        \item \textbf{Negation of Statement:} $\forall x, \exists y \in E, \neg Rich(x) \lor \neg SameDept(x,y)$
        \item In above negation, only $y$ needs to be chosen.
    \end{itemize}

\end{enumerate}

\newpage

\section*{Question 2}
\begin{enumerate}[a.]
    \item

    $f(x) = 10$, where $f:\mathbb{R} \to \mathbb{R}$

    \bigskip

    \begin{mdframed}
        \underline{\textbf{Correct Solution:}}

        \bigskip

        $\color{red}\exists x \in \mathbb{R}\color{black},\:f(x) = 10$, where $f:\mathbb{R} \to \mathbb{R}$

    \end{mdframed}

    \item

    $\forall y \in codomain(\mathbb{R})$, $\exists x \in domain(\mathbb{R})$, $f(x) = y$, where $f:\mathbb{R} \to \mathbb{R}$

    \bigskip

    \begin{mdframed}
        \underline{\textbf{Correct Solution:}}

        \bigskip

        \color{red}$\forall y \in \mathbb{R}$, $\exists x \in \mathbb{R}$\color{black}, $f(x) = y$, where $f:\mathbb{R} \to \mathbb{R}$

    \end{mdframed}

    \bigskip

    \textbf{Notes:}

    \begin{itemize}
        \item Noticed professor doesn't label sets using $codomain$ or $domain$.
    \end{itemize}

    \item

    \textbf{Negation of Onto:} $\exists y \in \mathbb{R}$, $\forall x \in \mathbb{R}$, $f(x) \neq y$

    \bigskip

    A counter example of $f$ not being onto is $y = -1$.

\end{enumerate}

\section*{Question 3}
\begin{enumerate}[a.]
    \item $\{n \mid n \in S,\:n > 1\}$
    \item

    $P(n): n > 3$
\end{enumerate}

\end{document}