\documentclass[12pt]{article}
\usepackage{enumerate}
\usepackage{amsfonts}
\usepackage{amsmath}
\usepackage{fancyhdr}

\begin{document}
\title{Worksheet 2 Review}
\maketitle

\section*{Question 1}
\begin{enumerate}[a.]
    \item

    A solution is $x = Aizah$ and $y = Aizah$.

    \bigskip

    There is more than one possible answer, and another solution is $x = Carlos$
    and $y = Carlos$

    \item

    One counter-example is $x = Aizah$ and $y = Betty$.

    \bigskip

    Another counter-example is $x = Flo$ and $y = Ellen$.

    \item

    The statement is true. For any employees, if the employee is in

    \bigskip

    1. Sales Department, Aizah can be chosen.

    2. HR Department, Carlos can be chosen.

    3. Design Department, Ellen can be chosen.

    \item

    Not true. There is an employee in all department.

    \bigskip

    A counter example to this statement is $x = Carlos$ and $y = Ellen$.

\end{enumerate}

\section*{Question 2}
\begin{enumerate}[a.]
    \item

    $\exists f: \mathbb{R} \to \mathbb{R},\: f(x) = 10$.

    \item

    $\forall y \in Codomain(\mathbb{R}),\:\exists x \in Domain(\mathbb{R}),\:
    f(x) = y$.

    \item

    A counter example is $f(x) = -1$. There is no value in domain that can be
    mapped to the value in codomain.

\end{enumerate}

\section*{Question 3}
\begin{enumerate}[a.]
    \item

    $S = \{n \mid \forall n \in \mathbb{N},\:n > 3\}$

    \item

    $\forall n \in \mathbb{N},\:n > 3 \Rightarrow n > 1$

    \item

    Every integer that is greater than 10 or less than -40 is not equal to 0

    $\forall n \in \mathbb{Z}, n > 10 \lor n < -40 \Rightarrow n \neq 0$

    \bigskip

    Every employee who is in the same department as Doug is rich.

    $\forall x \in E, SameDept(x, Doug) \Rightarrow Rich(x)$

\end{enumerate}

\end{document}