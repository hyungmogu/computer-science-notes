\documentclass[12pt]{article}
\usepackage{enumerate}
\usepackage{amsfonts}
\usepackage{amsmath}
\usepackage{fancyhdr}

\begin{document}
\title{Worksheet 2 Solution}
\maketitle

\section*{Question 1}

\begin{enumerate}[a)]
    \item
        $x = Aizah, y = Aizah$ is one solution.

        \bigskip

        Yes. There is more than one answer. Take example

        $x = Carlos, y = Carlos$

    \bigskip

    \item
        $x = Aizah, y = Betty$ is one solution.

        \bigskip

        Yes. There is more than one answer. Take example

        $x = Ellen, y = Flo$

    \item
        The statement is true

        \begin{tabular}{ c | c | c }
            $x$ & $y$ & $rich(x) \land sameDept(x,y)$ \\
            \hline
            Aizah & Aizah & True \\
            \hline
            Betty & Aizah & True \\
            \hline
            Carlos & Carlos & True \\
            \hline
            Doug & Aizah & True \\
            \hline
            Ellen & Ellen & True \\
            \hline
            Flo & Ellen & True \\
            \hline
        \end{tabular}

    \item
        False. Consider example $x = Ellen, y = Carlos$

\end{enumerate}

\section*{Question 2}

\begin{enumerate}[a)]
    \item
        $\forall x \in \mathbb{R}, f(x) = 10$

    \item
        $\forall y \in \mathbb{R}, \exists x \in \mathbb{R}, f(x) = y$ where $f: \mathbb{R} \to \mathbb{R}$

    \item
        A counter example : $x^2 = -1$

\end{enumerate}

\section*{Question 3}

\begin{enumerate}[a)]
    \item
        $S = \{\,n \mid \forall n \in \mathbb{N},\:n > 3\,\}$
    \item
        Predicate $P(n)$ is $n > 3$
    \item
        $\forall x \in \mathbb{Z},\:(-40 < x) \land (x > 10) \Rightarrow x \ne 0$

        $\forall x \in \mathbb{E},\:sameDept(x, Doug) \Rightarrow rich(x)$

\end{enumerate}

\end{document}