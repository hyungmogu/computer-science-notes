\documentclass[12pt]{article}
\usepackage[margin=2.5cm]{geometry}
\usepackage{enumerate}
\usepackage{amsfonts}
\usepackage{amsmath}
\usepackage{fancyhdr}
\usepackage{amsmath}
\usepackage{amssymb}
\usepackage{amsthm}
\usepackage{mdframed}
\usepackage{graphicx}
\usepackage{subcaption}

\begin{document}
\title{Worksheet 20 Solution}
\author{Hyungmo Gu}
\maketitle

\section*{Question 1}
\begin{enumerate}[a.]
    \item

    \bigskip

    \begin{mdframed}
        \underline{\textbf{Pseudoproof:}}

        \bigskip

        Let $V = \{1,2,3,4,5,6\}$, $E = \{(1,2),(1,6),(2,3),(3,4),(4,5),(5,6)\})$.

        \bigskip

        We need to prove the graph $G = (V,E)$ is bipartite by proving the following
        properties:

        \begin{enumerate}[1.]
            \item There exists subsets $V_1, V_2 \subset V$ such that
            $V_1 \neq \emptyset, V_2 \neq \emptyset$, and $V_1$ and $V_2$ form
            a partition of $V$.
            \item Every edge in $E$ has exactly one endpoint in $V_1$ and one in $V_2$.
        \end{enumerate}

        \bigskip

        We will prove the properties in parts.

        \bigskip

        \begin{enumerate}[1.]
            \item Show there exists subsets $V_1, V_2 \subset V$ such that
            $V_1 \neq \emptyset, V_2 \neq \emptyset$, and $V_1$ and $V_2$ form
            a partition of $V$

            \item Show every edge in $E$ has exactly one endpoint in $V_1$ and one in $V_2$.
        \end{enumerate}

    \end{mdframed}
\end{enumerate}

\end{document}