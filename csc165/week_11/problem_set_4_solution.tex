\documentclass[12pt]{article}
\usepackage[margin=2.5cm]{geometry}
\usepackage{enumerate}
\usepackage{amsfonts}
\usepackage{amsmath}
\usepackage{fancyhdr}
\usepackage{amsmath}
\usepackage{amssymb}
\usepackage{amsthm}
\usepackage{mdframed}
\usepackage{graphicx}
\usepackage{subcaption}
\usepackage{listings}
\usepackage{xcolor}


\begin{document}
\title{Problem Set 4 Solution}
\author{Hyungmo Gu}
\maketitle

\section*{Question 1}
\begin{enumerate}[a.]
    \item

    \textbf{Statement:} $\forall f,g:\mathbb{N} \to \mathbb{R}^{+}$,
    $b \in \mathbb{R}^{+}$, $(g(n) \in \Theta(f(n))) \land (n_0 \in \mathbb{N},\:
    n \geq n_0 \Rightarrow f(n) \geq b \land g(n) \geq b) \land (b > 1) \Rightarrow
    \log_b(g(n)) \in \Theta(\log_b(f(n)))$

    \bigskip

    \textbf{Statement Expanded:} $\forall f,g:\mathbb{N} \to \mathbb{R}^{+}$,
    $b \in \mathbb{R}^{+}$, $\Bigl(\exists c_1,c_2,n_0 \in \mathbb{R}^{+},\:\forall n \in \mathbb{N},\:
    n \geq n_0 \Rightarrow c_1 \cdot g(n) \leq f(n) \leq c_2 \cdot g(n)\Bigr) \land \Bigl(\exists n_1 \in \mathbb{N},\:
    n \geq n_1 \Rightarrow f(n) \geq b \land g(n) \geq b \Bigr) \land \Bigl( b > 1 \Bigr) \Rightarrow
    \Bigl(\exists d_1,d_2,n_2 \in \mathbb{R}^{+},\:\forall n \in \mathbb{N},\: n \geq n_2
    \Rightarrow d_1 \cdot \log_b(g(n)) \leq \log_b(f(n)) \leq d_2 \cdot \log_b(g(n) \Bigr)$

    \bigskip

    \begin{proof}
        Let $f,g:\mathbb{N} \to \mathbb{R}^{+}$, and $b \in \mathbb{R}^{+}$. Assume
        $c_1 = 1$, $c_2 = b$, and $n_0 = 1$, and $n \in \mathbb{N}$ such that
        $n \geq n_0$ and $c_1 \cdot g(n) \leq f(n) \leq c_2 \cdot g(n)$. Assume $f(n)$
        and $g(n)$ are eventually $\geq b$. Assume $b > 1$. Let $d_1 = 1$, $d_2 = 2$,
        and $n_2 = n_0$. Assume $n \geq n_2$.

        \bigskip

        We need to show $d_1 \cdot \log_b g(n) \leq \log_b f(n) \leq d_2 \cdot \log_b g(n)$.

        \bigskip

        We will do so in two parts. One for $(d_1 \cdot \log_b g(n) \leq \log_b f(n))$ and
        the other for $(\log_b f(n) \leq d_2 \cdot \log_b g(n))$.

        \bigskip

        \textbf{Part 1 ($d_1 \cdot \log_b g(n) \leq \log_b f(n)$):}

        \bigskip

        The assumption tell us

        \begin{align}
            c_1 \cdot g(n) \leq f(n)
        \end{align}

        \bigskip

        Then, it follows from the fact $\forall x,y \in \mathbb{R}^{+}, x \geq y
        \Leftrightarrow \log x \geq \log y$

        \begin{align}
            \log (c_1 \cdot g(n)) &\leq \log (f(n))
        \end{align}

        \bigskip

        Then, using the fact $b > 1$, we can calculate

        \begin{align}
            \frac{\log (c_1 \cdot g(n))}{\log b} &\leq \frac{\log (f(n))}{\log b}\\
            \frac{\log (c_1) + \log (g(n))}{\log b} &\leq \frac{\log (f(n))}{\log b}
        \end{align}

        \bigskip

        Then,

        \begin{align}
            \frac{\log (g(n))}{\log b} &\leq \frac{\log (f(n))}{\log b}
        \end{align}

        by the fact $c_1 = 1$ and $\log c_1 = 0$.

        \bigskip

        Then, since $\frac{\log f(x)}{\log b} = \log_b f(x)$,

        \begin{align}
            \log_b (g(n)) &\leq \log_b (f(n))
        \end{align}

        \bigskip

        Then, because we know $d_1 = 1$, we can conclude

        \begin{align}
            \log_b (g(n)) &\leq d_1 \cdot \log_b (f(n))
        \end{align}


        \bigskip

        \textbf{Part 2 ($\log_b f(n) \leq d_2 \cdot \log_b g(n)$):}

        \bigskip

        The assumption tells us

        \begin{align}
            f(n) &\leq c_2 \cdot g(n)
        \end{align}

        \bigskip

        Then, it follows from the fact $\forall x,y \in \mathbb{R}^{+}, x \geq y
        \Leftrightarrow \log x \geq \log y$

        \begin{align}
            \log (f(n)) &\leq \log (c_2 \cdot g(n))
        \end{align}

        \bigskip

        Then, using the fact $b > 1$, we can calculate

        \begin{align}
            \frac{\log (f(n))}{\log b} &\leq \frac{\log (c_2 \cdot g(n))}{\log b}\\
            \frac{\log (f(n))}{\log b} &\leq \frac{\log (c_2) + \log (g(n))}{\log b}
        \end{align}

        \bigskip

        Then, since $c_2 = b$,

        \begin{align}
            \frac{\log (f(n))}{\log b} &\leq \frac{\log (b) + \log (g(n))}{\log b}
        \end{align}

        \bigskip

        Then, using the fact $g(n)$ is eventually $\geq b$, we can write

        \begin{align}
            \frac{\log (f(n))}{\log b} &\leq \frac{\log (g(n)) + \log (g(n))}{\log b}\\
            \frac{\log (f(n))}{\log b} &\leq \frac{2 \cdot \log (g(n))}{\log b}
        \end{align}

        \bigskip

        Then, since $\frac{\log f(x)}{\log b} = \log_b f(x)$,

        \begin{align}
            \log_b (f(n)) &\leq 2 \cdot \log_b (g(n))
        \end{align}

        \bigskip

        Then, because we know $d_2 = 2$, we can conclude

        \begin{align}
            \log_b (f(n)) &\leq d_2 \cdot \log_b (g(n))
        \end{align}
    \end{proof}

    \textbf{Notes:}

    \begin{itemize}
        \item $\forall x,y \in \mathbb{R}^{+}, x \geq y \Leftrightarrow \log x \geq \log y$


        \item $\exists c_1,c_2,n_0 \in \mathbb{R}^{+},\:\forall n \in \mathbb{N},
        n \geq n_0 \Rightarrow c_1 \cdot g(n) \leq f(n) \leq c2 \cdot g(n)$

        \item \textbf{Definition of Eventually:} $\exists n_0 \in \mathbb{N},
        n \geq n_0 \Rightarrow P$, where $P:\mathbb{N} \to \{\text{True},\text{False}\}$
    \end{itemize}

    \item

    \begin{proof}
        Let $k \in \mathbb{N}$.

        \bigskip

        First, we will analyze the cost of loop 2 over iteration of loop 1.

        \bigskip

        The code tells us loop 2 starts at $j_k = 1$ with $j_k$ increasing by
        a factor of 3 per iteration until $j_k \geq 1$.

        \bigskip

        Using these facts, we can calculate that the terminating condition occurs
        when

        \setcounter{equation}{0}
        \begin{align}
            3^k &\geq i\\
            k &\geq \log_3 i
        \end{align}

        \bigskip

        Because we know the number of iterations is the smallest value of $k$
        satisfying the above inequality, we can conclude loop 2 has

        \begin{align}
            \lceil \log_3 i \rceil
        \end{align}

        iterations.

        \bigskip

        Next, we need to determine the total number of iterations of loop 2
        over all iterations of loop 1.

        \bigskip

        The code tells us loop 1 starts at $i = 1$ and ends at $i = n$ with each
        $i$ increasing by 1 per iteration.

        \bigskip

        Using these facts, we can conclude loop 2 has total of

        \begin{align}
            \lceil \log_3 1 \rceil + \lceil \log_3 2 \rceil + \cdots + \lceil \log_3 n \rceil &= \sum\limits_{i=1}^n \lceil \log_3 i \rceil
        \end{align}

        iterations.
    \end{proof}

\end{enumerate}

\section*{Question 2}

\section*{Question 3}

\section*{Question 4}

\end{document}