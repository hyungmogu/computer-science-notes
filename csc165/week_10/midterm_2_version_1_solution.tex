\documentclass[12pt]{article}
\usepackage{enumerate}
\usepackage{amsfonts}
\usepackage{amsmath}
\usepackage{fancyhdr}
\usepackage{amsmath}
\usepackage{amssymb}
\usepackage{amsthm}
\usepackage{listings}
\usepackage{mdframed}

\begin{document}
\title{Midterm 2 Version 1 Solution}
\maketitle

\section*{Question 1}
\begin{enumerate}
    \item

    \begin{align*}
        100 \div 2 = 50,\:&\textbf{Remainders}\:0\\
        50 \div 2 = 25,\:&\textbf{Remainders}\:0\\
        25 \div 2 = 12,\:&\textbf{Remainders}\:1\\
        12 \div 2 = 6,\:&\textbf{Remainders}\:0\\
        6 \div 2 = 3,\:&\textbf{Remainders}\:0\\
        3 \div 2 = 1,\:&\textbf{Remainders}\:1\\
        1 \div 2 = 0,\:&\textbf{Remainders}\:1
    \end{align*}

    \bigskip

    Then, it follows from above that the binary representation of 100 is $(1100100)_2$.

    \item

    The smallest number that can be expressed by an n-digit balanced ternary
    representation is

    \begin{align}
        \sum\limits_{i=0}^{n-1} d_i \cdot 3^i,\:\text{where}\:d_i \in \{0,1,2\}
    \end{align}

    \bigskip

    \begin{mdframed}
        \underline{\textbf{Correct Solution:}}

        \bigskip

        The smallest number that can be expressed by an n-digit balanced ternary
        representation is

        \begin{align}
            -\left[ \sum\limits_{i=0}^{n-1} 3^i \right]
        \end{align}
    \end{mdframed}

    \bigskip

    \textbf{Notes:}

    \begin{itemize}
        \item Realized professor is asking for an example of the smallest number.
        \item Learned a negative number could be expressed in in ternary or binary
        representation of numbers.
    \end{itemize}

\end{enumerate}

\section*{Question 2}

\section*{Question 3}

\section*{Question 4}

\end{document}