\documentclass[12pt]{article}
\usepackage{enumerate}
\usepackage{amsfonts}
\usepackage{amsmath}
\usepackage{fancyhdr}
\usepackage{amsmath}
\usepackage{amssymb}
\usepackage{amsthm}
\usepackage{listings}
\usepackage{mdframed}
\usepackage{graphicx}

\begin{document}
\title{Midterm 2 Version 1 Solution}
\maketitle

\section*{Question 1}
\begin{enumerate}[a.]
    \item

    \begin{align*}
        100 \div 2 = 50,\:&\text{Remainders}\:\textbf{0}\\
        50 \div 2 = 25,\:&\text{Remainders}\:\textbf{0}\\
        25 \div 2 = 12,\:&\text{Remainders}\:\textbf{1}\\
        12 \div 2 = 6,\:&\text{Remainders}\:\textbf{0}\\
        6 \div 2 = 3,\:&\text{Remainders}\:\textbf{0}\\
        3 \div 2 = 1,\:&\text{Remainders}\:\textbf{1}\\
        1 \div 2 = 0,\:&\text{Remainders}\:\textbf{1}
    \end{align*}

    \bigskip

    Then, it follows from above that the binary representation of 100 is $(1100100)_2$.

    \item The smallest number that can be expressed by an n-digit balanced ternary
    representation is

    \begin{align}
        \sum\limits_{i=0}^{n-1} d_i \cdot 3^i,\:\text{where}\:d_i \in \{0,1,2\}
    \end{align}

    \bigskip

    \begin{mdframed}
        \underline{\textbf{Correct Solution:}}

        \bigskip

        The smallest number that can be expressed by an n-digit balanced ternary
        representation is

        \setcounter{equation}{0}
        \color{red}
        \begin{align}
            -\left[ \sum\limits_{i=0}^{n-1} 3^i \right]
        \end{align}
        \color{black}
    \end{mdframed}

    \bigskip

    \textbf{Notes:}

    \begin{itemize}
        \item Realized professor is asking for an example of the smallest number.
        \item Ternary representation of a number

        \begin{align*}
            \sum\limits_{i=0}^{n-1} d_i \cdot 3^i,\:\text{where}\:d_i \in \{0,1,2\}
        \end{align*}

        \item Learned a negative number could be expressed in in ternary or binary
        representation of numbers.
    \end{itemize}

    \item

    \begin{tabular}{|c|c|c|c|c|c|}
        \hline
        $f(n) \in \Omega(n)$ & True & $g(n) \in \Omega(n)$ & False & $f(n) \in \mathcal{O}(g(n))$ & False\\
        \hline
        $f(n) \in \Theta(g(n))$ & False & $g(n) \in \Theta(\log_3 n)$ & True & $f(n) + g(n) \in \Theta(f(n))$ & True\\
        \hline
    \end{tabular}

    \bigskip

    \textbf{Notes:}

    \begin{itemize}
        \item

        $\forall g:\mathbb{N} \to \mathbb{R}^{\geq 0}$, and all numbers $a \in \mathbb{R}^{\geq 0}$,
        if $g \in \mathcal{O}(f)$, then $f + g \in \mathcal{O}(f)$

        \item
        $g \in \Theta(f):\: g \in \mathcal{O}(f) \land g \in \Omega(f)$

        or

        $g \in \Theta(f):\:\exists c_1,c_2,n_1 \in \mathbb{R}^{+}, \forall n \in \mathbb{N}, n \geq n_1
        \Rightarrow c_1g(n) \leq f(n) \leq c_2g(n)$, where $f,g:\:\mathbb{N} \to \mathbb{R}^{\geq 0}$

        \item
        $g \in \Omega(f):\:\exists c,n_o \in \mathbb{R}^{+},\:\forall n \in
        \mathbb{N},\:n \geq n_0 \Rightarrow g(n) \geq cf(n)$, where $f,g:\mathbb{N} \to \mathbb{R}^{\geq 0}$

        \item

        $g \in \mathcal{O}(f):\:\exists c,n_o \in \mathbb{R}^{+},\:\forall n \in
        \mathbb{N},\:n \geq n_0 \Rightarrow g(n) \leq cf(n)$, where $f,g:\mathbb{N} \to \mathbb{R}^{\geq 0}$

    \end{itemize}

    \item

    \begin{tabular}{|c|c|c|c|c|}
        \hline
        k & 0 & 1 & 2\\
        \hline
        $i_k$ & $3 = 3^1$ & $9 = 3^2$ & $81 = 3^4$\\
        \hline
    \end{tabular}

    \bigskip

    The value of $i_k$ is
    \setcounter{equation}{0}
    \begin{align}
        3^{2^k}
    \end{align}

    \bigskip

    \textbf{Notes:}

    \begin{itemize}
        \item Realized we are only concerned with the lines \textbf{i = i * i} and \textbf{i = 3}
    \end{itemize}

    \item The number of iterations the function's loop will run is

    \setcounter{equation}{0}
    \begin{align}
        \lceil \log_2 \log_3 n \rceil - 1
    \end{align}

    \textbf{Notes:}

    \begin{itemize}
        \item The loop terminates when $3^{2^{(k+1)}} = i_{k+1} = i_k \cdot i_k \geq n$.
        \item $\forall x \in \mathbb{Z},\:\forall y \in \mathbb{R},\:\lfloor x+y \rfloor = x + \lfloor y \rfloor$
        \item Feel more confident there is no need to add an extra \textbf{+1}. Done by
        playing with examples (i.e is $\lceil \log \log_3 (82) \rceil - 1$ true? Would the loop run only once?)
    \end{itemize}

\end{enumerate}

\section*{Question 2}
\begin{itemize}
    \item

    \textbf{Predicate Logic:} $\forall n \in \mathbb{N}, n \geq 3 \Rightarrow 5^n + 50 < 6^n$

    \bigskip

    \begin{proof}

    Let $n \in \mathbb{N}$.

    \bigskip

    We will prove the statement by induction on $n$.

    \bigskip

    \textbf{Base Case ($n = 3$):}

    \bigskip

    Let $n = 3$.

    \bigskip

    We want to show $5^3 + 50 < 6^3$.

    \bigskip

    Starting from $5^3 + 50$, we can calculate

    \setcounter{equation}{0}
    \begin{align}
        5^3 + 50 &= 125 + 50\\
        &= 175\\
        &< 216\\
        &< 6^3
    \end{align}

    \bigskip

    \textbf{Inductive Case:}

    \bigskip

    Let $n \in \mathbb{N}$. Assume $n \geq 3$ and $5^n + 50 < 6^n$.

    \bigskip

    We want to show $5^{n+1} + 50 < 6^{n+1}$.

    \bigskip

    Starting from $5^{n+1} + 50$, we can calculate

    \begin{align}
        50^{n+1} + 50 &= 5^n \cdot 5 + 50\\
        &< 5^n \cdot 5 + 50 \cdot 5\\
        &< 5(5^n + 50)
    \end{align}

    \bigskip

    Then,

    \begin{align}
        50^{n+1} + 5 &< 5 \cdot 6^n\\
        &< 6 \cdot 6^n\\
        &< 6^{n+1}
    \end{align}

    by using inductive hypothesis (i.e $5^n + 50 < 6^n$)

    \end{proof}

    \bigskip

    \begin{mdframed}
        \underline{\textbf{Correct Solution:}}

        \bigskip

        Let $n \in \mathbb{N}$.

        \bigskip

        We will prove the statement by induction on $n$.

        \bigskip

        \textbf{Base Case ($n = 3$):}

        \bigskip

        Let $n = 3$.

        \bigskip

        We want to show $5^3 + 50 < 6^3$.

        \bigskip

        Starting from $5^3 + 50$, we can calculate

        \setcounter{equation}{0}
        \begin{align}
            5^3 + 50 &= 125 + 50\\
            &= 175\\
            &< 216\\
            &< 6^3
        \end{align}

        \bigskip

        \textbf{Inductive Case:}

        \bigskip

        Let $n \in \mathbb{N}$. Assume $n \geq 3$ and $5^n + 50 < 6^n$.

        \bigskip

        We want to show $5^{n+1} + 50 < 6^{n+1}$.

        \bigskip

        Starting from $5^{n+1} + 50$, we can calculate

        \begin{align}
            50^{n+1} + 50 &= 5^n \cdot 5 + 50\\
            &\color{red}=\color{black} 5^n \cdot 5 + 50 \cdot 5\\
            &< 5(5^n + 50)
        \end{align}

        \bigskip

        Then,

        \begin{align}
            50^{n+1} + 5 &< 5 \cdot 6^n\\
            &< 6 \cdot 6^n\\
            &\color{red}=\color{black} 6^{n+1}
        \end{align}

        by using inductive hypothesis (i.e $5^n + 50 < 6^n$)
    \end{mdframed}

    \bigskip

    \textbf{Notes:}

    \begin{itemize}
        \item Noticed professor uses `\textbf{=}' sign if the expression's value
        remains unchanged from the one before

        \bigskip

        See equation 5 and 6 for example.
    \end{itemize}

\end{itemize}

\section*{Question 3}
\begin{itemize}
    \item

    \textbf{Statement:} $\exists a \in \mathbb{R}^{+}$, $an + 1 \in \Theta(n^3)$

    \bigskip

    \textbf{Negation of Statement:} $\forall a \in \mathbb{R}^{+}$,
    $\forall c_1,c_2,n_0 \in \mathbb{R}^{+}$, $\exists n \in \mathbb{N}$,
    $(n \geq n_0) \land \bigl( (an+1 < c_1n^3) \lor (an+1 > c_2n^3) \bigr)$

    \bigskip

    \begin{proof}

    Let $n = \left\lceil max(n_0, \sqrt{\frac{2a}{c_1}}, \sqrt[3]{\frac{1}{c_1}}) \right\rceil + 1$.

    \bigskip

    We will disprove the statement by showing $n \geq n_0$ and $an+1 < c_1n^3$

    \bigskip

    \textbf{Part 1 (Showing $n \geq n_0$):}

    \bigskip

    Using the fact that $\left\lceil max(n_0, \sqrt{\frac{2a}{c_1}}, \sqrt[3]{\frac{1}{c_1}}) \right\rceil$
    will result in a value greater than or equal to $n_0$, we can calculate

    \setcounter{equation}{0}
    \begin{align}
        n_0 &\leq \left\lceil max(n_0, \sqrt{\frac{2a}{c_1}}, \sqrt[3]{\frac{1}{c_1}}) \right\rceil\\
        &\leq \left\lceil max(n_0, \sqrt{\frac{2a}{c_1}}, \sqrt[3]{\frac{1}{c_1}}) \right\rceil + 1
    \end{align}

    \bigskip

    Then, because we know $n = \left\lceil max(n_0, \sqrt{\frac{2a}{c_1}},
    \sqrt[3]{\frac{1}{c_1}}) \right\rceil + 1$, we can conclude

    \begin{align}
       n_0 &\leq n
    \end{align}

    \end{proof}

    \bigskip

    \textbf{Part 2 (Showing $an+1 < c_1n^3$):}

    \bigskip

    We will prove $an +1 < c_1n^3$ by showing $an < \frac{c_1}{2}n^3$ and
    $1 < \frac{c_1}{2}n^3$, and then combining the two together.

    \bigskip

    For the first inequality, because we know $n = \left\lceil max(n_0,
    \sqrt{\frac{2a}{c_1}}, \sqrt[3]{\frac{1}{c_1}}) \right\rceil + 1 > \sqrt{\frac{2a}{c_1}}$,
    we can conclude

    \begin{align}
        \sqrt{\frac{2a}{c_1}} &< n\\
        \frac{2a}{c_1} &< n^2\\
        a &< \frac{c_1}{2}n^2\\
        an &< \frac{c_1}{2}n^3
    \end{align}

    \bigskip

    For the second inequality, because we know $n = \left\lceil max(n_0,
    \sqrt{\frac{2a}{c_1}}, \sqrt[3]{\frac{1}{c_1}}) \right\rceil + 1 > \sqrt[3]{\frac{1}{c_1}}$,
    we can conclude

    \begin{align}
        \sqrt[3]{\frac{1}{c_1}} &< n\\
        \frac{1}{c_1} &< n^3\\
        1 &< n^3
    \end{align}

    \bigskip

    Then,

    \begin{align}
        an +1 &< \frac{c_1}{2} \cdot n^3 + \frac{c_1}{2} \cdot n^3\\
        an + 1 &< c_1n^3
    \end{align}

    \bigskip

    \textbf{Notes:}

    \begin{itemize}
        \item I struggled on this question.
        \item Learned \textbf{+1} in $\left\lceil max(n_0, \sqrt{\frac{2a}{c_1}},
        \sqrt[3]{\frac{1}{c_1}}) \right\rceil + 1 > \sqrt[3]{\frac{1}{c_1}}$ is
        to allow the use of inequality sign `$<$'.
        \item Learned that when $c_1$ is in inequality, with multiple terms like
        $an + 1$ on the other side, and is asking to disprove it, I should first
        divide them up, find valid n for each term, and then recombine
        to create a valid $n$.

        \bigskip

        See figure 1 for example

        \begin{figure}
            \includegraphics[width=\linewidth]{images/midterm_1_v1_q3_comment.jpg}
            \caption{A sample work for question 3}
        \end{figure}
    \end{itemize}
\end{itemize}

\newpage

\section*{Question 4}
\begin{enumerate}[a.]
    \item

    \begin{proof}
        Let $n \in \mathbb{N}$.

        \bigskip

        We will determine the exact number of iterations by first evaluating the
        number of iterations of loop 2, and then evaluating the number of iterations
        of loop 1.

        \bigskip

        For loop 2, because we know it starts at $j = 0$, and ends at $j = i - 1$,
        with $j$ increasing by 3 per iteration, we can conclude the loop has

        \setcounter{equation}{0}
        \begin{align}
            \left\lceil \frac{i-1-0+1}{3} \right\rceil = \left\lceil \frac{i}{3} \right\rceil
        \end{align}

        iterations.

        \bigskip

        For loop 1, because we know it starts at $i = 0$ and ends at $i = n^2-1$,
        with each iteration taking $\left\lceil \frac{i}{3} \right\rceil$
        steps, we can conclude the loop takes total of

        \begin{align}
            \sum\limits_{i=0}^{n^2-1} \left\lceil \frac{i}{3} \right\rceil
        \end{align}

        iterations.

        \bigskip

        Then, because we know the floor and ceilings signs can be ignored,
        we can conclude the exact total number of iterations is

        \begin{align}
            \sum\limits_{i=0}^{n^2-1} \frac{i}{3} &= \frac{1}{3} \cdot \sum\limits_{i=0}^{n^2-1} i\\
            &= \frac{1}{3} \cdot \frac{(n^2 - 1)(n^2 -1 + 1)}{2}\\
            &= \frac{(n^2 - 1) \cdot n^2}{6}
        \end{align}
    \end{proof}

    \item

    \begin{proof}

        \underline{\textbf{Part 1 (Evaluating upper bound of the worst case running time)}}

        \bigskip

        Assume the algorithm runs on worst possible case.

        \bigskip

        We will evaluate the algorithm's upper bound running time by determining
        the total cost of the loop, and then the big-oh.

        \bigskip

        First, we will evaluate the total cost the cost of loop 2.

        \bigskip

        Because we know loop 2 starts at $j = 1$ and ends at $j = n-1$ with
        $j$ increasing by 1 per iteration, we can conclude loop 2 runs at most

        \setcounter{equation}{0}
        \begin{align}
            \left\lceil \frac{n-1-1+1}{1} \right\rceil &= n - 1
        \end{align}

        iterations.

        \bigskip

        Because we know each iteration in loop 2 costs a constant step, we can
        conclude loop 2 has at most

        \begin{align}
            (n - 1) \cdot 1 &= n - 1
        \end{align}

        steps.

        \bigskip

        Next, we will calculate the cost of loop 1.

        \bigskip

        Because we know loop 1 starts at $i = 0$ and ends at $i = n-1$,
        we can conclude the loop has at most

        \begin{align}
            n -1 - 0 + 1 &= n
        \end{align}

        iterations.

        \bigskip

        Because we know all the elements in $lst$ become even with loop 2 triggered,
        and because we know loop 2 will not be triggered for all other values of $i$,
        we can conclude each iteration of loop 1 will cost 1 step.

        \bigskip

        Then, we can conclude loop 1 will cost at most

        \begin{align}
            n \cdot 1 &= n
        \end{align}

        steps.

        \bigskip

        Now, we will combine the costs together.

        \bigskip

        Since loop 2 costs $n$ steps, and loop 1 cost $n$ steps, we can conclude
        the algorithm has total cost of at most

        \begin{align}
            n + n &= 2n
        \end{align}

        steps.

        \bigskip

        Then, it follows from above that the algorithm has upper bound worst-case
        running time of $\mathcal{O}(n)$.

        \bigskip

        \underline{\textbf{Part 2 (Evaluating lower bound of the worst case running time)}}

        \bigskip

        Let $n \in \mathbb{N}$, and $lst = [1,2,\cdots,n-1]$

        \bigskip

        Because we know loop 2 will be triggered when $i = 0$, and because we know
        the rest of elements in $lst$ will become even, we can conclude
        the algorithm with this input group will run the same as when evaluating the big oh.

        \bigskip

        Then, we can conclude the algorithm has total cost of $2n$.

        \bigskip

        Then, we can conclude the algorithm has lower bound worst-case running
        time of $\Omega(n)$.

        \bigskip

        Because we know the value of $\mathcal{O}$ and $\Omega$ are the same,
        we can conclude algorithm has running time of $\Theta(n)$.
    \end{proof}

\end{enumerate}

\end{document}