\documentclass[12pt]{article}
\usepackage[margin=2.5cm]{geometry}
\usepackage{enumerate}
\usepackage{amsfonts}
\usepackage{amsmath}
\usepackage{fancyhdr}
\usepackage{amsmath}
\usepackage{amssymb}
\usepackage{amsthm}
\usepackage{listings}
\usepackage{mdframed}
\usepackage{graphicx}

\begin{document}
\title{Midterm 2 Version 3 Solution}
\author{Hyungmo Gu}
\maketitle

\section*{Question 1}
\begin{enumerate}[a.]
    \item

    \begin{align*}
        165 \div 2 = 82,\:&\text{remainders}\:\textbf{1}\\
        82 \div 2 = 41,\:&\text{remainders}\:\textbf{0}\\
        41 \div 2 = 20,\:&\text{remainders}\:\textbf{1}\\
        20 \div 2 = 10,\:&\text{remainders}\:\textbf{0}\\
        10 \div 2 = 5,\:&\text{remainders}\:\textbf{0}\\
        5 \div 2 = 2,\:&\text{remainders}\:\textbf{1}\\
        2 \div 2 = 1,\:&\text{remainders}\:\textbf{0}\\
        1 \div 2 = 0,\:&\text{remainders}\:\textbf{1}
    \end{align*}

    \bigskip

    From the above, we can conclude the binary representation of the decimal
    number 165 is $(10100101)_2$

    \item

    The largest number that can be expressed by an $n$-digit balanced ternary
    representation is

    \begin{align}
        \sum\limits_{i=0}^{n-1} 3^i &= \frac{1}{2} \cdot (3^n - 1)
    \end{align}

    \bigskip

    \textbf{Notes:}

    \begin{itemize}
        \item Geometric Series

        \setcounter{equation}{0}
        \begin{align*}
            \sum\limits_{i=0}^n r^i = \frac{1-r^{n+1}}{1-r},\:\text{where}\:\lvert r \rvert > 1
        \end{align*}
    \end{itemize}

    \item

    \begin{tabular}{|l|c|l|c|l|c|}
        \hline
        $f(n) \in \mathcal{O}(n)$ & True & $g(n) \in \Omega(n)$ & True & $f(n) \in \Omega(g(n))$ & True\\
        \hline
        $f(n) \in \Theta(g(n))$ & False & $g(n) \in \Theta(n)$ & False & $f(n) + g(n) \in \Theta(g(n))$ & False\\
        \hline
    \end{tabular}

    \begin{mdframed}
        \underline{\textbf{Correct Solution:}}

        \bigskip

        \begin{tabular}{|l|c|l|c|l|c|}
            \hline
            $f(n) \in \mathcal{O}(n)$ & True & $g(n) \in \Omega(n)$ & True & $f(n) \in \Omega(g(n))$ & True\\
            \hline
            $f(n) \in \Theta(g(n))$ & False & $g(n) \in \Theta(n)$ & False & $f(n) + g(n) \in \Theta(g(n))$ & \color{red}True\color{black}\\
            \hline
        \end{tabular}

    \end{mdframed}

    \bigskip

    \textbf{Notes:}

    \begin{itemize}
        \item Note that for $f(n) + g(n) \in \Theta(g(n))$, large values of $n$
        causes $g(n) = n^{\log_2 n}$ to dominate $f(n) = \frac{3n}{\log_2 n + 8}$.
        This causes the inequality to be simplified to

        \setcounter{equation}{0}
        \begin{align}
            c_1 \cdot n^{\log_2 n} \leq n^{\log_2 n} \leq c_2 \cdot n^{\log_2 n}
        \end{align}

        \bigskip

        It follows from above the answer is True.
    \end{itemize}

    \item


    \begin{tabular}{|c|c|c|c|}
        \hline
        $k$ & 0 & 1 & 2 \\
        \hline
        $i * i$ & $3 = 3^{2^0}$ & $9 = 3^{2^1}$ & $81 = 3^{2^4}$\\
        \hline
    \end{tabular}

    From the rough work, we can deduce the value of $i$ after $k$ iterations
    is

    \setcounter{equation}{0}
    \begin{align}
        3^{2^k}
    \end{align}

    \item

    Loop termination occurs when $i_k \geq n^3$.

    \bigskip

    We need to find the smallest value of $k$, and the value is

    \setcounter{equation}{0}
    \begin{align}
        \lceil \log_2 3\log_3 n \rceil
    \end{align}

\end{enumerate}

\section*{Question 2}

\section*{Question 3}

\section*{Question 4}

\end{document}