\documentclass[12pt]{article}
\usepackage[margin=2.5cm]{geometry}
\usepackage{enumerate}
\usepackage{amsfonts}
\usepackage{amsmath}
\usepackage{fancyhdr}
\usepackage{amsmath}
\usepackage{amssymb}
\usepackage{amsthm}
\usepackage{listings}
\usepackage{mdframed}
\usepackage{graphicx}

\begin{document}
\title{Midterm 2 Version 3 Solution}
\author{Hyungmo Gu}
\maketitle

\section*{Question 1}
\begin{enumerate}[a.]
    \item

    \begin{align*}
        165 \div 2 = 82,\:&\text{remainders}\:\textbf{1}\\
        82 \div 2 = 41,\:&\text{remainders}\:\textbf{0}\\
        41 \div 2 = 20,\:&\text{remainders}\:\textbf{1}\\
        20 \div 2 = 10,\:&\text{remainders}\:\textbf{0}\\
        10 \div 2 = 5,\:&\text{remainders}\:\textbf{0}\\
        5 \div 2 = 2,\:&\text{remainders}\:\textbf{1}\\
        2 \div 2 = 1,\:&\text{remainders}\:\textbf{0}\\
        1 \div 2 = 0,\:&\text{remainders}\:\textbf{1}
    \end{align*}

    \bigskip

    From the above, we can conclude the binary representation of the decimal
    number 165 is $(10100101)_2$
\end{enumerate}

\section*{Question 2}

\section*{Question 3}

\section*{Question 4}

\end{document}