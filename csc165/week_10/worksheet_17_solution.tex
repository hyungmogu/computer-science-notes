\documentclass[12pt]{article}
\usepackage[margin=2.5cm]{geometry}
\usepackage{enumerate}
\usepackage{amsfonts}
\usepackage{amsmath}
\usepackage{fancyhdr}
\usepackage{amsmath}
\usepackage{amssymb}
\usepackage{amsthm}
\usepackage{mdframed}

\begin{document}
\title{Worksheet 17 Solution}
\author{Hyungmo Gu}
\maketitle

\section*{Question 1}
\begin{enumerate}[a.]
    \item

    We need to determine $\lvert \mathcal{I}_n \rvert$.

    \bigskip

    The problem tells that the values in inputs are either 1 or 0, and we know
    $\mathcal{I}_n$ represents all possible inputs of size $n$ containing binary values.

    \bigskip

    After watching lecture videos, and reading notes, I do not yet understand the details
    of how to evaluate the $\mathcal{I}_n$, but from the pattern below

    \begin{align*}
        [0], [1], [1,0], [0,1], [1,1], [0,0], [0,0,0],[0,0,1],[0,1,0],[1,0,0],
        [1,1,0],[1,0,1],[0,1,1],[1,1,1]
    \end{align*}

    we can see the inputs of size $1$ have $2$ different inputs, the inputs of size $2$ have $4$
    different inputs, and the inputs of size $3$ have $8$ different inputs.

    \bigskip

    Using this pattern, I can make an educated guess that $\lvert \mathcal{I}_n \rvert = 2^n$.

    \bigskip

    \textbf{Notes:}

    \begin{itemize}
     \item The idea of average-case analysis is that some data structures and
     algorithms have poor worst-case performance but perform well in vast majority
     of others.

    \item Average-case analysis looks at running time on sets of inputs
    \item Average case: $AVG_{func}(n) = avg \{ \text{runtime of func(x)} \mid x \in \mathcal{I}_n \}$
    \item Worst case: $WC_{func}(n) = max \{ \text{runtime of func(x)} \mid x \in \mathcal{I}_n \}$

    \end{itemize}

    \item

    \begin{tabular}{|c|c|l|c|}
        \hline
        $n$ & $i$ & Sets & $\lvert S_{n,i} \rvert$\\
        \hline
        2 & 0 & $\{[0]\}$ & 1\\
        \hline
        2 & 0 & $\{[0,1],[0,0]\}$ & 2\\
        \hline
        2 & 1 & $\{[1,0]\}$ & 1\\
        \hline
        3 & 0 & $\{[0,1,1],[0,0,1],[0,0,0]\}$ & 3\\
        \hline
        3 & 1 & $\{[1,0,1],[1,0,0]\}$ & 2\\
        \hline
        3 & 2 & $\{[1,1,0]\}$ & 1\\
        \hline
    \end{tabular}

    \bigskip

    By the pattern outlined above, we can deduce that $\lvert S_{n,i} \rvert = n - i$.

    \bigskip

    \begin{mdframed}
        \underline{\textbf{Correct Solution:}}

        \bigskip

        \begin{tabular}{|c|c|l|c|}
            \hline
            $n$ & $i$ & Sets & $\lvert S_{n,i} \rvert$\\
            \hline
            \color{red}1 & 0 & $\{[0]\}$ & 1\\
            \hline
            2 & 0 & $\{[0,1],[0,0]\}$ & 2\\
            \hline
            2 & 1 & $\{[1,0]\}$ & 1\\
            \hline
            3 & 0 & \color{red}$\{[0,1,1],[0,0,1],[0,1,0],[0,0,0]\}$ & \color{red}4\\
            \hline
            3 & 1 & $\{[1,0,1],[1,0,0]\}$ & 2\\
            \hline
            3 & 2 & $\{[1,1,0]\}$ & 1\\
            \hline
        \end{tabular}

        \bigskip

        By the pattern outlined above, we can deduce that \color{red}$\lvert S_{n,i} \rvert = 2^{n-i-1}$\color{black}.

    \end{mdframed}

    \item

    Because we know there is only one list in a set $S_{n}$ containing all 1s, we can conclude
    $\lvert S_{n,n} \rvert = 1$.

    \item

    We will prove the statement informally using proof by cases.

    \bigskip

    \underline{\textbf{Case 1 (when list doesn’t have 0s):}}

    \bigskip

    The definition of $S_{n,i}$ tells us $0 \leq i \leq n$, $S_{n,i}$ contains
    all lists with 0 starting at ith position.

    \bigskip

    Using the fact, we can conclude $S_{n,n}$ is a set of lists containing 0 at $n^{th}$
    position.

    \bigskip

    Since $i$ in a list starts at $i = 0$ and ends at $i = n-1$, there are no 0
    in the list of a set $S_{n,n}$.

    \bigskip

    Then, we can conclude $S_{n,n}$ is one and the only that contains a list of only 1s.

    \bigskip

    \underline{\textbf{Case 2 (when list has one or more 0s):}}

    \bigskip

    Since this list has 0 starting at $i^{th}$ position, we can conclude this list
    exists in the set $S_{n,i}$.

    \bigskip

    \begin{mdframed}
        \underline{\textbf{Attempt 2:}}

        \bigskip

        We will prove the statement informally using proof by cases.

        \bigskip

        \underline{\textbf{Case 1 (when list doesn’t have 0s):}}

        \bigskip

        The definition of $S_{n,i}$ tells us $0 \leq i \leq n$, $S_{n,i}$ contains
        all lists with 0 starting at ith position.

        \bigskip

        Using the facts, we can conclude $S_{n,n}$ is a set of lists containing 0 at $n^{th}$
        position.

        \bigskip

        Since $i$ in a list starts at $i = 0$ and ends at $i = n-1$, there are no 0
        in the list of a set $S_{n,n}$.

        \bigskip

        Then, we can conclude $S_{n,n}$ is one and the only that contains a list of only 1s.

        \bigskip

        \underline{\textbf{Case 2 (when list has one or more 0s):}}

        \bigskip

        \color{red}
        The definition of $S_{n,i}$ tells us, the set $S_{n,i}$ contains
        all lists with 0 starting at $i^{th}$ position.
        \color{black}

        \bigskip

        Because we know this list has 0 starting at $i^{th}$ position, \color{red}using the
        fact\color{black}, we can conclude this list exists in the set $S_{n,i}$.

    \end{mdframed}


    \item

    The definition of exact expression for average-case running time is

    \setcounter{equation}{0}
    \begin{align}
        AVG_{\text{has\_even}}(n) &= \frac{1}{\lvert \mathcal{I}_n \rvert} \cdot \sum\limits_{lst \in \mathcal{I}_n} \text{Runtime of has\_even(lst)}
    \end{align}

    \bigskip

    Then,

    \begin{align}
        AVG_{\text{has\_even}}(n) &= \frac{1}{2^n} \cdot \sum\limits_{lst \in \mathcal{I}_n} \text{Runtime of has\_even(lst)}
    \end{align}

    by the fact $\lvert \mathcal{I}_n \rvert = 2^n$ from the solution of question 1.a.

    \bigskip

    Then,

    \begin{align}
        AVG_{\text{has\_even}}(n) &= \frac{1}{2^n} \cdot \sum\limits_{i=0}^{n-1} \sum\limits_{\substack{lst \in S_{n,i}\\ lst[i]=0}} \text{Runtime of has\_even(lst)}
    \end{align}

    by the fact $\sum\limits_{lst \in \mathcal{I}_n}$ can be re-expressed as $\sum\limits_{i=0}^{n-1} \sum\limits_{\substack{lst \in S_{n,i}\\ lst[i]=0}}$.

    \bigskip

    Then,

    \begin{align}
        AVG_{\text{has\_even}}(n) &= \frac{1}{2^n} \cdot \sum\limits_{i=0}^{n-1} \sum\limits_{\substack{lst \in S_{n,i}\\ lst[i]=0}} (i+1)
    \end{align}

    by the fact the loop starts at $0$ and ends at $i$.

    \bigskip

    Then,

    \begin{align}
        AVG_{\text{has\_even}}(n) &= \frac{1}{2^n} \cdot \sum\limits_{i=0}^{n-1} 2^{n-i-1} (i+1)\\
        &= \sum\limits_{i=0}^{n-1} \frac{i+1}{2^{i+1}}
    \end{align}

    by the fact there are total of $2^{n-i-1}$ many lists in each $S_{n,i}$ from
    the solution of question 1.b.

\end{enumerate}

\end{document}