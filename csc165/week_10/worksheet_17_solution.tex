\documentclass[12pt]{article}
\usepackage[margin=2.5cm]{geometry}
\usepackage{enumerate}
\usepackage{amsfonts}
\usepackage{amsmath}
\usepackage{fancyhdr}
\usepackage{amsmath}
\usepackage{amssymb}
\usepackage{amsthm}
\usepackage{mdframed}

\begin{document}
\title{Worksheet 17 Solution}
\author{Hyungmo Gu}
\maketitle

\section*{Question 1}
\begin{enumerate}[a.]
    \item

    We need to determine $\lvert \mathcal{I}_n \rvert$.

    \bigskip

    The problem tells that the values in inputs are either 1 or 0, and we know
    $\mathcal{I}_n$ represents all possible inputs of size $n$ containing binary values.

    \bigskip

    After watching lecture videos, and reading notes, I do not yet understand the details
    of how to evaluate the $\mathcal{I}_n$, but from the pattern below

    \begin{align*}
        [0], [1], [1,0], [0,1], [1,1], [0,0], [0,0,0],[0,0,1],[0,1,0],[1,0,0],
        [1,1,0],[1,0,1],[0,1,1],[1,1,1]
    \end{align*}

    we can see the inputs of size $1$ have $2$ different inputs, the inputs of size $2$ have $4$
    different inputs, and the inputs of size $3$ have $8$ different inputs.

    \bigskip

    Using this pattern, I can make an educated guess that $\lvert \mathcal{I}_n \rvert = 2^n$.

    \bigskip

    \textbf{Notes:}

    \begin{itemize}
     \item The idea of average-case analysis is that some data structures and
     algorithms have poor worst-case performance but perform well in vast majority
     of others.

    \item Average-case analysis looks at running time on sets of inputs
    \item Average case: $AVG_{func}(n) = avg \{ \text{runtime of func(x)} \mid x \in \mathcal{I}_n \}$
    \item Worst case: $WC_{func}(n) = max \{ \text{runtime of func(x)} \mid x \in \mathcal{I}_n \}$

    \end{itemize}


\end{enumerate}

\end{document}