\documentclass[12pt]{article}
\usepackage{enumerate}
\usepackage{amsfonts}
\usepackage{amsmath}
\usepackage{fancyhdr}
\usepackage{amsmath}
\usepackage{amssymb}
\usepackage{amsthm}
\usepackage{listings}
\usepackage{mdframed}
\usepackage{graphicx}

\begin{document}
\title{Midterm 2 Version 2 Solution}
\author{Hyungmo Gu}
\maketitle

\section*{Question 1}
\begin{enumerate}[a.]
    \item

    \begin{align*}
        100 \div 3 = 33,&\:\text{Remainder}\:\textbf{1}\\
        33 \div 3 = 11,&\:\text{Remainder}\:\textbf{0}\\
        11 \div 3 = 3,&\:\text{Remainder}\:\textbf{2}\\
        3 \div 3 = 1,&\:\text{Remainder}\:\textbf{0}\\
        1 \div 3 = 0,&\:\text{Remainder}\:\textbf{1}
    \end{align*}

    \bigskip

    It follows from above that the ternary representation of 100 is $(10201)_3$.

    \item The largest number expressible by an n-digit binary representation is

    \setcounter{equation}{0}
    \begin{align}
        \sum\limits_{i=0}^{n-1} 2^i
    \end{align}


\end{enumerate}

\section*{Question 2}

\section*{Question 3}

\section*{Question 4}

\end{document}