\documentclass[12pt]{article}
\usepackage[margin=2.5cm]{geometry}
\usepackage{enumerate}
\usepackage{amsfonts}
\usepackage{amsmath}
\usepackage{fancyhdr}
\usepackage{amsmath}
\usepackage{amssymb}
\usepackage{amsthm}
\usepackage{listings}
\usepackage{mdframed}
\usepackage{graphicx}

\begin{document}
\title{Midterm 2 Version 2 Solution}
\author{Hyungmo Gu}
\maketitle

\section*{Question 1}
\begin{enumerate}[a.]
    \item

    \begin{align*}
        100 \div 3 = 33,&\:\text{Remainder}\:\textbf{1}\\
        33 \div 3 = 11,&\:\text{Remainder}\:\textbf{0}\\
        11 \div 3 = 3,&\:\text{Remainder}\:\textbf{2}\\
        3 \div 3 = 1,&\:\text{Remainder}\:\textbf{0}\\
        1 \div 3 = 0,&\:\text{Remainder}\:\textbf{1}
    \end{align*}

    \bigskip

    It follows from above that the ternary representation of 100 is $(10201)_3$.

    \item The largest number expressible by an n-digit binary representation is

    \setcounter{equation}{0}
    \begin{align}
        \sum\limits_{i=0}^{n-1} 2^i
    \end{align}

    \item

    \begin{tabular}{|l|c|l|c|l|c|}
        \hline
        $f(n) \in \mathcal{O}(n)$ & True & $g(n) \in \Omega(n)$ & False & $f(n) \in \Omega(g(n))$ & True\\
        \hline
        $f(n) \in \Theta(g(n))$ & False & $g(n) \in \Theta(\log_3 n)$ & False & $f(n) + g(n) \in \Theta(f(n))$ & True\\
        \hline
    \end{tabular}

    \bigskip

    \textbf{Notes:}

    \begin{itemize}
        \item Learned $\sqrt{n}$ rises faster than $\log n$.
        \item Learned if $g(n) \in \Theta(f(n))$ is true then
        $f(n) + g(n) \in \Theta(f(n))$ is true.
    \end{itemize}



\end{enumerate}

\section*{Question 2}

\section*{Question 3}

\section*{Question 4}

\end{document}