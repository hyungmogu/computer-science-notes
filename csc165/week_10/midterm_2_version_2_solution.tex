\documentclass[12pt]{article}
\usepackage[margin=2.5cm]{geometry}
\usepackage{enumerate}
\usepackage{amsfonts}
\usepackage{amsmath}
\usepackage{fancyhdr}
\usepackage{amsmath}
\usepackage{amssymb}
\usepackage{amsthm}
\usepackage{listings}
\usepackage{mdframed}
\usepackage{graphicx}

\begin{document}
\title{Midterm 2 Version 2 Solution}
\author{Hyungmo Gu}
\maketitle

\section*{Question 1}
\begin{enumerate}[a.]
    \item

    \begin{align*}
        100 \div 3 = 33,&\:\text{Remainder}\:\textbf{1}\\
        33 \div 3 = 11,&\:\text{Remainder}\:\textbf{0}\\
        11 \div 3 = 3,&\:\text{Remainder}\:\textbf{2}\\
        3 \div 3 = 1,&\:\text{Remainder}\:\textbf{0}\\
        1 \div 3 = 0,&\:\text{Remainder}\:\textbf{1}
    \end{align*}

    \bigskip

    It follows from above that the ternary representation of 100 is $(10201)_3$.

    \bigskip

    \begin{mdframed}
        \underline{\textbf{Attempt 2:}}

        \color{red}
        \setcounter{equation}{0}
        \begin{align*}
            100 + (-1 \cdot 3^4) &= 100 - 81 = 19\\
            19 + (-1 \cdot 3^3) &= 19 - 27 = -8\\
            -8 + (+1 \cdot 3^2) &= -8 + 9 = 1\\
            1 + (0 \cdot 3^1) &= 1 + 0 = 1\\
            1 + (-1 \cdot 3^0) &= 1 - 1 = 0
        \end{align*}

        So by flipping the signs, and reading from top to bottom, we can conclude
        the balanced ternary representation of 100 is $(11T101)_{bt}$
        \color{black}
    \end{mdframed}

    \textbf{Notes:}

    \begin{itemize}
        \item Balanced ternary representation expresses a decimal using $1$, $0$ and $-1$
        \item \textbf{T} represents negative sign in balanced ternary representation.
        \item Is my way of calculating balanced ternary representation correct? My
        approach was `which sign should be used given $3^n$ so the calculation stops
        at $3^0$?'
    \end{itemize}


    \item The largest number expressible by an n-digit binary representation is

    \setcounter{equation}{0}
    \begin{align}
        \sum\limits_{i=0}^{n-1} 2^i
    \end{align}

    \begin{mdframed}
        \underline{\textbf{Correct Solution:}}

        \setcounter{equation}{0}
        \begin{align}
            \sum\limits_{i=0}^{n-1} 2^i \color{red}= \frac{1 - 2^{n-1+1}}{1-2} = 2^n-1\color{black}
        \end{align}
    \end{mdframed}

    \bigskip

    \textbf{Notes:}

    \begin{itemize}
        \item Noticed professor simplified solution using geometric series
        \item Geometric series with finite sum

        \begin{align}
            \sum\limits_{i=0}^{n} r^k = \frac{1-r^{n+1}}{1-r},\:\text{where}\:\lvert r \rvert > 1
        \end{align}
    \end{itemize}

    \item

    \begin{tabular}{|l|c|l|c|l|c|}
        \hline
        $f(n) \in \mathcal{O}(n)$ & True & $g(n) \in \Omega(n)$ & False & $f(n) \in \Omega(g(n))$ & True\\
        \hline
        $f(n) \in \Theta(g(n))$ & False & $g(n) \in \Theta(\log_3 n)$ & False & $f(n) + g(n) \in \Theta(f(n))$ & True\\
        \hline
    \end{tabular}

    \bigskip

    \textbf{Notes:}

    \begin{itemize}
        \item Learned $\sqrt{n}$ rises faster than $\log n$.
        \item Learned if $g(n) \in \Theta(f(n))$ is true then
        $f(n) + g(n) \in \Theta(f(n))$ is true.
    \end{itemize}

    \item

    \begin{tabular}{|c|c|c|c|c|}
        \hline
        $k$ & 0 & 1 & 2 & 3\\
        \hline
        $i \cdot i \cdot i$ & $2 = 2^{3^0}$ & $2^3 = 2^{3^1}$ & $2^9 = 2^{3^2}$ & $2^27 = 2^{3^3}$\\
        \hline
    \end{tabular}

    We can deduce from above that $i_k = 2^{3^k}$

    \item

    $\lceil \log_3 (\log_2 (n) - 1) \rceil$

    \newpage

    \begin{mdframed}
        \underline{\textbf{Correct Solution:}}

        \bigskip

        \color{red} We want to find the smallest value of $k$ satisfying $2 \cdot
        i_k \geq n$, and the value is\color{black}

        \begin{align*}
            \lceil \log_3 (\log_2 (n) - 1) \rceil
        \end{align*}
    \end{mdframed}

\end{enumerate}

\section*{Question 2}
\begin{itemize}
    \item

    \textbf{Statement:} $\forall n \in \mathbb{N}, n \geq 2 \Rightarrow
    \prod\limits_{i=1}^n \frac{2^i - 1}{2^i} \geq \frac{1}{2n}$

    \bigskip

    \begin{proof}
        Let $n \in \mathbb{N}$. Assume $n \geq 2$.

        \bigskip

        We will prove the statement using induction on $n$.

        \bigskip

        \textbf{Base Case (n = 2):}

        \bigskip

        Let $n = 2$.

        \bigskip

        We want to show $\prod\limits_{i=1}^2 \frac{2^i - 1}{2^i} \geq \frac{1}{2 \cdot (2)}$

        \bigskip

        Starting from $\prod\limits_{i=1}^2 \frac{2^i - 1}{2^i}$, we can conclude

        \setcounter{equation}{0}
        \begin{align}
            \prod\limits_{i=1}^2 \frac{2^i - 1}{2^i} = \left( \frac{1}{2} \right) \cdot \left( \frac{3}{4} \right) &= \frac{3}{8}\\
            &\geq \frac{2}{8}\\
            &\geq \frac{1}{4}
        \end{align}

        \bigskip

        \textbf{Inductive Case:}

        \bigskip

        Let $n \in \mathbb{N}$. Assume $\prod\limits_{i=1}^n \frac{2^i - 1}{2^i}
        \geq \frac{1}{2n}$.

        \bigskip

        We want to show $\prod\limits_{i=1}^{n+1} \frac{2^i - 1}{2^i} \geq \frac{1}{2(n+1)}$.

        \bigskip

        Starting from $\frac{1}{2(n+1)}$, because we know $n \geq 1$, we can conclude

        \begin{align}
            \frac{1}{2(n+1)} &\leq \frac{1}{2 \cdot (n+n)}\\
            &= \frac{1}{2 \cdot 2n}
        \end{align}

        \bigskip

        Then, using inductive hypothesis $\prod\limits_{i=1}^n \frac{2^i - 1}{2^i}
        \geq \frac{1}{2n}$, we can conclude that

        \begin{align}
            \frac{1}{2 \cdot (n+1)} &\leq \prod\limits_{i=1}^n \frac{2^i - 1}{2^i} \cdot \frac{1}{2}\\
            &= \prod\limits_{i=1}^n \frac{2^i - 1}{2^i} \cdot \left( 1 - \frac{1}{2} \right)\\
            &< \prod\limits_{i=1}^n \frac{2^i - 1}{2^i} \cdot \left( 1 - \frac{1}{2^{n+1}} \right)\\
            &= \prod\limits_{i=1}^n \frac{2^i - 1}{2^i} \cdot \left( \frac{2^{n+1}}{2^{n+1}} - \frac{1}{2^{n+1}} \right)\\
            &= \prod\limits_{i=1}^n \frac{2^i - 1}{2^i} \cdot \left( \frac{2^{n+1} - 1}{2^{n+1}} \right)\\
            &= \prod\limits_{i=1}^{n+1} \frac{2^i - 1}{2^i}
        \end{align}
    \end{proof}

    \bigskip

    \textbf{Notes:}

    \begin{itemize}
        \item Solved the inductive part of the problem by starting from the left
        hand side and calculating to the right as far as I can, and then repeating
        the same procedure from the right to the left until there were few missing
        steps left in between.
    \end{itemize}


\end{itemize}

\section*{Question 3}
\begin{itemize}
    \item

    \textbf{Statement:} $\exists a \in \mathbb{R}^{+}, an^2 + 1 \in \Theta(n^4)$

    \textbf{Negation of Statement:} $\forall a,c_1,c_2,n_0 \in \mathbb{R}^{+},\:
    \exists n \in \mathbb{N},\:(n \geq n_0)\land \Bigl( (c_1n^4 > an^2 + 1) \lor
    (an^2 + 1 > c_2n^4) \Bigr)$

    \begin{proof}
    Let $a,c_1,n_0 \in \mathbb{R}^{+}$, and $n = \left\lceil max(n_0,
    \sqrt[4]{\frac{2}{c_1}},\sqrt{\frac{2a}{c_1}}) \right\rceil + 1$.

    \bigskip

    We will disprove the statement by showing $(n \geq n_0)$ and $(c_1n^4 > an^2 + 1)$.

    \bigskip

    \textbf{Part 1 ($n \geq n_0$):}

    \bigskip

    Because we know $\left\lceil max(n_0,\sqrt[4]{\frac{2}{c_1}},\sqrt{\frac{2a}{c_1}}) \right\rceil \geq n_0$,
    we can conclude

    \setcounter{equation}{0}
    \begin{align}
        n = \left\lceil max(n_0, \sqrt[4]{\frac{2}{c_1}},\sqrt{\frac{2a}{c_1}}) \right\rceil + 1 > n_0
    \end{align}

    \bigskip

    \textbf{Part 2 (Showing $c_1n^4 > an^2 + 1$):}

    \bigskip

    We need to show $c_1n^4 > an^2 + 1$.

    \bigskip

    We will do so by showing $\frac{c_1n^4}{2} > an^2$ and $\frac{c_1n^4}{2} > 1$,
    and then combining them.

    \bigskip

    For the first inequality, using the fact that $\sqrt{\frac{2a}{c_1}} <
    \left\lceil max(n_0, \sqrt[4]{\frac{2}{c_1}},\sqrt{\frac{2a}{c_1}}) \right\rceil + 1 = n$,
    we can calculate

    \begin{align}
        \sqrt{\frac{2a}{c_1}} &< n\\
        \frac{2a}{c_1} &< n^2\\
        2a &< c_1n^2\\
        a &< \frac{c_1n^2}{2}\\
        an &< \frac{c_1n^3}{2}
    \end{align}

    \bigskip

    For the second inequality, using the fact that $\sqrt[4]{\frac{2}{c_1}} <
    \left\lceil max(n_0, \sqrt[4]{\frac{2}{c_1}},\sqrt{\frac{2a}{c_1}}) \right\rceil + 1 = n$,
    we can calculate

    \begin{align}
        \sqrt[4]{\frac{2}{c_1}} &< n\\
        \frac{2}{c_1} &< n^4\\
        1 &< \frac{c_1 \cdot n^4}{2}
    \end{align}

    \bigskip

    Then, by combining the two results together,

    \begin{align}
        1 + an^2 &< \frac{c_1n^4}{2} + \frac{c_1n^4}{2}\\
        &< c_1n^4
    \end{align}

    \end{proof}

\end{itemize}

\section*{Question 4}
\begin{enumerate}[a.]
    \item

    Let $n \in \mathbb{N}$.

    \bigskip

    We need to evaluate the total number of iterations.

    \bigskip

    We will do so by calculating the number of iterations in loop 2, and then calculating
    the number of iterations in loop 1.

    \bigskip

    For loop 2, the code tells us it starts at $j = 0$ and ends at $j = i^2 - 1$,
    and $j$ increases by 2 per iteration.

    \bigskip

    Using this fact, we can conclude that loop 2 has
    \setcounter{equation}{0}
    \begin{align}
        \left\lceil \frac{i^2 - 1 - 0 + 1}{2} \right\rceil
    \end{align}

    iterations.

    \bigskip

    Furthermore, since each iteration takes 1 step, we can conclude loop 2 has total of

    \begin{align}
        \left\lceil \frac{i^2}{2} \right\rceil \cdot 1 =  \left\lceil \frac{i^2}{2} \right\rceil
    \end{align}

    iterations.

    \bigskip

    For loop 1, the code tells us that it starts at $i = n$ and ends at $i = 1$
    with $i$ decreasing by 1 per iteration.

    \bigskip

    Because we know each iteration takes $\left\lceil \frac{i^2}{2} \right\rceil$ steps,
    we can conclude the loop 1 has

    \begin{align}
        \left\lceil \frac{n^2}{2} \right\rceil + \left\lceil \frac{(n - 1)^2}{2} \right\rceil + \cdots
        + \left\lceil \frac{1^2}{2} \right\rceil = \sum\limits_{i=1}^{n} \left\lceil \frac{i^2}{2} \right\rceil
    \end{align}

    iterations.

    \bigskip

    Since the signs of ceilings and floors can be ignored, $\sum\limits_{i=1}^{n} \left\lceil \frac{i^2}{2} \right\rceil$
    can be simplified to

    \begin{align}
        \sum\limits_{i=1}^{n} \frac{i^2}{2} &= \frac{1}{2} \sum\limits_{i=1}^{n} i^2
    \end{align}

    \bigskip

    Then, since $\sum\limits_{i=1}^{n} i^2 = \frac{n(n+1)(2n+1)}{6}$,

    \begin{align}
        \frac{1}{2} \sum\limits_{i=1}^{n} i^2 &= \frac{1}{2} \cdot \frac{n(n+1)(2n+1)}{6}\\
        &= \frac{n(n+1)(2n+1)}{12}
    \end{align}

    \item

    \begin{proof}
        \underline{\textbf{Part 1 (Determining upper bound worst-case running time):}}

        \bigskip

        Let $n \in \mathbb{N}$. Assume the algorithm runs on worst possible case.

        \bigskip

        We will evaluate the upper bound worst-case running time by calculating the
        total cost of the algorithm, and then the big-oh.

        \bigskip

        First, we will evaluate the cost of loop 2.

        \bigskip

        The code tells us loop 2 starts at $j = i + 1$ and ends at $j = n - 1$, and
        $j$ increases incrementally.

        \bigskip

        Using the fact, we can calculate loop 2 has at most
        \setcounter{equation}{0}
        \begin{align}
            \left\lceil \frac{n-1-(i+1)+1}{2} \right\rceil &= n - i - 1
        \end{align}

        iterations.

        \bigskip

        Because we know each iteration in loop 2 costs a single step, we can conclude
        loop 2 has cost of at most

        \begin{align}
            (n - i - 1) \cdot 1 &= n - i - 1
        \end{align}

        steps.

        \bigskip

        Now, we will calculate the cost of loop 1.

        \bigskip

        Because we know the if condition \textbf{if lst[i] \% 2 == 0} will be satisfied
        at most $n$ times from $i = 0$ to $i = n - 1$, and because we know each
        iteration costs $(n - i - 1)$ steps, we can conclude the loop has cost of at most

        \begin{align}
            \sum\limits_{i=0}^{n-1} (n-i-1) &= \sum\limits_{i=0}^{n-1} \left[ (n-1) - i \right]\\
            &= \sum\limits_{i=0}^{n-1} (n-1) - \sum\limits_{i=0}^{n-1} i\\
            &= n \cdot (n-1) - \frac{n \cdot (n-1)}{2}\\
            &= \frac{n \cdot (n-1)}{2}
        \end{align}

        steps.

        \bigskip

        Then, it follows from above the loop has upper bound running time of
        $\mathcal{O}(n^2)$.

        \bigskip

        \underline{\textbf{Part 2 (Evaluating Lower Bound Running Time):}}

        \bigskip

        Let $n \in \mathbb{N}$, and $lst = [2,3,4,5,\dots,n]$.

        \bigskip

        We need to calculate the total cost of the algorithm given the input group,
        and the omega.

        \bigskip

        Since we know the line \textbf{list[j] = list[j] + 1} will turn all $i^{th}$
        value to an even number, we can conclude the if condition \textbf{if list[i] \% 2 == 0}
        and it's inner loop will run $n$ many times.

        \bigskip

        Since, this is equivalent to finding the total cost of algorithm for big-oh,
        the algorithm has total cost of

        \begin{align}
            \frac{n \cdot (n-1)}{2}
        \end{align}

        \bigskip

        Then, it follows from above the algorithm has lower bound running time of
        $\Omega(n^2)$.

        \bigskip

        Then, since the value in $\mathcal{O}$ and $\Omega$ are equal, $\Theta(n^2)$
        is true.
    \end{proof}

\end{enumerate}

\end{document}