\documentclass[12pt]{article}
\usepackage[margin=2.5cm]{geometry}
\usepackage{enumerate}
\usepackage{amsfonts}
\usepackage{amsmath}
\usepackage{fancyhdr}
\usepackage{amsmath}
\usepackage{amssymb}
\usepackage{amsthm}
\usepackage{listings}
\usepackage{mdframed}
\usepackage{graphicx}

\begin{document}
\title{Midterm 2 Version 2 Solution}
\author{Hyungmo Gu}
\maketitle

\section*{Question 1}
\begin{enumerate}[a.]
    \item

    \begin{align*}
        100 \div 3 = 33,&\:\text{Remainder}\:\textbf{1}\\
        33 \div 3 = 11,&\:\text{Remainder}\:\textbf{0}\\
        11 \div 3 = 3,&\:\text{Remainder}\:\textbf{2}\\
        3 \div 3 = 1,&\:\text{Remainder}\:\textbf{0}\\
        1 \div 3 = 0,&\:\text{Remainder}\:\textbf{1}
    \end{align*}

    \bigskip

    It follows from above that the ternary representation of 100 is $(10201)_3$.

    \bigskip

    \begin{mdframed}
        \underline{\textbf{Attempt 2:}}

        \color{red}
        \setcounter{equation}{0}
        \begin{align*}
            100 + (-1 \cdot 3^4) &= 100 - 81 = 19\\
            19 + (-1 \cdot 3^3) &= 19 - 27 = -8\\
            -8 + (+1 \cdot 3^2) &= -8 + 9 = 1\\
            1 + (0 \cdot 3^1) &= 1 + 0 = 1\\
            1 + (-1 \cdot 3^0) &= 1 - 1 = 0
        \end{align*}

        So by flipping the signs, and reading from top to bottom, we can conclude
        the balanced ternary representation of 100 is $(11T101)_{bt}$
        \color{black}
    \end{mdframed}

    \textbf{Notes:}

    \begin{itemize}
        \item Balanced ternary representation expresses a decimal using $1$, $0$ and $-1$
        \item \textbf{T} represents negative sign in balanced ternary representation.
        \item Is my way of calculating balanced ternary representation correct? My
        approach was `which sign should be used given $3^n$ so the calculation stops
        at $3^0$?'
    \end{itemize}


    \item The largest number expressible by an n-digit binary representation is

    \setcounter{equation}{0}
    \begin{align}
        \sum\limits_{i=0}^{n-1} 2^i
    \end{align}

    \begin{mdframed}
        \underline{\textbf{Correct Solution:}}

        \setcounter{equation}{0}
        \begin{align}
            \sum\limits_{i=0}^{n-1} 2^i \color{red}= \frac{1 - 2^{n-1+1}}{1-2} = 2^n-1\color{black}
        \end{align}
    \end{mdframed}

    \bigskip

    \textbf{Notes:}

    \begin{itemize}
        \item Noticed professor simplified solution using geometric series
        \item Geometric series with finite sum

        \begin{align}
            \sum\limits_{i=0}^{n} r^k = \frac{1-r^{n+1}}{1-r},\:\text{where}\:\lvert r \rvert > 1
        \end{align}
    \end{itemize}

    \item

    \begin{tabular}{|l|c|l|c|l|c|}
        \hline
        $f(n) \in \mathcal{O}(n)$ & True & $g(n) \in \Omega(n)$ & False & $f(n) \in \Omega(g(n))$ & True\\
        \hline
        $f(n) \in \Theta(g(n))$ & False & $g(n) \in \Theta(\log_3 n)$ & False & $f(n) + g(n) \in \Theta(f(n))$ & True\\
        \hline
    \end{tabular}

    \bigskip

    \textbf{Notes:}

    \begin{itemize}
        \item Learned $\sqrt{n}$ rises faster than $\log n$.
        \item Learned if $g(n) \in \Theta(f(n))$ is true then
        $f(n) + g(n) \in \Theta(f(n))$ is true.
    \end{itemize}

    \item

    \begin{tabular}{|c|c|c|c|c|}
        \hline
        $k$ & 0 & 1 & 2 & 3\\
        \hline
        $i \cdot i \cdot i$ & $2 = 2^{3^0}$ & $2^3 = 2^{3^1}$ & $2^9 = 2^{3^2}$ & $2^27 = 2^{3^3}$\\
        \hline
    \end{tabular}

    We can deduce from above that $i_k = 2^{3^k}$

    \item

    $\lceil \log_3 (\log_2 (n) - 1) \rceil$

\end{enumerate}

\section*{Question 2}

\section*{Question 3}

\section*{Question 4}

\end{document}