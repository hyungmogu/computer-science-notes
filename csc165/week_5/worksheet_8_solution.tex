\documentclass[12pt]{article}
\usepackage{enumerate}
\usepackage{amsfonts}
\usepackage{amsmath}
\usepackage{fancyhdr}
\usepackage{amssymb}

\begin{document}
\title{Worksheet 8 Solution}
\maketitle

\section*{Question 1}
\begin{enumerate}[a.]
    \item

    $P(n): \forall n \in \mathbb{N},\: n \leq 2^{n}$.

    $\forall k \in \mathbb{N}, P(0) \land P(k) \Rightarrow P(k+1)$

    \bigskip

    Or, with $P$ fully expanded,

    $\forall k \in \mathbb{N},\:0 \leq 2^0 \land k \leq 2^k \Rightarrow k+1 \leq
    2^{k+1}$

    \item

    \textbf{Base Case:}

    \bigskip

    Let $n = 0$.

    \bigskip

    Then,
    \setcounter{equation}{0}
    \begin{align}
        (0) &\leq 2^0\\
        0 &\leq 1
    \end{align}

    \bigskip

    Since, $n \leq 2^n$ is true for $n = 0$, the base case holds.

    \bigskip

    \textbf{Inductive Case:}

    \bigskip

    Let $k \in \mathbb{N}$, and assume that $P(k)$ is true.

    \bigskip

    Then,
    \setcounter{equation}{0}
    \begin{align}
        2^{k+1} &= 2^k + 2^k\\
        &\geq k + k\\
    \end{align}

    Then,

    \begin{align}
        2^{k+1} &\geq k + k\\
        &\geq k + 1
    \end{align}

    by the fact that $k \in \mathbb{N}$ and $k \geq 1$.

    \bigskip

    Then, it follows from proof by induction that the statement $k \leq 2^k$ is
    true.

\end{enumerate}


\section*{Question 2}

\section*{Question 3}

\section*{Question 4}

\end{document}