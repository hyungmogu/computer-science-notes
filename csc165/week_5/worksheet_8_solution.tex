\documentclass[12pt]{article}
\usepackage{enumerate}
\usepackage{amsfonts}
\usepackage{amsmath}
\usepackage{fancyhdr}
\usepackage{amssymb}

\begin{document}
\title{Worksheet 8 Solution}
\maketitle

\section*{Question 1}
\begin{enumerate}[a.]
    \item

    $P(n): \forall n \in \mathbb{N},\: n \leq 2^{n}$.

    $\forall k \in \mathbb{N}, P(0) \land P(k) \Rightarrow P(k+1)$

    \bigskip

    Or, with $P$ fully expanded,

    $\forall k \in \mathbb{N},\:0 \leq 2^0 \land k \leq 2^k \Rightarrow k+1 \leq
    2^{k+1}$

    \item

    \textbf{Base Case:}

    \bigskip

    Let $n = 0$.

    \bigskip

    Then,
    \setcounter{equation}{0}
    \begin{align}
        (0) &\leq 2^0\\
        0 &\leq 1
    \end{align}

    \bigskip

    Since, $n \leq 2^n$ is true for $n = 0$, the base case holds.

    \bigskip

    \textbf{Inductive Case:}

    \bigskip

    Let $k \in \mathbb{N}$, and assume that $P(k)$ is true.

    \bigskip

    Then,
    \setcounter{equation}{0}
    \begin{align}
        2^{k+1} &= 2^k + 2^k\\
        &\geq k + k\\
    \end{align}

    Then,

    \begin{align}
        2^{k+1} &\geq k + k\\
        &\geq k + 1
    \end{align}

    by the fact that $k \in \mathbb{N}$ and $k \geq 1$.

    \bigskip

    Then, it follows from proof by induction that the statement $k \leq 2^k$ is
    true.

\end{enumerate}


\section*{Question 2}
\begin{itemize}
    \item

    \textbf{Base Case:}

    Let $n = 0$.

    \bigskip

    Then,
    \setcounter{equation}{0}
    \begin{align}
        \sum\limits_{j=0}^0 T_j &= \frac{(0)(0 + 1)(0 + 2)}{6}\\
        &= 0
    \end{align}

    \bigskip

    Since $T_0 = 0$, the base case holds.

    \bigskip

    \textbf{Inductive Case:}

    \bigskip

    Let $k \in \mathbb{N}$, and assume that $\sum\limits_{j=0}^k T_j =
    \frac{k(k+1)(k+2)}{6}$ is true.

    \bigskip

    Then,
    \setcounter{equation}{0}
    \begin{align}
        \sum\limits_{j=0}^k T_j + T_{k+1} &= \frac{k(k+1)(k+2)}{6} + \frac{(k+1)(k+2)}{2}\\
        &= \frac{k(k+1)(k+2)}{6} + \frac{3(k+1)(k+2)}{6}\\
        &= \frac{(k+1)(k+2)(k+3)}{6}
    \end{align}

    \bigskip

    Then, it follows from proof by induction that the statement $\forall n \in
    \mathbb{N},\:\sum\limits_{j=0}^k T_j = \frac{k(k+1)(k+2)}{6}$ is true.

\end{itemize}

\section*{Question 3}
\begin{enumerate}[a.]
    \item

    Let $x \in \mathbb{R}^{+}$, and let $n \in \mathbb{N}$. Assume $(1+x)^n \geq
    1 + nx$.

    \bigskip

    Then,
    \setcounter{equation}{0}
    \begin{align}
        (1+x)^{n+1} &= (1+x)^n(1+x)\\
        &\geq (1+nx)(1+x)
    \end{align}

    by the assumption $(1+x)^n \geq 1 + nx$.

    \bigskip

    Then,

    \begin{align}
        (1+x)^{n+1} &\geq (1+nx)(1+x)\\
        &\geq 1 + x + nx + nx^2\\
        &\geq 1 + x(n+1) + nx^2\\
        &\geq 1 + x(n+1)
    \end{align}

    Then, it follows from proof by induction that the statement $\forall x \in
    \mathbb{R}^{+},\:\forall n \in \mathbb{N},\:(1+x)^n \geq 1 + nx$ is true.

\end{enumerate}

\section*{Question 4}
\begin{enumerate}[a.]
    \item $\forall n \in \mathbb{N},\:n \geq 8 \Rightarrow 30n \leq 2^n$.
\end{enumerate}

\end{document}