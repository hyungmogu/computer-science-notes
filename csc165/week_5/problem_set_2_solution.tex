\documentclass[12pt]{article}
\usepackage{enumerate}
\usepackage{amsfonts}
\usepackage{amsmath}
\usepackage{fancyhdr}
\usepackage{amssymb}

\begin{document}
\title{Problem Set 2 Solution}
\maketitle

\section*{Question 1}
\begin{enumerate}[a.]
    \item

    \item

    Let $k,n \in \mathbb{Z}^{+}$, and $p \in \mathbb{N}$. Assume $Prime(p)$, and
    $p^k < n < p^k + p$.

    \bigskip

    Then, $p^k$ can either be divided by 1 or $p$ by fact 3.

    \bigskip

    Since, $p^k < n < p^k + p$, $n$ cannot be written in multiples of $p$.

    \bigskip

    Then, it follows from the definition of divisibility that $p \nmid n$.

    \bigskip

    Since $p \nmid n$, but $1 \mid p^k$ and $1 \mid n$, $gcd(p^k, n) = 1$.

    \item

    \textbf{Predicate Logic:} $\forall m \in \mathbb{Z}$, $\forall n_0 \in \mathbb{N}$,
    $\exists n \in \mathbb{N}$ $n > n_0 \land gcd(n, n+m) = 1$

    \bigskip

    Since there are infinitely many primes by fact 4, let $Prime(n)$ and $n > m$.

    \bigskip

    Since $Prime(n)$, by fact 3, $n$ can either be divided by 1 or $n$.

    \bigskip

    Since $n \mid n$, but $n \nmid m$, $n \nmid (n+m)$, and $n$ can't be chosen
    as the greatest common divisor of $n$ and $n+m$.

    \bigskip

    Since $gcd(n,n+m) \neq n$ but $1 \mid n$ and $1 \mid (n+m)$, $gcd(n,n+m)=1$.

    \bigskip

    Then, it follows from above that the statement $\forall m \in \mathbb{Z}$,
    $\forall n_0 \in \mathbb{N}$, $\exists n \in \mathbb{N}$ $n > n_0 \land
    gcd(n, n+m) = 1$ is true.

\end{enumerate}

\section*{Question 2}

\section*{Question 3}

\end{document}