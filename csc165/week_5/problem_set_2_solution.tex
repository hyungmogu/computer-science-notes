\documentclass[12pt]{article}
\usepackage{enumerate}
\usepackage{amsfonts}
\usepackage{amsmath}
\usepackage{fancyhdr}
\usepackage{amssymb}

\begin{document}
\title{Problem Set 2 Solution}
\maketitle

\section*{Question 1}
\begin{enumerate}[a.]
    \item

    \item

    \textbf{Predicate Logic:} $\forall k,n \in \mathbb{Z}^{+},\:\forall p \in
    \mathbb{N}$, $Prime(p) \land p^k < n < p^k + p \Rightarrow gcd(p^k,n) = 1$

    \bigskip

    Let $k,n \in \mathbb{Z}^{+}$, and $p \in \mathbb{N}$. Assume $Prime(p)$, and
    $p^k < n < p^k + p$.

    \bigskip

    Then, $p^k$ can either be divided by 1 or $p$ by fact 3.

    \bigskip

    Since, $p^k < n < p^k + p$, $n$ cannot be written in multiples of $p$.

    \bigskip

    Then, it follows from the definition of divisibility that $p \nmid n$.

    \bigskip

    Since $p \nmid n$, but $1 \mid p^k$ and $1 \mid n$, $gcd(p^k, n) = 1$.

    \item

    \textbf{Predicate Logic:} $\forall m \in \mathbb{Z}$, $\forall n_0 \in \mathbb{N}$,
    $\exists n \in \mathbb{N}$ $n > n_0 \land gcd(n, n+m) = 1$

    \bigskip

    Since there are infinitely many primes by fact 4, let $Prime(n)$ and $n > m$.

    \bigskip

    Since $Prime(n)$, by fact 3, $n$ can either be divided by 1 or $n$.

    \bigskip

    Since $n \mid n$, but $n \nmid m$, $n \nmid (n+m)$, and $n$ can't be chosen
    as the greatest common divisor of $n$ and $n+m$.

    \bigskip

    Since $gcd(n,n+m) \neq n$ but $1 \mid n$ and $1 \mid (n+m)$, $gcd(n,n+m)=1$.

    \bigskip

    Then, it follows from above that the statement $\forall m \in \mathbb{Z}$,
    $\forall n_0 \in \mathbb{N}$, $\exists n \in \mathbb{N}$ $n > n_0 \land
    gcd(n, n+m) = 1$ is true.

    \item

    \textbf{Definition of Primary Gap:} Let $a \in \mathbb{N}$. We say that $a$
    is a prime gap when there exists a prime $p$ such that $p+a$ is also prime,
    and none of the numbers between $p$ and $p + a$ (exclusive) are prime.

    \bigskip

    \textbf{Case 1 ($a > 2$):}

    \bigskip

    Let $a,p \in \mathbb{Z}^{+}$. Assume $PrimaryGap(a)$, $Primary(p)$, and $a > 2$.

    \bigskip

    Then, $2 \nmid p$ and $2 \nmid p + a$.

    \bigskip

    Then,
    \setcounter{equation}{0}
    \begin{align}
        2 &\mid (p + a) - a\\
        2 &\mid a
    \end{align}

    by fact 1.

    \bigskip

    Then it follows from above that in case $a > 2$, primary gap is divisible by
    2.

    \bigskip

    \textbf{Case 2 ($a \leq 2$):}

    \bigskip

    Let $a,p \in \mathbb{Z}^{+}$. Assume $PrimaryGap(a)$, $Primary(p)$, and
    $a \leq 2$.

    \bigskip

    Then, only two primary numbers in $\mathbb{Z}^{+}$ exist - 1 and 2.

    \bigskip

    Then,
    \setcounter{equation}{0}
    \begin{align}
        a &= 2 - 1\\
        a &= 1
    \end{align}

    \bigskip

    Then, it follows from above that in case $a \leq 2$, the value of primary gap
    is 1.

\end{enumerate}

\section*{Question 2}

\begin{enumerate}[a.]
    \item

    Let $n \in \mathbb{N}$, and $x \in \mathbb{R}$.

    \bigskip

    Because we know $\forall x \in \mathbb{R},\: 0 \leq x - \lfloor x \rfloor < 1$ from
    fact 1, we can conclude $\lfloor x \rfloor \leq x < 1 + \lfloor x \rfloor$.

    \bigskip

    Then,
    \setcounter{equation}{0}
    \begin{align}
        \lfloor nx \rfloor - n \lfloor x \rfloor &\leq nx - n\lfloor x \rfloor\\
        &\leq n(x - \lfloor x \rfloor)
    \end{align}

    by using the above.

    \bigskip

    Then,

    \begin{align}
        \lfloor nx \rfloor - n \lfloor x \rfloor &\leq n(x - \lfloor x \rfloor)\\
        &< n\\
        &< k
    \end{align}

    by using fact 1 and choosing $k = n$.

    \bigskip

    Then, it follows that the statement the statement
    $\forall n \in \mathbb{N},\:\exists k \in \mathbb{N},\:x \in \mathbb{R},\:\lfloor
    nx \rfloor - n \lfloor x \rfloor \leq k$ is true.

    \item

    \textbf{Negation of statement:} $\forall k \in \mathbb{N},\: \exists m \in \mathbb{N},\:
    \exists x \in \mathbb{R},\: \lfloor nx \rfloor - n \lfloor x \rfloor > k$

    \bigskip

    Let $x = 0.5$ and $n = 2(k+1)$.

    \bigskip

    Then,
    \setcounter{equation}{0}
    \begin{align}
        \lfloor nx \rfloor - n \lfloor x \rfloor &= \lfloor \frac{2(k+1)}{2} \rfloor - n\lfloor 0.5 \rfloor\\
        &= k + 1 - 0\\
        &= k + 1\\
        &> k
    \end{align}

    \bigskip

    Then it follows that the statement $\exists k \in \mathbb{N},\:\forall n \in \mathbb{N},\:
    \forall x \in \mathbb{R},\: \lfloor nx \rfloor - n \lfloor x \rfloor \leq k$ is false.


\end{enumerate}

\section*{Question 3}

\begin{enumerate}[a.]
    \item

    \textbf{Predicate Logic}: $\forall f: \mathbb{R} \to \mathbb{R},\: f(x) = f(-x) \land
    -f(-x) = f(x) \leftrightarrow f = 0$

    \bigbreak

    \textbf{Part 1: Proving in $\Rightarrow$ direction}

    \bigbreak

    Let $f: \mathbb{R} \to \mathbb{R}$. Assume $f(x) = f(-x) \land -f(-x)=f(x)$.

    \bigbreak

    Then,
    \setcounter{equation}{0}
    \begin{align}
        f(-x) - f(-x) &= 2f(x)\\
        0 &= 2f(x)
    \end{align}

    by adding $f(x) = f(-x)$ and$-f(-x) = f(x)$ together.

    \bigbreak

    Then,

    \begin{align}
        0 &= f(x)
    \end{align}

    \bigbreak

    Then it follows that the statement $\forall f: \mathbb{R} \to \mathbb{R},\:
    f(x) = f(-x) \land -f(-x) = f(x) \Rightarrow f = 0$ is true.

    \bigbreak

    \textbf{Part 2: Proving in $\Leftarrow$ direction}

    \bigbreak

    Let $f: \mathbb{R} \to \mathbb{R}$. Assume $f(x) = 0$.

    \bigbreak

    Then,
    \setcounter{equation}{0}
    \begin{align}
        -f(-x) &= -(-0)\\
        &= 0\\
        &= f(x)
    \end{align}

    It follows from above that $f(x) = 0$ is an odd function.

    \bigbreak

    Also,

    \begin{align}
        f(-x) &= (-0)\\
        &= 0\\
        &= f(x)
    \end{align}

    It follows from above that $f(x) = 0$ is an odd function.

    \bigbreak

    Because we know $f(x) = 0$ is both even and odd, we can conclude that the statement
    $\forall f: \mathbb{R} \to \mathbb{R},\:f = 0 \Rightarrow f(x) = f(-x) \land -f(-x)
    = f(x)$ is true.

    \item

    \textbf{Predicate Logic:} $\forall f:\mathbb{R} \to \mathbb{R},\:\exists f_1,
    f_2: \mathbb{R} \to \mathbb{R}, -f_1(x) = f_1(x) \land f_2(-x) = f_2(fx)
    \land f(x) = f_1(x) + f_2(x)$

    \bigskip

    \textbf{Negation:} $\exists f: \mathbb{R} \to \mathbb{R},\:\forall f_1,f_2:
    \mathbb{R} \to \mathbb{R},\:-f_1(-x) \neq f_1(x) \lor f_2(x) \neq f_2(x) \lor
    f(x) \neq f_1(x) + f_2(x)$

    \bigskip

    Let $f$ be an even function. Assume $Even(f_1)$ and $Odd(f_2)$.

    \bigskip

    Then,
    \setcounter{equation}{0}
    \begin{align}
        f(-x) &= (f_1(-x) + f_2(-x))\\
        &= f_1(x) - f_2(x)\\
        &\neq f(x)
    \end{align}

    \bigskip

    Then, it follows from negation of the statement that every function cannot be
    written as a sum of an even function and an odd function.

    \item

    \textbf{Predicate Logic:} $\forall f: \mathbb{R} \to \mathbb{R},\:\exists f_1,
    f_2: \mathbb{R} \to \mathbb{R},\:Even(f_1) \land Odd(f_2) \land f(x) = f_1(x)
    + f_2(x)$.

    \bigskip

    \textbf{Negation:} $\exists f: \mathbb{R} \to \mathbb{R},\:\forall f_1,f_2:
    \mathbb{R} \to \mathbb{R}, \neg Even(f_1) \lor \neg Odd(f_2) \lor f(x) \neq f_1(x)f_2(x)$

    \bigskip

    Let $f$ be an even function. Assume $Even(f_1)$ and $Odd(f_2)$.

    \bigskip

    Then,

    \begin{align}
        f(-x) &= f_1(-x)f_2(-x)\\
        &= -f_1(x)f_2(x)\\
        &\neq f(x)
    \end{align}

    Then, it follows from negation of the statement that every function cannot be
    written as a product of an even function and an odd function.


\end{enumerate}

\end{document}