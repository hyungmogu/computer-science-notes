\documentclass[12pt]{article}
\usepackage{enumerate}
\usepackage{amsfonts}
\usepackage{amsmath}
\usepackage{fancyhdr}
\usepackage{amssymb}
\usepackage{mdframed}

\begin{document}
\title{Worksheet 9 Review}
\maketitle

\section*{Question 1}
\begin{enumerate}[a.]
    \item

    For every set of size 0 has 0 subsets of size 2.

    \item

    Let $n = 0$. Let $S$ be an arbitrary set. Assume $S$ has size 0.

    \bigskip

    Since $S$ has size 0, empty subsets are the \textbf{only} subsets that can be included
    in S.

    \bigskip

    Then, because we know empty subsets have size 0, we can conclude there are 0
    subsets of size 2.

    \bigskip

    It follows from above that the base case holds.


    \begin{mdframed}
        \underline{\textbf{Correct Solution:}}

        \color{red}
        \textbf{We want to show every set S of size 0 has 0 subsets of size 2}.
        \color{black}

        \bigskip

        Since $S$ has size 0, empty subsets are the \textbf{only} subsets that can be included
        in S.

        \bigskip

        Then, because we know an empty subset have size 0, we can conclude there are 0
        subsets of size 2.

    \end{mdframed}

    \textbf{Notes:}
    \begin{itemize}
        \item Professor specifically mentions \textbf{We want to show every set
        S of size 0 has 0 subsets of size 2}
        \item Professor doesn't include conclusion at the end of proof
        \item Under which cases conclusion to a proof are included.
    \end{itemize}

\end{enumerate}

\section*{Question 2}

\section*{Question 3}

\end{document}