\documentclass[12pt]{article}
\usepackage{enumerate}
\usepackage{amsfonts}
\usepackage{amsmath}
\usepackage{fancyhdr}
\usepackage{amssymb}
\usepackage{mdframed}
\usepackage{tabularx}

\begin{document}
\title{Worksheet 9 Review}
\maketitle

\section*{Question 1}
\begin{enumerate}[a.]
    \item

    For every set of size 0 has 0 subsets of size 2.

    \item

    Let $n = 0$. Let $S$ be an arbitrary set. Assume $S$ has size 0.

    \bigskip

    Since $S$ has size 0, empty subsets are the \textbf{only} subsets that can be included
    in S.

    \bigskip

    Then, because we know empty subsets have size 0, we can conclude there are 0
    subsets of size 2.

    \bigskip

    It follows from above that the base case holds.


    \begin{mdframed}
        \underline{\textbf{Correct Solution:}}

        \color{red}
        \textbf{We want to show every set S of size 0 has 0 subsets of size 2}.
        \color{black}

        \bigskip

        Since $S$ has size 0, empty subsets are the \textbf{only} subsets that can be included
        in S.

        \bigskip

        Then, because we know an empty subset have size 0, we can conclude there are 0
        subsets of size 2.

    \end{mdframed}

    \textbf{Notes:}
    \begin{itemize}
        \item Professor specifically mentions \textbf{We want to show every set
        S of size 0 has 0 subsets of size 2}
        \item Professor doesn't include conclusion at the end of proof
        \item Under which cases conclusion to a proof are included.
    \end{itemize}

    \item

    Now we will prove inductive step.

    \bigskip

    Let $k \in \mathbb{N}$. Assume every set of size $k$ has $\frac{k(k-1)}{2}$
    subsets of size 2.

    \bigskip

    We want to show a set of size $k+1$ has $\frac{(k+1)k}{2}$ subsets of size 2.

    \bigskip

    \underline{\textbf{Part 1: counting subsets of $S$ of size 2 that contain $s_{k+1}$}}

    \bigskip

    It follows from the table below,

    \begin{tabular}{|c|c|c|}
        \hline
        k & Sets & Subsets of Size 2 with $s_{k+1}$\\
        \hline
        0 & $\{0,1\}$ & 1\\
        \hline
        1 & $\{0,1,2\}$ & 2\\
        \hline
        2 & $\{0,1,2,3\}$ & 3\\
        \hline
        2 & $\{0,1,2,3,4\}$ & 4\\
        \hline
    \end{tabular}

    that the number of subsets of size 2 that contain $s_{k+1}$ is $k+1$.

    \bigskip

    \underline{\textbf{Part 2: counting subsets of $S$ of size 2 that doesn't contain $s_{k+1}$}}

    \bigskip

    Because we know the subset of $S$ that doesn't contain $s_{k+1}$ is a set S
    of size k, we can conclude using induction hypothesis that there are

    \begin{align}
        \frac{k(k+1)}{2}
    \end{align}

    subsets of size 2.

    \bigskip

    \underline{\textbf{Part 3: Putting the counts together}}

    \bigskip

    Then,

    \begin{align}
        \frac{k(k+1)}{2} + (k+1) &= \frac{(k+1)(k+2)}{2}
    \end{align}

    \bigskip

    Then, it follows from proof by induction that the statement '$\forall n \in \mathbb{N}$,
    every set of size n has $\frac{n(n-1)}{2}$ subsets of size 2' is true for all
    natural numbers n.


    \begin{mdframed}
        \underline{\textbf{Correct Solution:}}

        \underline{\textbf{Part 1: counting subsets of $S$ of size 2 that contain $s_{k+1}$}}

        \bigskip

        It follows from the table below,

        \color{red}
        \begin{tabular}{|c|c|c|}
            \hline
            k & Sets & Subsets of Size 2 with $s_{k+1}$\\
            \hline
            2 & $\{0,1\}$ & 1\\
            \hline
            3 & $\{0,1,2\}$ & 2\\
            \hline
            4 & $\{0,1,2,3\}$ & 3\\
            \hline
            5 & $\{0,1,2,3,4\}$ & 4\\
            \hline
        \end{tabular}
        \color{black}

        that the number of subsets of size 2 that contain $s_{k+1}$ is \color{red}\textbf{$k$}\color{black}.

        \bigskip

        \underline{\textbf{Part 2: counting subsets of $S$ of size 2 that doesn't contain $s_{k+1}$}}

        \bigskip

        Because we know the subset of $S$ that doesn't contain $s_{k+1}$ is a set S
        of size $k$, we can conclude using induction hypothesis that there are

        \color{red}
        \begin{align}
            \frac{k(k-1)}{2}
        \end{align}
        \color{black}

        subsets of size 2.

        \bigskip

        \underline{\textbf{Part 3: Putting the counts together}}

        \bigskip

        Then,

        \color{red}
        \begin{align}
            \frac{k(k-1)}{2} + k &= \frac{(k+1)k}{2}
        \end{align}
        \color{black}

        \bigskip

        Then, it follows from proof by induction that the statement '$\forall \color{red}k\color{black} \in \mathbb{N}$,
        every set of size \color{red}$k$\color{black} has \color{red}$\frac{k(k-1)}{2}$\color{black} subsets of size 2' is true for all
        natural numbers \color{red}$k$\color{black}.

    \end{mdframed}

    \textbf{Notes:}
    \begin{itemize}
        \item I forgot that k represents number of elements in a set.
    \end{itemize}

\end{enumerate}

\section*{Question 2}
\begin{itemize}
    \item

    \textbf{Statement:} For every $n \in \mathbb{N}$, every finite set $S$ of size $n$,
    has

    \setcounter{equation}{0}
    \begin{align}
        \frac{n(n-1)(n-2)}{6}
    \end{align}

    subsets of size 3.

    \bigskip

    We will prove this statement by using induction on $n$.

    \bigskip

    \textbf{Base Case:}

    \bigskip

    Let $n = 0$.

    \bigskip

    Then, only the empty subsets can be included in $S$.

    \bigskip

    Because an empty subset has size 0, there are 0 subsets of size 3 in S.

    \bigskip

    Then, since

    \begin{align}
        \frac{0 \cdot (0-1)(0-2)}{6} &= 0
    \end{align}

    the base case holds.

    \bigskip

    \textbf{Inductive Case:}

    \bigskip

    Let $k \in \mathbb{N}$. Assume every finite set $S$ of size $k$ has exactly
    $\frac{k(k-1)(k-2)}{6}$ subsets of size 3.

    \bigskip

    We want to show finite set $S$ of size $k+1$ contains $\frac{(k+1)k(k-1)}{6}$
    subsets of size 3.

    \bigskip

    It follows from the table below

    \bigskip

    \begin{tabular}{|c|c|c|c|}
        \hline
        k & Sets & \# of Subsets of Size 3 & \# of Subsets of Size 2\\
        \hline
        0 & $\{\}$ & 0 & 0\\
        \hline
        1 & $\{s_0\}$ & 0 & 0\\
        \hline
        2 & $\{s_0,s_1\}$ & 0 & 1\\
        \hline
        3 & $\{s_0,s_1,s_2\}$ & 1 & 3\\
        \hline
        4 & $\{s_0,s_1,s_2,s_3\}$ & 4 & 6\\
        \hline
        5 & $\{s_0,s_1,s_2,s_3,s_4\}$ & 10 & 10\\
        \hline
    \end{tabular}

    we can deduce that given a set S size $k+1$, the number of subsets of size 3
    containing $s_{k+1}$ is the sum of \# of subsets of size 3 that doesn't
    contain $s_{k+1})$ and \# of subsets of size 2 that doesn't contain $s_{k+1}$.

    \bigskip

    Then,

    \begin{align}
        \frac{k(k-1)(k-2)}{6} + \frac{k(k-1)}{2} &= \frac{k(k-1)(k-2)}{6} + \frac{3k(k-1)}{6}\\
        &= \frac{k(k-1)(k-2+3)}{6}\\
        &= \frac{k(k-1)(k+1)}{6}
    \end{align}

    \bigskip

    \textbf{Notes:}
    \begin{itemize}
        \item I wonder if table like above can be used in a proof. If not, why
        can't it be done? If yes, when is the use of table not valid? How should
        it be constructed that it's valid?
    \end{itemize}

\end{itemize}

\section*{Question 3}

\end{document}