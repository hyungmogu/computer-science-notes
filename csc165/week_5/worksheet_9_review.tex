\documentclass[12pt]{article}
\usepackage{enumerate}
\usepackage{amsfonts}
\usepackage{amsmath}
\usepackage{fancyhdr}
\usepackage{amssymb}
\usepackage{mdframed}
\usepackage{tabularx}

\begin{document}
\title{Worksheet 9 Review}
\maketitle

\section*{Question 1}
\begin{enumerate}[a.]
    \item

    For every set of size 0 has 0 subsets of size 2.

    \item

    Let $n = 0$. Let $S$ be an arbitrary set. Assume $S$ has size 0.

    \bigskip

    Since $S$ has size 0, empty subsets are the \textbf{only} subsets that can be included
    in S.

    \bigskip

    Then, because we know empty subsets have size 0, we can conclude there are 0
    subsets of size 2.

    \bigskip

    It follows from above that the base case holds.


    \begin{mdframed}
        \underline{\textbf{Correct Solution:}}

        \color{red}
        \textbf{We want to show every set S of size 0 has 0 subsets of size 2}.
        \color{black}

        \bigskip

        Since $S$ has size 0, empty subsets are the \textbf{only} subsets that can be included
        in S.

        \bigskip

        Then, because we know an empty subset have size 0, we can conclude there are 0
        subsets of size 2.

    \end{mdframed}

    \textbf{Notes:}
    \begin{itemize}
        \item Professor specifically mentions \textbf{We want to show every set
        S of size 0 has 0 subsets of size 2}
        \item Professor doesn't include conclusion at the end of proof
        \item Under which cases conclusion to a proof are included.
    \end{itemize}

    \item

    Now we will prove inductive step.

    \bigskip

    Let $k \in \mathbb{N}$. Assume every set of size $k$ has $\frac{k(k-1)}{2}$
    subsets of size 2.

    \bigskip

    We want to show a set of size $k+1$ has $\frac{(k+1)k}{2}$ subsets of size 2.

    \bigskip

    \underline{\textbf{Part 1: counting subsets of $S$ of size 2 that contain $s_{k+1}$}}

    \bigskip

    It follows from the table below,

    \begin{tabular}{|c|c|c|}
        \hline
        k & Sets & Subsets of Size 2 with $s_{k+1}$\\
        \hline
        0 & $\{0,1\}$ & 1\\
        \hline
        1 & $\{0,1,2\}$ & 2\\
        \hline
        2 & $\{0,1,2,3\}$ & 3\\
        \hline
        2 & $\{0,1,2,3,4\}$ & 4\\
        \hline
    \end{tabular}

    that the number of subsets of size 2 that contain $s_{k+1}$ is $k+1$.

    \bigskip

    \underline{\textbf{Part 2: counting subsets of $S$ of size 2 that doesn't contain $s_{k+1}$}}

    \bigskip

    Because we know the subset of $S$ that doesn't contain $s_{k+1}$ is a set S
    of size k, we can conclude using induction hypothesis that there are

    \begin{align}
        \frac{k(k+1)}{2}
    \end{align}

    subsets of size 2.

    \bigskip

    \underline{\textbf{Part 3: Putting the counts together}}

    \bigskip

    Then,

    \begin{align}
        \frac{k(k+1)}{2} + (k+1) &= \frac{(k+1)(k+2)}{2}
    \end{align}

    \bigskip

    Then, it follows from proof by induction that the statement '$\forall n \in \mathbb{N}$,
    every set of size n has $\frac{n(n-1)}{2}$ subsets of size 2' is true for all
    natural numbers n.

\end{enumerate}

\section*{Question 2}

\section*{Question 3}

\end{document}