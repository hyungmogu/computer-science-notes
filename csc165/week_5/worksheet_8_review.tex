\documentclass[12pt]{article}
\usepackage{enumerate}
\usepackage{amsfonts}
\usepackage{amsmath}
\usepackage{fancyhdr}
\usepackage{amssymb}
\usepackage{mdframed}
\usepackage{tabularx}
\usepackage{xcolor}

\begin{document}
\title{Worksheet 8 Review}
\maketitle

\section*{Question 1}
\begin{enumerate}[a.]
    \item

    $\forall n \in \mathbb{N,} (0 \leq 1) \land (n \leq 2^n) \Rightarrow (n+1) \leq 2^{n+1}$

    \bigskip

    \textbf{Note:}
    \begin{itemize}
        \item \textbf{Induction:} $\forall n \in \mathbb{N},\:P(0) \land P(n) \Rightarrow P(n+1)$
    \end{itemize}

    \item

    We will prove this statement by induction on n.

    \textbf{Base Case:}

    \bigskip

    Let $n = 0$.

    \bigskip

    Then,

    \begin{align}
        0 &\leq 2^0\\
        0 &\leq 1
    \end{align}

    \bigskip

    Since the above inequality is true, the base case holds.

    \bigskip

    \textbf{Inductive Case:}

    \bigskip

    Let $n \in \mathbb{N}$. Assume $P(n)$.

    \bigskip

    Then,

    \begin{align}
        n &\leq 2^n\\
        n+1 &\leq 2^n + 1\\
        n+1 &\leq 2^n + 2^n\\
        n+1 &\leq 2^{n+1}
    \end{align}

    by the fact $2^k + 2^k = 2^{k+1}$.

    \bigskip

    Then, it follows from proof by induction that the statement $n \leq 2^n$ is
    true for all $n$.

    \begin{mdframed}
        \underline{\textbf{Correct Solution:}}

        We will prove this statement by induction on n.

        \textbf{Base Case:}

        \bigskip

        Let $n = 0$.

        \bigskip

        Then,

        \begin{align}
            0 &\leq 2^0\\
            0 &\leq 1
        \end{align}

        \bigskip

        Since the above inequality is true, the base case holds.

        \bigskip

        \textbf{Inductive Case:}

        \bigskip

        Let $n \in \mathbb{N}$. Assume $P(n)$.

        \bigskip

        \textbf{We want to show $n+1 \leq 2^{n+1}$}.

        \bigskip

        Then,

        \begin{align}
            n &\leq 2^n\\
            n+1 &\leq 2^n + 1\\
            n+1 &\leq 2^n + 2^n\\
            n+1 &\leq 2^{n+1}
        \end{align}

        by the fact $2^n + 2^n = 2^{n+1}$.

        \bigskip

        Then, it follows from proof by induction that the statement $n \leq 2^n$ is
        true for all $n$.

    \end{mdframed}

    \bigskip

    \textbf{Notes:}
    \begin{itemize}
        \item professor specifically states what we want to show in inductive
        case part of the proof. I thought it was obvious, and not necessary.

        \item When are the times 'we want to show x' in proof can be omitted?

    \end{itemize}
\end{enumerate}

\section*{Question 2}

\begin{itemize}
    \item

    We will prove the statement by induction on natural number $n$.

    \bigskip

    \textbf{Base Case:}

    \bigskip

    Let $n = 1$.

    \bigskip

    Then,
    \setcounter{equation}{0}
    \begin{align}
        \sum\limits_{j=1}^{1} T_j &= 1 \frac{\cdot (1+1)(1+2)}{6}\\
        &= 1
    \end{align}

    \bigskip

    Since the data also shows value 1 at $n = 1$, the base case holds.

    \bigskip

    \textbf{Inductive Case:}

    \bigskip

    Let $n \in \mathbb{N}$. Assume $\sum\limits_{j=0}^n T_j = \frac{n \cdot (n+1)(n+2)}{6}$.

    \bigskip

    We want to show $\sum\limits_{j=0}^{n+1} T_j = \frac{(n+1)(n+2)(n+3)}{6}$.

    \bigskip

    It follows from the following table

    \begin{tabularx}{\textwidth}{|c|X|X|X|X|X|}
        \hline
        n & 1 & 2 & 3 & 4 & 5\\
        \hline
        $T_i = \frac{n \cdot (n+1)}{2}$ & 1 & 3 & 6 & 10 & 15\\
        \hline
        $\sum\limits_{j=1}^n T_j$ & 1 & 4 & 10 & 20 & 35\\
        \hline
    \end{tabularx}

    that $n+1^{th}$ value of the summation is $\frac{(n+1)(n+2)}{2}$ more than
    the $n^{th}$ sum.

    \bigskip

    Then,

    \begin{align}
        \sum\limits_{j=0}^{n+1} T_j &= \frac{n \cdot (n+1)(n+2)}{6} + \frac{(n+1)(n+2)}{2}\\
        &= \frac{n \cdot (n+1)(n+2)}{6} + \frac{3(n+1)(n+2)}{6}\\
        &= \frac{(n+1)(n+2)(n+3)}{6}
    \end{align}

    \begin{mdframed}
        \underline{\textbf{Correct Solution:}}

        We will prove the statement by induction on natural number $n$.

        \bigskip

        \textbf{Base Case:}

        \bigskip

        \color{red}
        \textbf{Let $n = 0$.}
        \color{black}

        \bigskip

        Then,
        \setcounter{equation}{0}
        \begin{align}
            \sum\limits_{j=0}^{1} T_j &= \frac{0 \cdot (0+1)(0+2)}{6}\\
            &= 0
        \end{align}

        \bigskip

        \color{red}
        \textbf{Since}

        \begin{align}
            \sum\limits_{j=0}^0 T_j &= T_0
        \end{align}

        \textbf{and,}

        \begin{align}
            T_0 &= \frac{0 \cdot (0+1)}{2}\\
            &= 0
        \end{align}

        \textbf{, the base case holds.}

        \color{black}

        \bigskip

        \textbf{Inductive Case:}

        \bigskip

        Let $n \in \mathbb{N}$. Assume $\sum\limits_{j=0}^n T_j = \frac{n \cdot (n+1)(n+2)}{6}$.

        \bigskip

        We want to show $\sum\limits_{j=0}^{n+1} T_j = \frac{(n+1)(n+2)(n+3)}{6}$.

        \bigskip

        It follows from the following table

        \begin{tabularx}{\textwidth}{|c|X|X|X|X|X|}
            \hline
            n & 1 & 2 & 3 & 4 & 5\\
            \hline
            $T_i = \frac{n \cdot (n+1)}{2}$ & 1 & 3 & 6 & 10 & 15\\
            \hline
            $\sum\limits_{j=1}^n T_j$ & 1 & 4 & 10 & 20 & 35\\
            \hline
        \end{tabularx}

        that $n+1^{th}$ value of the summation is $\frac{(n+1)(n+2)}{2}$ more than
        the $n^{th}$ sum.

        \bigskip

        Then,

        \begin{align}
            \sum\limits_{j=0}^{n+1} T_j &= \frac{n \cdot (n+1)(n+2)}{6} + \frac{(n+1)(n+2)}{2}\\
            &= \frac{n \cdot (n+1)(n+2)}{6} + \frac{3(n+1)(n+2)}{6}\\
            &= \frac{(n+1)(n+2)(n+3)}{6}
        \end{align}

    \end{mdframed}

    \bigskip

    \textbf{Notes:}
    \begin{itemize}
        \item I wasn't explicit about where the value 1 in data came from.
    \end{itemize}

\end{itemize}

\section*{Question 3}

\end{document}