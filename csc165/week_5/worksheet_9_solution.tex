\documentclass[12pt]{article}
\usepackage{enumerate}
\usepackage{amsfonts}
\usepackage{amsmath}
\usepackage{fancyhdr}
\usepackage{amssymb}

\begin{document}
\title{Worksheet 9 Solution}
\maketitle

\section*{Question 1}
\begin{enumerate}[a.]
    \item

    Every set $S$ of size 0 has $\frac{0(0-1)}{2} = 0$ subsets of size 2

    \item

    Let $n = 0$, and S be an arbitrary set. Assume $S$ has size 0.

    \bigskip

    Then, $S$ only has empty subsets by the fact that $S$ has size 0.

    \bigskip

    Since empty subset has size 0, there are 0 subsets with size 2.

    \item

    \textbf{Section 1:}

    \bigskip

    Every set of size $k$ has $\frac{k(k-1)}{2}$ subsets of size 2.

    \bigskip

    \textbf{Section 2:}

    \bigskip

    Every set of sie $k+1$ has $\frac{(k+1)k}{2}$ subsets of size 2.

    \bigskip

    \textbf{Section 3.1:}

    \bigskip

    Because we know

    \begin{tabular}{c | c | c}
        Index & Set & \# of subsets of size 2 containing last element\\
        \hline
        2 & $\{s_1,s_2\}$ & has 1 subset containing $s_2$\\
        \hline
        3 & $\{s_1,s_2,s_3\}$ & has 2 subsets containing $s_3$\\
        \hline
        4 & $\{s_1,s_2,s_3,s_4\}$ & has 3 subsets containing $s_4$
    \end{tabular}

    \bigskip

    , we can deduce from above that the number of subsets of size 2 containing
    $s_{k+1}$ is $k$.

    \bigskip

    \textbf{Section 3.2:}

    \bigskip

    P(n): $\forall n \in \mathbb{N}$, every set of size $n$ has $\frac{n(n-1)}{2}$
    subsets of size 2

    \bigskip

    Let $k \in \mathbb{N}$, and assume P(k).

    \bigskip

    Then, the number of subsets of $S$ of size 2 that don't contain $s_{k+1}$ is
    $\frac{k(k-1)}{2}$.

    \bigskip

    \textbf{Section 3.3:}

    \begin{align}
        \frac{k(k-1)}{2} + k &= \frac{k(k-1)}{2} + \frac{2k}{2}\\
        &= \frac{k}{2}\left[(k-1) + 2\right]\\
        &= \frac{k(k+1)}{2}
    \end{align}

    \bigskip

    Then, it follows from the proof of induction that the statement $\forall n
    \in \mathbb{N}$, every set of size $n$ has $\frac{n(n-1)}{2}$ is true.

\end{enumerate}

\section*{Question 2}
\begin{enumerate}[a.]
    \item

    P(n): Every finite set $S$ of size $n$ has exactly $\frac{n(n-1)(n-2)}{6}$
    subsets of size 3.

    \textbf{Base Case ($n = 0$):}

    \bigskip

    Let the size of $S$ be 0.

    \bigskip

    Then, $S$ only contains empty subsets.

    \bigskip

    Since an empty subset has size 0, $S$ has 0 subsets of size 3.

    \bigskip

    \textbf{Inductive Step:}

    \bigskip

    Let $n \in \mathbb{N}$.

    \bigskip

    By the table below

    \begin{tabular}{c | c | c}
        Index & Set & \# of subsets of size 3 containing last element\\
        \hline
        0 & $\{\}$ & 0\\
        \hline
        1 & $\{s_1\}$ & 0\\
        \hline
        2 & $\{s_1,s_2\}$ & 0\\
        \hline
        3 & $\{s_1,s_2,s_3\}$ & 1\\
        \hline
        4 & $\{s_1,s_2,s_3,s_4\}$ & 3\\
        \hline
        5 & $\{s_1,s_2,s_3,s_4,s_5\}$ & 6\\
    \end{tabular}

    ,we can deduce that the number of subsets of size 3 containing
    $s_{k+1}$ is $\frac{k(k-1)}{2}$.

    \bigskip

    Since the number of subsets of $S$ of size 3 that doesn't contain $s_{k+1}$ is
    $\frac{(k)(k-1)(k-2)}{6}$,
    \setcounter{equation}{0}
    \begin{align}
        \frac{k(k-1)(k-2)}{6} + \frac{k(k-1)}{2} &= \frac{k(k-1)(k-2)}{6} + \frac{3k(k-2)}{6}\\
        &= \frac{k(k-1)}{6}(k-2 + 3)\\
        &= \frac{(k+1)k(k-1)}{6}
    \end{align}

    Then, it follows from the proof of induction that the statement every finite
    set $S$ of size $n$ has exactly $\frac{n(n-1)(n-2)}{6}$ subsets of size 3 is true.

\end{enumerate}

\section*{Question 3}

\end{document}