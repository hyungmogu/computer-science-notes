\documentclass[12pt]{article}
\usepackage[margin=2.5cm]{geometry}
\usepackage{enumerate}
\usepackage{amsfonts}
\usepackage{amsmath}
\usepackage{fancyhdr}
\usepackage{amsmath}
\usepackage{amssymb}
\usepackage{amsthm}
\usepackage{mdframed}
\usepackage{graphicx}
\usepackage{subcaption}
\usepackage{listings}
\usepackage{xcolor}


\begin{document}
\title{Worksheet 1 Review}
\author{Hyungmo Gu}
\maketitle

\section*{Question 1}
\begin{enumerate}[a.]
    \item $A^c = \{1,3,4,6\}$
    \item $A = U \setminus A$
    \item

    $A^c \cap B^c = \{ x \mid x \in U,\:x \leq 0 \text{ and } x \geq 4 \}$

    $A^c \cup B^c = \{ x \mid x \in U,\:x < 1 \text{ and } x > 2 \}$

    $(A \cap B)^c = \{ x \mid x \in U,\:x < 1 \text{ and } x > 2\}$

    $(A \cup B)^c = \{ x \mid x \in U,\:x \leq 0 \text{ and } x \geq 4\}$

    \bigskip

    \begin{mdframed}
        \underline{\textbf{Correct Solution:}}

        \bigskip

        $A^c \cap B^c = \{ x \mid x \in U,\:x \leq 0 \color{red}\text{ or }\color{black} x \geq 4 \}$

        $A^c \cup B^c = \{ x \mid x \in U,\:x < 1 \color{red}\text{ or }\color{black} x > 2 \}$

        $(A \cap B)^c = \{ x \mid x \in U,\:x < 1 \color{red}\text{ or }\color{black} x > 2\}$

        $(A \cup B)^c = \{ x \mid x \in U,\:x \leq 0 \color{red}\text{ or }\color{black} x \geq 4\}$

        \color{red}It follows from above that $A^c \cap B^c = (A \cup B)^c$ and
        $A^c \cup B^c = (A \cap B)^c$\color{black}

    \end{mdframed}

\end{enumerate}

\section*{Question 2}
\begin{enumerate}[a.]
    \item

    $T_0 = \{3,6,9,\dots\}$

    $T_1 = \{1,4,7,\dots\}$

    $T_2 = \{2,5,8,\dots\}$

    $T_3 = \{6,12,18,\dots\}$

    \item

    A partition of $\mathbb{Z}$ is $\{T_0,T_1,T2\}$.

    \bigskip

    All four sets can't be used because elements in $T_3$ overlaps with $T_0$.
    A partition cannot have any elements in common.

    \bigskip

    \textbf{Notes:}

    \begin{itemize}
        \item \textbf{Definition of Partition:} Let $A$ be a set. A (finite or
        infinite) collection of nonempty sets $\{A_1,A_2,A_3\}$ is called a
        \textbf{partition} of $A$ when (1) $A$ is the union of all of the $A_i$,
        and (2) the sets $A_1,A_2,A_3,\dots$ do not have any element in common.

    \end{itemize}

\end{enumerate}


\section*{Question 3}
\begin{enumerate}[a.]
    \item

    All strings over the alphabet $\{0,1\}$ of length three are

    \begin{align*}
        000,100,010,001,110,101,011,111
    \end{align*}

    \item

    $S_1 = \{aa,ab,ac,ba,bb,bc,ca,cb,cc\}$

    $S_2 = \{a,b,c,aa,bb,cc,\dots\}$

    $S_1 \cap S_2 = \{aa,bb,cc\}$

    $S_1 \setminus S_2 = \{ab,ac,ba,bc,ca,cb\}$

    \item

    $S_1 = (S_1 \cap S_2) \cup (S_1 \setminus S_2)$

\end{enumerate}

\section*{Question 4}
\begin{enumerate}[a.]

    \item

    \begin{tabular}{|c|c|c|}
        \hline
         & $\lfloor x \rfloor$ & $\lceil x \rceil$\\
        \hline
        $\frac{25}{4}$ & 6 & 7\\
        \hline
        0.99 & 0 & 1\\
        \hline
        -2.01 & -3.0 & -2.0\\
        \hline
    \end{tabular}

    \bigskip

    \textbf{Notes:}

    \begin{itemize}
        \item floor of a negative number: ceiling but with negative sign
        \item ceiling of a negative number: floor but with negative sign
    \end{itemize}

    \item

    \textbf{Domain of the floor \& ceiling function:} $\mathbb{R}$

    \textbf{Codomain of the floor \& ceiling function:} $\mathbb{N}$

    \item

    The statement is false. Consider example $x = -0.5$ and $y = 0.5$.

    \bigskip

    Then, $\lfloor x + y \rfloor = 0$ and $\lfloor x \rfloor + \lfloor y \rfloor = -1 + 0 = -1$.

\end{enumerate}

\section*{Question 5}
\begin{enumerate}[a.]

    \item

    $\sum\limits_{k=1}^3 (k+1) = (1 + 1) + (2 + 1) + (3 + 1)$

    $\sum\limits_{m=0}^1 \frac{1}{2^m} = \frac{1}{2^0}$

    $\sum\limits_{k=-1}^2 (k^2 + 3) = ((-1)^2 + 3) + (0^2 + 3) + (1^2 + 3) + (2^2 + 3)$

    $\sum\limits_{j=0}^4 (-1)^j \frac{j}{j + 1} = (-1)^0 \cdot \frac{0}{0 + 1} +
    (-1) \cdot \frac{1}{1 + 1} + (-1)^2 \cdot \frac{2}{2 + 1} + (-1)^3 \cdot \frac{3}{3 + 1} +
    (-1)^4 \cdot \frac{4}{4 + 1}$

    $\sum\limits_{k=1}^5 (2 \cdot k) = (2 \cdot 1) + (2 \cdot 2) + (2 \cdot 3) +
    (2 \cdot 4) + (2 \cdot 5)$

    $\prod\limits_{i=2}^4 \frac{i(i+2)}{(i-1)(i+1)} = \left( \frac{0 \cdot (0+2)}{(0-1)(0+1)} \right)
    \left( \frac{1 \cdot (1+2)}{(1-1)(1+1)} \right) \left( \frac{2 \cdot (2+2)}{(2-1)(2+1)} \right)
    \left( \frac{3 \cdot (3+2)}{(3-1)(3+1)} \right) \left( \frac{4 \cdot (4+2)}{(4-1)(4+1)} \right)$

\end{enumerate}


\end{document}