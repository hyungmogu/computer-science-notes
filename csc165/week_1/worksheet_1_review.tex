\documentclass[12pt]{article}
\usepackage[margin=2.5cm]{geometry}
\usepackage{enumerate}
\usepackage{amsfonts}
\usepackage{amsmath}
\usepackage{fancyhdr}
\usepackage{amsmath}
\usepackage{amssymb}
\usepackage{amsthm}
\usepackage{mdframed}
\usepackage{graphicx}
\usepackage{subcaption}
\usepackage{listings}
\usepackage{xcolor}


\begin{document}
\title{Worksheet 1 Review}
\author{Hyungmo Gu}
\maketitle

\section*{Question 1}
\begin{enumerate}[a.]
    \item $A^c = \{1,3,4,6\}$
    \item $A = U \setminus A$
    \item

    $A^c \cap B^c = \{ x \mid x \in U,\:x \leq 0 \text{ and } x \geq 4 \}$

    $A^c \cup B^c = \{ x \mid x \in U,\:x < 1 \text{ and } x > 2 \}$

    $(A \cap B)^c = \{ x \mid x \in U,\:x < 1 \text{ and } x > 2\}$

    $(A \cup B)^c = \{ x \mid x \in U,\:x \leq 0 \text{ and } x \geq 4\}$

    \bigskip

    \begin{mdframed}
        \underline{\textbf{Correct Solution:}}

        \bigskip

        $A^c \cap B^c = \{ x \mid x \in U,\:x \leq 0 \color{red}\text{ or }\color{black} x \geq 4 \}$

        $A^c \cup B^c = \{ x \mid x \in U,\:x < 1 \color{red}\text{ or }\color{black} x > 2 \}$

        $(A \cap B)^c = \{ x \mid x \in U,\:x < 1 \color{red}\text{ or }\color{black} x > 2\}$

        $(A \cup B)^c = \{ x \mid x \in U,\:x \leq 0 \color{red}\text{ or }\color{black} x \geq 4\}$

        \color{red}It follows from above that $A^c \cap B^c = (A \cup B)^c$ and
        $A^c \cup B^c = (A \cap B)^c$\color{black}

    \end{mdframed}

\end{enumerate}

\section*{Question 2}
\begin{enumerate}[a.]
    \item

    $T_0 = \{3,6,9,\dots\}$

    $T_1 = \{1,4,7,\dots\}$

    $T_2 = \{2,5,8,\dots\}$

    $T_3 = \{6,12,18,\dots\}$

    \item

    A partition of $\mathbb{Z}$ is $\{T_0,T_1,T2\}$.

    \bigskip

    All four sets can't be used because elements in $T_3$ overlaps with $T_0$.
    A partition cannot have any elements in common.

    \bigskip

    \textbf{Notes:}

    \begin{itemize}
        \item \textbf{Definition of Partition:} Let $A$ be a set. A (finite or
        infinite) collection of nonempty sets $\{A_1,A_2,A_3\}$ is called a
        \textbf{partition} of $A$ when (1) $A$ is the union of all of the $A_i$,
        and (2) the sets $A_1,A_2,A_3,\dots$ do not have any element in common.

    \end{itemize}

\end{enumerate}


\section*{Question 3}

\section*{Question 4}

\end{document}