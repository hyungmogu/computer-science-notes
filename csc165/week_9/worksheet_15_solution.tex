\documentclass[12pt]{article}
\usepackage{enumerate}
\usepackage{amsfonts}
\usepackage{amsmath}
\usepackage{fancyhdr}
\usepackage{amssymb}
\usepackage{listings}
\usepackage{mdframed}

\begin{document}
\title{Worksheet 15 Solution}
\maketitle

\section*{Question 1}
\begin{enumerate}[a.]
    \item

    \textbf{Inner Loop Iterations (upper bound):} $n$

    \textbf{Inner Loop Step Size:} 1

    \textbf{Inner Loop Steps Total:} $n$

    \bigskip

    \textbf{Outer Loop Iterations (upper bound):} $n$

    \textbf{Outer Loop Step Size:} 1

    \textbf{Outer Loop Steps Total:} $n$

    \bigskip

    \textbf{Steps Total:} $n \cdot n = n^2$

    \begin{mdframed}

    \bigskip

    \underline{\textbf{Correct Solution:}}

    Since the inner loop starts at $i+1$ and ends at $n-1$, where $i$ represents
    the variable in outer loop, the inner loop has $(n-1 - (i+1) + 1) = n - i - 1$
    iterations.

    \bigskip

    Since each iteration takes 1 step, the total steps taken by inner loop is:

    \begin{align}
        (n - i - 1) \cdot 1 &= (n - i - 1)
    \end{align}

    \bigskip

    Now, we will evaluate total steps taken by outer loop.

    \bigskip

    Since the outer loop starts at $i = 0$, and ends at $n-1$, the loop runs
    at most $n$ iterations.

    \bigskip

    Since each iteration takes $(n -i - 1)$ steps, the total steps of outer
    loop is:

    \begin{align}
        \sum\limits_{i=0}^{n-1} (n - i + 1) &= \sum\limits_{i=0}^{n-1} \left[ (n-1) - i \right]\\
        &= \sum\limits_{i=0}^{n-1} (n-1) - \sum\limits_{i=0}^{n-1} i\\
        &= n(n-1) - \frac{n(n-1)}{2}\\
        &= \frac{n^2-n}{2}
    \end{align}

    \bigskip

    Then, since the last \textbf{return} statement takes 1 step, it follows
    that the total number of steps of this algorithm is at most
    $\frac{n^2-n}{2} + 2$, or $\mathcal{O}(n^2)$.

    \end{mdframed}

    \item

    Consider the input family where none of the values in a list are the same
    (i.e. $[1,2,3,4,5,6,7,8,9]$).

    \bigskip

    Since all values in the input list are not matching, both the inner and
    the outer loop will run, giving the loops the total number of steps
    of $\frac{n^2-n}{2}$.

    \bigskip

    Since the last \textbf{return} statement takes 1 step, the total number of
    steps of this algorithm is $\frac{n^2-n}{2} + 1$, or $\Omega(n^2)$.

    \begin{mdframed}

    \bigskip

    \underline{\textbf{Correct Solution:}}

    \bigskip

    Let $n \in \mathbb{N}$ and $lst = [1,2,3,\dots,n-1,n-1]$.

    \bigskip

    Since the inner loop will run without interruptions until the end, the
    inner loop has

    \setcounter{equation}{0}
    \begin{align}
        n-1-(i+1) + 1 &= n-i-1
    \end{align}

    iterations.

    \bigskip

    Then, since the inner loop takes 1 step per iteration, the total steps
    taken by the inner loop is

    \begin{align}
        (n-i-1) \cdot 1 &= (n-i-1)
    \end{align}

    \bigskip

    Since the \textbf{if condition} $lst[i] == lst[j]$ and the \textbf{return}
    statement are activated when $i=n-2$, the outer loop will run until $i=n-2$,
    where $j$ is the variable of the inner loop and $i$ is the variable of
    the outer loop.

    \bigskip

    Since the outer loop starts at 0 and ends at $n-2$, it has

    \begin{align}
        n-2+1 &= n-1
    \end{align}

    iterations.

    \bigskip

    Since each iteration in the outer loop takes $(n-i-1)$ steps, the outer
    loop has total cost of

    \begin{align}
        \sum\limits_{i=0}^{n-2} (n-i-1) &= \sum\limits_{i=0}^{n-2} (n-1) + \sum\limits_{i=0}^{n-2} i\\
        &= (n-1)(n-1) - \frac{(n-2)(n-1)}{2}\\
        &= \frac{(n-1)n}{2}
    \end{align}

    Since each of the \textbf{if condition} and \textbf{return} statement has cost of 1,
    the total cost of algorithm is $\frac{n(n-1)}{2} + 2$, or $\Omega(n^2)$

    \end{mdframed}

    \item

    Let $n \in \mathbb{N}$, and $lst = [1,2,3,\dots,n-1,1]$

    \bigskip

    Since the inner loop will run from $j = i + 1$ until the end without interruptions,
    the loop has

    \setcounter{equation}{0}
    \begin{align}
        (n-1)-(i+1)+1 = n-i-1
    \end{align}

    iterations.

    \bigskip

    Since the inner loop takes 1 step per iteration, the loop takes total of

    \begin{align}
        (n-i-1) \cdot 1 = (n-i-1)
    \end{align}

    steps.

    \bigskip

    Now, because we know that the \textbf{if condition} and \textbf{return}
    statement will occur at $i = 0$, the outer loop has at most 1 iteration.

    \bigskip

    Because we know that the outer loop terminates at $i=0$, the total cost of
    inner loop can be simplified to

    \begin{align}
        (n-i-1) &= n-1
    \end{align}

    \bigskip

    Since the outer loop has 1 iteration and takes $n-1$ steps, the loop has
    total cost of $n-1$.

    \bigskip

    Lastly, since each of the \textbf{if condition} and \textbf{return} statement
    has cost of 1, the total cost of the algorithm is

    \begin{align}
        n-1+2 &= n+1
    \end{align}

    steps, or $\Theta(n)$.

    \bigskip

    \textbf{Note}

    \begin{itemize}
        \item
        What's the lower/upper bound of this input family? How can I find them?

        \item $[1,2,3,...,1,n-1]$ returns total cost of algorithm of $n$. Does it
        imply $[1,2,3,...,1,n-1]$ is in different input family than $[1,2,3,...,n-1,1]$?
        \item

        $g \in \mathcal{O}(f):\:\exists c,n_o \in \mathbb{R}^{+},\:\forall n \in
        \mathbb{N},\:n \geq n_0 \Rightarrow g(n) \leq cf(n)$, where $f,g:\mathbb{N} \to \mathbb{R}^{\geq 0}$

        \item

        $g \in \Omega(f):\:\exists c,n_o \in \mathbb{R}^{+},\:\forall n \in
        \mathbb{N},\:n \geq n_0 \Rightarrow g(n) \geq cf(n)$, where $f,g:\mathbb{N} \to \mathbb{R}^{\geq 0}$

        \item

        $g \in \Theta(f):\: g \in \mathcal{O}(f) \land g \in \Omega(f)$
    \end{itemize}

\end{enumerate}

\section*{Question 2}
\begin{enumerate}[a.]
    \item

    Since $j=len(lst)=n$ and $i=0$ initially, the initial value of $r$ is

    \setcounter{equation}{0}
    \begin{align}
        r &= j - i\\
        &= n - 0\\
        &= n
    \end{align}

    \item

    The loop terminates when $r \leq 0$.

    \item

    Let $k \in \mathbb{N}$, and $j,i \in \mathbb{Z}$. Assume $j > i$.

    \bigskip

    We will prove the statement by separating into two cases, and combining them
    at the end.

    \bigskip

    \textbf{Case1} ($lst[mid] < x$):

    \bigskip

    Assume $lst[mid] < x$.

    \bigskip

    Then, it follows from the fact $i = \lfloor \frac{i+j}{2} \rfloor + 1$ and
    $j = j$ that the value of $r$ at $k+1^{th}$ step is

    \setcounter{equation}{0}
    \begin{align}
        r_{k+1} &= j - \left( \left\lfloor \frac{i + j}{2} \right\rfloor + 1 \right)
    \end{align}

    \bigskip

    Then,

    \begin{align}
        r_{k+1} &\leq j - \left( \left\lfloor \frac{i + j}{2} \right\rfloor \right)\\
        &\leq \frac{2j}{2} - \left( \left\lfloor \frac{i + j}{2} \right\rfloor \right)\\
        &\leq - \left( \left\lfloor \frac{i + j}{2} \right\rfloor + \frac{(-2j)}{2} \right)\\
        &\leq - \left( \left\lfloor \frac{i + j}{2} \right\rfloor + \frac{(-2j)}{2} \right)\\
        &\leq - \left( \left\lfloor \frac{i + j}{2} + \frac{(-2j)}{2} \right\rfloor \right)
    \end{align}

    by using the fact $\forall x \in \mathbb{Z},\:\forall y \in \mathbb{R},\:
    \lfloor x+y \rfloor = x + \lfloor y \rfloor$.

    \bigskip

    Then,

    \begin{align}
        - \left( \left\lfloor \frac{i + j}{2} + \frac{(-2j)}{2} \right\rfloor \right) &\leq - \left( \left\lfloor \frac{i - j}{2} \right\rfloor \right)\\
        &\leq - \left( \frac{i - j}{2} \right)\\
        &\leq \left( \frac{j - i}{2} \right)\\
        &\leq \frac{1}{2} r_k
    \end{align}

    \bigskip

    \textbf{Case2} ($lst[mid]> x$):

    \bigskip

    Assume $lst[mid] \geq x$.

    \bigskip

    Then, it follows from the fact $i = 1$ and $j = \left\lfloor \frac{i+j}{2} \right\rfloor$
    that the value of $r$ at $k+1^{th}$ step is

    \begin{align}
        r_{k+1} &= \left\lfloor \frac{i+j}{2} \right\rfloor - i\\
    \end{align}

    \bigskip

    Then,

    \begin{align}
        \left\lfloor \frac{i+j}{2} \right\rfloor - i &\leq \left( \frac{i+j}{2} \right) - i
    \end{align}

    by the fact $\forall x \in \mathbb{R}, \lfloor x \rfloor \leq x < 1 + \lfloor x \rfloor$.

    \bigskip

    Then,

    \begin{align}
        \left( \frac{i+j}{2} \right) &\leq \frac{i + j}{2} - \frac{2i}{2}\\
        &\leq \frac{j - i}{2}\\
        &\leq \frac{1}{2} r_k
    \end{align}


    Then, it follows from proof by cases that the statement $r_{k+1} \leq \frac{1}{2} r_k$
    is true.

    \bigskip

    \textbf{Notes:}

    \begin{itemize}
        \item External properties of ceiling and floor

        \begin{enumerate}[1.]
            \item $\forall x \in \mathbb{R},\: 0 \leq x - \lfloor x \rfloor < 1$
            \item $\forall x \in \mathbb{R}^{\geq 0},\:x \geq 4 \Rightarrow (\lfloor x \rfloor)^2 \geq \frac{1}{2}x^2$
            \item $\forall x \in \mathbb{R}^{\geq 0},\:x \geq 4 \Rightarrow \frac{1}{2}x^2 \geq 2x$
            \item $\forall x \in \mathbb{Z},\:\forall y \in \mathbb{R},\:\lfloor x+y \rfloor = x + \lfloor y \rfloor$
        \end{enumerate}
    \end{itemize}

    \item

    Let $n \in \mathbb{N}$, $k \in \mathbb{Z}^{\geq 0}$ and $r = j - i$.
    Assume $j > i$.

    Because we know $r_{k+1} \geq \frac{1}{2}r_k$, we can conclude that the
    size of $r$ at $k+1^{th}$ step is

    \setcounter{equation}{0}
    \begin{align}
        \left\lfloor \frac{n}{2^{k+1}} \right\rfloor
    \end{align}

    \bigskip

    Then, because we know the loop terminates when $0 < \frac{n}{2^k} < 1$, we can
    also conclude that loop terminates when

    \begin{align}
        2^{k+1} &> n\\
        k + 1 &> \log n\\
        k + 1 &> \lfloor \log n \rfloor
    \end{align}

    \bigskip

    Since $\lfloor \log n \rfloor$ is the lower bound of loop termination,
    $\lfloor \log n \rfloor + 1$ is the value of iteration that results in termination.

    \bigskip

    Since each iteration in loop takes 1 step, the loop has total cost of

    \begin{align}
        (\lfloor \log n \rfloor + 1) \cdot 1 &= \lfloor \log n \rfloor + 1
    \end{align}

    steps.

    \bigskip

    Since the \textbf{return} statement occurs at the end and since it has cost of 1,
    the total cost of algorithm is at most $\lfloor \log n \rfloor + 2$, or $\mathcal{O}(\log n)$.

    \textbf{Note}

    \begin{itemize}
        \item A through understanding of algorithm is required. Maybe try few examples before proof?
        \item iteration is in terms of k, but we want in terms of n.
    \end{itemize}


    \item

    Let $n \in \mathbb{N}, k \in \mathbb{Z}^{\geq 0}$, $lst=[1,2,3,4,5,...,n-1]$,
    and $x = 0$.

    \bigskip

    Because we know $r_{k+1} \geq \frac{1}{2}r_k$, we can conclude that the value
    of $r$ at $k+1^{th}$ step is

    \setcounter{equation}{0}
    \begin{align}
        \left\lfloor \frac{n}{2^k} \right\rfloor
    \end{align}

    \bigskip

    Because we know the loop terminates when $j \leq i$, or when $r \leq 0$, we
    can also conclude that, in terms of equation 1, the termination occurs when

    \begin{align}
        2^k &> n\\
        k &> \log n
    \end{align}

    \bigskip

    Since $\log n$ represents the lower bound of loop termination, the loop
    terminates when $k = \log n + 1$.

    \bigskip

    Since each loop iteration takes 1 step, the loop has total cost of

    \begin{align}
        (\log n + 1) \cdot 1 = \log n + 1
    \end{align}

    steps.

    \bigskip

    Since \textbf{return} statement occurs at the end and since it has cost of 1,
    the total cost of algorithm is $\log n + 2$, or $\Omega(\log n)$.

    \bigskip

    Since $\log n + 2$ is the same for both the upper bound and the lower bound,
    $\Theta(\log n)$ is true.


\end{enumerate}

\end{document}