\documentclass[12pt]{article}
\usepackage{enumerate}
\usepackage{amsfonts}
\usepackage{amsmath}
\usepackage{fancyhdr}
\usepackage{amssymb}
\usepackage{listings}
\usepackage{tcolorbox}

\begin{document}
\title{Worksheet 14 Solution}
\maketitle

\section*{Question 1}
\begin{enumerate}[a.]
    \item

    \textbf{Inner Loop Iterations (upper bound):} $n$

    \textbf{Inner Loop Step Size:} 1

    \textbf{Inner Loop Steps Total:} $n$

    \bigskip

    \textbf{Outer Loop Iterations (upper bound):} $n$

    \textbf{Outer Loop Step Size:} 1

    \textbf{Outer Loop Steps Total:} $n$

    \bigskip

    \textbf{Steps Total:} $n \cdot n = n^2$

    \bigskip

    \begin{tcolorbox}
        \underline{\textbf{Correct Solution:}}

        Since the inner loop starts at $i+1$ and ends at $n-1$, where $i$ represents
        the variable in outer loop, the inner loop has $(n-1 - (i+1) + 1) = n - i - 1$
        iterations.

        \bigskip

        Since each iteration takes 1 step, the total steps taken by inner loop is:

        \begin{align}
            (n - i - 1) \cdot 1 &= (n - i - 1)
        \end{align}

        \bigskip

        Now, we will evaluate total steps taken by outer loop.

        \bigskip

        Since the outer loop starts at $i = 0$, and ends at $n-1$, the loop runs
        at most $n$ iterations.

        \bigskip

        Since each iteration takes $(n -i - 1)$ steps, the total steps of outer
        loop is:

        \begin{align}
            \sum\limits_{i=0}^{n-1} (n - i + 1) &= \sum\limits_{i=0}^{n-1} \left[ (n-1) - i \right]\\
            &= \sum\limits_{i=0}^{n-1} (n-1) - \sum\limits_{i=0}^{n-1} i\\
            &= n(n-1) - \frac{n(n-1)}{2}\\
            &= \frac{n^2-n}{2}
        \end{align}

        \bigskip

        Then, since the last \textbf{return} statement takes 1 step, it follows
        from that the total number of steps of this algorithm is at most
        $\frac{n^2-n}{2} + 1$, or $\mathcal{O}(n^2)$.
    \end{tcolorbox}

    \item

    Consider the input family where none of the values in a list are the same
    (i.e. $[1,2,3,4,5,6,7,8,9]$).

    \bigskip

    Since all values in the input list are not matching, both the inner and
    the outer loop will run, giving the loops the total number of steps
    of $\frac{n^2-n}{2}$.

    \bigskip

    Since the last \textbf{return} statement takes 1 step, the total number of
    steps of this algorithm is $\frac{n^2-n}{2} + 1$, or $\Omega(n^2)$.

\end{enumerate}

\section*{Question 2}


\end{document}