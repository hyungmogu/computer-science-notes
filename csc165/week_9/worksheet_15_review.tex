\documentclass[12pt]{article}
\usepackage[margin=2.5cm]{geometry}
\usepackage{enumerate}
\usepackage{amsfonts}
\usepackage{amsmath}
\usepackage{fancyhdr}
\usepackage{amssymb}
\usepackage{listings}
\usepackage{listings}
\usepackage{xcolor}
\usepackage{mdframed}

\definecolor{codegreen}{rgb}{0,0.6,0}
\definecolor{codegray}{rgb}{0.5,0.5,0.5}
\definecolor{codepurple}{rgb}{0.58,0,0.82}
\definecolor{backcolour}{rgb}{0.95,0.95,0.92}

\lstdefinestyle{mystyle}{
    backgroundcolor=\color{backcolour},
    commentstyle=\color{codegreen},
    keywordstyle=\color{magenta},
    numberstyle=\tiny\color{codegray},
    stringstyle=\color{codepurple},
    basicstyle=\ttfamily\footnotesize,
    breakatwhitespace=false,
    breaklines=true,
    captionpos=b,
    keepspaces=true,
    numbers=left,
    numbersep=5pt,
    showspaces=false,
    showstringspaces=false,
    showtabs=false,
    tabsize=2
}

\begin{document}
\title{Worksheet 15 Review}
\maketitle

\section*{Question 1}
\begin{enumerate}[a.]
    \item

    First, we will evaluate the cost of he inner most loop.

    \bigskip

    Because the loop runs from $j = i = 1$ to $j = n - 1$, with each iteration costing
    1 step, we can conclude that the inner most loop has cost of at most

    \begin{align}
        \lceil (n-1) - (i + 1) + 1 \rceil &= n - i - 1
    \end{align}

    steps.

    \bigskip

    Next, we will evaluate the cost of the outer most loop.

    \bigskip

    Because the loop runs from $i = 0$ to $i = n - 1$ with each iteration
    costing $(n - i - 1)$ steps, we can conclude the outer most loop has cost of
    at most

    \begin{align}
        \sum\limits_{i=0}^{n-1} (n - i - 1) &= \left[ \sum\limits_{i=0}^{n-1} (n-1) - \sum\limits_{i=0}^{n-1} i \right]\\
        &= \left[ \frac{2n(n-1)}{2} - \frac(n(n-1)){2} \right]\\
        &= \frac{n(n-1)}{2}
    \end{align}

    steps.

    \bigskip

    Next, we will bring everything together.

    \bigskip

    Since the lines \textbf{n = len(lst)} and \textbf{return\:False} have cost of 1 step each,
    the total cost of the algorithm is

    \begin{align}
        \frac{n(n-1)}{2} + 2
    \end{align}

    steps.

    \bigskip

    Then, it follows from above that the algorithm has runtime of $\Theta(n^2)$.

    \begin{mdframed}
        \underline{\textbf{Correct Solution:}}

        \bigskip

        First, we will evaluate the cost of he inner most loop.

        \bigskip

        Because the loop runs from $j = i = 1$ to $j = n - 1$, with each iteration costing
        1 step, we can conclude that the inner most loop has cost of at most

        \begin{align}
            \lceil (n-1) - (i + 1) + 1 \rceil &= n - i - 1
        \end{align}

        steps.

        \bigskip

        Next, we will evaluate the cost of the outer most loop.

        \bigskip

        Because the loop runs from $i = 0$ to $i = n - 1$ with each iteration
        costing $(n - i - 1)$ steps, we can conclude the outer most loop has cost of
        at most

        \begin{align}
            \sum\limits_{i=0}^{n-1} (n - i - 1) &= \left[ \sum\limits_{i=0}^{n-1} (n-1) - \sum\limits_{i=0}^{n-1} i \right]\\
            &= \left[ \frac{2n(n-1)}{2} - \frac(n(n-1)){2} \right]\\
            &= \frac{n(n-1)}{2}
        \end{align}

        steps.

        \bigskip

        Next, we will bring everything together.

        \bigskip

        Since the lines \textbf{n = len(lst)} and \textbf{return\:False} have cost of 1 step each,
        the total cost of the algorithm is \color{red} at most \color{black}

        \begin{align}
            \frac{n(n-1)}{2} + 2
        \end{align}

        steps.

        \bigskip

        Then, it follows from above that the algorithm has runtime of \color{red}$\mathcal{O}(n^2)$\color{black}.

    \end{mdframed}

\end{enumerate}

\section*{Question 2}

\end{document}