\documentclass[12pt]{article}
\usepackage[margin=2.5cm]{geometry}
\usepackage{enumerate}
\usepackage{amsfonts}
\usepackage{amsmath}
\usepackage{fancyhdr}
\usepackage{amssymb}
\usepackage{listings}
\usepackage{listings}
\usepackage{xcolor}
\usepackage{mdframed}

\definecolor{codegreen}{rgb}{0,0.6,0}
\definecolor{codegray}{rgb}{0.5,0.5,0.5}
\definecolor{codepurple}{rgb}{0.58,0,0.82}
\definecolor{backcolour}{rgb}{0.95,0.95,0.92}

\lstdefinestyle{mystyle}{
    backgroundcolor=\color{backcolour},
    commentstyle=\color{codegreen},
    keywordstyle=\color{magenta},
    numberstyle=\tiny\color{codegray},
    stringstyle=\color{codepurple},
    basicstyle=\ttfamily\footnotesize,
    breakatwhitespace=false,
    breaklines=true,
    captionpos=b,
    keepspaces=true,
    numbers=left,
    numbersep=5pt,
    showspaces=false,
    showstringspaces=false,
    showtabs=false,
    tabsize=2
}

\begin{document}
\title{Worksheet 15 Review}
\maketitle

\section*{Question 1}
\begin{enumerate}[a.]
    \item

    First, we will evaluate the cost of he inner most loop.

    \bigskip

    Because the loop runs from $j = i = 1$ to $j = n - 1$, with each iteration costing
    1 step, we can conclude that the inner most loop has cost of at most

    \begin{align}
        \lceil (n-1) - (i + 1) + 1 \rceil &= n - i - 1
    \end{align}

    steps.

    \bigskip

    Next, we will evaluate the cost of the outer most loop.

    \bigskip

    Because the loop runs from $i = 0$ to $i = n - 1$ with each iteration
    costing $(n - i - 1)$ steps, we can conclude the outer most loop has cost of
    at most

    \begin{align}
        \sum\limits_{i=0}^{n-1} (n - i - 1) &= \left[ \sum\limits_{i=0}^{n-1} (n-1) - \sum\limits_{i=0}^{n-1} i \right]\\
        &= \left[ \frac{2n(n-1)}{2} - \frac(n(n-1)){2} \right]\\
        &= \frac{n(n-1)}{2}
    \end{align}

    steps.

    \bigskip

    Next, we will bring everything together.

    \bigskip

    Since the lines \textbf{n = len(lst)} and \textbf{return\:False} have cost of 1 step each,
    the total cost of the algorithm is

    \begin{align}
        \frac{n(n-1)}{2} + 2
    \end{align}

    steps.

    \bigskip

    Then, it follows from above that the algorithm has runtime of $\Theta(n^2)$.

    \begin{mdframed}
        \underline{\textbf{Correct Solution:}}

        \bigskip

        First, we will evaluate the cost of he inner most loop.

        \bigskip

        Because the loop runs from $j = i = 1$ to $j = n - 1$, with each iteration costing
        1 step, we can conclude that the inner most loop has cost of at most

        \begin{align}
            \lceil (n-1) - (i + 1) + 1 \rceil &= n - i - 1
        \end{align}

        steps.

        \bigskip

        Next, we will evaluate the cost of the outer most loop.

        \bigskip

        Because the loop runs from $i = 0$ to $i = n - 1$ with each iteration
        costing $(n - i - 1)$ steps, we can conclude the outer most loop has cost of
        at most

        \begin{align}
            \sum\limits_{i=0}^{n-1} (n - i - 1) &= \left[ \sum\limits_{i=0}^{n-1} (n-1) - \sum\limits_{i=0}^{n-1} i \right]\\
            &= \left[ \frac{2n(n-1)}{2} - \frac(n(n-1)){2} \right]\\
            &= \frac{n(n-1)}{2}
        \end{align}

        steps.

        \bigskip

        Next, we will bring everything together.

        \bigskip

        Since the lines \textbf{n = len(lst)} and \textbf{return\:False} have cost of 1 step each,
        the total cost of the algorithm is \color{red} at most \color{black}

        \begin{align}
            \frac{n(n-1)}{2} + 2
        \end{align}

        steps.

        \bigskip

        Then, it follows from above that the algorithm has runtime of \color{red}$\mathcal{O}(n^2)$\color{black}.

    \end{mdframed}

    \bigskip

    \textbf{Notes:}

    \begin{itemize}
        \item Noticed that in here, professor considers the cost of loop variables and
        other lines with constant time.

        \item $\mathcal{O}$ used since we are determining the upper bound.

        \item In worksheet 14, the cost of loop variables is not required.

    \end{itemize}

    \item

    Let $n \in \mathbb{N}$, and $lst = [0,1,2,3,\dots,n-3,n-1,n-1]$.

    \bigskip

    First, we will calculate the cost of the inner most loop.

    \bigskip

    Because we know the inner most loop will terminate when \textbf{if lst[i] == lst[j]} and because we know
    this condition occurs when $i = n-2$, we can conclude the loop will start
    at $i = 0$ and run until $i = n - 2$.

    \bigskip

    Because we know the condition of the inner most loop \textbf{for j in range(i+1,n)}
    stays true until $i = n - 2$ even at the worst case, we can conclude that the cost of
    inner loop is the same as the cost of the inner loop at worst case, that is

    \setcounter{equation}{0}
    \begin{align}
        n - i - 1
    \end{align}

    \bigskip

    Next, we will evaluate the cost of the outer most loop.

    \bigskip

    Since the outer most loop starts at $i = 0$ and ends at $i = n-1$ with each iteration
    costing $(n - i -1)$ steps, the outer most loop has cost of

    \begin{align}
        \sum\limits_{i=0}^{n-1} (n-i-1) &= \frac{n(n-1)}{2}
    \end{align}

    steps.

    \bigskip

    Next, we will combine everything together.

    \bigskip

    Since each of the lines \textbf{n = len(lst)} and \textbf{return True} have
    cost of 1 step, we can conclude that the algorithm has total cost of

    \begin{align}
        \frac{n(n-1)}{2} + 2
    \end{align}

    steps.

    \bigskip

    Then, we can conclude the algorithm has runtime of $\Omega(n^2)$.

    \bigskip

    Because we know both $\mathcal{O}(n^2)$ and $\Omega(n^2)$ are true, we can
    also conclude the algorithm has runtime of $\Theta(n^2)$.

\end{enumerate}

\section*{Question 2}

\end{document}