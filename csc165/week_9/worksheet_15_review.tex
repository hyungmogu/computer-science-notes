\documentclass[12pt]{article}
\usepackage[margin=2.5cm]{geometry}
\usepackage{enumerate}
\usepackage{amsfonts}
\usepackage{fancyhdr}
\usepackage{amsmath}
\usepackage{amssymb}
\usepackage{amsthm}
\usepackage{listings}
\usepackage{listings}
\usepackage{xcolor}
\usepackage{mdframed}

\definecolor{codegreen}{rgb}{0,0.6,0}
\definecolor{codegray}{rgb}{0.5,0.5,0.5}
\definecolor{codepurple}{rgb}{0.58,0,0.82}
\definecolor{backcolour}{rgb}{0.95,0.95,0.92}

\lstdefinestyle{mystyle}{
    backgroundcolor=\color{backcolour},
    commentstyle=\color{codegreen},
    keywordstyle=\color{magenta},
    numberstyle=\tiny\color{codegray},
    stringstyle=\color{codepurple},
    basicstyle=\ttfamily\footnotesize,
    breakatwhitespace=false,
    breaklines=true,
    captionpos=b,
    keepspaces=true,
    numbers=left,
    numbersep=5pt,
    showspaces=false,
    showstringspaces=false,
    showtabs=false,
    tabsize=2
}

\begin{document}
\title{Worksheet 15 Review}
\maketitle

\section*{Question 1}
\begin{enumerate}[a.]
    \item

    First, we will evaluate the cost of he inner most loop.

    \bigskip

    Because the loop runs from $j = i = 1$ to $j = n - 1$, with each iteration costing
    1 step, we can conclude that the inner most loop has cost of at most

    \begin{align}
        \lceil (n-1) - (i + 1) + 1 \rceil &= n - i - 1
    \end{align}

    steps.

    \bigskip

    Next, we will evaluate the cost of the outer most loop.

    \bigskip

    Because the loop runs from $i = 0$ to $i = n - 1$ with each iteration
    costing $(n - i - 1)$ steps, we can conclude the outer most loop has cost of
    at most

    \begin{align}
        \sum\limits_{i=0}^{n-1} (n - i - 1) &= \left[ \sum\limits_{i=0}^{n-1} (n-1) - \sum\limits_{i=0}^{n-1} i \right]\\
        &= \left[ \frac{2n(n-1)}{2} - \frac(n(n-1)){2} \right]\\
        &= \frac{n(n-1)}{2}
    \end{align}

    steps.

    \bigskip

    Next, we will bring everything together.

    \bigskip

    Since the lines \textbf{n = len(lst)} and \textbf{return\:False} have cost of 1 step each,
    the total cost of the algorithm is

    \begin{align}
        \frac{n(n-1)}{2} + 2
    \end{align}

    steps.

    \bigskip

    Then, it follows from above that the algorithm has runtime of $\Theta(n^2)$.

    \begin{mdframed}
        \underline{\textbf{Correct Solution:}}

        \bigskip

        First, we will evaluate the cost of he inner most loop.

        \bigskip

        Because the loop runs from $j = i = 1$ to $j = n - 1$, with each iteration costing
        1 step, we can conclude that the inner most loop has cost of at most

        \begin{align}
            \lceil (n-1) - (i + 1) + 1 \rceil &= n - i - 1
        \end{align}

        steps.

        \bigskip

        Next, we will evaluate the cost of the outer most loop.

        \bigskip

        Because the loop runs from $i = 0$ to $i = n - 1$ with each iteration
        costing $(n - i - 1)$ steps, we can conclude the outer most loop has cost of
        at most

        \begin{align}
            \sum\limits_{i=0}^{n-1} (n - i - 1) &= \left[ \sum\limits_{i=0}^{n-1} (n-1) - \sum\limits_{i=0}^{n-1} i \right]\\
            &= \left[ \frac{2n(n-1)}{2} - \frac(n(n-1)){2} \right]\\
            &= \frac{n(n-1)}{2}
        \end{align}

        steps.

        \bigskip

        Next, we will bring everything together.

        \bigskip

        Since the lines \textbf{n = len(lst)} and \textbf{return\:False} have cost of 1 step each,
        the total cost of the algorithm is \color{red} at most \color{black}

        \begin{align}
            \frac{n(n-1)}{2} + 2
        \end{align}

        steps.

        \bigskip

        Then, it follows from above that the algorithm has runtime of \color{red}$\mathcal{O}(n^2)$\color{black}.

    \end{mdframed}

    \bigskip

    \textbf{Notes:}

    \begin{itemize}
        \item Noticed that in here, professor considers the cost of loop variables and
        other lines with constant time.

        \item $\mathcal{O}$ used since we are determining the upper bound.

        \item In worksheet 14, the cost of loop variables is not required.

    \end{itemize}

    \item

    Let $n \in \mathbb{N}$, and $lst = [0,1,2,3,\dots,n-3,n-1,n-1]$.

    \bigskip

    First, we will calculate the cost of the inner most loop.

    \bigskip

    Because we know the inner most loop will terminate when \textbf{if lst[i] == lst[j]} and because we know
    this condition occurs when $i = n-2$, we can conclude the loop will start
    at $i = 0$ and run until $i = n - 2$.

    \bigskip

    Because we know the condition of the inner most loop \textbf{for j in range(i+1,n)}
    stays true until $i = n - 2$ even at the worst case, we can conclude that the cost of
    inner loop is the same as the cost of the inner loop at worst case, that is

    \setcounter{equation}{0}
    \begin{align}
        n - i - 1
    \end{align}

    \bigskip

    Next, we will evaluate the cost of the outer most loop.

    \bigskip

    Since the outer most loop starts at $i = 0$ and ends at $i = n-1$ with each iteration
    costing $(n - i -1)$ steps, the outer most loop has cost of

    \begin{align}
        \sum\limits_{i=0}^{n-1} (n-i-1) &= \frac{n(n-1)}{2}
    \end{align}

    steps.

    \bigskip

    Next, we will combine everything together.

    \bigskip

    Since each of the lines \textbf{n = len(lst)} and \textbf{return True} have
    cost of 1 step, we can conclude that the algorithm has total cost of

    \begin{align}
        \frac{n(n-1)}{2} + 2
    \end{align}

    steps.

    \bigskip

    Then, we can conclude the algorithm has runtime of $\Omega(n^2)$.

    \bigskip

    Because we know both $\mathcal{O}(n^2)$ and $\Omega(n^2)$ are true, we can
    also conclude the algorithm has runtime of $\Theta(n^2)$.

    \bigskip

    \textbf{Notes:}

    \begin{itemize}
        \item Does \textbf{if condition} counted towards the total cost?

    \end{itemize}

    \item

    Let $n \in \mathbb{N}$, and $lst = [1,2,3,\dots,n-2,0]$.

    \bigskip

    We will prove that given the input group, the runtime of the algorithm is
    $\Theta(n)$ by computing the total cost of the algorithm, starting off
    with the inner most loop, then the the outer most loop, and then the rest.

    \bigskip

    First, we will calculate the cost of the inner most loop.

    \bigskip

    Because we know the loop starts at $j = i + 1$ and finishes at $j = n-1$ with
    each iteration costing 1 step, we can conclude the inner most loop has cost of

    \setcounter{equation}{0}
    \begin{align}
        n-1-(i+1)+1 &= n-i
    \end{align}

    steps.

    \bigskip

    Next, we will calculate the cost of the outer most loop.

    \bigskip

    Because we know the outer most loop terminates by the \textbf{if condition} at
    $i = 0$, we can conclude the outer most loop runs only 1 iteration.

    \bigskip

    Then, since each iteration costs $(n-i-1)$ steps, the cost of the outer most loop is

    \begin{align}
        1 \cdot (n-0-1) &= n-1
    \end{align}

    steps.

    \bigskip

    Next, we will combine everything together, and evaluate theta.

    \bigskip

    Because we know both of the lines \textbf{n = len(lst)} and \textbf{return True} have cost of 1,
    the total cost of the algorithm is

    \begin{align}
        n - 1 + 2 &= n + 1
    \end{align}

    steps.

    \bigskip

    Then, it follows from above that the runtime of the algorithm is $\Theta(n)$.

\end{enumerate}

\section*{Question 2}
\begin{enumerate}[a.]
    \item
    Because we know the loop starts at $ i = 0$ and $j = n$, we can conclude the
    initial value of $r$, in terms of $n$ is $n$.

    \item
    Since the loop condition is true as long as $i < j$, the loop termination will
    occur when $i \geq j$, or $r \leq 0$.

    \item

    Let $k \in \mathbb{N}$, $x \in \mathbb{R}$, and $r_k = j - i$.

    \bigskip

    We will prove the statement $r_{k+1} \leq \frac{1}{2} r_k$ using proof by cases.

    \bigskip

    \textbf{Case 1} ($lst[mid] < x$):

    \bigskip

    \begin{proof}
        Assume $lst[mid] < x$.

        \bigskip

        Because we know $i_{k+1} = \lfloor \frac{i + j}{2} \rfloor + 1$
        and $j_{k+1} = j$, we can conclude the new value of $r$ or $r_{k+1}$ is

        \setcounter{equation}{0}
        \begin{align}
            r_{k+1} &= j_{k+1}  - i_{k+1}\\
            r_{k+1} &= j - \left( \left\lfloor \frac{i+j}{2} \right\rfloor + 1 \right)\\
            &\leq j - \left\lfloor \frac{(i+j)}{2} \right\rfloor
        \end{align}

        \bigskip

        Then, using a fact about floor/ceil $\forall x \in \mathbb{Z},\:\forall y \in
        \mathbb{R},\:\lfloor x+y \rfloor = x + \lfloor y \rfloor$, we can calculate that

        \begin{align}
            r_{k+1} &\leq - \left( \left\lfloor \frac{i + j}{2} \right\rfloor + (-j) \right)\\
            &\leq - \left( \left\lfloor \frac{i + j}{2} - j \right\rfloor \right)\\
            &\leq - \left( \left\lfloor \frac{i - j}{2} \right\rfloor \right)
        \end{align}

        \bigskip

        Then,

        \begin{align}
            r_{k+1} &\leq - \left( \frac{i - j}{2} \right)\\
            &\leq \left( \frac{j - i}{2} \right)
        \end{align}

        by using the fact $\forall x \in \mathbb{R},\: \lfloor x \rfloor \leq x < 1 + \lfloor x \rfloor$.

        \bigskip

        Because we know $r_k = j - i$, we can conclude

        \begin{align}
            r_{k+1} &\leq \frac{1}{2} r_k
        \end{align}
    \end{proof}

    \bigskip

    \textbf{Case 2} ($lst[mid] \geq x$):

    \bigskip

    \begin{proof}

        Assume $lst[mid] \geq x$.

        \bigskip

        Because we know $j_{k+1} = \lfloor \frac{i + j}{2} \rfloor$
        and $i_{k+1} = i$, we can conclude the new value of $r$ or $r_{k+1}$ is

        \setcounter{equation}{0}
        \begin{align}
            r_{k+1} &= j_{k+1}  - i_{k+1}\\
            r_{k+1} &= \left\lfloor \frac{i+j}{2} \right\rfloor - i
        \end{align}

        \bigskip

        Then, using a fact about floor/ceil $\forall x \in \mathbb{Z},\:\forall y \in
        \mathbb{R},\:\lfloor x+y \rfloor = x + \lfloor y \rfloor$, we can calculate that

        \begin{align}
            r_{k+1} &\leq \left\lfloor \frac{i + j}{2} \right\rfloor + (-i)\\
            &\leq \left\lfloor \frac{i + j}{2} + (-i) \right\rfloor\\
            &\leq \left\lfloor \frac{j - i}{2} \right\rfloor
        \end{align}

        Then,

        \begin{align}
            r_{k+1} &\leq \frac{j - i}{2}
        \end{align}

        by using the fact $\forall x \in \mathbb{R},\: \lfloor x \rfloor \leq x < 1 + \lfloor x \rfloor$.

        \bigskip

        Because we know $r_k = j - i$, we can conclude

        \begin{align}
            r_{k+1} &\leq \frac{1}{2} r_k
        \end{align}
    \end{proof}

    \textbf{Notes:}

    \begin{itemize}
        \item

        How do we introduce the loop variables $i$ and $j$ with $r$?

        \bigskip

        \begin{itemize}
            \item Ah. professor assumes that we know $i$ and $j$. Same with $r_k$
            and $r_{k+1}$. He introduces $i_k$, $j_k$, $i_{k+1}$, $j_{k+1}$ as follows:

            \bigskip

            \begin{mdframed}
                Let $i_k$, $j_k$, $i_{k+1}$, and $j_{k+1}$ As the values of $i$
                and $j$ immediately before and after the $k^{th}$ iteration, respectively.
                So then $r_k = j_k - i_k$ and $r_{k+1} = j_{k+1} - i_{k+1}$.
            \end{mdframed}

        \end{itemize}
        \item $\forall x \in \mathbb{Z},\:\forall y \in \mathbb{R},\:\lfloor x+y \rfloor = x + \lfloor y \rfloor$
        \item Professor wrote $- \lfloor x \rfloor = \lceil -x \rceil$ is true. Is $x \in \mathbb{R}$? Also, how
        does $j$ get embedded to $- \left\lfloor \frac{j + i}{2} \right\rfloor$?

    \end{itemize}



\end{enumerate}

\end{document}