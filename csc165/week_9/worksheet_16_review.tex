\documentclass[12pt]{article}
\usepackage{enumerate}
\usepackage{amsfonts}
\usepackage{amsmath}
\usepackage{fancyhdr}
\usepackage{amssymb}
\usepackage{listings}
\usepackage{mdframed}

\begin{document}
\title{Worksheet 16 Review}
\maketitle

\section*{Question 1}
\begin{enumerate}[a.]
    \item

    Let $k \in \mathbb{N}$.

    \bigskip

    Here, the minimum possible change occurs for the loop variable in a single
    iteration when $i = i + 1$.

    \bigskip

    The maximum possible change occurs for the loop variable in a single
    iteration when $i = i + 6$.

    \bigskip

    The exact upper bound of the variable after k iteration is

    \begin{align}
        i_k &\leq 6k
    \end{align}

    \bigskip

    The exact lower bound of the variable after k iteration is

    \begin{align}
        k &\leq i_k
    \end{align}

    \bigskip

    Using the fact that the termination occurs when $i_k = n$, we can calculate
    that for the upper bound, the loop terminates when

    \begin{align}
        6k &\geq n\\
        k &\geq \frac{n}{6}
    \end{align}

    Because we know $\frac{n}{6}$ may be a decimal, we can conclude the closest value at
    which the loop terminates is when

    \begin{align}
        k &= \left\lceil \frac{n}{6} \right\rceil
    \end{align}

    \bigskip

    Using the same fact, we can calculate that for the lower bound, the loop
    terminates when

    \begin{align}
        k &\geq n
    \end{align}

    It follows from above that for the lower bound, the smallest value of $k$ at which
    the loop termination occurs is when

    \begin{align}
        k &= n
    \end{align}

    \bigskip

    Then, we can conclude the function has asymptotic lower bound of $\Omega(n)$, and
    asymptotic upper bound of $\mathcal{O}(n)$.

    \bigskip

    Then, since both $\Omega$ and $\mathcal{O}$ have the same value, $\Theta(n)$
    is also true.


\end{enumerate}

\section*{Question 2}

\section*{Question 3}

\end{document}