\documentclass[12pt]{article}
\usepackage{enumerate}
\usepackage{amsfonts}
\usepackage{amsmath}
\usepackage{fancyhdr}
\usepackage{amssymb}
\usepackage{listings}
\usepackage{mdframed}

\begin{document}
\title{Worksheet 16 Solution}
\maketitle

\section*{Question 1}
\begin{enumerate}[a.]
    \item

    \textbf{Part 1.a - Finding minimum possible change for a loop in a single iteration}

    The minimum possible change in a loop occurs when $i$ increments by 1.

    \textbf{Part 1.b - Finding maximum possible change for a loop in a single iteration}

    The maximum possible change in a loop occurs when $i$ increments by 6.

    \textbf{Part 2.a - Determine formula for an exact lower bound on the value}

    Since the loop starts at $i=0$ and ends at $n-1$, the loop has

    \begin{align}
        n-1+1 &= n
    \end{align}

    iterations.

    \bigskip

    Since the smallest step increases by 1 per iteration, the total cost of the
    loop at minimum possible change is

    \begin{align}
        (n) \cdot 1 = n
    \end{align}

    steps.


    \textbf{Part 2.a - Determine formula for an exact upper bound on the value}

    Since the loop starts at $i=0$ and ends at $n-1$, the loop has

    \begin{align}
        n-1+1 &= n
    \end{align}

    iterations.

    \textbf{Part 2.b - Determine formula for an exact lower bound on the value}

    \bigskip

    Since the largest step increases by 6 per iteration, the total cost of the
    loop at minimum possible change is

    \begin{align}
        \left\lceil \frac{n}{6} \right\rceil
    \end{align}

    steps.

    \textbf{Part 3.a - Determine formula for an exact upper bound on the value}
    Is it $n$?

    \textbf{Part 3.a - Determine formula for an exact upper bound on the value}
    Is it $\left\lceil \frac{n}{6} \right\rceil$?

    \textbf{Part 4 - Determine Big Oh and Big Omega}

    The big Oh bound of running time is $\mathcal{O}(n)$, and the big theta of
    running time is $\Omega(n)$.

    \bigskip

    Since $n$ in $\mathcal{O}(n)$ and  $\Omega(n)$ are the same, $\Theta(n)$ is
    also true.

    \begin{mdframed}

    \bigskip

    \underline{\textbf{Correct Solution:}}

    \bigskip

    \textbf{Part 1.a - Finding minimum possible change for a loop in a single iteration}

    The minimum possible change in a loop occurs when $i$ increments by 1.

    \textbf{Part 1.b - Finding maximum possible change for a loop in a single iteration}

    The maximum possible change in a loop occurs when $i$ increments by 6.

    \color{red}
    \textbf{Part 2.a - Determine formula for an exact upper bound on the value}

    The upper bound of loop termination is when $k\geq n$


    \textbf{Part 2.b - Determine formula for an exact lower bound on the value}

    The lower bound of loop termination is when $6k \leq n$

    \textbf{Part 3.a - Use the formula to determine the exact number of loops that will occur for upper bound}

    \bigskip

    Since the loop starts from 0 and ends at $n-1$, the loop has total of

    \begin{align}
        n-1 - 0 + 1 &= n
    \end{align}

    iterations.

    \bigskip

    Since 1 step is taken for each iteration, the upper bound total cost of loop
    iteration is

    \begin{align}
        n \cdot 1 &= n
    \end{align}

    \bigskip

    Since the statement on line 2 has cost of 1, the upper bound total cost of
    the algorithm is $n + 1$, or $\mathcal{O}(n)$.

    \bigskip

    \textbf{Part 3.b - Use the formula to determine the exact number of loops that will occur for lower bound}

    \bigskip

    Since the loop starts from 0 and ends at $n-1$, the loop has total of

    \begin{align}
        n-1 - 0 + 1 &= n
    \end{align}

    iterations.

    \bigskip

    Since 6 steps are taken for each iteration, the lower bound total cost of loop
    iteration is

    \begin{align}
        \left\lceil \frac{n}{6} \right\rceil
    \end{align}

    \bigskip

    Since the statement on line 2 has cost of 1, the lower bound total cost of
    the algorithm is $\left\lceil \frac{n}{6} \right\rceil + 1$, or $\Omega(n)$
    \color{black}

    \textbf{Part 4 - Determine Big Oh and Big Omega}

    The big Oh bound of running time is $\mathcal{O}(n)$, and the big theta of
    running time is $\Omega(n)$.

    \bigskip

    Since $n$ in $\mathcal{O}(n)$ and  $\Omega(n)$ are the same, $\Theta(n)$ is
    also true.

    \end{mdframed}

    \item

    \textbf{Part 1.a - Finding minimum possible change for a loop in a single iteration}

    \bigskip

    The minimum possible change for a look in a single iteration is when $i$
    increases by a factor of 2

    \bigskip

    \textbf{Part 1.b - Finding maximum possible change for a loop in a single iteration}

    \bigskip

    The maximum possible change for a look in a single iteration is when $i$
    increases by a factor of 3

    \bigskip

    \textbf{Part 2.a - Determine formula for an exact upper bound of the loop variable after k iterations}

    \bigskip

    The exact upper bound of the loop variable after k iteration is $2^k \geq n$

    \bigskip

    \textbf{Part 2.b - Determine formula for an exact lower bound of the loop variable after k iterations}

    \bigskip

    The exact lower bound of the loop variable after k iteration is $3^k \geq n$

    \bigskip

    \textbf{Part 3.a - Use the formula to determine the exact number of loops that will occur for upper bound}

    \bigskip

    The upper bound of loop iteration is $\lceil \log n \rceil$, or $\mathcal{O}(\log n)$

    \bigskip

    \textbf{Part 3.b - Use the formula to determine the exact number of loops that will occur for lower bound}

    \bigskip

    The lower bound of loop iteration is $\lceil \log_3 n \rceil$, or $\Omega(\log n)$

    \bigskip

    \textbf{Part 4 - Determine Big Oh and Big Omega}

    \bigskip

    For the upper bound, we have $\mathcal{O}(\log n)$.

    \bigskip

    For the lower bound, we have $\Omega(\log n)$

    \bigskip

    Since Big Oh and Big Omega have the same value, $\Theta (\log n)$ is also true.

\end{enumerate}

\section*{Question 2}
\begin{enumerate}[a.]
    \item

    Since \textbf{helper1} has cost of $n$ steps, and \textbf{helper2} has cost of
    $n^2$ steps, the algorithm has total runtime of $n^2 + n$ steps, or $\Theta (n^2)$

    \begin{mdframed}
        \underline{\textbf{Attempt \#2:}}

        \bigskip

        Since \textbf{helper1} has cost of $n$ steps, and \textbf{helper2} has cost of
        $n^2$ steps, the algorithm has total \color{red}\textbf{cost}\color{black}
        \:of $n^2 + n$ steps, or $\Theta (n^2)$

    \end{mdframed}

    \textbf{Notes:}

    \begin{itemize}
     \item Noticed professor uses \textbf{runtime} for $\Theta(n^2)$ or $\Theta(n)$
     and \textbf{cost} for the exact cost of helper functions (i.e. $n^2 + n$)

    \end{itemize}


\end{enumerate}

\section*{Question 3}

\end{document}