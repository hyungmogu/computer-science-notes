\documentclass[12pt]{article}
\usepackage[margin=2.5cm]{geometry}
\usepackage{enumerate}
\usepackage{amsfonts}
\usepackage{amsmath}
\usepackage{fancyhdr}
\usepackage{amsmath}
\usepackage{amssymb}
\usepackage{amsthm}
\usepackage{mdframed}
\usepackage{graphicx}
\usepackage{subcaption}
\usepackage{adjustbox}
\usepackage{listings}
\usepackage{xcolor}
\usepackage{booktabs}
\usepackage[utf]{kotex}
\usepackage{hyperref}

\definecolor{codegreen}{rgb}{0,0.6,0}
\definecolor{codegray}{rgb}{0.5,0.5,0.5}
\definecolor{codepurple}{rgb}{0.58,0,0.82}
\definecolor{backcolour}{rgb}{0.95,0.95,0.92}

\lstdefinestyle{mystyle}{
    backgroundcolor=\color{backcolour},
    commentstyle=\color{codegreen},
    keywordstyle=\color{magenta},
    numberstyle=\tiny\color{codegray},
    stringstyle=\color{codepurple},
    basicstyle=\ttfamily\footnotesize,
    breakatwhitespace=false,
    breaklines=true,
    captionpos=b,
    keepspaces=true,
    numbers=left,
    numbersep=5pt,
    showspaces=false,
    showstringspaces=false,
    showtabs=false,
    tabsize=1
}

\lstset{style=mystyle}

\pagestyle{fancy}
\renewcommand{\headrulewidth}{0.4pt}
\lhead{CSC 373}
\rhead{Worksheet 2}

\begin{document}
\title{CSC373 Worksheet 2}
\maketitle

\bigskip

Source: \href{http://www.cs.toronto.edu/~denisp/csc373/material.html}{link}

\begin{enumerate}[1.]
    \item \textbf{CLRS 16.1-2:} Suppose that instead of always selecting the first activity to finish, we instead select
    the last activity to start that is compatible with all previously selected activities. Describe
    how this approach is a greedy algorithm, and prove that it yields an optimal
    solution.

    \item \textbf{CLRS 16.2-5:} Describe an efficient algorithm that, given a set $\{x_1, x_2, ..., x_n\}$ of points on the
    real line, determines the smallest set of unit-length closed intervals that contains
    all of the given points. Argue that your algorithm is correct.

    \item \textbf{CLRS 16.3-2:} Prove that a binary tree that is not full cannot correspond to an optimal prefix code.

    \item \textbf{CLRS 16.3-3:} What is an optimal Huffman code for the following set of frequencies, based on
    the first 8 Fibonacci numbers?

    \bigskip

    $a:1 b:1 c:2 d:3 e:5 f:8 g:13 h:21$

    \item \textbf{CLRS 16.3-7:} Generalize Huffman’s algorithm to ternary codewords (i.e., codewords using the
    symbols 0, 1, and 2), and prove that it yields optimal ternary codes.
\end{enumerate}

\end{document}