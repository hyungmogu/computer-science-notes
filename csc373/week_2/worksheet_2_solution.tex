\documentclass[12pt]{article}
\usepackage[margin=2.5cm]{geometry}
\usepackage{enumerate}
\usepackage{amsfonts}
\usepackage{amsmath}
\usepackage{fancyhdr}
\usepackage{amsmath}
\usepackage{amssymb}
\usepackage{amsthm}
\usepackage{mdframed}
\usepackage{graphicx}
\usepackage{subcaption}
\usepackage{adjustbox}
\usepackage{listings}
\usepackage{xcolor}
\usepackage{booktabs}
\usepackage[utf]{kotex}
\usepackage{hyperref}

\definecolor{codegreen}{rgb}{0,0.6,0}
\definecolor{codegray}{rgb}{0.5,0.5,0.5}
\definecolor{codepurple}{rgb}{0.58,0,0.82}
\definecolor{backcolour}{rgb}{0.95,0.95,0.92}

\lstdefinestyle{mystyle}{
    backgroundcolor=\color{backcolour},
    commentstyle=\color{codegreen},
    keywordstyle=\color{magenta},
    numberstyle=\tiny\color{codegray},
    stringstyle=\color{codepurple},
    basicstyle=\ttfamily\footnotesize,
    breakatwhitespace=false,
    breaklines=true,
    captionpos=b,
    keepspaces=true,
    numbers=left,
    numbersep=5pt,
    showspaces=false,
    showstringspaces=false,
    showtabs=false,
    tabsize=1
}

\lstset{style=mystyle}

\pagestyle{fancy}
\renewcommand{\headrulewidth}{0.4pt}
\lhead{CSC 373}
\rhead{Worksheet 2 Solution}

\begin{document}
\title{CSC373 Worksheet 2 Solution}
\maketitle

\bigskip

\begin{enumerate}[1.]
    \item

    \bigskip

    \underline{\textbf{Notes:}}

    \bigskip

    \begin{itemize}
        \item Greedy Algorithm

        \begin{itemize}
            \item Always makes the choice that looks best at the moment

            \begin{itemize}
                \item Locally optimal solution leads to globally optimal solution
            \end{itemize}

        \end{itemize}

        \item Activity-selection Problem (Greedy algorithm using dynamic programming)

        \begin{itemize}
            \item Goal: Selecting maximum size set of mutually compatible activities

            \bigskip

            \underline{\textbf{Example:}}

            \bigskip

            \begin{tabular}{c|ccccccccccc}
                $i$ & 1 & 2 & 3 & 4 & 5 & 6 & 7 & 8 & 9 & 10 & 11\\
                \hline
                $s_i$ & 1 & 3 & 0 & 5 & 3 & 5 & 6 & 8 & 8 & 2 & 12\\
                $f_i$ & 4 & 5 & 6 & 7 & 9 & 9 & 10 & 11 & 12 & 14 & 16
            \end{tabular}

            \bigskip

            \item Suppose a set exists $S = \{a_1 = [s_1, f_1), a_2 = [s_2, f_2), ..., a_n = [s_n, f_n)\}$

            \begin{itemize}
                \item $a_i$ represents an $i^{th}$ activity
                \item $s_i$ represents starting time
                \item $f_i$ represents finishing time
                \item $0 \leq s_i < f_i < \infty$
                \item $a_1, ..., a_n$ sorted in monotonically increasing order of finish time

                \bigskip

                \quad i.e.

                \bigskip

                \quad $f_1 \leq f_2 \leq f_3 \leq ... \leq f_{n-1} \leq f_n$

                \bigskip

                \item $a_i$ and $a_j$ are \textbf{compatible}, if intervals $[s_i, f_i)$ and $[s_j, f_j)$
                don't overlap

                \bigskip

                \quad i.e

                \bigskip

                \quad $s_i \geq f_j$ and $s_j \geq f_i$

                \bigskip

            \end{itemize}

            \item Steps

            \begin{enumerate}[1.]
                \item Think about dynamic programming solution

                \begin{itemize}
                    \item \textbf{$S_{ij}$:} activities that start after activity $a_i$ finishes and before
                    activity $a_j$ starts

                    \bigskip

                    i.e.

                    \bigskip

                    $S_{19} = \{a_4 = [5,7), a_6 = [5, 9), a_7 = [6, 10)\}$

                    \bigskip

                    \item \textbf{$A_{ij}$:} maximum set of mutually compatible activities in $S_{ij}$ (including $a_k$)

                    \begin{itemize}
                        \item $A_{ik} = A_{ij} \cap S_{ik}$
                        \item $A_{kj} = A_{ij} \cap S_{kj}$
                        \item $A_{ij} = A_{ik} \cup \{a_k\} \cup A{kj}$
                        \item So, $\lvert A_{ij} \rvert  = \lvert A_{ik} \rvert + \lvert A_{kj} \rvert + 1$
                    \end{itemize}

                \end{itemize}

                \item Observe that only one choice - greedy choice, and that when we make the greedy choice, only one subproblem remains
                \item Develop recursive greedy solution
                \item Convert the recursive algorithm into iterative one
            \end{enumerate}
        \end{itemize}
    \end{itemize}
\end{enumerate}

\end{document}