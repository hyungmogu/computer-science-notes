\documentclass[12pt]{article}
\usepackage[margin=2.5cm]{geometry}
\usepackage{enumerate}
\usepackage{amsfonts}
\usepackage{amsmath}
\usepackage{fancyhdr}
\usepackage{amsmath}
\usepackage{amssymb}
\usepackage{amsthm}
\usepackage{mdframed}
\usepackage{graphicx}
\usepackage{subcaption}
\usepackage{adjustbox}
\usepackage{listings}
\usepackage{xcolor}
\usepackage{booktabs}
\usepackage[utf]{kotex}
\usepackage{hyperref}

\definecolor{codegreen}{rgb}{0,0.6,0}
\definecolor{codegray}{rgb}{0.5,0.5,0.5}
\definecolor{codepurple}{rgb}{0.58,0,0.82}
\definecolor{backcolour}{rgb}{0.95,0.95,0.92}

\lstdefinestyle{mystyle}{
    backgroundcolor=\color{backcolour},
    commentstyle=\color{codegreen},
    keywordstyle=\color{magenta},
    numberstyle=\tiny\color{codegray},
    stringstyle=\color{codepurple},
    basicstyle=\ttfamily\footnotesize,
    breakatwhitespace=false,
    breaklines=true,
    captionpos=b,
    keepspaces=true,
    numbers=left,
    numbersep=5pt,
    showspaces=false,
    showstringspaces=false,
    showtabs=false,
    tabsize=1
}

\lstset{style=mystyle}

\pagestyle{fancy}
\renewcommand{\headrulewidth}{0.4pt}
\lhead{CSC 373}
\rhead{Worksheet 5}

\begin{document}
\title{CSC373 Worksheet 5}
\maketitle

\begin{enumerate}[1.]
    \item \textbf{CLRS 26.1-3:} Suppose that a flow network $G = (V,E)$ violates the assumption that the network
    contains a path $s \rightsquigarrow v \rightsquigarrow t$ for all vertices  $v \in V$ . Let $u$ be a vertex for which there
    is no path $s \rightsquigarrow u \rightsquigarrow t$ . Show that there must exist a maximum flow $f$ in $G$ such
    that $f(u,v) = f(v,u) = 0$ for all vertices $v \in V$.

    \item \textbf{CLRS 26.1-6:} Professor Adam has two children who, unfortunately, dislike each other. The problem
    is so severe that not only do they refuse to walk to school together, but in fact
    each one refuses to walk on any block that the other child has stepped on that day.
    The children have no problem with their paths crossing at a corner. Fortunately
    both the professor’s house and the school are on corners, but beyond that he is not
    sure if it is going to be possible to send both of his children to the same school.
    The professor has a map of his town. Show how to formulate the problem of determining
    whether both his children can go to the same school as a maximum-flow
    problem.

    \item \textbf{CLRS 26.1-7:} Suppose that, in addition to edge capacities, a flow network has \textbf{vertex capacities}.
    That is each vertex $v$ has a limit $l(v)$ on how much flow can pass though $v$. Show
    how to transform a flow network $G = (V,E)$ with vertex capacities into an equivalent
    flow network $G' = (V',E')$ without vertex capacities, such that a maximum
    flow in $G'$ has the same value as a maximum flow in $G$. How many vertices and
    edges does $G'$ have?

    \item \textbf{CLRS 26.2-6:} Suppose that each source $s_i$ in a flow network with multiple
    sources and sinks produces exactly $p_i$ units of flow, so that $\sum\limits_{v \in V} f(s_i,v) = p_i$.
    Suppose also that eeach sink $t_j$ consumes exactly $q_j$ units, so that $\sum\limits_{v \in V} f(v, t_j) = q_j$,
    where $\sum\limits_{i} p_i = \sum\limits_{j} q_j$. Show how to convert the problem of finding a flow $f$
    that obeys these additional constraints into the problem of finding a maximum flow in a single source, single-sink
    flow network.
\end{enumerate}

\end{document}