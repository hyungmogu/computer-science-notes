\documentclass[12pt]{article}
\usepackage[margin=2.5cm]{geometry}
\usepackage{enumerate}
\usepackage{amsfonts}
\usepackage{amsmath}
\usepackage{fancyhdr}
\usepackage{amsmath}
\usepackage{amssymb}
\usepackage{amsthm}
\usepackage{mdframed}
\usepackage{graphicx}
\usepackage{subcaption}
\usepackage{adjustbox}
\usepackage{listings}
\usepackage{xcolor}
\usepackage{booktabs}
\usepackage[utf]{kotex}
\usepackage{hyperref}

\definecolor{codegreen}{rgb}{0,0.6,0}
\definecolor{codegray}{rgb}{0.5,0.5,0.5}
\definecolor{codepurple}{rgb}{0.58,0,0.82}
\definecolor{backcolour}{rgb}{0.95,0.95,0.92}

\lstdefinestyle{mystyle}{
    backgroundcolor=\color{backcolour},
    commentstyle=\color{codegreen},
    keywordstyle=\color{magenta},
    numberstyle=\tiny\color{codegray},
    stringstyle=\color{codepurple},
    basicstyle=\ttfamily\footnotesize,
    breakatwhitespace=false,
    breaklines=true,
    captionpos=b,
    keepspaces=true,
    numbers=left,
    numbersep=5pt,
    showspaces=false,
    showstringspaces=false,
    showtabs=false,
    tabsize=1
}

\lstset{style=mystyle}

\pagestyle{fancy}
\renewcommand{\headrulewidth}{0.4pt}
\lhead{CSC 373}
\rhead{Worksheet 5 Solution}

\begin{document}
\title{CSC373 Worksheet 5 Solution}
\maketitle

\bigskip

\begin{enumerate}[1.]
    \item

    \bigskip

    \underline{\textbf{Rough Works:}}

    \bigskip

    Assume that a flow network $G = (V,E)$ violates the assumption that the
    network contains a path $s \rightsquigarrow v \rightsquigarrow t$ for all
    vertices $v \in V$. Let $u$ be a vertex for which there is no path $s \rightsquigarrow u \rightsquigarrow t$.

    \bigskip

    I must show such that there exists a maximum flow $f$ in $G$ such that
    $f(u,v) = f(v,u) = 0$ for all vertices $v \in V$

    \bigskip

    \begin{enumerate}[1.]
        \item
    \end{enumerate}

    \bigskip

    \underline{\textbf{Notes}}

    \begin{itemize}
        \item \textbf{Maximum Flow:}

        \begin{itemize}
            \item Is the maximum amount of flow that the network would allow to flow from source to sink. $^{[1]}$
        \end{itemize}

        \bigskip

        \item \textbf{Flow Network:}
        \begin{itemize}
            \item $G = (V,E)$ is a directed graph in which each edge $(u,v) \in E$
            has a nonnegative capacity $c(u,v) \geq 0$.
            \item Two vertices must exist: \textbf{source} s and \textbf{sink} t
            \item \textbf{path} from source $s$ to vertax $v$ to sink $t$ is represented by $s \rightsquigarrow v \rightsquigarrow t$

        \end{itemize}

        \bigskip

        \begin{center}
        \includegraphics[width=0.9\linewidth]{images/worksheet_5_solution_1.png}
        \includegraphics[width=0.9\linewidth]{images/worksheet_5_solution_2.png}
        \end{center}

        \item \textbf{Capacity:}

        \begin{itemize}
            \item Is a non-negative function $f: V \times V \to \mathbb{R}_{\geq 0}$
            \item Has \textbf{capacity constraint} where for all $u,v \in V$ $0 \leq f(u,v) \leq c(u,v)$

            \begin{itemize}
                \item Means flow cannot be above capacity constraint
            \end{itemize}
        \end{itemize}

        \item \textbf{Flow:}

        \begin{itemize}
            \item Is a real valued function $f: V \times V \to \mathbb{R}$ in $G$
            \item Satisfies \textbf{capacity constraint} (i.e for all $u,v \in V$, $0 \leq f(u,v) \leq c(u,v)$)
            \item Satisfies \textbf{flow conservation}

            \bigskip

            For all $u \in V - \{s,t\}$, we require

            \bigskip

            \begin{align}
            \sum\limits_{v \in V} f(v,u) = \sum\limits_{v \in V} f(u,v)
            \end{align}

            \bigskip

            Means total flow forward is the same as total flow backward


        \end{itemize}
    \end{itemize}

    \bigskip

    \underline{\textbf{References}}

    \bigskip

    \begin{enumerate}[1)]
        \item Hackerearth, Maximum Flow, \href{https://www.hackerearth.com/practice/algorithms/graphs/maximum-flow/tutorial/#:~:text=It%20is%20defined%20as%20the,Fulkerson%20algorithm%20and%20Dinic's%20Algorithm.}{link}
    \end{enumerate}

\end{enumerate}

\end{document}