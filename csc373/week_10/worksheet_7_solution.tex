\documentclass[12pt]{article}
\usepackage[margin=2.5cm]{geometry}
\usepackage{enumerate}
\usepackage{amsfonts}
\usepackage{amsmath}
\usepackage{fancyhdr}
\usepackage{amsmath}
\usepackage{amssymb}
\usepackage{amsthm}
\usepackage{mdframed}
\usepackage{graphicx}
\usepackage{subcaption}
\usepackage{adjustbox}
\usepackage{listings}
\usepackage{xcolor}
\usepackage{soul}
\usepackage{booktabs}
\usepackage[utf]{kotex}
\usepackage{hyperref}

\definecolor{codegreen}{rgb}{0,0.6,0}
\definecolor{codegray}{rgb}{0.5,0.5,0.5}
\definecolor{codepurple}{rgb}{0.58,0,0.82}
\definecolor{backcolour}{rgb}{0.95,0.95,0.92}

\lstdefinestyle{mystyle}{
    backgroundcolor=\color{backcolour},
    commentstyle=\color{codegreen},
    keywordstyle=\color{magenta},
    numberstyle=\tiny\color{codegray},
    stringstyle=\color{codepurple},
    basicstyle=\ttfamily\footnotesize,
    breakatwhitespace=false,
    breaklines=true,
    captionpos=b,
    keepspaces=true,
    numbers=left,
    numbersep=5pt,
    showspaces=false,
    showstringspaces=false,
    showtabs=false,
    tabsize=1
}

\lstset{style=mystyle}

\pagestyle{fancy}
\renewcommand{\headrulewidth}{0.4pt}
\lhead{CSC 373}
\rhead{Worksheet 7 Solution}

\begin{document}
\title{CSC373 Worksheet 7 Solution}
\maketitle

\bigskip

\begin{enumerate}[1.]
    \item

    \bigskip

    \begin{mdframed}
        \underline{\textbf{My Work}}

        \bigskip

        The longest simple cycle problem is the problem of finding a cycle of maximum
        length in a graph $^{[5]}$.
        \end{mdframed}

        \bigskip

        The decision problem is, given $k$, to determine whether or not the instance graph
        has a simple cycle of length at least $k$. If yes, output 1. Otherwise, output 0.

        \bigskip

        \begin{mdframed}
        \underline{\textbf{My Work}}

        \bigskip

        The language corresponding to the decision problem is as follows:

        \begin{align*}
            \begin{split}
                \text{LONGEST-SIMPLE-CYCLE} = \{\langle G,v_0,v_1,...,v_k,k\rangle: &\text{$G = (V,E)$ is an undirected graph}\\
                &\text{$k \geq 3$ is an integer,}\\
                &\text{$v_0,v_1,...,v_k \in V$ are distinct,}\\
                &\text{$v_0 = v_k$,}\\
                &\text{There should exist a simple cycle in G}\\
                &\text{with at least $k$ edges}\}
            \end{split}
        \end{align*}
        \end{mdframed}

    % \bigskip

    % \underline{\textbf{Rough Works:}}

    % \bigskip

    % \begin{mdframed}
    % \underline{\textbf{My Work}}

    % \bigskip

    % The longest simple cycle problem is the problem of finding a cycle of maximum
    % length in a graph $^{[5]}$.
    % \end{mdframed}

    % \bigskip

    % The decision problem is, given $k$, to determine whether or not the instance graph
    % has a simple cycle of length at least $k$ edges. If yes, output 1. Otherwise, output 0.

    % \bigskip

    % \begin{mdframed}
    % \underline{\textbf{My Work}}

    % \bigskip

    % The language corresponding to the decision problem is as follows:

    % \begin{align*}
    %     \begin{split}
    %         \text{LONGEST-SIMPLE-CYCLE} = \{\langle G,v_0,v_1,...,v_k,k\rangle: &\text{$G = (V,E)$ is an undirected graph}\\
    %         &\text{$k \geq 3$ is an integer,}\\
    %         &\text{$v_0,v_1,...,v_k \in V$ are distinct,}\\
    %         &\text{$v_0 = v_k$,}\\
    %         &\text{There should exist a simple cycle in G}\\
    %         &\text{with at least $k$ edges}\}
    %     \end{split}
    % \end{align*}
    % \end{mdframed}

    \underline{\textbf{Notes}}

    \begin{itemize}
        \item \textbf{A Cycle in an Undirected Graph}

        \begin{itemize}
            \item A path $\langle v_0,v_1,...,v_k$ forms a cycle if $k \geq 3$, and $v_0 = v_k$.
        \end{itemize}

        \item \textbf{Simple Cycle}

        \begin{itemize}
            \item A cycle is simple if $v_1, v2, ..., v_k$ are distinct
        \end{itemize}
        \item \textbf{Decision Problem}

        \begin{itemize}
            \item Is the problem with yes/no solution
        \end{itemize}

        \bigskip

        \item \textbf{Alphabet}

        \begin{itemize}
            \item Is a finite set of symbols
            \item Is denoted $\Sigma$

            \bigskip

            \underline{\textbf{Example:}}

            \bigskip

            For decision problem, its alphabet is: $\Sigma = \{0,1\}$

            \begin{itemize}
                \item 1 means `yes'
                \item 0 means `no'
            \end{itemize}
        \end{itemize}

        \item \textbf{Language}

        \begin{itemize}
            \item Is any set of strings made of symbols from $\Sigma$
            \item Is denoted $L$

            \bigskip

            \underline{\textbf{Example:}}

            \bigskip

            $L = \{10,11,101,111,1011,1101,10001\}$

            \bigskip

            \item Is denoted $\Sigma^*$ for language of all strings over $\Sigma$ plus empty string $\epsilon$.

            \bigskip

            \underline{\textbf{Example:}}

            \bigskip

            $\Sigma^* = \{\epsilon, 0,1,00,01,11,000,...\}$

            \bigskip

            \underline{\textbf{Example 2:}}

            \bigskip

            The decision problem PATH has the corresponding language

            \bigskip

            \begin{align*}
                \begin{split}\text{PATH} = \{\langle G,U,v,k \rangle:&  \text{$G = (V,E)$ is an undirected graph,}\\&\text{$u,v \in V$,}\\&\text{$k \geq 0$ is an integer, and}\\&\text{tere exists a path from $u$ to $v$ in $G$}\\&\text{consisting of at most $k$ edges}\}\end{split}
            \end{align*}
        \end{itemize}

        \item \textbf{P}

        \begin{itemize}
            \item Is set of problems that can be solved by a deterministic Turing machine in Polynomial time (i.e. $\mathcal{O}(n^k)$) $^{[2]}$.

            \bigskip

            \underline{\textbf{Example:}}

            \bigskip

            \begin{enumerate}[1)]
                \item Shortest path problems
                \item Calculating the greatest common divisor
                \item Finding maximum bipartite matching
            \end{enumerate}

            \bigskip

            \begin{center}
            \includegraphics[width=0.7\linewidth]{images/worksheet_7_solution_1.png}
            \end{center}
        \end{itemize}

        \bigskip

        \item \textbf{NP (Non-deterministic Polynominal):}

        \begin{center}
        \includegraphics[width=0.7\linewidth]{images/worksheet_7_solution_2.png}
        \end{center}

        \begin{itemize}
            \item Is set of decision problems that can be solved by a Non-deterministic Turing Machine in Polynomial time.$^{[2]}$
            \item Has no particular rule is followed to make a guess $^{[1]}$.
            \item Can be solved in polynominal time via a ``lucky algorithm'', a magical algorithm that always make a right guess $^{[2]}$
            \item $P \subseteq NP$
        \end{itemize}

        \bigskip

        \underline{\textbf{Examples:}}

        \begin{itemize}
            \item Longest-path problems
            \item Hamiltonian Cycle
            \item Graph coloring
        \end{itemize}

        \bigskip

        \item \textbf{NP-Complete Problems:}

        \begin{center}
        \includegraphics[width=0.7\linewidth]{images/worksheet_7_solution_3.png}
        \end{center}

        \begin{itemize}
            \item A decision problem A is NP-complete (NPC) if

            \begin{enumerate}[1)]
                \item $A \in NP$ and
                \item Every (other) problems $A'$ in NP is reducible to $A$
            \end{enumerate}
            \item Has no efficient solution in polynominal number of steps (not yet) $^{[3]}$
            \item Is not likely that there is an algorithm to make it efficient $^{[3]}$
        \end{itemize}

        \item \textbf{NP-Hard:}

        \bigskip

        \begin{center}
        \includegraphics[width=0.7\linewidth]{images/worksheet_7_solution_4.png}
        \end{center}

        \begin{itemize}
            \item A decision problem A is NP-hard if

            \begin{enumerate}[1)]
                \item $A \in NP$ (Not necessarily) and
                \item Every (other) problems $A'$ in NP is reducible to $A$
            \end{enumerate}
            \item NP-Hard means ``at least as hard as any problems in NP''
            \item Does not have to be about decision problems
        \end{itemize}

        \underline{\textbf{Example:}}

        \bigskip

        \begin{enumerate}[1)]
            \item Alan Turing's Halting Problem
        \end{enumerate}

    \end{itemize}

    \bigskip

    \underline{\textbf{References}}

    \bigskip

    \begin{enumerate}[1)]
        \item Encyclopedia Britannica, NP-Complete Problem, \href{https://www.britannica.com/science/NP-complete-problem}{link}
        \item Geeks for Geeks, NP-Completeness, \href{https://www.geeksforgeeks.org/np-completeness-set-1/}{link}
        \item Wikipedia, NP-complete, \href{https://simple.wikipedia.org/wiki/NP-complete}{link}
        \item UCLA UC-Davis, ECS122A Handout on NP-Completeness, \href{https://web.cs.ucdavis.edu/~bai/ECS122A/npcnotes.pdf}{link}
    \end{enumerate}
\end{enumerate}


\end{document}