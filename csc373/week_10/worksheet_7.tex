\documentclass[12pt]{article}
\usepackage[margin=2.5cm]{geometry}
\usepackage{enumerate}
\usepackage{amsfonts}
\usepackage{amsmath}
\usepackage{fancyhdr}
\usepackage{amsmath}
\usepackage{amssymb}
\usepackage{amsthm}
\usepackage{mdframed}
\usepackage{graphicx}
\usepackage{subcaption}
\usepackage{adjustbox}
\usepackage{listings}
\usepackage{xcolor}
\usepackage{booktabs}
\usepackage[utf]{kotex}
\usepackage{hyperref}

\definecolor{codegreen}{rgb}{0,0.6,0}
\definecolor{codegray}{rgb}{0.5,0.5,0.5}
\definecolor{codepurple}{rgb}{0.58,0,0.82}
\definecolor{backcolour}{rgb}{0.95,0.95,0.92}

\lstdefinestyle{mystyle}{
    backgroundcolor=\color{backcolour},
    commentstyle=\color{codegreen},
    keywordstyle=\color{magenta},
    numberstyle=\tiny\color{codegray},
    stringstyle=\color{codepurple},
    basicstyle=\ttfamily\footnotesize,
    breakatwhitespace=false,
    breaklines=true,
    captionpos=b,
    keepspaces=true,
    numbers=left,
    numbersep=5pt,
    showspaces=false,
    showstringspaces=false,
    showtabs=false,
    tabsize=1
}

\lstset{style=mystyle}

\pagestyle{fancy}
\renewcommand{\headrulewidth}{0.4pt}
\lhead{CSC 373}
\rhead{Worksheet 7}

\begin{document}
\title{CSC373 Worksheet 7}
\maketitle

\begin{enumerate}[1.]
    \item \textbf{CLRS 34.1-2:} Give a formal definition for the problem of finding the longest simple cycle in an
    undirected graph. Give a related decision problem. Give the language corresponding
    to the decision problem.

    \item \textbf{CLRS 34.1-3:} Give a formal encoding of directed graphs as binary strings using an adjacencymatrix
    representation. Do the same using an adjacency-list representation. Argue
    that the two representations are polynomially related.

    \item \textbf{CLRS 34.1-4:} Is the dynamic-programming algorithm for the 0-1 knapsack problem that is asked
    for in Exercise 16.2-2 a polynomial-time algorithm? Explain your answer.

    \item \textbf{CLRS 34.1-5:} Show that if an algorithm makes at most a constant number of calls to polynomialtime
    subroutines and performs an additional amount of work that also takes polynomial
    time, then it runs in polynomial time. Also show that a polynomial number of
    calls to polynomial-time subroutines may result in an exponential-time algorithm.

    \item \textbf{CLRS 34.1-6:} Show that the class P, viewed as a set of languages, is closed under union, intersection,
    concatenation, complement, and Kleene star. That is, if $L_1, L_2 \in P$, then
    $L_1 \cup L_2 \in P$, $L_1 \cap L_2 \in P$, $L_1L_2 \in P$, $L_1 \in P$ and $L_1^* \in P$.

    \item \textbf{CLRS 34.2-1:}
    Consider the language GRAPH-ISOMORPHISM D fhG1; G2i W G1 and G2 are
    isomorphic graphsg. Prove that GRAPH-ISOMORPHISM 2 NP by describing a
    polynomial-time algorithm to verify the language.

    \item \textbf{CLRS 34.2-3:}
    Show that if HAM-CYCLE 2 P, then the problem of listing the vertices of a
    hamiltonian cycle, in order, is polynomial-time solvable.

    \item \textbf{CLRS 34.2-5:}
    Show that any language in NP can be decided by an algorithm running in
    time $2^{\mathcal{O}(n^k)}$ for some constant k.

    \item \textbf{CLRS 34.2-8:}
    Let  be a boolean formula constructed from the boolean input variables $x_1, x_2,
    ,..., x_k$, negations ($\neq$), ANDs ($\land$), ORs ($\lor$), and parentheses. The formula $\Phi$ is a
    tautology if it evaluates to 1 for every assignment of 1 and 0 to the input variables.
    Define TAUTOLOGY as the language of boolean formulas that are tautologies.
    Show that TAUTOLOGY $\in$ co-NP.

    \item \textbf{CLRS 34.2-9:}
    Prove that P $\subseteq$ co-NP.

    \item \textbf{CLRS 34.2-10:}
    Prove that if NP $\neq$ co-NP, then P $\neq$ NP.

    \item \textbf{CLRS 34.3-3:}
    Prove that $L \leq_p \bar{L}$ if and only if $\bar{L} \leq_p L$.

    \item \textbf{CLRS 34.3-7:}
    Show that, with respect to polynomial-time reductions (see Exercise 34.3-6), $L$ is
    complete for NP if and only if $\bar{L}$ is complete for co-NP.

    \item \textbf{CLRS 34.5-7:}
    The \textbf{longest-simple-cycle} problem is the problem of determining a simple cycle
    (no repeated vertices) of maximum length in a graph. Formulate a related decision
    problem, and show that the decision problem is NP-complete.

    \item \textbf{CLRS 34.5-8:}
    In the \textbf{half 3-CNF satisfiability problem}, we are given a 3-CNF formula  with n
    variables and m clauses, where m is even. We wish to determine whether there
    exists a truth assignment to the variables of  such that exactly half the clauses
    evaluate to 0 and exactly half the clauses evaluate to 1. Prove that the half 3-CNF
    satisfiability problem is NP-complete.
\end{enumerate}

\end{document}