\documentclass[12pt]{article}
\usepackage[margin=2.5cm]{geometry}
\usepackage{enumerate}
\usepackage{amsfonts}
\usepackage{amsmath}
\usepackage{fancyhdr}
\usepackage{amsmath}
\usepackage{amssymb}
\usepackage{amsthm}
\usepackage{mdframed}
\usepackage{graphicx}
\usepackage{subcaption}
\usepackage{adjustbox}
\usepackage{listings}
\usepackage{xcolor}
\usepackage{booktabs}
\usepackage[utf]{kotex}
\usepackage{hyperref}

\definecolor{codegreen}{rgb}{0,0.6,0}
\definecolor{codegray}{rgb}{0.5,0.5,0.5}
\definecolor{codepurple}{rgb}{0.58,0,0.82}
\definecolor{backcolour}{rgb}{0.95,0.95,0.92}

\lstdefinestyle{mystyle}{
    backgroundcolor=\color{backcolour},
    commentstyle=\color{codegreen},
    keywordstyle=\color{magenta},
    numberstyle=\tiny\color{codegray},
    stringstyle=\color{codepurple},
    basicstyle=\ttfamily\footnotesize,
    breakatwhitespace=false,
    breaklines=true,
    captionpos=b,
    keepspaces=true,
    numbers=left,
    numbersep=5pt,
    showspaces=false,
    showstringspaces=false,
    showtabs=false,
    tabsize=1
}

\lstset{style=mystyle}

\pagestyle{fancy}
\renewcommand{\headrulewidth}{0.4pt}
\lhead{CSC 373}
\rhead{Worksheet 0}

\begin{document}
\title{CSC373 Worksheet 0}
\maketitle

\begin{enumerate}[1.]
    \item \textbf{CLRS 4.3-1:} Show that the solution of $T(n) = T(n - 1) + n$ is $\mathcal{O}(n^2)$.
    \item \textbf{CLRS 4.3-2:} Show that the solution of $T(n) = T(\lceil n/2 \rceil)  + 1$ is $\mathcal{O}(\lg n)$.
    \item \textbf{CLRS 4.3-3:} We saw that the solution of $T(n) = 2T(\lfloor n/2 \rfloor) + n$ is
    $O(n \lg n)$. Show that the solution of this recurrence is also $\Omega(n \lg n)$. Conclude that the solution is $\Theta(n \lg n)$.
    \item \textbf{CLRS 4.3-5:} Show that $\Theta(n \lg n)$ is the solution to the “exact” recurrence (4.3) for merge sort.
    \item \textbf{CLRS 4.3-6:} Show that the solution to $T(n) = 2T(\lfloor n/2 \rfloor + 17) + n$ is $O(n \lg n)$.
    \item \textbf{CLRS 4.3-7:}  Using the master method in Section 4.5, you can show that the solution to the
    recurrence $T(n)= 4T(n/3) + n$ is $T(n) = \Theta(n^{\log_3 4})$. Show that a substitution
    proof with the assumption $T(n) \leq cn^{\log_3 4}$ fails. Then show how to subtract off a
    lower-order term to make a substitution proof work.
    \item \textbf{CLRS 4.3-8:}  Using the master method in Section 4.5, you can show that the solution to the
    recurrence $T(n)= 4T(n/2) + n$ is $T(n) = \Theta(n^2)$. Show that a substitution
    proof with the assumption $T(n) \leq cn^2$ fails. Then show how to subtract off a
    lower-order term to make a substitution proof work.
    \item \textbf{CLRS 4.4-1:}  Use a recursion tree to determine a good asymptotic upper bound on the
    recurrrence $T(n) = 3T(\lfloor n/2 \rfloor) +n$. Use the subtitution method to verify your answer.
    \item \textbf{CLRS 4.4-2:}  Use a recursion tree to determine a good asymptotic upper bound on the
    recurrrence $T(n) = 3T(n/2) + n^2$. Use the subtitution method to verify your answer.
    \item \textbf{CLRS 4.4-3:}  Use a recursion tree to determine a good asymptotic upper bound on the recurrence
    $T(n) = 4T(n/2 +2) + n$. Use the subtitution method to verify your answer.
    \item \textbf{CLRS 4.4-4:}  Use a recursion tree to determine a good asymptotic upper bound on the recurrence
    $T(n) = 2T(n - 1) + 1$. Use the subtitution method to verify your answer.
    \item \textbf{CLRS 4.4-5:}  Use a recursion tree to determine a good asymptotic upper bound on the recurrence
    $T(n) = 2T(n - 1) + T(n/2) + n$. Use the subtitution method to verify your answer.
    \item \textbf{CLRS 4.4-6:}  Argue that the solution to the recurrent $T(n) = 4 T(\lfloor n/2 \rfloor) + cn$ where
    $c$ is a constant, and provide a tight asymptotic bound on its solution. Verify your bound
    on the subtitution method.
    \item \textbf{CLRS 4.4-7:}  Draw the recursion tree for $T(n) = 4T(n/2) + cn$, where $c$ is a constant, and
    provide a tight asymptotic bound on its solution. Verify your bound by the substitution method.
\end{enumerate}

\end{document}