\documentclass[12pt]{article}
\usepackage[margin=2.5cm]{geometry}
\usepackage{enumerate}
\usepackage{amsfonts}
\usepackage{amsmath}
\usepackage{fancyhdr}
\usepackage{amsmath}
\usepackage{amssymb}
\usepackage{amsthm}
\usepackage{mdframed}
\usepackage{graphicx}
\usepackage{subcaption}
\usepackage{adjustbox}
\usepackage{listings}
\usepackage{xcolor}
\usepackage{booktabs}
\usepackage[utf]{kotex}
\usepackage{hyperref}

\definecolor{codegreen}{rgb}{0,0.6,0}
\definecolor{codegray}{rgb}{0.5,0.5,0.5}
\definecolor{codepurple}{rgb}{0.58,0,0.82}
\definecolor{backcolour}{rgb}{0.95,0.95,0.92}

\lstdefinestyle{mystyle}{
    backgroundcolor=\color{backcolour},
    commentstyle=\color{codegreen},
    keywordstyle=\color{magenta},
    numberstyle=\tiny\color{codegray},
    stringstyle=\color{codepurple},
    basicstyle=\ttfamily\footnotesize,
    breakatwhitespace=false,
    breaklines=true,
    captionpos=b,
    keepspaces=true,
    numbers=left,
    numbersep=5pt,
    showspaces=false,
    showstringspaces=false,
    showtabs=false,
    tabsize=1
}

\lstset{style=mystyle}

\pagestyle{fancy}
\renewcommand{\headrulewidth}{0.4pt}
\lhead{CSC 373}
\rhead{Worksheet 0 Solution}

\begin{document}
\title{CSC373 Worksheet 0 Solution}
\maketitle

\bigskip

\begin{enumerate}[1.]
    \item

    \bigskip

    \underline{\textbf{Notes:}}

    \bigskip

    \begin{itemize}
        \item Substitution method

        \begin{itemize}
            \item Solves recurrences
            \begin{itemize}
                \item Recurrence characters the running time of divide-and-conquer algorithm
            \end{itemize}
            \item How it works:

            \begin{enumerate}[1.]
                \item Make a guess for the solution
                \item Use mathematical induction to prove the guess is correct or incorrect.
            \end{enumerate}

            \bigskip

            \underline{\textbf{Example:}}

            \bigskip

            \underline{Recurrence:} $T(n) = 2T(\lfloor n/2 \rfloor) + n$

            \bigskip

            \underline{Guess:} $T(n) = \mathcal{O}(n\log n)$,

            \bigskip

            We need to show $T(n) \leq cn \lg n$.

            \bigskip

            \begin{enumerate}[1.]
                \item Assume the bound holds for all positive $m < n$, in particular $m = \lfloor n/2 \rfloor$
                \item Find the upper bound of $T(m)$

                \bigskip

                $T(\lfloor n/2 \rfloor) \leq c \lfloor n/2 \rfloor \lg (\lfloor n/2 \rfloor)$

                \bigskip

                \item Show $T(n) = 2T(\lfloor n/2 \rfloor) + n$ leads to $T(n) \leq cn \lg n$

                \bigskip

                \begin{align}
                    T(n) &\leq 2(c \lfloor n/2 \rfloor \lg (\lfloor n/2 \rfloor)) + n\\
                    &\leq cn \lg (n/2) + n\\
                    &= cn \lg (n) - cn \lg 2 + n\\
                    &= cn \lg (n) - cn + n\\
                    &\leq cn \lg (n) - cn + cn\\
                    &\leq cn \lg (n)
                \end{align}

                \item Show that the boundary holds using mathematical induction

                \bigskip

                \color{red}Doesn't have information in detail. Skipping this for now.\color{black}
            \end{enumerate}

            \item Making good guess

            \begin{itemize}
                \item Three suggestions
                \begin{enumerate}[1.]
                    \item Using recursion tree
                    \item Through practice
                    \item prove loose upper and lower bounds on the recurrence and then reduce the range of uncertainty
                \end{enumerate}
            \end{itemize}
        \end{itemize}
    \end{itemize}

\end{enumerate}

\end{document}