\documentclass[12pt]{article}
\usepackage[margin=2.5cm]{geometry}
\usepackage{enumerate}
\usepackage{amsfonts}
\usepackage{amsmath}
\usepackage{fancyhdr}
\usepackage{amsmath}
\usepackage{amssymb}
\usepackage{amsthm}
\usepackage{mdframed}
\usepackage{graphicx}
\usepackage{subcaption}
\usepackage{adjustbox}
\usepackage{listings}
\usepackage{xcolor}
\usepackage{booktabs}
\usepackage[utf]{kotex}
\usepackage{hyperref}

\definecolor{codegreen}{rgb}{0,0.6,0}
\definecolor{codegray}{rgb}{0.5,0.5,0.5}
\definecolor{codepurple}{rgb}{0.58,0,0.82}
\definecolor{backcolour}{rgb}{0.95,0.95,0.92}

\lstdefinestyle{mystyle}{
    backgroundcolor=\color{backcolour},
    commentstyle=\color{codegreen},
    keywordstyle=\color{magenta},
    numberstyle=\tiny\color{codegray},
    stringstyle=\color{codepurple},
    basicstyle=\ttfamily\footnotesize,
    breakatwhitespace=false,
    breaklines=true,
    captionpos=b,
    keepspaces=true,
    numbers=left,
    numbersep=5pt,
    showspaces=false,
    showstringspaces=false,
    showtabs=false,
    tabsize=1
}

\lstset{style=mystyle}

\pagestyle{fancy}
\renewcommand{\headrulewidth}{0.4pt}
\lhead{CSC 373}
\rhead{Worksheet 1}

\begin{document}
\title{CSC373 Worksheet 1}
\maketitle

\bigskip

Source: \href{http://www.cs.toronto.edu/~denisp/csc373/material.html}{link}

\begin{enumerate}[1.]
    \item \textbf{CLRS 4.2-2:} Write Pseudocode for Strassen's algorithm
    \item \textbf{CLRS 4.2-4:} What is the largest $k$ such that if you can multiply $3 \times 3$ matrics
    using $k$ multiplications (not assuming commutativity of multiplciation), then you can multiply $n \times n$
    matrices in time $o(n^{\lg 7})$? What would the runing time of this algorithm be?
    \item \textbf{CLRS 4.2-5:} V.Pan has discovered a way of multiplying $68 \times 68$ matrices, using 132,464 multiplications,
    a way of multiplying $70 \times 70 matrices$ using $143,640$ multiplications, and a way of
    multiplying $72 \times 72$ matricsusing $155,424$ multiplicaiotns. Which method yields the
    best asymptotic running time when used in a divide-and-conquer matric-multiplication algorithm? How does it
    compare to Strassen's algorithm?
    \item \textbf{CLRS 4.2-7:} Show how to multiply the complex numbers $a + bi$ and $c + di$
    using only three multiplications of real numbers. The algorithm should take $a,b,c$ and $d$ as
    input and product the real component $ac - bd$ and the imaginary component $ad + bc$
    separately.
    \item \textbf{CLRS 4-1:} Give asymptotic upper and lower bounds for $T(n)$ in each of the following
    recurrences. Assume that $T(n)$ is constant for $n \leq 2$. Make your bounds as tight
    as possible, and justify your answers

    \bigskip

    \begin{enumerate}[a)]
        \item $T(n) = 2T(n/2) + n^4$
        \item $T(n) = T(7n/10) + n$
        \item $T(n) = 16T(n/4) + n^2$
        \item $T(n) = 7T(n/3) + n^2$
        \item $T(n) = 7T(n/2) + n^2$
        \item $T(n) = 2T(n/4) + \sqrt{n}$
        \item $T(n) = 2T(n - 2) + n^2$
    \end{enumerate}


    \item \textbf{CLRS 33.4-2:} Show that it actually suffices to check only the
    points in the 5 array positions following each point in the array $Y'$

    \item \textbf{CLRS 33.4-4:} Give two points $p_1$ and $p_2$ in the plane, the $L_{\infty}$-distance
    between them is given by $max(\vert x_1 - x_2 \vert, \vert y_1 - y_2 \vert)$. Modify the closest-pair
    algorithm to use the $L_{\infty}$-distance.

    \item \textbf{CLRS 33.4-6:} Auggest a change to the closest-pair algorithm that avoids presorting the $Y$
    array but leaves the running time as $O(n \lg n)$. (\textit{Hint:} Merge sorted arrays $Y_L$ and $Y_R$ to
    form the sorted array $Y$.

\end{enumerate}


\end{document}