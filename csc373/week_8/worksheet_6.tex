\documentclass[12pt]{article}
\usepackage[margin=2.5cm]{geometry}
\usepackage{enumerate}
\usepackage{amsfonts}
\usepackage{amsmath}
\usepackage{fancyhdr}
\usepackage{amsmath}
\usepackage{amssymb}
\usepackage{amsthm}
\usepackage{mdframed}
\usepackage{graphicx}
\usepackage{subcaption}
\usepackage{adjustbox}
\usepackage{listings}
\usepackage{xcolor}
\usepackage{booktabs}
\usepackage[utf]{kotex}
\usepackage{hyperref}

\definecolor{codegreen}{rgb}{0,0.6,0}
\definecolor{codegray}{rgb}{0.5,0.5,0.5}
\definecolor{codepurple}{rgb}{0.58,0,0.82}
\definecolor{backcolour}{rgb}{0.95,0.95,0.92}

\lstdefinestyle{mystyle}{
    backgroundcolor=\color{backcolour},
    commentstyle=\color{codegreen},
    keywordstyle=\color{magenta},
    numberstyle=\tiny\color{codegray},
    stringstyle=\color{codepurple},
    basicstyle=\ttfamily\footnotesize,
    breakatwhitespace=false,
    breaklines=true,
    captionpos=b,
    keepspaces=true,
    numbers=left,
    numbersep=5pt,
    showspaces=false,
    showstringspaces=false,
    showtabs=false,
    tabsize=1
}

\lstset{style=mystyle}

\pagestyle{fancy}
\renewcommand{\headrulewidth}{0.4pt}
\lhead{CSC 373}
\rhead{Worksheet 6}

\begin{document}
\title{CSC373 Worksheet 6}
\maketitle

\begin{enumerate}[1.]
    \item \textbf{CLRS 29.1-4:} Convert the following linear program into standard form:

    \bigskip

    Minimize

    \begin{align*}
        2x_1 + 7 x_2 + x_3
    \end{align*}

    Subject to

    \begin{align*}
        x_1 - x_3  &= 7\\
        3x_1 + x_2 &\geq 7\\
        x_2 &\geq 0\\
        x_3 &\leq 0
    \end{align*}

    \item \textbf{CLRS 29.1-5:} Convert the following linear program into slack form:

    \bigskip

    Maximize

    \begin{align*}
        2x_1 - 6x_3
    \end{align*}

    Subject to

    \begin{align*}
        x_1 + x_2 - x_3 &\leq 7\\
        3x_1 - x_2 &\geq 7\\
        -x_1 + 2x_2 + 2x_3 &\geq 0\\
        x_1,x_2,x_3 &\geq 0
    \end{align*}


    \item \textbf{CLRS 29.1-6:} Show the following linear program is infeasible:

    \bigskip

    Maximize

    \begin{align*}
        3x_1 - 2x_2
    \end{align*}

    Subject to

    \begin{align*}
        x_1 + x_2 &\leq 2\\
        -2x_1 - 2x_2 &\leq -10\\
        x_1, x_2 &\geq 0
    \end{align*}

    \item \textbf{CLRS 29.1-7:} Show that the following linear program is unbounded:

    \bigskip

    Maximize

    \begin{align*}
        x_1 - x_2
    \end{align*}

    Subject to

    \begin{align*}
        -2x_1 + x_2 &\leq -1\\
        -x_1 - 2x_2 &\leq -2\\
        x_1,x_2 &\geq 0
    \end{align*}

    \item \textbf{CLRS 29.1-8:} Suppose that we have a general linear program with n variables and m constraints,
    and suppose that we convert it into standard form. Give an upper bound on the
    number of variables and constraints in the resulting linear program.

    \item \textbf{CLRS 29.1-9:} Give an example of a linear program for which the feasible region is not bounded,
    but the optimal objective value is finite.

    \item \textbf{CLRS 29.2-3:} In the single-source shortest-paths problem, we want to find the shortest-path
    weights from a source vertex $s$ to all vertices  $v \in V$. Given a graph G, write a
    linear program for which the solution has the property that $d_v$ is the shortest-path
    weight from $s$ to $v$ for each vertex $v \in V$.

    \item \textbf{CLRS 29.2-7:} In the \textbf{minimum-cost multicommodity-folow problem}, we are given directed
    graph $G = (V,E)$ in which each edge $(u,v) \in E$ has a nonnegative capacity $c(u,v) \geq 0$
    and a cost $a(u,v)$. As in the multicommodity-flow problem, we are given $k$
    different commodities, $K_1, K_2, ..., K_k$, where we specify commodify $i$ by
    the triple $K_i = (s_i, t_i, d_i)$. We define the flow $f_i$ for commodity $i$ and
    the aggregate flow $f_{uv}$ in which the aggregate flow on each ege $(u,v)$ is no more than the
    capacity of edge $(u,v)$. The cost of a flow is $\sum\limits_{u,v \in V} a(u,v)f_{uv}$, and
    the goal is to find the feasible flow of minimum cost. Express this problem as a linear
    program.

\end{enumerate}

\end{document}