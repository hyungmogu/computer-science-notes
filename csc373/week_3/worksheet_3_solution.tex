\documentclass[12pt]{article}
\usepackage[margin=2.5cm]{geometry}
\usepackage{enumerate}
\usepackage{amsfonts}
\usepackage{amsmath}
\usepackage{fancyhdr}
\usepackage{amsmath}
\usepackage{amssymb}
\usepackage{amsthm}
\usepackage{mdframed}
\usepackage{graphicx}
\usepackage{subcaption}
\usepackage{adjustbox}
\usepackage{listings}
\usepackage{xcolor}
\usepackage{booktabs}
\usepackage[utf]{kotex}
\usepackage{hyperref}

\definecolor{codegreen}{rgb}{0,0.6,0}
\definecolor{codegray}{rgb}{0.5,0.5,0.5}
\definecolor{codepurple}{rgb}{0.58,0,0.82}
\definecolor{backcolour}{rgb}{0.95,0.95,0.92}

\lstdefinestyle{mystyle}{
    backgroundcolor=\color{backcolour},
    commentstyle=\color{codegreen},
    keywordstyle=\color{magenta},
    numberstyle=\tiny\color{codegray},
    stringstyle=\color{codepurple},
    basicstyle=\ttfamily\footnotesize,
    breakatwhitespace=false,
    breaklines=true,
    captionpos=b,
    keepspaces=true,
    numbers=left,
    numbersep=5pt,
    showspaces=false,
    showstringspaces=false,
    showtabs=false,
    tabsize=1
}

\lstset{style=mystyle}

\pagestyle{fancy}
\renewcommand{\headrulewidth}{0.4pt}
\lhead{CSC 373}
\rhead{Worksheet 3 Solution}

\begin{document}
\title{CSC373 Worksheet 3 Solution}
\maketitle

\bigskip

\begin{enumerate}[1.]
    \item

    \bigskip

    \underline{\textbf{Notes:}}

    \bigskip

    \begin{itemize}
        \item Dynamic Programming

        \begin{itemize}
            \item Is applied to optimization problems
            \item Applies when the subproblems overlap
            \item Uses the following sequence of steps

            \begin{enumerate}[1.]
                \item Characterize the structure of an optimal solution
                \item Recursively define the value of an optimal solution
                \item Construct an optimal solution from computed information
            \end{enumerate}
        \end{itemize}

        \bigskip

        \item Matrix-chain Multiplication

        \begin{itemize}
            \item Is an optimization problem solved using dynamic programming
            \item Goal is to find matrix parenthesis with fewest number of operations

            \bigskip

            \underline{\textbf{Example:}}

            \bigskip

            Given chain of matrices $<A,B,C>$, it's fully parenthesized product is:

            \bigskip

            \begin{itemize}
                \item $(AB)C$ needs $(10 \times 30 \times 5) + (10 \times 5 \times 60) = 1500 + 3000 = 4500$ operations
                \item $A(BC)$ needs $(30 \times 5 \times 60) + (10 \times 30 \times 60) = 27000$ operations
            \end{itemize}

            \bigskip

            Thus, $(AB)C$ performs more efficiently than $A(BC)$.

            \bigskip

            \item Is stated as: given a chain $<A_1, A_2, ..., A_n>$ of $n$ matrices,
            where for $i = 1,2,...,n$ matrix $A_i$ has dimension $p_{i-1} \times p_i$,
            fully parenthesize the product $A_1A_2...A_n$ in a way that minimizes the number of scalar multiplications.

            \item Steps

            \begin{enumerate}[1.]
                \item The Structure of an Optimal Parenthesization
                \item Recursive Solution
                \item Computing the Estimated Cost
                \item Constructing the Optimal Solution
            \end{enumerate}
        \end{itemize}
    \end{itemize}
\end{enumerate}

\end{document}