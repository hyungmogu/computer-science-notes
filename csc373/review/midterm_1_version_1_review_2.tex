\documentclass[12pt]{article}
\usepackage[margin=2.5cm]{geometry}
\usepackage{enumerate}
\usepackage{amsfonts}
\usepackage{amsmath}
\usepackage{fancyhdr}
\usepackage{amsmath}
\usepackage{amssymb}
\usepackage{amsthm}
\usepackage{mdframed}
\usepackage{graphicx}
\usepackage{subcaption}
\usepackage{adjustbox}
\usepackage{listings}
\usepackage{xcolor}
\usepackage{booktabs}
\usepackage[utf]{kotex}
\usepackage{hyperref}
\usepackage{accents}

\definecolor{codegreen}{rgb}{0,0.6,0}
\definecolor{codegray}{rgb}{0.5,0.5,0.5}
\definecolor{codepurple}{rgb}{0.58,0,0.82}
\definecolor{backcolour}{rgb}{0.95,0.95,0.92}

\lstdefinestyle{mystyle}{
    backgroundcolor=\color{backcolour},
    commentstyle=\color{codegreen},
    keywordstyle=\color{magenta},
    numberstyle=\tiny\color{codegray},
    stringstyle=\color{codepurple},
    basicstyle=\ttfamily\footnotesize,
    breakatwhitespace=false,
    breaklines=true,
    captionpos=b,
    keepspaces=true,
    numbers=left,
    numbersep=5pt,
    showspaces=false,
    showstringspaces=false,
    showtabs=false,
    tabsize=1
}

\lstset{style=mystyle}

\pagestyle{fancy}
\renewcommand{\headrulewidth}{0.4pt}
\lhead{CSC 373}
\rhead{Midterm 1 Version 1 Review 2}

\begin{document}
\title{Midterm 1 Version 1 Review 2}
\maketitle

\begin{enumerate}[1.]
    \item

    \begin{enumerate}[a)]
        \item ${aaa,aab,aac,bba,bbb,bbc,cca,ccb,ccc}$
        \item

        \underline{\textbf{Solution:}}

        \begin{tabular}{|c|c|c|c|c|}
            \hline
            $p$ & $q$ & $r$ & $p \lor q$ & $(p \lor q) \Rightarrow \neg r$\\
            T & T & T & T & T \\
            \hline
            T & T & F & T & F \\
            \hline
            T & F & T & T & T \\
            \hline
            F & T & T & T & T \\
            \hline
            T & F & F & T & F \\
            \hline
            F & T & F & T & F \\
            \hline
            F & F & T & F & T \\
            \hline
            F & F & F & F & T \\
            \hline
        \end{tabular}

        \item

        $\exists x \in \mathbb{N}, \forall y \in \mathbb{N}, \neg P(x,y) \land \neg Q(x,y)$

        \bigskip

        \begin{mdframed}
            \underline{\textbf{Correct Solution:}}

            \bigskip

            Let $x = \_\_\_\_$. Let $y \in P$.

            \bigskip

            \color{red}We need to prove $\neg P(x,y)$ and $\neg Q(x,y)$ are true.\color{black}

        \end{mdframed}

    \end{enumerate}

    \item

    \begin{enumerate}[a)]
        \item $\exists x \in P, Student(x) \land Attends(x)$
        \item $\forall x \in P, \exists y \in P, Student(y) \land Attends(y) \Rightarrow Loves(x,y)$

        \bigskip

        \begin{mdframed}
            \underline{\textbf{Correct Solution:}}

            \bigskip

            $\forall x \in P, \exists y \in P, Student(y) \land Attends(y) \color{red} \land Loves(x,y)$
        \end{mdframed}

        \item $\forall x \in P, Student(x) \land Attends(x) \Rightarrow Loves(x,x)$
        \item $\forall x,y \in P, x \neq y \Rightarrow Loves(x,y) \Rightarrow Attends(x) \lor Attends(y)$
    \end{enumerate}

    \item

    \begin{enumerate}[a)]
        \item $\forall a,b,c \in \mathbb{Z}, \exists k_1, k_2, k_3 \in \mathbb{Z}, a = k_1b \land b = k_2c \Rightarrow a = k_3c$
        \item

        \begin{proof}
        Let $a,b,c \in \mathbb{Z}$, and there is some $k1,k2,k3 \in \mathbb{Z}$ such that
        $a = k_1b, b = k_2c$.

        \bigskip

        I need to prove $a = k_3 c$.

        \bigskip

        Let $k_3 = k_1k_2$.

        \bigskip

        Then, we can conclude

        \begin{align}
            a &= k_1b    & [\text{By header}]\\
            &= k_1k_2c    & [\text{By replacing $b$ with $k_2c$}]\\
            &= k_3c    & [\text{By $k_3 = k_1k_2$}]
        \end{align}
        \end{proof}
    \end{enumerate}

    \item
    \setcounter{equation}{0}
    \begin{proof}

    Let $x,y \in \mathbb{R}$. Assume $\lfloor x + y \rfloor$.

    \bigskip

    I need to show $\lfloor x + y \rfloor \geq \lfloor x \rfloor + \lfloor y \rfloor$.

    \bigskip

    Indeed we have

    \begin{align}
        \lfloor x + y \rfloor &= \lfloor \lfloor x \rfloor + \epsilon + y \rfloor & [\text{By fact 1}]\\
        &= \lfloor x \rfloor + \lfloor \epsilon + y \rfloor & [\text{By fact 2}]\\
        &\geq \lfloor x \rfloor + \lfloor y \rfloor & [\text{Since $\epsilon \geq 0$}]
    \end{align}
    \end{proof}

    \bigskip

    \begin{mdframed}
        \underline{\textbf{Correct Solution:}}
        \setcounter{equation}{0}
        \begin{proof}

        \bigskip

        Let $x,y \in \mathbb{R}$. Assume $\lfloor x + y \rfloor$.

        \bigskip

        I need to show $\lfloor x + y \rfloor \color{red}\geq\color{black} \lfloor x \rfloor + \lfloor y \rfloor$.

        \bigskip

        Indeed we have

        \begin{align}
            \lfloor x + y \rfloor &= \lfloor \lfloor x \rfloor + \epsilon + y \rfloor & [\text{By fact 1}]\\
            &= \lfloor x \rfloor + \lfloor \epsilon + y \rfloor & [\text{By fact 2}]\\
            &\geq \lfloor x \rfloor + \lfloor y \rfloor & [\text{Since $\epsilon \geq 0$}]
        \end{align}
        \end{proof}
    \end{mdframed}
\end{enumerate}

\end{document}