\documentclass[12pt]{article}
\usepackage[margin=2.5cm]{geometry}
\usepackage{enumerate}
\usepackage{amsfonts}
\usepackage{amsmath}
\usepackage{fancyhdr}
\usepackage{amsmath}
\usepackage{amssymb}
\usepackage{amsthm}
\usepackage{mdframed}
\usepackage{graphicx}
\usepackage{subcaption}
\usepackage{adjustbox}
\usepackage{listings}
\usepackage{xcolor}
\usepackage{booktabs}
\usepackage[utf]{kotex}
\usepackage{hyperref}
\usepackage{accents}

\definecolor{codegreen}{rgb}{0,0.6,0}
\definecolor{codegray}{rgb}{0.5,0.5,0.5}
\definecolor{codepurple}{rgb}{0.58,0,0.82}
\definecolor{backcolour}{rgb}{0.95,0.95,0.92}

\lstdefinestyle{mystyle}{
    backgroundcolor=\color{backcolour},
    commentstyle=\color{codegreen},
    keywordstyle=\color{magenta},
    numberstyle=\tiny\color{codegray},
    stringstyle=\color{codepurple},
    basicstyle=\ttfamily\footnotesize,
    breakatwhitespace=false,
    breaklines=true,
    captionpos=b,
    keepspaces=true,
    numbers=left,
    numbersep=5pt,
    showspaces=false,
    showstringspaces=false,
    showtabs=false,
    tabsize=1
}

\lstset{style=mystyle}

\pagestyle{fancy}
\renewcommand{\headrulewidth}{0.4pt}
\lhead{CSC 373}
\rhead{Midterm 2 Version 1 Review}

\begin{document}
\title{Midterm 2 Version 1 Review}
\maketitle

\begin{enumerate}[1.]
    \item

    \begin{enumerate}[a)]
        \item 1100100
        \item $- \sum\limits_{i=0}^{n-1} 3^i$

        \bigskip

        \underline{\textbf{Notes:}}

        \bigskip

        \begin{itemize}
            \item Balanced Ternary

            \begin{itemize}
                \item is a way of representing numbers
                \item balanced ternary is in base 3, and has values 1,0 or -1

                \bigskip

                \begin{align}
                    \sum\limits_{i=0}^{n-1} d_i \cdot 3^i \text{ where $d_i \in \{0,1,-1\}$}
                \end{align}
            \end{itemize}
        \end{itemize}

        \item

        \begin{enumerate}[i.]
            \item \underline{$f(n) \in \Omega(n)$}

            \bigskip

            True (since $n^2 + 10n + 2 \geq cn$)

            \bigskip

            \item \underline{$g(n) \in \Omega(n)$}

            \bigskip

            False (Let $c = 100, n_0 = 100$. Then $100 \log_2 n < 100 n$)

            \bigskip

            \item \underline{$f(n) \in \mathcal{O}(g(n))$}

            \bigskip

            False ($f(n) = n^2 + 10n + 2$ grows faster than $g(n) = 100 \log_2 n$)

            \bigskip

            \item \underline{$f(n) \in \Theta(g(n))$}

            \bigskip

            True (Set $c_1 = -1 , c_2 = 1, n_1 = 100$. Then $c_1f(n) \leq g(n) \leq c_2f(n)$)

            \bigskip

            \item \underline{$g(n) \in \Theta(\log_3 n)$}

            \bigskip

            True (set $c_1  = -1, c_2 = 1, n_1 = 2$. Then $c_1 g(n) \leq \log_3 n \leq c_2 g(n)$)

            \bigskip

            \item \underline{$g(n) \in \Theta(\log_3 n)$}

            \bigskip

            False (set $c_1  = -1, c_2 = 1, n_1 = 2$. Then $c_1 g(n) \leq \log_3 n \leq c_2 g(n)$)

            \bigskip

            \item \underline{$f(n) + g(n) \in \Theta(f(n))$}

            \bigskip

            True (set $c_1  = -2, c_2 = 2, n_1 = 1$. Then $c_1 (f(n) + g(n)) \leq f(n) \leq c_2 (f(n) + g(n))$)

            \bigskip

        \end{enumerate}

        \bigskip

        \underline{\textbf{Notes:}}

        \bigskip

        \begin{itemize}
            \item
            $g \in \Omega(f):\:\exists c,n_o \in \mathbb{R}^{+},\:\forall n \in
            \mathbb{N},\:n \geq n_0 \Rightarrow g(n) \geq cf(n)$, where $f,g:\mathbb{N} \to \mathbb{R}^{\geq 0}$

            \item

            $g \in \mathcal{O}(f):\:\exists c,n_o \in \mathbb{R}^{+},\:\forall n \in
            \mathbb{N},\:n \geq n_0 \Rightarrow g(n) \leq cf(n)$, where $f,g:\mathbb{N} \to \mathbb{R}^{\geq 0}$

            \item

            $g \in \Theta(f):\: g \in \mathcal{O}(f) \land g \in \Omega(f)$

            or

            $g \in \Theta(f):\:\exists c_1,c_2,n_1 \in \mathbb{R}^{+}, \forall n \in \mathbb{N}, n \geq n_1
            \Rightarrow c_1g(n) \leq f(n) \leq c_2g(n)$, where $f,g:\:\mathbb{N} \to \mathbb{R}^{\geq 0}$
        \end{itemize}

        \item $i = 3^{2^k}$

        \bigskip

        Since

        \bigskip

        \begin{tabular}{|c|c|c|c|}
            \hline
            k & 0 & 1 & 2\\
            \hline
            i & 3 & 9 & 81\\
            \hline
            & $3^1$ & $3^2$ & $3^4$\\
            \hline
        \end{tabular}

        \item $k = \lceil \log_3(\log_2 n) - 1\rceil$
        \setcounter{equation}{0}
        \bigskip

        Since

        \begin{align}
            i^2 &\geq n\\
            3^{2^k} &\geq n^{1/2}\\
            2^k &\geq \log_3(n^{1/2})\\
            2^k &\geq (1/2)\log_3(n)\\
            k &\geq \log_2((1/2)\log_3(n))\\
            &\geq \log_2(\log_3(n)) - 1
        \end{align}

        \bigskip

        which gives $k = \lceil \log_2(\log_3(n)) - 1 \rceil$

    \end{enumerate}

    \item Let $n \in \mathbb{N}$. Assume $n \geq 3$.
    \setcounter{equation}{0}
    \bigskip

    I will prove $5^n + 50 < 6^n$ by induction.

    \bigskip

    \underline{\textbf{Base Step ($n = 3$):}}

    \bigskip

    Let $n = 3$.

    \bigskip

    Then,

    \begin{align}
        5^3 + 50 = 715 < 6^3 = 216
    \end{align}

    \bigskip

    So, the base case holds.

    \bigskip

    \underline{\textbf{Inductive Step}}

    \bigskip

    Let $n \in \mathbb{N}$. Assume ($5^n + 50 < 6^n$).

    \bigskip

    I need to show $5^{n+1} + 50 < 6^{n+1}$.

    \bigskip

    Indeed we have

    \begin{align}
        5^{n+1} + 50 &= 5^n5 + 50\\
        &= 5(5^n + 10)\\
        &< 5(5^n + 50)\\
        &< 56^n\\
        &< 66^n\\
        &< 6^{n=1}
    \end{align}

    \item
    \setcounter{equation}{0}
    \textbf{Negation(expanded):} $\forall a \in \mathbb{R}, \forall c_1, c_2, n_1 \in \mathbb{R}^+, \exists n \in \mathbb{N},
    (n \geq n_1) \land (c_1 g(n) > f(n)) \lor (f(n) > c_2g(n))$

    \bigskip
    \begin{proof}
        Let $a \in \mathbb{R}$.

        \bigskip

        I need to show $an+1 \notin \Theta(n^3)$. That is, $an + 1 \notin \mathcal{O}(n^3) \lor an+1 \notin \Omega(n^3)$.
        In other words, $\forall c, n_0 \in \mathbb{R}^+, \exists n \in \mathbb{N}, (n \geq n_0) \land (an+1 > c \cdot n^3)$ or
        $\forall c_1, n_1 \in \mathbb{R}^+, \exists n \in \mathbb{N}, (n \geq n_1) \land (an+1 < c_1 \cdot n^3)$.

        \bigskip

        Let $c_1, c_2, n_1 \in \mathbb{R}^+$, and let $n = \bigl\lceil max\bigl( n_1, \sqrt{\frac{2a}{c_1}}, \sqrt[3]{\frac{2}{c_1}}) \bigr) \bigr\rceil + 1$.

        \bigskip

        Then, we can write

        \begin{align}
            n &= \bigl\lceil max\bigl( n_1, \sqrt{\frac{2a}{c_1}}, \sqrt[3]{\frac{2}{c_1}}) \bigr) \bigr\rceil + 1 > \sqrt{\frac{2a}{c_1}}\\
            n^2 &> \frac{2a}{c_1}\\
            \frac{c_1 n^3}{2} &> an
        \end{align}

        \bigskip

        And

        \begin{align}
            n &= \bigl\lceil max\bigl( n_1, \sqrt{\frac{2a}{c_1}}, \sqrt[3]{\frac{2}{c_1}}) \bigr) \bigr\rceil + 1 > \sqrt[3]{\frac{2}{c_1}}\\
            \frac{c_1n^3}{2} &> 1
        \end{align}

        \bigskip

        Thus, we can conclude

        \begin{align}
            \frac{c_1n^3}{2} + \frac{c_1n^3}{2} &> an + 1\\
            c_1 \cdot n^3 &> an + 1
        \end{align}
    \end{proof}

    \bigskip

    \underline{\textbf{Notes:}}

    \bigskip

    \begin{itemize}
        \item
        $g \in \Omega(f):\:\exists c,n_o \in \mathbb{R}^{+},\:\forall n \in
        \mathbb{N},\:n \geq n_0 \Rightarrow g(n) \geq cf(n)$, where $f,g:\mathbb{N} \to \mathbb{R}^{\geq 0}$

        \item

        $g \in \mathcal{O}(f):\:\exists c,n_o \in \mathbb{R}^{+},\:\forall n \in
        \mathbb{N},\:n \geq n_0 \Rightarrow g(n) \leq cf(n)$, where $f,g:\mathbb{N} \to \mathbb{R}^{\geq 0}$

        \item

        $g \in \Theta(f):\: g \in \mathcal{O}(f) \land g \in \Omega(f)$

        or

        $g \in \Theta(f):\:\exists c_1,c_2,n_1 \in \mathbb{R}^{+}, \forall n \in \mathbb{N}, n \geq n_1
        \Rightarrow c_1g(n) \leq f(n) \leq c_2g(n)$, where $f,g:\:\mathbb{N} \to \mathbb{R}^{\geq 0}$
    \end{itemize}

    \item

    \begin{enumerate}[a)]
        \item

        I need to evaluate the total number of iterations of loop 2.

        \bigskip

        First, I need to evaluate the number of iterations of loop 2 per iteration of loop 1.

        \bigskip

        The code tells us that the value of $j$ increases by 3 per iteration $k$. That is, $j = 3k$.

        \bigskip

        Since the inner loop ends when $j \geq i$, the earliest iteration at which
        the loop ends per iteration of the outerloop is $k = \lceil \frac{i}{3} \rceil$.

        \bigskip

        Finally, the outer loop starts from $i = 0$ to $i = n^2$.

        \bigskip

        Thus, the total number of iterations in loop 2 is:

        \setcounter{equation}{0}
        \begin{align}
            \sum\limits_{i=0}^{n^2} \frac{i}{3} = \frac{n^2 (n^2 + 1)}{6}
        \end{align}

        \bigskip

        \begin{mdframed}
            \underline{\textbf{Correct Solution:}}

            \bigskip

            I need to evaluate the total number of iterations of loop 2.

            \bigskip

            First, I need to evaluate the number of iterations of loop 2 per iteration of loop 1.

            \bigskip

            The code tells us that the value of $j$ increases by 3 per iteration $k$. That is, $j = 3k$.

            \bigskip

            Since the inner loop ends when $j \geq i$, the earliest iteration at which
            the loop ends per iteration of the outerloop is $k = \lceil \frac{i}{3} \rceil$.

            \bigskip

            Finally, the outer loop starts from $i = 0$ to $i = n^2$.

            \bigskip

            Thus, the total number of iterations in loop 2 is:

            \setcounter{equation}{0}
            \begin{align}
                \sum\limits_{i=0}^{\color{red}n^2-1\color{black}} \frac{i}{3} = \frac{\color{red}(n^2-1)n^2\color{black}}{6}
            \end{align}

            \bigskip

        \end{mdframed}

        \item

        \underline{\textbf{Finding upperbound:}}

        \bigskip

        Let $n \in \mathbb{N}$, and \textbf{lst} be arbitrary list of integers of length $n$.

        \bigskip

        The \textbf{if} condition occurs when $i$ value is odd.

        \bigskip

        Then, the inner loop triggers, causing the values in \textbf{lst[$i+1$]} to \textbf{lst[$n-1$]}
        to become even, and the inner loop to iterate $n - (i+1) - 1 = n - i$ times.

        \bigskip

        Then, the outer loop iterates until the end, giving the outerloop
        to have $n$ iterations.
        \bigskip

        Thus, the alg \textbf{$my\_alg$} has upper bound worst case of $\mathcal{O}(2n-i)$ or $\mathcal{O}{n}$

        \bigskip

        So, the upperbound of $my\_alg$ is $\mathcal{O}(n)$

        \bigskip

        \underline{\textbf{Finding worst-case lowerbound:}}

        \bigskip

        Let $n \in \mathbb{N}$, and let $n = [1,2,3,...,n]$.

        \bigskip

        Then, at $n[i] = 3$, the \textbf{if} statement will occur.

        \bigskip

        Then, the inner loop will run from $i + 1$ to $n$, causing the rest of elements
        in \textbf{lst} to have even values, and the inner loop to have $n - (2+1) + 1 = n - 2$ iterations.

        \bigskip

        Then, the outer loop runs until $i = 3$ to $i = n$ without the \textbf{if} condition,
        resulting in the outloop to have $n$ iterations.

        \bigskip
        Thus, the worst-case lower bound of \textbf{$my\_alg$} is $\Omega(2n - 2)$ or $\Omega(n)$.

        \bigskip

        \underline{\textbf{Notes:}}

        \bigskip

        \begin{itemize}
            \item Upperbound and lowerbound worstcase is determined by input :)

            \bigskip

            Upperbound $\to$ arbitrary input

            \bigskip

            Lowerbound $\to$ not arbitrary, but produces wost case values

            \bigskip

            i.e. $[1,2,3,4,...,n]$
        \end{itemize}

    \end{enumerate}

    \bigskip

\end{enumerate}

\end{document}