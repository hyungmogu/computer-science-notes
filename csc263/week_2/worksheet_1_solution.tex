\documentclass[12pt]{article}
\usepackage[margin=2.5cm]{geometry}
\usepackage{enumerate}
\usepackage{amsfonts}
\usepackage{amsmath}
\usepackage{fancyhdr}
\usepackage{amsmath}
\usepackage{amssymb}
\usepackage{amsthm}
\usepackage{mdframed}
\usepackage{graphicx}
\usepackage{subcaption}
\usepackage{adjustbox}
\usepackage{listings}
\usepackage{xcolor}
\usepackage{booktabs}
\usepackage[utf]{kotex}

\definecolor{codegreen}{rgb}{0,0.6,0}
\definecolor{codegray}{rgb}{0.5,0.5,0.5}
\definecolor{codepurple}{rgb}{0.58,0,0.82}
\definecolor{backcolour}{rgb}{0.95,0.95,0.92}

\lstdefinestyle{mystyle}{
    backgroundcolor=\color{backcolour},
    commentstyle=\color{codegreen},
    keywordstyle=\color{magenta},
    numberstyle=\tiny\color{codegray},
    stringstyle=\color{codepurple},
    basicstyle=\ttfamily\footnotesize,
    breakatwhitespace=false,
    breaklines=true,
    captionpos=b,
    keepspaces=true,
    numbers=left,
    numbersep=5pt,
    showspaces=false,
    showstringspaces=false,
    showtabs=false,
    tabsize=1
}

\lstset{style=mystyle}

\begin{document}
\title{CSC263 Worksheet 1 Solution}
\author{Hyungmo Gu}
\maketitle

\section*{Question 1}
\begin{enumerate}[a.]
    \item

    \begin{proof}
        Assume the statement $P(115)$ is true. That is, $\sum\limits_{i=0}^{i=115} 2^i = 2^{115+1}$.

        \bigskip

        We need to prove $\sum\limits_{i=0}^{i=116} 2^i = 2^{116+1}$.

        \bigskip

        Starting from the left, we can write

        \bigskip

        \begin{align}
            \sum\limits_{i=0}^{i=116} 2^i = \sum\limits_{i=0}^{i=115} 2^i + 2
        \end{align}

        \bigskip

        Then, using the assumption $\sum\limits_{i=0}^{i=115} 2^i = 2^{115+1}$, we
        can conclude

        \begin{align}
            \sum\limits_{i=0}^{i=116} 2^i &= 2^{115+1} + 2^{116}\\
            &= 2^{116} + 2^{116}\\
            &= 2^{116}(1 + 1)\\
            &= 2 \cdot 2^{116}\\
            &= 2^{116+1}
        \end{align}
    \end{proof}

    \item

    \begin{proof}
        No. The statement is not true for every natural natural number.

        \bigskip

        We will prove this by counter example. That is, $\exists n \in \mathbb{N},\:
        \sum\limits_{i=0}^{i=n} 2^i \neq 2^{n+1}$.

        \bigskip

        Let $n = 0$.

        \bigskip

        Then, starting from the left hand side, it follows from the fact $n = 0$ that

        \setcounter{equation}{0}
        \begin{align}
            \sum\limits_{i=0}^{i=n} 2^i &= \sum\limits_{i=0}^{i=0} 2^i\\
            &= 0
        \end{align}


        Now, for the right hand side, using the same fact, we can write

        \begin{align}
            2^{0+1} &= 2^1\\
            &= 2
        \end{align}
    \end{proof}

\end{enumerate}

\bigskip

\section*{Question 2}

\section*{Question 3}

\section*{Question 4}

\end{document}