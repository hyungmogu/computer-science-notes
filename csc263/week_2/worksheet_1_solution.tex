\documentclass[12pt]{article}
\usepackage[margin=2.5cm]{geometry}
\usepackage{enumerate}
\usepackage{amsfonts}
\usepackage{amsmath}
\usepackage{fancyhdr}
\usepackage{amsmath}
\usepackage{amssymb}
\usepackage{amsthm}
\usepackage{mdframed}
\usepackage{graphicx}
\usepackage{subcaption}
\usepackage{adjustbox}
\usepackage{listings}
\usepackage{xcolor}
\usepackage{booktabs}
\usepackage[utf]{kotex}

\definecolor{codegreen}{rgb}{0,0.6,0}
\definecolor{codegray}{rgb}{0.5,0.5,0.5}
\definecolor{codepurple}{rgb}{0.58,0,0.82}
\definecolor{backcolour}{rgb}{0.95,0.95,0.92}

\lstdefinestyle{mystyle}{
    backgroundcolor=\color{backcolour},
    commentstyle=\color{codegreen},
    keywordstyle=\color{magenta},
    numberstyle=\tiny\color{codegray},
    stringstyle=\color{codepurple},
    basicstyle=\ttfamily\footnotesize,
    breakatwhitespace=false,
    breaklines=true,
    captionpos=b,
    keepspaces=true,
    numbers=left,
    numbersep=5pt,
    showspaces=false,
    showstringspaces=false,
    showtabs=false,
    tabsize=1
}

\lstset{style=mystyle}

\begin{document}
\title{CSC236 Worksheet 1 Solution}
\author{Hyungmo Gu}
\maketitle

\section*{Question 1}
\begin{enumerate}[a.]
    \item

    \begin{proof}
        Assume the statement $P(115)$ is true. That is, $\sum\limits_{i=0}^{i=115} 2^i = 2^{115+1}$.

        \bigskip

        We need to prove $\sum\limits_{i=0}^{i=116} 2^i = 2^{116+1}$.

        \bigskip

        Starting from the left, we can write

        \bigskip

        \begin{align}
            \sum\limits_{i=0}^{i=116} 2^i = \sum\limits_{i=0}^{i=115} 2^i + 2
        \end{align}

        \bigskip

        Then, using the assumption $\sum\limits_{i=0}^{i=115} 2^i = 2^{115+1}$, we
        can conclude

        \begin{align}
            \sum\limits_{i=0}^{i=116} 2^i &= 2^{115+1} + 2^{116}\\
            &= 2^{116} + 2^{116}\\
            &= 2^{116}(1 + 1)\\
            &= 2 \cdot 2^{116}\\
            &= 2^{116+1}
        \end{align}
    \end{proof}

    \item

    \begin{proof}
        No. The statement is not true for every natural natural number.

        \bigskip

        We will prove this by counter example. That is, $\exists n \in \mathbb{N},\:
        \sum\limits_{i=0}^{i=n} 2^i \neq 2^{n+1}$.

        \bigskip

        Let $n = 0$.

        \bigskip

        Then, starting from the left hand side, it follows from the fact $n = 0$ that

        \setcounter{equation}{0}
        \begin{align}
            \sum\limits_{i=0}^{i=n} 2^i &= \sum\limits_{i=0}^{i=0} 2^i\\
            &= 0
        \end{align}


        Now, for the right hand side, using the same fact, we can write

        \begin{align}
            2^{0+1} &= 2^1\\
            &= 2
        \end{align}
    \end{proof}

\end{enumerate}

\bigskip

\section*{Question 2}
\begin{itemize}
    \item

    \underline{\textbf{Statement:}} $\forall n \in \mathbb{N},\:\exists d \in \mathbb{Z},\:8^n-1 = 7d$

    \bigskip

    \begin{proof}

    We will prove this statement by induction on $n$.

    \bigskip

    \underline{\textbf{Base Case:}}

    \bigskip

    Let $n = 0$.

    \bigskip

    We need to prove $8^n-1 = 7 \cdot 0$.

    \bigskip

    Starting from the left hand side, using the fact $n = 0$, we can conclude,
    \setcounter{equation}{0}
    \begin{align}
        8^{0} - 1 &= 1 - 1\\
        &= 0\\
        &= 7 \cdot 0
    \end{align}

    \bigskip

    \underline{\textbf{Inductive Step:}}

    \bigskip

    Let $n \in \mathbb{N}$. Assume there is an integer $d$ such that $8^n-1 = 7d$.

    \bigskip

    We need to prove there is an integer $\tilde{d}$ such that $8^{n+1} - 1 = 7 \tilde{d}$.

    \bigskip

    Let $\tilde{d} = 8^n+d$.

    \bigskip

    Starting from the left hand side, we can write

    \bigskip
    \begin{align}
        8^{n+1} - 1 &= 8^n + 8^n + 8^n + 8^n + 8^n + 8^n + 8^n + 8^n - 1\\
        &= 8^n + 8^n + 8^n + 8^n + 8^n + 8^n + 8^n + (8^n - 1)
    \end{align}

    \bigskip

    Then, using inductive hypothesis, i.e. $8^n-1 = 7d$, we can conclude

    \begin{align}
        8^{n+1} - 1 &= 8^n + 8^n + 8^n + 8^n + 8^n + 8^n + 8^n + 7d\\
        &= 7 \cdot 8^n + 7d\\
        &= 7 \cdot (8^n + d)\\
        &= 7 \cdot \tilde{d}
    \end{align}

    \end{proof}

    % \bigskip

    % \begin{mdframed}
    %     \underline{\textbf{Rough Work:}}

    %     \bigskip

    %     \begin{enumerate}[1.]
    %         \item Base Case

    %         \begin{mdframed}
    %         \underline{\textbf{Base Case:}}

    %         \bigskip

    %         Let $n = 0$.

    %         \bigskip

    %         We need to prove $8^n-1 = 7 \cdot 0$.

    %         \bigskip

    %         Starting from the left hand side, using the fact $n = 0$, we can conclude,
    %         \setcounter{equation}{0}
    %         \begin{align}
    %             8^{0} - 1 &= 1 - 1\\
    %             &= 0\\
    %             &= 7 \cdot 0
    %         \end{align}

    %         \end{mdframed}

    %         \item Inductive Step

    %         \begin{mdframed}
    %         \underline{\textbf{Inductive Step:}}

    %         \bigskip

    %         Let $n \in \mathbb{N}$. Assume there is an integer $d$ such that $8^n-1 = 7d$.

    %         \bigskip

    %         We need to prove there is an integer $\tilde{d}$ such that $8^{n+1} - 1 = 7 \tilde{d}$.

    %         \bigskip

    %         Let $\tilde{d} = 8^n+d$.

    %         \bigskip

    %         Starting from the left hand side, we can write

    %         \bigskip
    %         \begin{align}
    %             8^{n+1} - 1 &= 8^n + 8^n + 8^n + 8^n + 8^n + 8^n + 8^n + 8^n - 1\\
    %             &= 8^n + 8^n + 8^n + 8^n + 8^n + 8^n + 8^n + (8^n - 1)
    %         \end{align}

    %         \bigskip

    %         Then, using inductive hypothesis, i.e. $8^n-1 = 7d$, we can conclude

    %         \begin{align}
    %             8^{n+1} - 1 &= 8^n + 8^n + 8^n + 8^n + 8^n + 8^n + 8^n + 7d\\
    %             &= 7 \cdot 8^n + 7d\\
    %             &= 7 \cdot (8^n + d)\\
    %             &= 7 \cdot \tilde{d}
    %         \end{align}

    %         \bigskip
    %         \end{mdframed}
    %     \end{enumerate}
    % \end{mdframed}
\end{itemize}

\section*{Question 3}
\begin{itemize}
    \item

    \underline{\textbf{Statement:}} $\forall n \in \mathbb{N},\:\exists m \in \mathbb{N}$,
    the units digit of $7^n$ is the same as the units digit of $3^m$.

    \bigskip

    \begin{proof}

    We will prove this statement by induction on $n$.

    \bigskip

    \underline{\textbf{Base Case:}}

    \bigskip

    Let $n = 0$.

    \bigskip

    We need to prove there is a natural number $m$ such that the
    units digit of $7^0 = 1$ is the same as the units digit of $3^m$. That is,
    the ones place of the number $3^m$ is 1.

    \bigskip

    Let $m = 0$.

    \bigskip

    Then, using this fact, we can conclude
    \setcounter{equation}{0}
    \begin{align}
        3^0 = 1
    \end{align}

    \bigskip

    \underline{\textbf{Inductive Step:}}

    \bigskip

    Let $n \in \mathbb{N}$. Assume there exists a natural number $m$ such that
    the units digit of $7^n$ is the same as the units digit of $3^m$.

    \bigskip

    We need to prove there is a natural number $\tilde{m}$ such that the
    units digit of $7^{n+1}$ is the same as the units digit of $3^{\tilde{m}}$.
    That is, the ones digit of $7^{n+1}$ is the same as the ones digit of $3^{\tilde{m}}$.

    \bigskip

    Let $\tilde{m} = m + 9$.

    \bigskip

    Starting with $7^{n+1}$, we can write

    \begin{align}
        7^{n+1} = 7 \cdot 7^n.
    \end{align}

    \bigskip

    Then, it follows from above fact that the ones digit of $7^{n+1}$
    is 7 times the ones digit of $7^n$.

    \bigskip

    Now, for $3^{\tilde{m}}$, using the fact $\tilde{m} = m + 9$, we
    can write

    \begin{align}
        3^{\tilde{m}} &= 3^{m+9}\\
        &= 3^9 \cdot 3^m\\
        &= 27 \cdot 3^m
    \end{align}

    \bigskip

    Then, it follows from above fact that that the ones digit of $3^{\tilde{m}}$
    is 7 times the ones digit of $3^m$.

    \bigskip

    Then, using inductive hypothesis, we can conclude the ones digit of $3^{\tilde{m}}$
    is 7 times the ones digit of $7^n$.

    \end{proof}

    % \begin{mdframed}
    %     \underline{\textbf{Rough Work:}}

    %     \bigskip

    %     We will prove this statement by induction on $n$.

    %     \bigskip

    %     \begin{enumerate}[1.]
    %         \item Base Case

    %         \begin{mdframed}

    %         \underline{\textbf{Base Case:}}

    %         \bigskip

    %         Let $n = 0$.

    %         \bigskip

    %         We need to prove there is a natural number $m$ such that the
    %         units digit of $7^0 = 1$ is the same as the units digit of $3^m$. That is,
    %         the ones place of the number $3^m$ is 1.

    %         \bigskip

    %         Let $m = 0$.

    %         \bigskip

    %         Then, using this fact, we can conclude
    %         \setcounter{equation}{0}
    %         \begin{align}
    %             3^0 = 1
    %         \end{align}

    %         \end{mdframed}

    %         \item Inductive Step

    %         Let $n \in \mathbb{N}$. Assume there exists a natural number $m$ such that
    %         the units digit of $7^n$ is the same as the units digit of $3^m$.

    %         \bigskip

    %         We need to prove there is a natural number $\tilde{m}$ such that the
    %         units digit of $7^{n+1}$ is the same as the units digit of $3^{\tilde{m}}$.
    %         That is, the ones digit of $7^{n+1}$ is the same as the ones digit of $3^{\tilde{m}}$.

    %         \bigskip

    %         Let $\tilde{m} = m + 9$.

    %         \bigskip

    %         \begin{itemize}
    %             \item Show $7^{n+1} = 7 \cdot 7^n$.

    %             \begin{mdframed}
    %             Starting with $7^{n+1}$, we can write

    %             \begin{align}
    %                 7^{n+1} = 7 \cdot 7^n.
    %             \end{align}

    %             \end{mdframed}

    %             \item Show the ones digit of $7^{n+1}$ is 7 times the ones digit
    %             of $7^n$.

    %             \begin{mdframed}
    %             Then, it follows from above fact that the ones digit of $7^{n+1}$
    %             is 7 times the ones digit of $7^n$.
    %             \end{mdframed}

    %             \item Show $3^{\tilde{m}}$ also has 7 times the ones digit of $7^n$

    %             \begin{mdframed}
    %             Now, for $3^{\tilde{m}}$, using the fact $\tilde{m} = m + 9$, we
    %             can write

    %             \begin{align}
    %                 3^{\tilde{m}} &= 3^{m+9}\\
    %                 &= 3^9 \cdot 3^m\\
    %                 &= 27 \cdot 3^m
    %             \end{align}

    %             \bigskip

    %             Then, it follows from above fact that that the ones digit of $3^{\tilde{m}}$
    %             is 7 times the ones digit of $3^m$.

    %             \bigskip

    %             Then, using inductive hypothesis, we can conclude the ones digit of $3^{\tilde{m}}$
    %             is 7 times the ones digit of $7^n$.
    %             \end{mdframed}

    %         \end{itemize}

    %         \begin{mdframed}

    %         \underline{\textbf{Inductive Step:}}

    %         \bigskip

    %         Let $n \in \mathbb{N}$. Assume there exists a natural number $m$ such that
    %         the units digit of $7^n$ is the same as the units digit of $3^m$.

    %         \bigskip

    %         We need to prove there is a natural number $\tilde{m}$ such that the
    %         units digit of $7^{n+1}$ is the same as the units digit of $3^{\tilde{m}}$.
    %         That is, the ones digit of $7^{n+1}$ is the same as the ones digit of $3^{\tilde{m}}$.

    %         \bigskip

    %         Let $\tilde{m} = m + 9$.

    %         \bigskip

    %         Starting with $7^{n+1}$, we can write

    %         \begin{align}
    %             7^{n+1} = 7 \cdot 7^n.
    %         \end{align}

    %         \bigskip

    %         Then, it follows from above fact that the ones digit of $7^{n+1}$
    %         is 7 times the ones digit of $7^n$.

    %         \bigskip

    %         Now, for $3^{\tilde{m}}$, using the fact $\tilde{m} = m + 9$, we
    %         can write

    %         \begin{align}
    %             3^{\tilde{m}} &= 3^{m+9}\\
    %             &= 3^9 \cdot 3^m\\
    %             &= 27 \cdot 3^m
    %         \end{align}

    %         \bigskip

    %         Then, it follows from above fact that that the ones digit of $3^{\tilde{m}}$
    %         is 7 times the ones digit of $3^m$.

    %         \bigskip

    %         Then, using inductive hypothesis, we can conclude the ones digit of $3^{\tilde{m}}$
    %         is 7 times the ones digit of $7^n$.

    %         \end{mdframed}
    %     \end{enumerate}

    % \end{mdframed}

    \bigskip

    \begin{mdframed}
        \underline{\textbf{Correct Solution:}}

        \bigskip

        \underline{\textbf{Statement:}} $\forall n \in \mathbb{N},\:\exists m \in \mathbb{N}$,
        the units digit of $7^n$ is the same as the units digit of $3^m$.

        \bigskip

        \begin{proof}

        We will prove this statement by induction on $n$.

        \bigskip

        \underline{\textbf{Base Case:}}

        \bigskip

        Let $n = 0$.

        \bigskip

        We need to prove there is a natural number $m$ such that the
        units digit of $7^0 = 1$ is the same as the units digit of $3^m$. That is,
        the ones place of the number $3^m$ is 1.

        \bigskip

        Let $m = 0$.

        \bigskip

        Then, using this fact, we can conclude
        \setcounter{equation}{0}
        \begin{align}
            3^0 = 1
        \end{align}

        \bigskip

        \underline{\textbf{Inductive Step:}}

        \bigskip

        Let $n \in \mathbb{N}$. Assume there exists a natural number $m$ such that
        the units digit of $7^n$ is the same as the units digit of $3^m$.

        \bigskip

        We need to prove there is a natural number $\tilde{m}$ such that the
        units digit of $7^{n+1}$ is the same as the units digit of $3^{\tilde{m}}$.
        That is, the ones digit of $7^{n+1}$ is the same as the ones digit of $3^{\tilde{m}}$.

        \bigskip

        Let $\tilde{m} = m + \color{red}3\color{black}$.

        \bigskip

        Starting with $7^{n+1}$, we can write

        \begin{align}
            7^{n+1} = 7 \cdot 7^n.
        \end{align}

        \bigskip

        Then, it follows from above fact that the ones digit of $7^{n+1}$
        is 7 times the ones digit of $7^n$.

        \bigskip

        Now, for $3^{\tilde{m}}$, using the fact $\tilde{m} = m + \color{red}3\color{black}$, we
        can write

        \begin{align}
            3^{\tilde{m}} &= 3^{m+\color{red}3\color{black}}\\
            &= 3^{\color{red}3\color{black}} \cdot 3^m\\
            &= 27 \cdot 3^m
        \end{align}

        \bigskip

        Then, it follows from above fact that that the ones digit of $3^{\tilde{m}}$
        is 7 times the ones digit of $3^m$.

        \bigskip

        Then, using inductive hypothesis, we can conclude the ones digit of $3^{\tilde{m}}$
        is 7 times the ones digit of $7^n$.

        \end{proof}
    \end{mdframed}
\end{itemize}

\section*{Question 4}
\begin{itemize}
    \item

    \begin{proof}
        Let $m = 7$. Let $n \in \mathbb{N}$. Assume $n \geq m$.

        \bigskip

        We need to prove $4^n \geq 5n^4 + 6$.

        \bigskip

        We will do so by induction on $n$.

        \bigskip

        \underline{\textbf{Base Case ($n = 7$):}}

        \bigskip

        Let $n = 7$.

        \bigskip

        We need to prove $4^n \geq 5n^4 + 6$.

        \bigskip

        Starting with the left hand side, using the fact $n = 7$, we
        can calculate
        \setcounter{equation}{0}
        \begin{align}
            4^7 = 16384
        \end{align}

        \bigskip

        Now, for the right hand side, using the same fact, we can conclude

        \begin{align}
            5 \cdot 7^4 + 6 = 12011
        \end{align}

        \bigskip

        \underline{\textbf{Inductive Step:}}

        \bigskip

        Let $n \in \mathbb{N}$. Assume that $4^n \geq 5n^4 + 6$.

        \bigskip

        We need to prove $4^{n+1} \geq 5 \cdot (n+1)^4 + 6$.

        \bigskip

        Starting from the left hand side, we can write

        \begin{align}
        4^{n+1} &= 4^n + 4^n + 4^n + 4^n
        \end{align}

        \bigskip

        Then, by inductive hypothesis, i.e. $4^n \geq 5n^4 + 6$,

        \begin{align}
        4^{n+1} &\geq (5n^4 + 6) + (5n^4 + 6) + (5n^4 + 6) + (5n^4 + 6)\\
        &= 5n^4 + 5n^4 + 5n^4 + 5n^4 + 24\\
        &= (5n^4 + 5 \cdot n \cdot n^3 + 5 \cdot n^2 \cdot n^2 + 5 \cdot n^3 \cdot n) + 24
        \end{align}

        \bigskip

        Then, because we know $n \geq 7$ from the header, we can conclude

        \begin{align}
        4^{n+1} &\geq (5n^4 + 5 \cdot 7 \cdot n^3 + 5 \cdot 7^2 \cdot n^2 + 5 \cdot 7^3 \cdot n) + 24\\
        &\geq (5n^4 + 5 \cdot 4 \cdot n^3 + 5 \cdot 6 \cdot n^2 + 5 \cdot 4 \cdot n) + 5 + 19\\
        &\geq 5 \cdot (n^4 + 4n^3 + 6n^2 + 4n + 1) + 19\\
        &= 5 \cdot \bigl((n+1)^2 \cdot (n+1)^2 \bigr) + 19\\
        &= 5 \cdot (n+1)^4 + 19\\
        &> 5 \cdot (n+1)^4 + 6
        \end{align}
    \end{proof}


    % \begin{mdframed}

    %     \underline{\textbf{Rough Work:}}

    %     \bigskip

    %     Let $m = 7$. Let $n \in \mathbb{N}$. Assume $n \geq m$.

    %     \bigskip

    %     We need to prove $4^n \geq 5n^4 + 6$.

    %     \bigskip

    %     We will do so by induction on $n$.

    %     \begin{enumerate}[1.]
    %         \item Base Case ($n = 7$)

    %         \begin{mdframed}
    %         \underline{\textbf{Base Case ($n = 7$):}}

    %         \bigskip

    %         Let $n = 7$.

    %         \bigskip

    %         We need to prove $4^n \geq 5n^4 + 6$.

    %         \bigskip

    %         Starting with the left hand side, using the fact $n = 7$, we
    %         can calculate
    %         \setcounter{equation}{0}
    %         \begin{align}
    %             4^7 = 16384
    %         \end{align}

    %         \bigskip

    %         Now, for the right hand side, using the same fact, we can conclude

    %         \begin{align}
    %             5 \cdot 7^4 + 6 = 12011
    %         \end{align}

    %         \end{mdframed}

    %         \item Inductive Step

    %         \begin{mdframed}

    %         \underline{\textbf{Inductive Step:}}

    %         \bigskip

    %         Let $n \in \mathbb{N}$. Assume that $4^n \geq 5n^4 + 6$.

    %         \bigskip

    %         We need to prove $4^{n+1} \geq 5 \cdot (n+1)^4 + 6$.

    %         \bigskip

    %         Starting from the left hand side, we can write

    %         \begin{align}
    %         4^{n+1} &= 4^n + 4^n + 4^n + 4^n
    %         \end{align}

    %         \bigskip

    %         Then, by inductive hypothesis, i.e. $4^n \geq 5n^4 + 6$,

    %         \begin{align}
    %         4^{n+1} &\geq (5n^4 + 6) + (5n^4 + 6) + (5n^4 + 6) + (5n^4 + 6)\\
    %         &= 5n^4 + 5n^4 + 5n^4 + 5n^4 + 24\\
    %         &= (5n^4 + 5 \cdot n \cdot n^3 + 5 \cdot n^2 \cdot n^2 + 5 \cdot n^3 \cdot n) + 24
    %         \end{align}

    %         \bigskip

    %         Then, because we know $n \geq 7$ from the header, we can conclude

    %         \begin{align}
    %         4^{n+1} &\geq (5n^4 + 5 \cdot 7 \cdot n^3 + 5 \cdot 7^2 \cdot n^2 + 5 \cdot 7^3 \cdot n) + 24\\
    %         &\geq (5n^4 + 5 \cdot 4 \cdot n^3 + 5 \cdot 6 \cdot n^2 + 5 \cdot 4 \cdot n) + 5 + 19\\
    %         &\geq 5 \cdot (n^4 + 4n^3 + 6n^2 + 4n + 1) + 19\\
    %         &= 5 \cdot \bigl((n+1)^2 \cdot (n+1)^2 \bigr) + 19\\
    %         &= 5 \cdot (n+1)^4 + 19\\
    %         &> 5 \cdot (n+1)^4 + 6
    %         \end{align}

    %         \end{mdframed}
    %     \end{enumerate}
    % \end{mdframed}
\end{itemize}

\end{document}